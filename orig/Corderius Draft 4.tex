\documentclass{article}
\usepackage{fontspec}
\defaultfontfeatures{Mapping=tex-text}
\usepackage{xunicode}
\usepackage{metalogo}
\usepackage{ccicons}

\setmainfont[Numbers={OldStyle}]{Libertinus Serif}
\setsansfont{Libertinus Sans}

\usepackage{polyglossia}
\setdefaultlanguage[variant=modern,hyphenation=classic,usej=false]{latin}
\setotherlanguage[]{french}
\setotherlanguage[variant=british]{english}
\newcommand{\fr}[1]{\foreignlanguage{french}{\emph{#1}}}

\usepackage{hyperref}
\usepackage{multicol}
\usepackage{epigraph}
\usepackage{verse}


\usepackage{unnumberedtotoc}
\renewcommand{\thesection}{\Roman{section}}
\setcounter{tocdepth}{1}

\widowpenalty=5000
\clubpenalty=5000

\begin{document}
\thispagestyle{empty}
\setlength{\parindent}{0em}
\begin{center}
{\huge Maturini Corderii}

\vspace{2em}

{\huge \textbf{COLLOQUIORUM SCHOLASTICORUM LIBRI IIII}}

\vspace{2em}

{\huge ad pueros in sermone Latino paulatim exercendos}


\vfill

{\huge Recognovit brevique adnotatione critica instruxit}

\vspace{2em}

{\huge Fergus J. P. Walsh}
\end{center}

\vspace{2em}

{\begin{english}\huge Draft 4. \today\end{english}}

\pagebreak
\begin{english}
\tableofcontents

\section*{Note}
\textcopyright~ Fergus J. P. Walsh 2021

\ccPublicDomain~ The Latin text is in the Public Domain.

\ccbynd~ This edition, preface and notes are released under the Creative Commons Attribution-Share Alike 4.0 International Licence:

``The licensor permits others to create and distribute derivative works, but only under the same or a compatible license. The licensor permits others to copy, distribute, display, and perform the work, including for commerical purposes.''

See \url{www.creativecommons.org/licenses/by-sa/4.0/legalcode} for full details of this licence.
\vfill
\setlength{\parindent}{1em}

\epigraph{They argued a little about Sophocles, then in low water Durham said it was a pose in ``us undergraduates'' to ignore him and advised Fethersonhaugh to re-read the \emph{Ajax} with his eye on the characters rather than the author; he would learn more that way, both about Greek grammar and life.}{E. M. Forster \emph{Maurice}}

\pagebreak
\phantomsection
\addcontentsline{toc}{section}{Preface}
\section*{Preface}
\subsection*{Editing principles}

This is the first public release of a draft critical edition of Mathurin Cordier's \emph{Colloquia Scholastica}. Like many Early Modern Latin texts, the \emph{Colloquia Scholastica} is not available in any modern print edition. From its first printing in 1564 until the early 19\textsuperscript{th} century, the \emph{Colloquia} was a popular classroom text, and was reprinted, translated and edited countless times.\footnote{For an almost exhaustive list, see Bömer, A., \emph{Die lateinischen Schülergespräche der Humanisten}, Vol. 2, J. Harrwitz Nachfolger, Berlin, 1899, pp. 202--205} Many of these editions can be found online as scans, and a digital text is available thanks to the work of the University of Kentucky Institute for Latin Studies.\footnote{Formerly at \url{stoa.org/colloquia} and currently available at \url{daedalus.umkc.edu/sandbox/colloquia/index.html}}

Comparing the Kentucky text to earlier editions, however, I noticed that there was variation in the orthography, character names, place names and vernacular used in the text. Moreover some editions divided the work into five books rather than four, some had a different number and ordering of the dialogues and some omitted the introductory summaries (\emph{argumenta}). Yet after discovering the \emph{editio princeps} (S.) and Jean Durant's 1570 edition (D.), I set out to produce a new critical edition of the \emph{Colloquia Scholastica}.\footnote{The Kentucky text is cited as Corderius, Maturinus, d. 1564. \emph{Maturini Corderii Colloquiorum Scholasticorum Libri quattuor. ad pueros in quotidiano tempore paulatim exercendos.} Ed. stereotypa. Lipsiae. sumptibus Ottonis Holtze. 1867-1872. 771 p. (2 vols), however I have been unable to locate this edition online or in any library.}

In 1563, Henri Estienne encouraged Cordier, who like Estienne was living in Geneva as a Protestant refuge from France, to collect and publish the dialogues he had presumably been working on for much of his teaching career. Cordier died in the following year, but not before the first edition of the \emph{Colloquia Scholastica} was published. It proved a run away success and was immediately pirated in Paris by Gabriel Buon and Thomas de Straton in Lyon.\footnote{Buon famously removed all of Cordier's anti-Catholic sentiment from the work. An exemplar of Buon's edition is in the Bern Universitätsbibliothek Klein Z 404 \url{baselbern.swissbib.ch/Record/365126411} and de Straton's in the Bibliothèque Nationale De France FRBNF30269908 \url{catalogue.bnf.fr/ark:/12148/cb30269908c.public}}

In 1570, Jean Durant publised an expanded edition: the \emph{princeps} had contained 195 dialogues (with \emph{argumenta}) but Durant added twenty-six new dialogues and a number of corrections from Cordier's own notes, while also removing the \emph{argumenta}.

Most subsequent editions were based on Durant's, as is this edition, which includes the additional dialogues and follows the ordering in D. It is impossible to verify Durant's claim that they came from Cordier's own papers, but the reader will note in the \emph{apparatus criticus} that Durant's emendations often match those in the Emenda \& Corrigenda page of Estienne's (S. Corr.), and so I feel confident in following the text of D. when if differs from that of S. I have re-inserted the \emph{argumeta} from S. and the twenty-six new dialogues can be easily identified by their absence. Misprints and minor corrections have not been noted in the \emph{apparatus} and abbreviations have been silently expanded. Several dialogues feature text in the vernacular, which was changed from French to the local vernacular in editions published outside of France, and I have retained the French exactly as it is written in the two source editions and as of yet not provided an English gloss. I have modernised the punctuation (with the exception of the use of parentheses) and the Latin orthography, including the use of \emph{quum} for \emph{cum} as conjunction and the spelling \emph{Iohannes} for \emph{Ioannes}. The only original orthography I have retained is the use of \emph{-n-} for \emph{-m-} in words such as \emph{nunquam}.

I have attempted to provide an \emph{apparatus fontium} where quotations can be confidently identified. I retain the \emph{locus} as it appears in the source editions, but quote the text from a modern critical edition in the \emph{apparatus}: the Bible from \emph{Biblia Sacra Vulgata} (5\textsuperscript{th} edition, Deutsche Bibelgesellschaft, 2007); Pseudo-Cato from the edition of Marcus Boas (Amsterdam, 1952, accessed at \url{www.baldzuhn.de/cato.html}) and most other Latin texts from the Packhard Humanitites Institute database of Latin texts (\url{latin.packhum.org}).

\subsection*{Sigla}
\begin{description}
\item \textbf{S.} --- Colloquiorum Scholasticorum Libri IIII, ad pueros in sermone Latino paulatim exerendos. Authore \emph{[sic]} Maturino Corderio. Colloquiorum seu Dialogorum Graecorum Specimen. Authore Henr. Stephano. Anno M. D. LXIIII. Excudebat Henricus Stephanus, illustris viri Huldrici Fuggeri typographus.

Two exemplars were consulted: the first Geneva BGE Su 1049 (\url{dx.doi.org/10.3931/e-rara-6173}) (pp.73--74 missing) and Lyon Rés 306778 (missing the Errata \& Corrigenda page).

\item \textbf{S. Corr.} --- Errata \& Corrigenda page of S.

\item \textbf{D.} --- Colloquiorum Scholasticorum Libri IIII, autore Maturino Corderio ab ipso aucti \& recogniti. Argumentum huius operis per M. Corderium: Hic tibi purus inest sermo, brevis, atque Latinus; Hic bene vivendi sunt documenta simul. Apud Ioannem Durantium. M. D. L. XX. Cum Privilegio.

The exemlar of the Basel UB Hauptbibliothek, UBH VIII 5:2 (\url{doi.org/10.3931/e-rara-2018}), was consulted.
\end{description}

\phantomsection
\addcontentsline{toc}{section}{Concordance}
\section*{Concordance of S. and D.}
`0' denotes a new \emph{colloquium}. D.'s numbering is used for this edition.
\setlength{\parindent}{0em}
\begin{multicols}{4}
D 195 --- S 221

D 1.1 = S 1.1

1.2 = 1.2

1.3 = 0

1.4 = 1.3

1.5 = 0

1.6 = 0

1.7 = 1.4

1.8 = 1.5

1.9 = 1.6a

1.10 = 0

1.11 = 0

1.12 = 1.6b

1.13 = 1.7

1.14 = 1.8

1.15 = 1.9

1.16 = 1.10

1.17 = 1.11

1.18 = 1.12

1.19 = 0

1.20 = 1.13

1.21 = 1.14

1.22 = 1.15

1.23 = 1.16

1.24 = 1.17

1.25 = 1.18

1.26 = 1.19

1.27 = 1.22

1.28 = 1.23

1.29 = 1.24

1.30 = 1.25

1.31 = 1.26

1.32 = 1.27

1.33 = 1.28

1.34 = 0

1.35 = 1.29

1.36 = 1.30

1.37 = 1.31

1.38 = 1.32

1.39 = 1.33

1.40 = 1.34

1.41 = 1.35

1.42 = 1.36

1.43 = 1.37

1.44 = 1.38

1.45 = 1.39

1.46 = 1.40

1.47 = 1.41

1.48 = 0

1.49 = 0

1.50 = 1.42

1.51 = 1.43

1.52 = 1.44

1.53 = 1.45

1.54 = 1.46

1.55 = 1.47

1.56 = 1.48

1.57 = 1.49

1.58 = 1.50

1.59 = 0

1.60 = 1.51

1.61 = 1.52

1.62 = 1.53

1.63 = 1.54

1.64 = 0

1.65 = 1.55

1.66 = 1.56

1.67 = 1.57

1.68 = 0

1.69 = 1.58

1.70 = 1.59

2.1 = 2.1

2.2 = 2.2

2.3 = 2.3

2.4 = 2.4

etc.

2.17 = 0

2.18 = 2.17

2.19 = 2.18

2.20 = 0

2.21 = 2.19

2.22 = 2.20

2.23 = 0

2.24 = 2.21

etc.

2.31 = 2.28

2.32 = 0

2.33 = 2.29

2.34 = 0

2.35 = 2.30

etc.

2.47 = 2.42

2.48 = 0

2.49 = 2.43

2.50 = 2.44

etc.
2.62 = 2.56

2.63 = 2.57

2.64 = 0

2.65 = 2.58

2.66 = 0

2.67 = 2.59

2.68 = 2.60

2.69 = 2.61

2.70 = 2.62

2.71 = 1.20

2.72 = 1.21

3.1 = 3.1

etc.

3.33 = 0

3.34 = 3.33

3.35 = 3.34

3.36 = 3.35

3.37 = 3.36

3.38 = 3.37

3.39 = 0

3.40 = 3.38

3.41 = 3.39

3.42 = 3.40

4.1 = 4.1

4.2 = 4.2

4.3 = 0

4.4 = 4.3

4.5 = 4.4

4.6 = 0

4.7 = 4.5

4.8 = 4.6

4.9 = 4.7

4.10 = 4.8

4.11 = 4.9

4.12 = 0

4.13 = 4.10

4.14 = 4.11

4.15 = 4.12

4.16 = 0

4.17 = 4.13

4.18 = 4.14

etc.

4.39 = 4.33
\end{multicols}

\pagebreak
\phantomsection
\addcontentsline{toc}{section}{Frontmatter (D.)}
\section*{Frontmatter from Durant (1570) (D.)}
\end{english}

\poemtitle*{Argumentum huius operis per Maturinum Corderium}
\settowidth{\versewidth}{Hic tibi purus inest sermo, brevis, atque Latinus,}
\begin{verse}[\versewidth] Hic tibi purus inest sermo, brevis, atque Latinus,\\ Hic bene vivendi sunt documenta simul.\end{verse}

\subsection*{Typographus Lectori Salutem}
\setlength{\parindent}{1em}
Maturini Corderii colloquia, ut erant ab ipso autore recognita et emendate, nos qua potuimus fide tibi exhibemus, lector, quae insuper addidimus, ea ex schedulis ipsius manu conscriptis, huc transtulimus. Quod argumenta sustulerimus, in eo secuti sumus consilium eorum qui iam per annos aliquot enarrarunt pueris haec colloquia; quibus non aegre paruimus, quod adolescentulos tenuioris fortunae inutilibus sumptibus non libenter gravemus. Dictionibus trisyllabis et maioribus appingi curavimus accentus, ut saltem hoc modo pueri assuescant rectae pronuntiationis.

Vale.

\phantomsection
\addcontentsline{toc}{section}{Maturini Corderii in colloquiorum suorum libros praefatio}
\section*{Maturini Corderii in Colloquiorum Suorum Libros Praefatio}
Annus agitur minimum quinquagesimus ex quo, suscepta docendi pueros provincia, in hanc cogitationem totus incubui, qua possem ratione efficere ut pueri pietatem bonosque mores cum humanarum litterarum studiis coniungerent. Quanvis enim, cum Parisiis primum eo munere fungi coepi, (cum in aliis gymnasiis, tum in Rhemensi, Sanctae Barbarae, Lexoviensi, Marchiano, Navarraeo) nondum mihi verum Evangeii lumen illuxisset, sed in profundis superstitionum tenebris demersus iacerem, discipulos tamen meos bona fide semper non solum ad humanitatis studia, sed etiam ad cultum divinum adhortabar. (Si tamen eo nomine appellare licet profanos illos falsae Ecclesiae ritus, quos ego paene ab incunabulis hauseram, et Deo acceptos esse mihi persuaseram.)

Me autem in illo instituto constanter perseverasse, satis idonei sunt testes libelli aliquot a me diversis temporibus editi, in quibus scribendis semper mihi consilium fuit ad utrumque horum simul pueros formare. Idem testari possunt et mei discipuli, e quorum ingenti numero cum supersint ad hunc usque diem plerique celeberrimi viri, unus tamen potissimum in praesentia mihi occurrit, ex iis quos Parisiis docui, praestantissimus ille vir, Iohannes Calvinus, quem honoris causa nomino. Ex quo autem misertus Pater clementissimus, mentem vera sui evangelii cognitione illustravit, multo etiam ardentius id propositum persequutus sum. Quod et Nivernensis Schola, et aliquanto post etiam Burdegalensis (ad quam, Lutetia profugus propter evangelicae doctrinae professionem, me contuleram) per triennium experta est. Sed cum et plenior evangelii cognitio deinde accessisset, et liberior etiam, immo vero prorsus libera, mihi esset eius professio, tum vero voti mei campos reddi, vehementiore desiderio quam unquam antea concupivi. Atque id testari haec schola Genevensis iampridem potuit, in qua ego, relicta Burdegalensi, docui. Potuit et Neocomensis, cuius per annos circiter septem fui moderator (de Neocomo autem in Helvetiorum finibus sito loquor). Potuit et Lausanensis id testari, ubi gymnasiarchae partes annos totos duodecim, magnificentissimorum dominorum Bernatum auspiciis, sustinui. Potuerunt (inquam) una cum Genevensi hae quoque scholae id testari. Sed et nunc eadem mihi testis esse potest, cum in eam me secundo Pater ille benignissimus, senectutis meae misertus, (quae annum octogesimum quintum attigit) tanquam in portum tutissimum, post infinitos labores et multa pericula, receperit.

Ex quo tempore saepissime mecum cogitavi, qua potissimum re inservire illi possem, qui me per totam vitam tanta benignitate prosecutus esset, meque tot laboribus et periculis liberasset. Cum autem Robertus Stephanus, amicorum meorum intimus (quo primum doctore ad evangelii cognitionem usus fueram) me, ut alias saepe, ad scribendum aliquid pueris vehementer hortaretur, et adminicula quaecunque necessaria essent polliceretur, atque adeo iam mihi amanuensem suis sumptibus aleret; animum ad eam rem appellere coepi. Sed (proh dolor) Robertus ille meus haud multo post ex hac vita ad Christum, non sine maximo literarum detrimento, commigravit. Neque tamen ego incepto destiti, sed aliquot opuscula scribere sum aggressus, quibus me consulturum puerorum studiis sperabam, si quando mihi manum illis extremam imponere liceret. Verum enimvero, ex quo tempore ad hanc docendi provinciam (Deo ita volente) sum revocatus, sic animum abieci, ut vix sperare possem aliquid unquam edendi mihi datum iri spatium: propterea quod (ut semper antea feceram) me ad satisfaciendum discipulis ita dedebam,\footnote{S. dedebam D. debebam} ut ne successivas quidem horas mihi liberas relinquerem. Et quanvis permulti saepe mecum agerent, adeoque precibus efflagitarent ut tandem aliquid ad puerorum utilitatem in lucem emitterem: nemo tamen persuadere poterat ut mihi exirent e manibus quae iudicio meo nondum satis probarentur.

Anno tamen superiore, cum docendi adiutor ex animi sententia mihi dono Dei obtigisset, venit in mentem chartulas meas, praesertim vetustiores, diligentius excutere: inter quas occurrerunt haec nostra Colloquia, quae iam fere triennium (quasi sopita) in musaeolo meo latuerunt. Ea igitur, velut e somno excitata, cum in manus sumpsissem, placuit matutinis horis (id est, tempore mihi a docendi provincia vacuo) recognoscere, aliquot etiam augere Colloquiis, et, quantum res ipsa patiebatur, expolire. Peracta recognitione, visum est opusculum communicare cum quibusdam viris doctissimis: qui cum meum in eo consilium probassent, dignumque censuissent quod pueri cum grammatiae rudimentis haberent in manibus, ego persuasus frequenti eorum hortatione, permisi ut in publicum ederetur.

Sunt enim haec nostra Colloquia eiusmodi quae (nisi me fallit animus) bonae indolis pueros magnopere iuvare possint ad eas res consequendas quas ego semper in votis habui, et in quibus ut pueri exercerentur, omni professionis meae tempore potissimum laboravi: hoc est (ut supra dixi) ad pietatem et bonos mores cum literarum elegantia coniungendos. Equidem fateor aliis opus esse ad eam rem adiumentis, quam haec sint puerilia Colloquia: sed tamen si pueri, diligenti magistrorum exhortatione assidue incitati, in his lectitandis sese oblectabunt, futurum confido ut non solum purius et honestius sermones inter se conferre assuescant, sed a pravis etiam colloquiis abstinere: quibus (ut ait Apostolus ex Menandro) mores boni corrumpuntur. Hic etenim, praeter linguae Latinae simplicem puritatem, multae interiectae sunt modo ad pietatem et Dei timorem admonitiones, modo praecepta salutaria de moribus, passim vero documenta sive exempla ad recte vivendum accommodata. Ut interim taceam quot modis una et eadem opera excitari et acui possit puerorum ingenium et ad prudentiam et ad iudicium comparandum: ubi scilicet alter alterum hortatur, admonet, arguit, libereque reprehendit. Porro, in his ipsis libellis sic induxi loquentes pueros, ut etiam obiter et quasi aliud agens, parentes, paedagogos, praeceptores quoque minus diligentes, de suo quenque in tractandis puerilibus ingeniis officio, compluribus in locis commonefaciam.

Sed de colloquiorum usu atque utilitate hactenus. Reliquum est, ut illa pueri, in quorum usum conscripta sunt, eo quo scripsimus animo, amplectantur et legant: iisque quandiu opus erit, fruantur. Quod siquid fructus ex eorum lectione se percepisse intellexerint, ei gratias agant immortales qui nobis ad ea scribenda et animum et facultatem suppeditavit. Interea quoque meminerint preces ad ipsum Deum saepe et ex animo pro huius civitatis amplissimo Senatu et prudentissimis magistratibus fundere: sub quorum felici administratione, Deo sua gratia providente, tranquillam vitam agimus, studiaque nostra in ipsius gloriam feliciter prosequimur. 


Datum Genevae, \textsc{viii} Idus Februarias anno Christianae redemptionis \textsc{mdlxiii}, aetatis autem nostrae \textsc{lxxxv}.

\setlength{\parindent}{0em}
\section{Liber Primus}
\subsection{Colloquium 1}
\emph{In hoc colloquio quaesta est colloquendi materia ex simulato ludendi praetextu. Exemplum bonae indolis puerorum, ubi alter bene monet, alter paret sine controversia.}

\emph{Colloquuntur} Bernardus et Claudius.

B. Salve, Claudi.

C. Tu quoque salvus sis, Bernarde.

B. Ludamus paulisper.

C. Quid ais, ineptule? Vix scholam ingressus es, et iam de ludo loqueris?

B. Ne irascaris, quaeso.

C. Non irascor.

B. Quid ergo sic exclamas?

C. Accuso tuam stultitiam.

B. Non licet igitur ludere?

C. Immo licet, at cum tempus est.

B. Vah, tu nimium sapis.

C. Utinam tantum saperem satis. Sed mitte me, quaeso, ut repetam quae mox reddenda erunt praeceptori.

B. Aequum dicis, volo ego quoque tecum repetere, si tibi placet.

C. Eho, quid hoc est? Quid sibi vult ista tam subita mutatio? Nonne tu modo loquebaris de lusu?

B. Loquebar quidem, sed non serio.

C. Cur simulabas?

B. Ut paucis tecum fabularer.

C. Quid illud prodest?

B. Etiam rogas? Nunquam audivisti ex praeceptore?

C. Nunc mihi non occurrit. Quid, inquam, prodest confabulari?

B. Ad nos in Latina lingua exercendos.

C. Profecto recte putas, et ego te nunc magis amo.

B. Habeo tibi gratiam. Age, repetamus praelectionemm nam brevi praeceptor aderit.

\subsection{Colloquium 2}
\emph{Puer ultro salutatum venit praeceptorem. Comiter ab illo accipitur. Reddit quae praescripta fuerant. Laudatur a magistro. Deum laudat. Munusculis blande invitatur. Exemplum ad parvulos blande et comiter in schola tractandos, ne severitate disciplinae absterreantur.}

Stephanio, Praeceptor

S. Salve, praeceptor.

P. Salvus sis, mi Stephanio. Unde venis tam multo mane?

S. E cubiculo nostro.

P. Quando surrexisti?

S. Paulo ante sextam, praeceptor.

P. Quid ais?

S. Sic est ut dico.

P. Tu nimis es matutinus, quis te expergefecit?

S. Frater meus.

P. An precatus es Deum?

S. Cum primum frater me pexuit, precatus sum.

P. Quomodo?

S. Flexis genibus, et coniunctis manibus dixi precationem Dominicam, cum gratiarum actione.

P. Qua lingua?

S. Gallica.

P. O factum bene! Quis te misit ad me?

S. Nemo.

P. Quid ergo?

S. Ultro veni.

P. Mi animule, quam pulchrum est sapere! Nonne est ientandi tempus?

S. Nondum esurio.

P. Quid vis igitur?

S. Volo reddere nomina quotidiana, si tibi placet audire me.

P. Quidni placeret? Tenes igitur memoria?

S. Teneo, gratia Deo.

P. Age, pronuntia.

S. Sed mihi soles praeire Gallice, et ego Latine respondeo.

P. Bene mones, paene istud oblitus eram. Responde igitur.

S. Exspecto ut proponas.

P. \fr{La teste.}

S. Caput.

P. \fr{Le sommet de la teste.}

S. Vertex.

P. \fr{Le devant.}

S. Sinciput.

P. \fr{Le derriere.}

S. Occiput.

P. Nunc responde Gallice: caput.

S. \fr{La teste.}

P. Vertex.

S. \fr{Le sommet de la teste.}

P. Sinciput.

S. \fr{Le devant.}

P. Occiput.

S. \fr{Le derriere.}

P. Quid si nunc solus dicas omnia?

S. Facile dicam.

P. Ego vero te libenter audiam.

S. \fr{La teste}, caput; \fr{le sommet de la teste}, vertex; \fr{le devant}, sinciput; \fr{le derriere}, occiput. Nonne bene dixi, praeceptor?

P. Quam optime!

S. Laudetur Dominus Deus.

P. O pulchrum verbum! Ito nunc petitum ab ancilla ientaculum.

S. Malim abs te accipere, praeceptor, si tibi non est molestum.

P. O quam te amo de isto verbo! Age, sequere me; dabo tibi aliquid boni, quia tuum recte fecisti officium. Quid est hoc?

S. Panis candidus, \fr{du pain blanc.}

P. Quid haec sunt?

S. Ficus aridae, \fr{des figues seches.}

P. Numera.

S. Una, duae, tres, quattuor, quinque, sex.

P. O lepidum capitulum! \fr{O le gentil petit compagnon}! Ienta nunc otiose.

\subsection{Colloquium 3}
A. B.

A. Vis ientare mecum?

B. Non habeo ientaculum.

A. Quid? Non attulisti?

B. Ego domi ientaveram.

A. Itane semper facis?

B. Minime, sed quia bene mane surrexeram, sic matri placuit me tractare.

A. Prosit tibi, ego igitur solus ientabo.

B. Et ego interim studebo.

\subsection{Colloquium 4}
\emph{Paedagogus a puero rationem studii exigit. Reddit puer alacriter. Admonetur ut a mendacio caveat. Formula ad verborum declinationem pueris breviter indicandam, postquam exempla generalia didicerint.}

Paedagogus, Puer

Pae. Esne paratus ad reddendum studii tui rationem?

Pu. Paratus, ut mihi videor.

Pae. Redde igitur, et esto praesenti animo.

Pu. Hoc matutino tempore primum pronuntiavimus carmen ex Catone, deinde eius interpretationem Latine et Gallice reddidimus, postremo bini tractavimus singulas partes orationis, cum attributis et significatione.

Pae. Rectene fecisti officium tuum?

Pu. Puto me satisfecisse praeceptori magna ex parte.

Pae. Vide ne mentiaris, nam ego illum percontabor.

Pu. Ut voles, praeceptor, nihil hac in re metuo.

Pae. Age, pergamus. Meridie quid habetis reddere?

Pu. Habemus declinare verbum \emph{possum}, Latine et Gallice.

Pae. Nihil praeterea?

Pu. Nihil.

Pae. Ego te istud alias docui, tenesne memoria?

Pu. Non ausim affirmare, donec tentavero.

Pae. Declina in primas personas, cetera tibi erunt facillima.

Pu. Indicativus: possum, \fr{je puys}; poteram, \fr{je pouvoye}; potui, \fr{j' ay peu}; potueram, \fr{j' avoye peu}; potero, \fr{je pourray}.

Imperativus deest.

Optativus: utinam possim, \fr{Dieu vueille que je puisse}; utinam possem, \fr{pleust à Dieu que je puesse}; utinam potuerim, \fr{Dieu vueille que j' aye peu}; utinam potuissem, \fr{pleust à Dieu que j'eu eusse peu}.

Subiunctivus: ut possim, \fr{que je puisse}; ut possem, \fr{que je peusse}; quanvis potuerim, \fr{combienque j' aye peu}; quanvis potuissem, \fr{combienque j’ eusse peu}; cum potuero, \fr{quand j' auray peu}.

Pae. Declina totum infinitivum.

Pu. Infinitivi modi praesens et praeteritum imperfectum: posse, \fr{pouvoir}; praeteritum perfectum et plusquam perfectum: \fr{potuisse}, \fr{avoir peu}; cetera desunt.

Pae. Cur hoc verbum \emph{possum} caret futuro indefinito, cur item participio in \emph{-rus}?

Pu. Quia non habet supinum.

Pae. Quid tum?

Pu. Illae enim voces a supino formari solent.

Pae. Da exemplum in aliquo verbo integro.

Pu. Ut a supino \emph{lectum} fit \emph{lecturus}, et a \emph{lecturus} fit \emph{lecturum esse}.

Pae. Recte sane. Sed cur praetermisisti participium praesens a verbo \emph{possum}, cum sit in usu \emph{potens}, \emph{potentis}?

Pu. Quia (ut saepe nos docuisti) \emph{potens} non est participium, licet a \emph{possum} veniat.

Pae. Quid ergo est?

Pu. Nomen adiectivum.

Pae. Probe meministi, utinam sic pergas semper.

Pu. Spero in dies meliora per Dei gratiam.

Pae. Ego quoque idem tecum spero. Nunc restat ut dicas praeteritum cum prole.

Pu. \emph{Potui}, \emph{potueram}, \emph{potuerim}, \emph{potuero}, \emph{potuissem}, \emph{potuisse}.\footnote{S. potuero, potuisse S. Corr. et D. potuero, potuissem, potuisse.}

Pae. Dic terminationes.

Pu. \emph{-i}, \emph{-ram}, \emph{-rim}, \emph{-ro}, \emph{-ssem}, \emph{-sse}.

Pae. Dic significationes.

Pu. Possum: \fr{je puis}; posse: \fr{pouvoir}.

Pae. Haec hactenus.\footnote{D. Haec hactenus S. Hactenus} Ecce, ecce, vocamur ad prandium.\footnote{D. Ecce, ecce S. Ecce}

\subsection{Colloquium 5}
C., D.

C. Quando vis prandere?

D. Ego iam prandi.

C. Quota hora?

D. Sesquioctava.

C. Tam mane igitur prandetis?

D. Sic fere solemus in aestate. Vos autem?

C. Non prandemus ante sesquidecimam, interdum ab undecima.

D. Papae! Cur non citius?

C. Exspectandus est pater, dum e curia redierit.

D. Tu igitur non potes adesse aulae in cantione Psalmorum.

C. Raro admodum intersum.

D. Quomodo excusaris?

C. Exemptus sum illo munere.

D. Quis te exemit?

C. Didascalus, patris mei monitu.

D. Ergone omnes Senatorum filii habent eiusmodi privilegium?

C. Habent, modo patres iubeant.

D. Nonne mater posset dare tibi prandium ante reditum patris e Senatu?

C. Posset quidem, sed pater vult a me exspectari.

D. Quamobrem?

C. Quia sic illi placet.

D. Nunc mihi tacendum est. Os enim mihi occlusisti.

C. Cur tu es tam curiosus percontator?

D. Puer sum, et pueri semper cupiunt aliquid scire novi.

C. Fateor, sed est modus in rebus, ut praeceptor nos saepe docet.

D. Ergo discedamus, ut te pransum conferas.

C. Ignosce, quaeso, si qua in re te offenderim.

D. Ego abs te idem peto. Ego, inquam, potius, qui te offendere potui loquacitate mea, sed interim nihil mali cogitans.

\subsection{Colloquium 6}
F., G.

F. Ubi hodie cibum cepisti?

G. Apud hospitem meum.

F. Quanti prandisti?

G. Sex quadrantibus.

F. Quid cena, quanti constitit?

G. Tantidem. Tu vero quanti aleris quotidie?

F. Pluris quam tu.

G. Quanti igitur?

F. Quattuor assibus.

\subsection{Colloquium 7}
\emph{Amici libera reprehensio, si bene accipiatur, haud mediocriter prodest.}

Choletus, Colognerius

Ch. Unde nunc redis?

Co. Foris.

Ch. Cur prodieras?

Co. Ut irem domum.

Ch. Quid eo?

Co. Petitum libros meos.

Ch. Eho, cur non attuleras?

Co. Oblitus eram.

Ch. Siccine soles ientaculum aut merendam oblivisci?

Co. Rarissime.

Ch. Profecto magna fuit negligentia.

Co. Immo maxima. Sed quid agas? Pueri sumus.

Ch. Quid si praeceptor tuum factum sciret?

Co. Fortasse poenas darem.

Ch. Ain' tu ``fortasse''? Procul dubio vapulares. Non te pudet sine libris in scholam venire?

Co. Non solum pudet, sed piget etiam. Verumtamen ne me accuses, obsecro.

Ch. Nihil minus cogito. Sed non possum dissimulare quin ego te reprehendam.

Co. Istud (credo equidem) amice facis, itaque boni consulo.

Ch. Id satis est mihi. Eamus intro in auditorium.

Co. Tempus est, iam decuriones exigunt scripturae rationem.

\subsection{Colloquium 8}
\emph{Parvulus puer ludendi copiam sibi et condiscipulis impetrat. Quintilianus: ``Nec me offernderit lusus in pueris.''\footnote{\emph{Institutio Oratoria} I.3.10} Exemplum ad parvulos et ceteri ut in secundo colloquio.}

Puer, Paedagogus

Pu. Praeceptor, licetne pauca?

Pae. Loquere audacter.

Pu. Ego et condiscipuli mei hoc fere toto triduo libris affixi fuimus, licetne paulisper animum ludo relaxare?

Pae. Dic igitur aliquam sententiam.

\begin{verse}[\versewidth]\flagverse{Pu.}``Interpone tuis interdum gaudia curis, \\ Ut possis animo quemvis sufferre laborem.''\footnote{Disticha Catonis III.6}\end{verse}

Pae. Dic etiam versus Gallicos, si memoria tenes.

\begin{verse}[\versewidth]\flagverse{Pu.} ``\fr{Mesle par fois esbat en ton labeur,\\ Pour travailler apres de plus grand cœur.}''\footnote{M. Cordier, \emph{Disticha de moribus nomine Catonis inscripta, Latina et Gallica interpretatione epitome in singula fere disticha; dicta sapientium cum duplici sua quoque interpretatiuncula}, Robert Estienne, Paris, 1533}\end{verse}

Pae. Quam recte dixisti omnia!

Pu. Est Deo gratia.

Pae. Addendum posthac erit aliquid.

Pu. Quidnam, praeceptor?

Pae. ``Qui dedit mihi ingenium et mentem bonam.''

Pu. Sed quis me illa docebit verba?

Pae. Ea scribam tibi in commentariolo tuo, ut ediscas. Sed dic mihi, quaeso, quis te docuit istam orationem quam pronuntiasti?

Pu. Campanus heri dederat mihi scriptam, et ego memoriae mandavi.

Pae. Profecto ego te amo, mi Daniel, ob istam diligentiam.

Pu. Ago tibi gratias, praeceptor. Permittisne igitur ut ludamus?

Pae. Sane. Abi, renuntia condiscipulis tuis.

Pu. Faciam.

Pae. Quid dices illis?

Pu. Id quod me docuisti aliquando.

Pae. Sed volo prius ex te audire.

Pu. ``Gaudete, pueri! En, affero vobis iucundum nuntium. Ego vobis impetravi ludendi potestatem.''

Pae. Euge, probe meministi. Ito nunc iam.

\subsection{Colloquium 9}
\emph{Duo pueri diligenter repetunt quae sunt in auditorio redituri. Exemplum diligentiae in pueris, pro captu aetulae. Utinam aetate et doctrina provecti hoc imitarentur exemplum.}

Conradus, Daniel

C. Repetamus nomina quotidiana, ut certius reddamus ea praeceptori.

D. Bene mones, praeito mihi hesterna.

C. Dic Latine, \fr{un œil}.

D. Oculus.

C. \fr{L’œil dextre}.

D. Oculus dexter.

C. \fr{L’ œil gauche}.

D. Oculus sinister.

C. \fr{Les deux yeulx}.

D. Ambo oculi.

C. Probe tenes.

D. Nunc audi an recte solus dicam.

C. Age, audio.

D. \fr{Un œil}.

C. Debes enumerare in digitis, ut docet praeceptor.

D. Quid prodest istud?

C. Ad memoriam iuvandam.

D. Quid hoc sibi vult?

C. Non audisti multoties?

D. Ego sum obliviosus, quid agerem?

C. Esto diligentior ad ea retinenda quae perceperis.

D. Quod me fideliter mones, pergratum facis.

C. Age, ad rem redi.

D. \fr{Un œil}, oculus; \fr{l’œil dextre}, oculus dexter; \fr{l’œil gauche}, oculus sinister; \fr{les deux yeulx}, ambo oculi.

C. Quam recte omnia dixisti!

D. Repetamus etiam hodierna.

C. Placet, at tu vicissim praeito mihi.

D. \fr{Une main}.

C. Manus.

D. \fr{La main dextre}.

C. Manus dextra.

D. \fr{La main gauche}.

C. Manus sinistra.

D. \fr{Les deux mains}.

C. Ambae manus.

D. Restat, ut solus dicas.

C. \fr{Une main}, manus; \fr{la main dextre}, manus dextera; \fr{la main gauche}, manus sinistra; \fr{les deux mains}, ambae manus.

D. O si tam bene diceremus coram praeceptore!

C. Quid obstat?

D. Quia timemus.

C. Et tamen errata nostra satis humaniter corrigit.

D. Nescio quid hoc sibi velit, ego semper sum timidus in principio.

C. Istud est quodammodo naturale omnibus, ut audivimus ex praeceptore.

D. Nunc repetendum esset Latine et Gallice, sed praeceptorem venientem video.

C. Ingrediamur.

\subsection{Colloquium 10}
A., B.

A. Dic Latine, ``\fr{Recordons nostre leçon ensemble}.''

B. ``Repetamus una praelectionem.''

A. Haec oratio quot habet partes?

B. Tres.

A. Discerne singulas nominatim.

B. \emph{Repetamus} est verbum; \emph{una}: adverbium; \emph{praelectionem}: nomen.

A. Declara paulo planius.

B. Tu igitur praeito mihi, ut solet praeceptor.

A. ``Repetamus.''

B. \emph{Repeto,} \emph{repetis}, \emph{repetere}, coniugationis tertiae sicut \emph{lego}, \emph{legis}, \emph{legere}. Praeteritum: \emph{repetivi}; supinum: \emph{repetitum}; participia: \emph{repetens} et \emph{repetiturus}.

A. ``Una.''

B. Non declinatur, quia est adverbium hoc in loco. Gallice: \fr{ensemble}.

A. ``Praelectionem.''

B. \emph{Praelectio}, \emph{praelectionis}, generis feminii: \fr{la leçon, ou la lecture.}
\subsection{Colloquium 11}
A., B.

A. Eamus una repetitum.

B. Quid ego repeterem? Nonne satis est quod solus repetiverim?

A. Si tantum semel aut bis repetieris, id parum est ad ediscendum.

B. Immo circiter decies repetivi.

A. Id quidem sufficit.

B. Quid igitur vis amplius?

A. Si vis certissime reddere coram praeceptore, opus est cum aliquo repetivisse.

B. Istud ego nesciebam, sed tibi libenter assentior.

A. Faciamus ergo quod ego te monebam.

B. Equidem non recuso. Incipe.

\subsection{Colloquium 12}
\emph{Eiusdem est argumenti cum superiore.\footnote{D. I.11 = S. I.6a; D. I.12 = S. I.6b} Praeterea, qui errat, corrigitur ab altero, et admonetur. Laudatur memoria, et divinum esse donum agnoscitur.}

Gentilis, Isaacus

G. Tenesne memoria praelectionem?

I. Propemodum.

G. Visne repetamus una?

I. Maxime velim.

G. Incipe igitur.

I. Faciam libenter. Sed tu attente audi, ut me corrigas, si quid erravero.

G. Agedum.

\begin{verse}[\versewidth]\flagverse{I.} ``Fac sumptum propere.''\footnote{Disticha Catonis II.5:\\``Fac sumptum propere, cum res desiderat ipsa:\\dandum etenim est aliquid, cum tempus postulat aut res.''}\end{verse}

G. Iam errasti, incipiendum fuit ab hesterna.

I. Bene admones, nunc incipiam: \begin{verse}[\versewidth]``Irâtus de re incérta conténdere noli.\\ Impédit ira.''\footnote{Disticha Catonis II.4:\\``Iratus de re incerta contendere noli,\\impedit ira animum, ne possis cernere verum.''}\end{verse}

G. Peccas accentu. Itera.

\begin{verse}[\versewidth]\flagverse{I.} ``Ímpedit ira ánimum ne.''\end{verse}

G. Distingue post ``animum.''

\begin{verse}[\versewidth]\flagverse{I.}``Ímpedit ira animum, ne possit cérnere verum.\\ Fac sumptum próperè, cum res desiderat ipsa.\\ Dandum eténim.''\end{verse}

G. Iterum peccas accentu. Repete.

\begin{verse}[\versewidth]\flagverse{I.}``Dandum étenim est áliquid, cum tempus póstulat, aut res.''\end{verse}

G. Videsne te erravisse quater?

I. Video.

G. Et observasti locos?

I. Observavi.

G. Sic tibi facilius cavebis.

I. Me miserum! Putabam me recte tenere.

G. Sic mihi quoque solet accidere, quoties memoria non est bene confirmata.

I. Felix qui memoriam bonam habet.

G. Magnum Dei beneficium. Sed tempus abit, audi nunc me.

I. Audio, pronuntia.

\begin{verse}[\versewidth]\flagverse{G.}``Iratus de re incerta contendere noli.''\end{verse}

I. Hem, praeceptor adest. Tace, ut illum salutemus.

\subsection{Colloquium 13}
\emph{Felix arguit Clementem negligentiae. Clemens fatetur culpam. Promittit diligentiam in posterum. Exemplum bonae indolis in puero.}

Clemens, Felix

C. Nihilne est quod reddamus hodie praeceptori?

F. Nihil, nisi de grammaticae rudimentis.

C. Quidnam?

F. Inspice librum tuum; invenies notas in quinque lectiones, quas praeceptor nobis praescripsit.

C. Quando istud fuit?

F. Die Veneris, hora quarta.

C. At ego tunc non interfui.

F. Ergo plagas meruisti.

C. Siccine iudicas, severe iudex? Occupatus eram domi, nec abieram iniussu praeceptoris.

F. Esto. Sed tamen debuisti postridie quaerere quid pridie actum esset.

C. Meam culpam confiteor: sed cedo librum tuum, quaeso, ut videam quid nobis reddendum sit.

F. Accipe, et eadem opera signato quae a praeceptore nobis praescripta sunt.

C. Faciam diligenter, neque posthac (ut spero) me accusabis negligentiae.

\subsection{Colloquium 14}
\emph{Quidam, quod abfuerit, quaerit ex condiscipulo quid actum sit in auditorio.}

Comes, Olivarius

C. Quid actum est in auditorio hora tertia?

O. Tractatae sunt orationis partes ex praelectione.

C. Nihilne amplius?

O. Dixissem, nisi me interpellasses.

C. Erravi, perge.

O. Postea dictavit praeceptor argumentum Gallicum, hodie vertendum.

C. Quando reddendum?

O. Cras meridie.

C. Iamne vertisti?

O. Utcunque.

C. Dicta mihi, quaeso, vernaculum.

O. Excipe, festina, nam habeo scribere aliquid.

\subsection{Colloquium 15}
\emph{Exordium est ad repetendam praelectionem.}

Sylvius, Genasius

S. Quid agis?

G. Repeto mecum.

S. Quid repetis?

G. Praescriptum hodiernum praeceptoris.

S. Tenesne memoria.

G. Sic, opinor.

S. Repetamus una: sic fiet ut uterque\footnote{S. sic fiet ut uterque D. sic uterque} nostrum rectius pronuntiet coram praeceptore.

G. Tu igitur incipe, qui me provocasti.

S. Age, attentus esto, ne me sinas aberrare.

G. Sum promptior ad audiendum, quam tu ad pronuntiandum.

\subsection{Colloquium 16}
\emph{Ad idem cum proximo superiore.}

Accorratus, Curtius

A. Visne mecum repetere praelectionem?

C. Volo.

A. Tenesne?

C. Non satis recte, fortasse.

A. Age, faciamus periculum.

C. Quid igitur exspectamus?

A. Ubi voles, incipe.

C. Atqui tuum est potius incipere.

A. Quid ita?

C. Quia me invitasti.

A. Aequum dicis, attende igitur.

C. Istic sum.

\subsection{Colloquium 17}
\emph{Exordium est ad colloquendum. Qui bene agit, reprehendi non timet. Prudenter igitur facit qui praeceptum illud usurpat: ``Sic age, ne timeas.''}

Crispus, Sandrotus

C. Iamne tenes quae reddenda sunt hora tertia?

S. Teneo.

C. Ego quoque.

S. Ergo confabulemur paulisper.

C. Sed si intervenerit observator, putabit nos garrire.

S. Quid times, ubi nihil timendum est? Si venerit, non deprehendet nos in otio aut in re mala; audiat, si velit, nostrum colloquium.

C. Optime loqueris; secedamus aliquo in angulum, ne quis nos impediat.

\subsection{Colloquium 18}
\emph{Admonitio brevis, sed pueris aptissima.}

Trimondus, Messor

T. Non decet hic otiari aut garrire, dum praeceptor exspectatur.

M. Quid ais, non decet? Immo vero non licet, nisi volumus vapulare.

T. Tu igitur audi me, dum praelectionem pronuntio. Ego deinde te audiam.

M. Age, pronuntia.

\subsection{Colloquium 19}
E., F.

E. Cur non scribis?

F. Iam scripsi meam paginam. Tu vero?

E. Eo scriptum in area.

F. Quid ita?

E. Quia serenum est caelum.

F. Festina, tempus abit, et mox exigetur ratio.

\subsection{Colloquium 20}
\emph{Ab amico petitur aliquid operae. Promittitur dari per opportunitatem. Exemplum amici officiosi.}

Titus, Valerius

T. Cur non scribis?

V. Quia non libet.

T. Atqui praeceptor iusserat.

V. Scio, sed est mihi aliquid legendum prius. Praeterea, nihil habeo quod nunc scribam.

T. O si velles mihi scribere!

V. Quidnam?

T. Habeo describenda dictata praeceptoris.

V. Quae dictata?

T. In Ciceronis \emph{Epistolas}.

V. Libenter describam tibi. Sed exspecta feriarum diem proximum.

T. Exspectabo igitur. Sed ne fallas, quaeso.

V. Nec sciens, nec volens fallam.

\subsection{Colloquium 21}
\emph{Damon officium petit ab Audace. Hic occupationem causatus, belle denegat. Alter non urget ulterius. Exemplum modestiae tum in petendo, tum in denegando.}

Damon, Audax

D. Visne mihi describere praelectionem?

A. Cur non habes?

D. Quia hesterno die fui occupatus.

A. Accipe librum meum, et describe.

 
D. Non ignoras me lentius scribere, et tu citius totum descripseris quam ego quattuor aut quinque versiculos.

A. Quaere tibi alium scriptorem. Nunc ego tibi non possum dare operam.

D. Cur non?

A. Est mihi aliud negotium, idemque pernecessarium.

D. Nolo te urgere, nec possum quidem: sed saltem commoda tuum codicem.

A. Accipe, utere ut libet, modo ne abutare. 

D. Nihil est quod hic verearis.

\subsection{Colloquium 22}
\emph{Exemplum diligentiae in agendis rebus.}

Augustinus, Observator Domesticus

A. Licetne ire cubitum, condiscipuli?

O. Cur ante horam?

A. Quia tertia est mihi surgendum.

O. Quamobrem?

A. Scribendi causa.

O. Quid scripturus es?

A. Litteras ad patrem.
 
O. Cur non petis a praeceptore veniam?

A. Bene mones, estne in musaeolo?

O. Puto esse, vise.

\subsection{Colloquium 23}
\emph{Quidam puer studere mavult quam ludere. Exemplum in pueris rarissimum.}

Paedagogus, Puer

P. Unde venis?

Pu. Venio inferne.

 
P. Quod erat tibi negotium infra?

Pu. Iveram redditum urinam.

P. Sede nunc ad mensam, et mane in cubiculo donec rediero.\footnote{D. mane in cubiculo donec rediero S. mane in cubiculo.}

Pu. Quid agam interea?

P. Ediscito praelectionem in diem crastinum, ut eam mihi reddas ante cenam.

Pu. Iam edidici, praeceptor.

P. Lude igitur.

Pu. Sed nullos habeo collusores.

P. Satis multos invenies in hac vicinia, ex tuis etiam condiscipulis.

Pu. Nihil id curo nunc. Malim (si tibi placet) ediscere de Catechismo, in diem Dominicum.

P. Ut libet.

Pu. Si quis te quaeret, quid illi dicam?

P. Dic me prodiisse, sed mox reversurum.

\subsection{Colloquium 24}
\emph{Quidam paedagogus discipulum educit ambulatum. Exercitatio corporis. Socratis apophthegma. Ambulatio fit propter comites iucundior. Exemplum honestae remissionis in studiis.}

Paedagogus, Abrahamus puer

P. Heus, Abrahame.

A. Hem, praeceptor.

P. Pone libros, iam satis toto die studuisti. Para te, ut eamus ambulatum.

A. Nonne a cena praestaret?

P. Salubrior est ante cibum exercitatio corporis.

A. Memini ex te audire.

P. Narra Socratis dictum in eam sententiam.

A. Cum Socrates usque ad vesperam contentius ambularet, interrogatus qua re id faceret, respondit se, quo melius cenaret, obsonare famem ambulando.\footnote{Cicero \emph{Tusculanae Disputationes} V.97: ``Socraten ferunt, cum usque ad vesperum contentius ambularet quaesitumque esset ex eo, qua re id faceret, respondisse se, quo melius cenaret, obsonare ambulando famem.''}

P. Probe meministi, quis auctor?

A. Cicero. Sed quo prodibimus, praeceptor?

P. Extra urbem.

A. Mutabone calceos?

P. Muta, ne illos novos pulvere conspergas. Sume etiam umbellam, ne solis ardor infuscet tibi faciem.

A. Iam paratus adsum.

P. Nunc sane prodeamus.

A. Vocabone ex vicinia unum aut alterum comitem?

P. Recte admones. Sic enim iucundior erit deambulatio. Nam per viam sermones inter vos conferetis, et in umbra colludetis alicubi.

A. Sic etiam excitabitur cibi appetentia.

P. Ego lento gradu praecedam. Ubi nactus eris comites, vos me per portam Ripariam sequimini.

A. Nos igitur illic exspectabis?

P. Certo.

A. Quid si nullos invenero?

P. Nihilominus sequere me. Audistine?

A. Audivi, praeceptor.

\subsection{Colloquium 25}
\emph{De emptione chartae. Auctarium in emptione. Suam quisque utilitatem curat.}

Hersenius, Gimardus

H. A quo emisti istam chartam?

G. A Fatino.

H. Estne bona?

G. Melior quam ista tua, ut opinor.

H. Nihil miror.

G. Cur istud dicis?

H. Quia fortasse carior.

G. Nescio.

H. Quanti emisti scapum?

G. Solido et semisse, tu vero quanti?

H. Solido et pluris.

G. Quanti igitur?

H. Quinque quadrantibus.

G. Non male profecto emisti.

H. Quinetiam mercator dedit mihi auctarium.

G. Quodnam quaeso?

H. Schedam chartae bibulae.

G. O me imprudentem, qui oblitus sum petere!

H. Ego ne petivi quidem, sed ultro ille dedit. ``Et hoc,'' inquit, ``addo tibi, ut me revisas.''

G. Sic solent emptores allicere.

H. Nec mirum: suum quisque commodum quaerit.

G. Sed quid agimus, hodierni pensi immemores?

H. Exiguum est, satis temporis nobis restat.

\subsection{Colloquium 26}
\emph{Charta mutuo data reposcitur. Prudentis est, prius referre in codicem expensi quam aliquid credere, sive dare mutuo.}

Ancellus, Fontanus

A. Meministin' me tibi nuper dedisse chartam mutuo?

F. Quidni meminerim? Non adeo sum obliviosus.

A. Quot erant schedae?

F. Quattuor.

A. Cur non reddidisti?

F. Exspectavi dum haberem codicem.

A. Habuistine tandem?

F. Tantum hodie.

A. Unde nactus es?

F. Petivi a praeceptore.

A. Ubi?

F. In bibliotheca eius.

A. Quid ille, deditne libenter?

F. Misit me ad hypodidascalum, qui statim dedit codicem.

A. Non ante in suum codicem retulit?

F. Nihil dare solet quin prius inscribat.

A. Audivi ex patre id esse viri prudentis.

F. Praesertim si reddenda est ratio.

A. Sed quibus indiciis dare tibi ausus est?

F. Ostendi illi manu mea scriptum in libello meo. Sic enim (ut scis) facere solemus.

A. Reddes ergo mihi mutuum?

F. A prandio statim, ne dubites.

\subsection{Colloquium 27}
\emph{Pennae venales. Diversa vivendi ratio pro locorum diversitate. Hic observa commune emptorum morem, qui tot verbis tentant de pretio indicato aliquid detrahere.}

Francus, Marius

F. Pennae istae, quas circumfers, suntne venales?

M. Etiam, si se emptor obtulerit.

F. Ostende. Vah, quam sunt molles!

M. Tales deciderunt ex alis nostrorum anserum. Sed tenta diligenter, sunt aliae aliis firmiores.

F. Satis video quales sint. Quot vis dare pro quadrante?

M. Tantum sex.

F. Quid ais, ``sex''? Mallem emere a mercatoribus, qui Lutetia et Lugduno huc adferunt.

M. Quasi nesciam quanti vendantur. Audivi ex fratre, qui dat operam scribae huius civitatis, se emisse Lutetiae singulis assibus.

F. Aliter Lutetiae, aliter Genevae vivitur. Sed non opus est tot verbis, vis dare duodecim?

M. Hui, duodecim, quasi ego furatus sim!

F. Istud non dico, sed vide num tibi placet conditio.

M. Vis uno verbo dicam?

F. Dic, quaeso, satis iam garritum est.

M. Dabo tibi novem, modo promiscue de mea manu accipias.

F. Nugas agis. Ego sine delectu nollem accipere quindecim. Vale, alibi inveniam satis.

M. Per me licet. Heus! Heus! Redi.

F. Cur me revocas?

M. Accipe, si vis, octo; nec a me plures exspecta.

F. Cedo totum fasciculum, ut deligam arbitratu meo.

M. Tene, delige ut voles.

F. Vide nunc, et si libet, numera.

M. Sunt viginti quattuor. Constat numerus, sed miror te nullas accepisse ex ala extrema; sunt enim firmiores.

F. Scio, sed habent caulem breviorem. Accipe pretium.

M. Bene vertat Deus utrique nostrum.

F. Idem tecum opto atque precor. Sed quando adferes meliores pennas?

M. Nescio an meliores possim, sed (ut spero) brevi plures afferam, cum ad vos domum proficiscar.

F. Suntne vobis multi anseres?

M. Triginta et amplius.

F. Papae, quantus grex anserum! Ubi pascuntur?

M. Scies alias, non licet mihi diutius hic morari. Vale, France.

F. Cura ut valeas, Mari.

\subsection{Colloquium 28}
\emph{Othomanus a Philiberto pennam commodato petitam accipit.}

Othomanus, Philibertus

O. Visne mihi dare unicam pennam?

P. Non sic dantur mihi.

O. Hem, rem tantillam mihi negas? Quid si magnum quid rogarem?

P. Fortasse repulsam ferres.

O. Credo equidem. Age, non peto dono, saltem commodabis?\footnote{S. commodabis. D. commodabis?}

P. Non recuso, modo ne abutare.

O. Non abutar.

P. Cave hinc pedem moveas.

O. Nusquam moveo.

\subsection{Colloquium 29}
Mercator, Bertrandus

M. Acuistine pennam meam?

B. Iamdudum.

M. Qua forma scripturae?

B. Mediocri.

M. Maluissem ad minutas litteras.

B. Debuisti praedicere.

M. Oblitus eram.

B. Parum refert, mucronem facile mutabo. I petitum.

M. Sed ubi reliquisti?

B. Super mensam hypocausti.

M. In qua parte?

B. Ubi studere soleo.

\subsection{Colloquium 30}
\emph{Pictonus a Iosua pennam commodato petit. Hic primo fingit se commodare nolle, deine dono dat. Christi praeceptum. Christianus debet Christum praeceptorem imitari. Colloquendi occasion quaesita per simulationem.}

Pictonus, Iosuë\footnote{S. Iosuë D. Iosua}

P. Habesne duas aut tres pennas?

I. Sunt mihi tantum duae.

P. Da mihi unam commodato.

I. Non faciam.

P. Cur non?

I. Ne abutaris.

P. Memineris, fortasse aliquando me frustra rogabis aliquid.

I. Atqui iubet Christus malum bono compensandum.

P. Nondum istud didici.

I. Tamen discas oportet, si cupis esse Christi discipulus.

P. Quid cupio magis?

I. Disce igitur magistrum imitari.

P. Discam progressu temporis.

I. Praestaret nunc incipere, dum per tempus licet.

P. Nimis urges; nondum complevi annum octavum, ut ait mater.

I. Semper est bene agendi tempus. Sed interim ne mihi, quaeso, succenseas. Iocabar enim, ut te ad colloquendum invitarem tantisper dum sumus otiosi. Ecce tibi penna, eaque non omnino pessima.

P. Reddam tibi statim, cum aliquid descripsero.

I. Nolo mihi reddas.

P. Quid igitur faciam?

I. Quicquid voles, a me enim tibi dono datur.

P. Gratiam habeo maximam.

\subsection{Colloquium 31}
\emph{De emptione scalpelli librarii, et eius probatione. Solus Deus bona docet.}

Henricus, Gualtherus

H. Unde redis tam anhelus?

G. A foro.

H. Quid illinc affers?

G. Scalpellum.

H. Quanti emisti?

G. Duobus assibus.

H. Estne bonum?

G. Est a Germania, ut dixit mercator. Vide notam.

H. Ego minime novi, sed non satis prudenter facis, qui fidas cuilibet mercatori.

G. Quid facerem?

H. Debuisti aliquem peritum adhibere, qui tibi optimum deligeret.

G. Hic erravi, fateor; sed hoc me consolatur, quod mercator habetur vir bonus, utpote professionis evangelicae.

H. Quasi nulli sint fallaces eiusmodi.

G. Puto esse plurimos. Sed haec omittamus, quin potius experiamur ipsum scalpellum?

H. Experientia docebit nos.

G. Accipe, et tenta, obsecro. Non enim probavi, nisi levissime, idque inter emendum.

H. Papae, quis te docuit tam prudenter eligere?

G. Rogas? Non meministi praeceptorem nobis dicere tam saepe, Deum esse solum qui bona doceat?

H. Profecto hic optime te docuit.

G. Ago illi ex animo gratias, et precor ut me semper doceat parere suae voluntati.

H. Ego quoque idem precor nec solum nobis, sed etiam piis omnibus.

G. Facis ut pium decet puerum. Sed estne tempus ut conferamus nos in auditorium?

H. Sic est. Sume libros, et eamus una.

\subsection{Colloquium 32}
\emph{Scalpellum commodato petitur. Datur, sed addita conditione. Commodatarius non debet abuti commodato.}

Michael, Renatus

M. Habesne scalpellum?

R. Habeo.

M. Oro te, commoda mihi parumper.

R. Quando reddes?

M. Cum primum duas pennas exacuero.

R. Accipe, sed ea lege ut integrum reddas.

M. Ea conditione acceptum intelligo, etiamsi non addidisses.

R. ``Intelligenti,'' (ut vulgo dicitur), ``pauca sufficiunt.''

\subsection{Colloquium 33}
\emph{Consilii mutation propter spem lucre. Consilium bonum etiam inimicis dandum. Christus magister optimus. Doctrina Christi studiose sequenda. Spiritus divini instinctus.}

S. Emistine scalpellum, ut nuper volebas?

M. Non emi.

S. Quid obstitit? Dixeras enim mihi empturum te hodie.

M. Dixeram quidem; sed mihi postea in mentem venit, praestare ut exspectem mercatum proxime futurum in hac ipsa urbe.

S. Quid facies lucri?

M. Et minoris emam,\footnote{S. ema D. emam} et melioris notae, nempe ex Germaniae officinis.

S. Quis tibi istud consilium dedit?

M. Hieronymus noster.

S. Bene fecit. Debemus enim amicis bonum consilium semper dare.

M. Tantumne igitur amicis?

S. Immo et inimicis, fateor; quia sic iubet Christus praeceptor noster optimus.

M. Utinam doctrinam eius bene infixam memoriae conservemus, eamque perpetuo sequamur.

S. Faxit ille Spiritus bonus, cuius unius instinctu animi nostri ad bene agendum accenduntur.

M. Bene precaris.

\subsection{Colloquium 34}
Campanus, Languinus

C. Habesne multos libros?

L. Non admodum.

C. Sed quos habes?

L. \emph{Rudimenta Grammaticae}, \emph{Colloquia Scholastica}, Terentium, \emph{Epistolas} Ciceronis cum Gallica interpretatione, Catonem, Dictionarium, Testamentum Gallicum, Psalmos cum Catechismo, praeterea librum chartaceum ad scribendum dictata praeceptoris. Tu vero quos habes?

C. Omnes habeo quos enumerasti, praeter Catonem, Terentium et Ciceronis \emph{Epistolas.} Cur enim libros haberem, qui non praeleguntur in classe nostra?

L. At ego, dum sumus otiosi, lego interdum illos, ut semper aliquid addiscam novi; praesertim in lingua Latina et honestis moribus.

C. Prudenter facis, mi Languine. O me miserum, qui nunquam didici quid sit studiosum esse!

L. Disce igitur. Praestat enim sero quam nunquam discere.

\subsection{Colloquium 35}
\emph{Simeon ab Haggaeo librum commodato petit. Non dat Haggaeus, quia iam alteri dederat. Idem mavult iniuriam pati quam accusare condiscipulum ob privatam iniuriam. O caritatem rarissimam! Consilium bonum res est tutissima.}

Simeon, Haggaeus

S. Commoda mihi Vergilium tuum in duos dies, si nullo incommodo tuo id fieri potest.

H. Profecto non possum.

S. Cur non?\footnote{S. Cur non? D. Cur non!}

H. Cum Gerardus his diebus a me commodato accepisset, pignori apposuit.

S. Ain' tu, pignori?

H. Sic est ut dico.

S. Quanti oppigneravit?

H. Tribus, ut ait, assibus.

S. O hominem ingratum!

H. Tantumne ingratum?

S. Immo vero ingratum, et malum. Sed nunquid ille rem tuam oppignerare potuit, te inconsulto?

H. Potuit, ut factum vides.

S. Non tamen debuit.

H. Rem acu tetigisti. Sed quid facerem?

S. Rogas? Defer eum ad praeceptorem.

 
H. Malo istam pati iniuriam, quam committere ut miser vapulet.

S. Bene facis, dummodo tuum reddat.

H. Reddet, spero.

S. Unde redderet?

H. Ait se brevi accepturum a patre pecuniam.

S. Quid si te fallit?

H. Fieri potest, sed tamen aliquot dies exspectabo quid futurum sit, post deinde consilium capiam.

S. Consilio recto nihil est tutius.

H. Meministi probe, sic enim praeceptor dictitavit nobis. Sed nunquid vis aliud?

S. Ut bene tibi sit.

H. Et tibi optime.

\subsection{Colloquium 36}
\emph{Liber item commodato petitur, isque impetrantur, sed ab alio quodam repetendus. Beneficio non abutendum.}

Grauanus, Forestus

G. Visne mihi commodare tuum Terentium?

F. Volo equidem, modo illum repetas a Conrado, cui utendum dedi.

G. Quo signo vis repetam?

F. Nempe hoc, quod eius habeo \emph{Epistolas}.

G. Id mihi satis est.

F. Sed quando reddes?

G. Cum descripsero contextum in tres aut quattuor praelectiones.

F. Matura igitur, ne meo studio incommodes.

G. Maturabo.

F. Sed heus tu, cave macules, alioqui aegre commodabo posthac.

G. Nempe indignus essem beneficio.

\subsection{Colloquium 37}
\emph{Liber eleganter compactus ostentatur. Reprehenditur\footnote{S. Deprehenditur S. Corr. Reprehenditur} stulta quaestio. Excusatur quasi iocus fuerit. Iocus inter familiars usitatus. In iocis nostris Deus non est offendendus.}

Augustinus, Rodigus

A. Quis iste novus liber deauratus, quem tam magnifice ostentas?

R. Terentius.

A. Ubi impressus?

R. Lutetiae.

A. Quis tibi dedit eum?

R. Emi pecunia mea.

A. Unde nactus eras pecuniam?

R. Stulte, istud quaeris, quasi vero furatus sim?

A. Absit a me id cogitare, sed animi causa id rogabam.

R. Nec ego serio dictum tuum reprehendi, sed eo more iocari solemus cum familiaribus.

A. Nihil iocari prohibet, modo ne Deus offendatur. Sed age, ad propositum revertamur: de quo emisti Terentium istum?

R. De Clemente.

A. Illone bibliopola circumforaneo?

R. Maxime.

A. Quanti constitit?

R. Decem assibus.

A. Nihilne amplius?

R. Nihil omnino.

A. Profecto satis vile pretium.

R. Praesertim cum auratus, adeoque eleganter compactus sit.

A. Erantne codices alii similes?

R. Duo vel tres.

A. Deduc me quaeso ad illum.

R. Eamus.

\subsection{Colloquium 38}
\emph{Liber inventus redditur suo domino. Summo iure non semper est agendum. Leges scholasticae. Leges aequitate regendae. Causa sine periculo. Petitur gratiae relatio. Exemplum hic est caritatis.}

Alardus, Balbus

A. Nonne hic liber est tuus?\footnote{D. hic liber est tuus? S. hic liber tuus est?}

B. Ostende mihi; agnosco meum. Ubi invenisti?

A. In auditorio nostro.

B. Ago tibi gratias quod eum collegeris.

A. Atqui nunc notandus esses, si summo iure vellem tecum agere.

B. Quid ita?

A. Nescis leges nostras scholasticas?

B. Ipsae etiam leges cupiunt ut iure regantur.

A. Quo iure reguntur leges nostrae?

B. Aequitate et praeceptoris arbitrio, nempe qui nobis eas privatim condiderit. Praeterea, non solet tam severus esse in eo quod vel negligentia vel oblivione peccatum est.

A. Sic saepe expertus sum. Sed quoquo modo peccaveris, dicenda erit causa coram observatore.

B. Non timeo causam dicere ubi nihil est periculi.

A. Taceo.

B. Sed quaeso, quid opus est ut id sciat observator? Hic enim Deus nihil offensus est.

A. Age, celabo.

B. Bene facies.

A. Sed heus, memento par pari referre, si forte mihi acciderit aliquod delictum eiusmodi.

B. Aequum bonum dicis, meminero.

\subsection{Colloquium 39}
\emph{Repetitur commodatum. Dilationem petit commodatarius, eamque impetrat.}

Calliatus, Germanus

C. Cur non reddis mihi librum?

G. Exspecta in crastinum diem, nondum satis usus sum.

C. Libenter exspectabo.

G. Referam tibi gratiam, Deo volente.

C. Pro tantillo beneficio nullam exspectabo gratiam.

G. Tantum est meum agnoscere.

\subsection{Colloquium 40}
\emph{Quaeritur liber per oblivionem relictus. Arguitur neglegentiae qui reliquerat.}

Noaeus, Capellus

N. Vidistine librum meum?

C. Quem librum quaeris?

N. Ciceronis \emph{Epistolas}.

C. Ubi reliqueras?

N. Oblitus eram in auditorio.

C. Tua fuit negligentia.

N. Fateor. Sed interim indica, si quem scias accepisse.

C. Cur non adis praeceptorem? Solet enim (ut scis) quae a nobis relicta sunt aut ferre in musaeolum, aut alicui dare qui reddat.

N. Bene mones.

C. O me obliviosum, cui id in mentem non venit!

\subsection{Colloquium 41}
\emph{Beatus Ezechiel promptissime dat mutuam pecuniam. Exemplum expromptae liberalitatis.}

Ezechiel, Beatus

E. Vis a me magnam inire gratiam?

B. Nihil libentius fecerim, si quidem penes meres ipsa est. Sed quid est in quo tibi commodare possim?

E. Da mihi mutuo asses decem.

B. Non tantum nunc habeo, sed maiorem partem.

E. Quantum, quaeso?

B. Nescio, nisi crumenam inspexero. Ecce tibi octo asses cum semisse.

E. Solos septem accipio. Non enim te vacuare prorsus volo.

B. Parum refert; totum, si vis, accipe.

E. Ago tibi gratiam. Credo hoc pecuniae satis fore negotio meo, cum aliquantula quam ipse habeo.

B. Ut libet.

E. Amo te de ista tam exprompta benignitate.

B. Siquid aliud possim, ne parcas.

E. Reddam totum, Deo volente, cum primum pater ad me miserit.

B. Ne sis magnopere solicitus, nondum est opus mihi.

\subsection{Colloquium 42}
\emph{David a Nicolao quinque asses mutuo petit. Solum duos obtinet.}

David, Nicolaus

D. Potesne mihi mutuo dare aliquantulum pecuniae?

N. Quantum petis?

D. Quinque asses, si tibi est commodum.

N. Non tot habeo.

D. Quot igitur?

N. Tantum quattuor.

D. Bene sane, da mihi istos quattuor.

N. Dabo (si vis) dimidium.

D. Cur non totum?

N. Quia sunt mihi opus duo.

D. Da igitur duos, quaeso.

N. Sed tibi non sufficient.

D. Petam ab aliquo alio.

N. Accipe igitur hos duos. Quando reddes?

D. Die (ut spero) Sabbati cum pater ad forum venerit.

N. Esto igitur memor.

D. Ne timeas.

\subsection{Colloquium 43}
\emph{Custos habens pecuniam, Paqueto pauxillum petenti denegat. Spem tamen dat in postremum.}

Paquetus, Custos

P. Da mihi duos asses mutuo?

C. Nunc mihi non est promptum dare.

P. Quid obstat? Nam scio te his diebus accepisse pecuniam.

C. Accepi quidem, sed emendi sunt libri et alia mihi necessaria.

P. Nolo tuum commodum remorari.

C. Ubi emero quae mihi sunt opus, si quid supersit, faciam te libenter participem.

P. Interea igitur sperans exspectabo. Sed quid si nihil tibi superfuerit?

C. Statim dicam tibi, ne frustra diutius exspectes.

P. Quando emes ea quae decrevisti?

C. Cras (ut spero), aut ad summum perendie.

P. Bene habet, tempus est brevissimum.

\subsection{Colloquium 44}
\emph{Longa insinuatio petitur mutua in futurum pecunia. Ea statim promittitur. Exemplum astutiae, item animi prompti ad bene merendum.}

Morellus, Bobussardus

M. Abiitne pater tuus?

B. Abiit

M. Quota hora?\footnote{S. Quota? D. Quota hora?}

B. Prima pomeridiana.

M. Quid dixit tibi?

B. Multis verbis monuit me, ut diligenter studerem.

M. Utinam sic facias.

B. Faciam, Deo iuvante.

M. Ecquid pecuniae dedit tibi?

B. Dedit, ut fere solet.

M. Quantum?

B. Nihil ad te.

M. Fateor, sed tamen quid facies ista pecunia?

B. Emam chartam, et alia quae mihi sunt usui.

 
M. Quid si amiseris?

B. Aequo animo ferendum erit.

M. Quid si forte eguero? Dabisne mutuo?

B. Dabo, et equidem libenter.

M. Habeo tibi gratiam.

\subsection{Colloquium 45}
\emph{Redditur commodatum, cum excusatione dilationis. Non debemus offendi nisi ubi Deus offenditur.}

Columbanus, Fontanus

C. Satisne usus es scalpello meo?

F. Satis.

C. Redde igitur.

F. Accipe, ago tibi gratias.

C. Nihil est quod agas.

F. Sed ignosce quod non ultro et citius reddiderim.

C. Ea de re nihil sum offensus. Non enim debemus offendi, nisi cum Deum offendi videmus.

F. Recte sentis.

\subsection{Colloquium 46}
\emph{Cultellus datur commodato, sed aegre.}

Bergerius, Nepos

B. Commoda mihi parumper cultellum tuum.

N. Semper aliquid commodato petis. Accipe, quin tu emeres potius.

B. Non habeo pecuniam.

N. Cur non petis?

B. Unde peterem?

N. A patre.

B. Non est in hac urbe.

N. Ubi igitur?

B. Peregre profectus est.

N. Quo?

B. Bernam.

N. Quo die?

B. Nudiustertius.

N. Quando est reversurus?

B. Cras, ut speramus. Sic enim dixit proficiscens.

N. Bene vertat Deus.

\subsection{Colloquium 47}
\emph{Panis mutuo dandus appenditur.}

Columberius, Simo

C. Restatne tibi multum panis?

S. Satis, gratia Deo.

C. Visne dare mihi mutuo?

S. Libenter.

C. Sed fortasse tibi non sufficiet.

S. Immo, ut spero.

C. Ad quod usque tempus?

S. Ad diem Veneris.

C. Unde habebis postea?

S. Domo.

C. Quis afferet?

S. Egomet ibo petitum.

C. Quando?

S. Ipso die Veneris.

C. Da mihi mutuo sesquilibram.

S. Quis appendet?

C. Uxor praeceptoris aut ancilla.

S. Eamus petitum ex arca mea.

 
C. Quin ito solus, ego te in culina exspectabo.

\subsection{Colloquium 48}
I., L.

I. Oro te, da mihi ex pane tuo.

 
L. Mihi non habeo nimis. Tamen volo tibi impertire, accipe.

I. Gratias ago tibi.

L. Non est quod agas ob rem tantulam. Sed dic, quaeso, cur non attulisti?

I. Quia nemo erat domi nostrae qui mihi daret.

L. Tu vero cur non accipis?

I. Non audeo, nisi det mater.

L. Bene facis, sed audi bonum consilium.

I. Ausculto, ut audiam. Dic, quaeso.

L. Cum prandio finito tollantur mensae reliquiae, petito merendam tuam eamque in peram statim recondito. Ita fiet ut nunquam inanis venias.

I. De ientaculo autem quid suades?

L. Ut petas in exitu cenae, et idem facias quod dixi tibi de merenda.

I. Nunquam vidi melius consilium dari.

L. Fac igitur ut memineris, et cum voles, utere.

I. Ego vero utar quoties opus erit.

\subsection{Colloquium 49}
A., B.

A. Da mihi frustum panis.

B. Non habes?

A. Si haberem, non peterem.

B. Cur non attulisti?

A. Dicam postea; sed da interim, quaeso, nam esurio vehementer.

B. Cape.

A. Hui! Tantillum?

B. Etiam quereris?

A. Non immerito, das parce nimis.

B. Vide quantulum restat, dedi fere dimidium.

A. Ago tibi gratias; dedisti abunde, sed iocabar.

B. Nunc responde, cur non attulisti panem domo?

A. Nemo erat qui daret.

B. Nemo?

A. Prorsus nemo.

B. Quid mater?

A. Aberat domo.

B. Quid ceteri?

A. Omnes erant occupati.

B. Cur tute non accipiebas?

A. Nunquam auderem tale quidpiam.

B. Cur non?

A. Mater perpetuo vetat ne quid attingam sine permissu eius.

B. Dura mater.

A. Tuo quidem iudicio, qui indulgentiorem habes.

B. Non dico indulgentem, sed certe liberalem.

A. Quomodo te tractat?

B. Suavissime, omninoque ex animi sententia.

A. Fortasse in tuam perniciem.

B. Avertat Deus optimus maximus.

A. Non equidem invideo.

B. Cur ergo istud dicis?

A. Ut interim te moneam, ``Omnes licentia deteriores fieri.''

B. Bene facis. Sed quid censes? Nonne uti licet parentum bonitate?

A. Certe licet, modo ne abutaris.

B. Quomodo abutimur?

A. Rogas? Cum aut patris aut matris indulgentiam in malum vertimus.

B. Recte dicis: sed quotusquisque id facit?

A. Immo fere omnes, nisi qui a Domino Deo prohibentur.

B. Quis potest bonus esse, nisi per Dei gratiam?

A. Ergo (ut saepe monemur a praeceptore) precandus est, ut Spiritu suo nos bonos et sanctos efficiat.

B. Gaudeo te non attulisse ientaculum.

A. Quamobrem?

B. Quia hoc nostro colloquio mihi videor multum profecisse.

A. Ego quoque non parum.

B. Tua opera id factum est.

 
A. Immo beneficio Dei, qui quidem ita voluit.

B. Credo equidem.

A. Et hic igitur et in ceteris agnoscamus bonitatem eius.

B. Valde id aequum est.

A. Immo valde necessarium, siquidem volumus ingrati animi crimen effugere.

B. O sermonem iucundissimum!

A. Gratiae Deo immortales per Iesum Christum.

B. Ita velim.

\subsection{Colloquium 50}
\emph{Praemium victoriae causa petunt pueri. Datur a magistro, sed non sine admonition. Obligatur quis ex promisso. Solus Deus successum dat studiis. Sine Deo omnia vana.}

Discipulus primus ex victoribus, Praeceptor, Nomenclator

Discipulus. Praeceptor, visne dare praemiolum?

Praeceptor. Quamobrem?

D. Victoriae causa.

P. Ubi sunt compares tui?

D. Adsunt Hugo et Audax.

P. Heus, Nomenclator, suntne hi victores hac hebdomade?

N. Habent notas omnium paucissimas.

P. Ergo sunt victores, quid aliud ex te quaero? Vos igitur quod praemium petitis?

D. Quod tibi placuerit.

P. Quo tandem iure debeo?

D. Ex promisso.

P. Aequum dicitis. Quicquid enim recte promissum est, praestari debet.

D. Sic ex te didicimus.

P. Ecce vobis pennae singulae ad scribendum; ac ne putetis vulgares esse, ex earum sunt genere quae vulgo Hollandinae appellantur.

D. Gratias agimus, praeceptor.

P. Quin potius agite Deo, omnium bonorum auctori, qui dat studiis vestris successus prosperos. Vos autem in litterarum studio pergite diligenter.

D. Dabimus operam, quantum iuvabit ille Pater optimus.

P. Sine eius ope vana sunt nostra omnia.

\subsection{Colloquium 51}
\emph{Observatoris diligentia in exigenda a pueris ratione. Bene sibi conscius securo est animo. Omnis homo, mendax.}

Observator, Capperonus

O. Unde venis, Capparone?

C. Domo.

O. Quid affers illinc?

C. Merendam.

O. Quis tibi permiserat exire?

C. Praeceptor ipse.

O. Unde istud probabis?

C. Adeamus illum ut consulamus.

O. At vide quid agas.

C. Hac in re nihil timeo.

O. Adeone securus es?

C. Qui verum dicit, nihil timere debet.

O. Vera quidem ista est sententia, sed quotusquisque non mentitur?

C. Certus sum me nihil mentiri nunc.

O. Propemodum persuades mihi. Abi, credo tibi, quia in mendacio nunquam te deprehendi.

C. Est Deo gratia, quem precor ut me integrum et purum custodiat.

O. Utinam ex animo omnes precarentur! Recipe nunc te, ut edas merendam tuam.

\subsection{Colloquium 52}
\emph{Praemii parvitas contemnitur a sordido. Praemium non lucre sed honoris causa datur.}

Giraldus, Eliel

G. Qui sunt victores hac hebdomade?

E. Ubi eras cum rationes redderentur?

G. Accersitus a patre fueram. Sed qui sunt victores? Dic sodes.

E. Ego et Puteanus.

G. Iamne habuistis praemium?

E. Habuimus.

G. Quodnam?

E. Duodenas iuglandes.

G. Hui! Quale praemium!

E. Eho, inepte, aestimas ergo praemium ex rei pretio?

G. Hic nihil aliud video aestimandum.

E. Sordius es, qui lucro sic inhias. Non meministi verbum praeceptoris?

G. Quod verbum?

E. ``Non lucri, sed honoris causa datur praemium.''

G. Nunc reminiscor, quasi per nebulam. Posthac ero diligentior.

E. Sic tandem sapies.
\subsection{Colloquium 53}
\emph{Galatinus et Burcardus ante lusum tractant studiosum aliquid. A lusu ad studium reditur difficilius. Exemplum puerorum studiosorum.}

Galatinus, Burcandus

G. Euge, dimissi sumus ad lusum! Audistin'?

B. Quidni audirem, cum egomet adfuerim?

G. Placetne paulisper confabulari, deinde ludemus una?

B. Mallem ego prius ludere.

G. Atqui difficile est ludum abrumpere.

B. Plane verum dicis, et ego in me sic experior.

G. Quoniam igitur placet tibi mea ratio, da nobis aliquid argumenti ad confabulandum.

B. Immo tuum est dare, ut qui me lacessiveris.

G. Aequum dicis. Redde nomina Latine, quae tibi Gallice proponam.

B. Qua de re propones?

G. De supellectili.

B. Tentabo respondere, modo ne plura quam decem proponas.

G. Enumerabo in digitis, ne forte numerum excedam. Audi igitur.

B. Istic sum.

G. \fr{Un buffet}.

B. Abacus.

G. \fr{Un banc}.

B. Sella.

G. \fr{Un chandelier}.

B. Candelabrum.

G. \fr{Un coquemard}.

B. Ahenum.

G. \fr{Un soufflet}.

B. Follis.

G. \fr{Un coussin de lict}.

B. Pulvinus.

G. \fr{Un oreillier}. 

B. Cervical.

G. \fr{Un linceul}.

B. Linteum.

G. \fr{Un pot à cuire}.

B. Olla.

G. \fr{Un pot à vin}.

B. Oenophorum.
 
G. Errasti semel.

B. Ubi?

G. Dixisti ``linteum'' pro ``lodice.''

B. Fateor. Debeo tibi semel victoriam.

G. Nunc vicissim propone, ut redimas, si potes.

B. Vis respondere de eduliis?

G. Ut libet.

B. \fr{De la chair fresche}.

G. Caro recens.

B. \fr{Du porc}.

G. Suilla.

B. \fr{De la venaison}.

G. Ferina.

B. \fr{Venaison de sanglier}.

G. Aprugna.

B. \fr{Du laict boulli.}

G. Lac decoctum.

B. \fr{Du petit laict.}

G. Serum, vel serum lactis.

B. \fr{Du fromage nouveau.}

G. Caseus recens.

B. \fr{Brouet de chair.}

G. Ius carnium.

B. \fr{Poisson boulli}.

G. Piscis elixus.

B. \fr{De la sauce}.

G. Condimentum.

B. Falleris.

G. Quid ergo est?

B. Intinctus.

G. Condimentum volo.

B. At ego nolo contendere.

G. Quis contendit? Consulamus.

B. Quin prius ludamus, illud fiet posterius.

G. Age, fiat, ne amittamus ludendi occasionem.

\subsection{Colloquium 54}
\emph{Invitatio ad ambulationem, lusionis loco. Dei opera invitari nos debent ad eum laudandum.}

Moses, Olivetanus

M. Iamdudum taedet me toties repetere lusus scholasticos.

O. Quid facias igitur?

M. Eamus in hortum nostrum.

O. Quid agemus?

M. Ambulabimus, colloquemur, Dei beneficia in eius operibus commemorabimus.

O. Nihil sane iucundius. Sed interim petenda est a praeceptore venia.

M. Iam impetravi mihi, et item uni quem vellem ducere.

O. Bene res habe. Eamus, ducente Deo.

M. Precor ut nos custodiat.

O. Ego quoque idem precor tecum.

\subsection{Colloquium 55}
\emph{Sulpitius et Munchius parant se ad ludendum pila palmaria.}

Sulpitius, Munchius

S. Impetrastis ludendi facultatem?

M. Impetravimus.

S. Ad quod usque tempus?

M. Ad cenam usque.

S. Qui dederunt versus?

M. Primi et secundi.

S. Quid ceterae classes?

M. Primus quisque decurio trium proximarum classium pronuntiavit unam e Sacris Litteris sententiam.

S. Nonne precati estis, ut solemus?

M. Precati, et quidem praesente ludimagistro. Tu vero ubi eras?

S. Iveram domum, a matre accersitus.

M. Nunc igitur quid agere cogitas?

S. Ludere sesquihoram, deinde ad studium me recipere.

M. Vin' tu ut tibi sim collusor?

S. Quidni velim?

M. Quo lusu nos exercebimus?

S. Nullus est mihi iucundior pila palmaria.

M. Nec mihi quidem.

S. Visamus igitur an ceteri partes sortiti sint. Nam si soli luderemus, minus esset voluptatis.

M. Visamus sane.

\subsection{Colloquium 56}
\emph{Invitatio ad animum relaxandum. Corporis exerciatio confert valetudini.}

Miconius, Raverius

M. Visne ire mecum?

R. Quo properas?

M. Ad lacum.

R. Quid eo?

M. Lotum pedes.

R. I sane, nunc lotione mihi opus non est.

M. Sed interim parum fabulabimur.

R. Ne fabulari quidem nunc velim.

M. Atqui utilis est confabulatio, duntaxat de honestis rebus.

R. At mihi utilior est ad valetudinem exercitatio corporis.

M. Quid si mansero tecum?

R. Prudenter facies, et nos pila palmaria colludemus.

M. Bene vertat Deus, maneo.

R. Alias lotum una tecum ibo, cum scilicet longius erit temporis spatium.

M. Ad ludum igitur nos accingamus.

R. Nulla est in me mora.

\subsection{Colloquium 57}
\emph{Ob negotium lusus praetermissus. Exemplum diligentiae.}

Vincentius, Bonus

V. Cur hodie non lusisti nobiscum?

B. Non erat mihi ludendi spatium.

V. Quid habes negotii?

B. Non absolveram pensum meum.

V. Quod pensum?

B. Dimidium exemplaris restabat mihi perscribendum.

V. Perfecistine?

B. Perfeci.

V. Laudo tuam diligentiam, ludes alias otiose.

B. Cum voluerit Deus.

V. Recte dicis. Nam absque voluntate eius fieri nihil potest.

\subsection{Colloquium 58}
\emph{Pueri elementarii, extra ordinem, ludendi veniam petunt.}

Primus puer, Praeceptor, Secundus puer et Tertius.

Pueri. Salve, praeceptor.

Prae. Sit vobis salus a Christo.

Pueri. Amen.

Prae. Iamne repetivistis?

Pri. Etiam, praeceptor.

Prae. Quis docuit vos?

Pri. Subdoctor.

Prae. Quid nunc vultis?

Sec. Ut per te liceat nobis parumper ludere?

Prae. Non est ludendi tempus.

Ter. Non petimus omnibus, sed nobis parvulis tantum.

Prae. Atqui pluit, ut videtis.

Sec. Ludemus in pergula.

Prae. Quo lusu?

Sec. Aciculis vel iuglandibus.

Prae. Quid mihi dabitis?

Pri. Dicemus nomina.

Prae. Quot dicetis singuli?

Sec. Duo.

Prae. Dicite igitur.

Pri. \fr{Du papier}: charta; \fr{de l'encre}: atramentum. Dixi.

Sec. \fr{Un livre}: liber; \fr{un petit livre}: libellus. Dixi.

Ter. \fr{Une cerise}: cerasum; \fr{des noix}, iuglandes. Diximus.

Prae. Quam belli estis homunculi. Ludite ad cenam usque.

Pueri. Gratias agimus, praeceptor.

\subsection{Colloquium 59}
A., B.

A. Ubi nunc est pater tuus?

B. Puto eum esse Lugduni.

A. Quid illic agit?

B. Negotiatur.

A. Ex quo tempore?

B. Ab ipso initio mercatus.

A. Valde miror qui audeat illic commorari tot dies, cum pestilentia tanta sit per totam urbem.

B. Non est adeo mirandum.

A. Itane tibi videtur?

B. Ita profecto, fuit enim alias in maiore periculo, sed Dominus Deus semper eum custodivit.

A. Credo equidem, et adhuc custodiet. Sed quando est reversurus?

B. Nescio. In horas exspectamus.

A. Reducat illum Deus.

B. Ita precor.

\subsection{Colloquium 60}
\emph{Invitatur ad lusum quidam; at ille studium praefert lusui. Exemplum rarum pueri stuidosi. Omnia tempus habent. Utendum remissionibus in tempore.}

Elisaeus, Delphinus

E. Qua de re sic elatus es laetitia?

D. Pater meus modo advenit.\footnote{S. advenit modo D. modo advenit}

E. Quid mea refert?

D. Immo plurimum; quia nobis impretravit ludendi veniam.

E. Ain' tu?

D. Vide pueros iam ludentes in area.

E. Ludant sane alii, ego studere malim quam ludere.

D. Nec minus ego, sed in tempore. Nam, ut est in proverbio, ``Omnia tempus habent.''\footnote{Ecclesiasticus 3:1: ``Omnia tempus habent, et suis spatiis transeunt universa sub caelo.''} Unde et nos recte monet Cato noster:\begin{verse} ``Interpone tuis interdum gaudia curis;\\ut possis animo quemvis sufferre laborem.''\footnote{Disticha Catonis III.6}\end{verse}

E. Vera sunt quae dicis, fateor. Sed interim omitte me, ut serio studeam.

D. Per me studeas licet. Nihil impedio, at ego hac utar occasione.

E. Utere sane.

\subsection{Colloquium 61}
\emph{Pueri a lusu revocantur.}

Nomenclator, Quidam puer ex turba, item Alius

N. Heus, pueri, heus, heus!

Q. Quid clamitas?

N. Desistendum est a lusu.

Q. Eho, inepte, nondum quarta exacta est.

N. Immo fere semihora post quartam.

A. Cur non dedisti signum?

N. Quia tintinnabuli funis fractus est.

A. Clama iterum, sed attolle vocem.

 
N. Heus, pueri, recipite vos omnes. Festinate, festinate, inquam; urget praeceptor.

Q. Desine clamare, accurrunt omnes.

\subsection{Colloquium 62}
Orontius, Quintus

O. Quid ita laetus es?

Q. Quia venit pater.

O. Ain' tu? Unde venit?

Q. Lutetia.

O. Quando advenit?

Q. Modo.

O. Iamne salutasti?

Q. Salutavi cum ex equo descenderet.

O. Quid amplius illi fecisti?

Q. Calcaria detraxi et ocreas.

O. Miror te non mansisse domi propter eius adventum.

Q. Nec ille permisisset, nec ego vellem; praesertim nunc, cum audienda est praelectio.

O. Bene tibi consulis, qui temporis rationem habeas. Sed quid pater? Valetne?

Q. Recte, Dei beneficio.

O. Equidem gaudeo plurimum tua et eius causa, quod salvus peregre redierit.

Q. Facis ut amicum decet. Sed cras pluribus verbis colloquemur. Vide praeceptorem, qui iam ingreditur auditorium.

O. Eamus auditum praelectionem.

\subsection{Colloquium 63}
\emph{Marcus Aharonem Christinane et humaniter admonet. Eidem bene ultro facit. Caritas. Verecundia in petendo. Liberalis in dando, petere quoque non dubitat.}

Marcus, Aharon

M. Miseret me tui.

A. Quid ita?

M. Quod penna tua tam misere abutaris.

A. Quomodo abutor?

M. Quia pessime tractas eam in acuendo.

A. Non est culpa mea, ne quid erres.

M. Cuius\footnote{S. cuia D. cuius} igitur?

A. Scalpelli mei, cuius acies obtusa est.

M. Scalpellum in culpa non est, sed tu ipse.

A. Cur istud dicis?

M. Quia debuisti vel scalpellum tuum acuere, vel aliud alicunde rogare commodato, saltem ad praesens negotium.

A. Non audeo petere.

M. Quid times?

A. Ne mihi denegetur.

M. Ecce tibi meum.

A. Gratias ago.

M. Utere quantum voles, sed recte.

A. Sciens non abutar.

M. Nec sis posthac tam verecundus in petendo.

A. Sic est ingenium meum, soleo libentius dare quam petere.

M. Utinam multi essent tui similes! Sed tamen qui libenter dat beneficium, is petere quoque libere potest. Sed ego te nimis detineo, perfice quod coeperas.

\subsection{Colloquium 64}
C., D.\footnote{Sic `C' et `D' nominantur personae esse, sed `A.' et `B.' quoque scribuntur.}

A. Cur non venit Petrus in scholam?

B. Est occupatus.

A. In quo negotio?

B. In ligno struendo.

A. Qui scis?

B. Dictum est mihi.

A. A quo?

B. A patre eius.

A. Ubi eum vidisti?

B. Fuit mihi obvius cum venirem.

A. Vide ne mentiaris, nam ex illo quaeram si forte occurrat mihi per vicos.

B. Reperies sic ut dico.

\subsection{Colloquium 65}
\emph{Interrogationes curiosae de rebus minutissimis, nec tamen inutiles. Praeparatio in rebus omnibus. Patrisfamilias diligentia in re familiari.}

Sulpitius, Rogetus

S. Cur hodie mane abfuisti?

R. Occupatus eram.

S. In quo negotio?

R. In scribendis ad matrem litteris.

S. Quid opus erat illi scribere?

R. Quia ad me scripserat.

S. Ergo rescripsisti?

R. Proprie loqueris.

S. Unde tibi miserat litteras?

R. Rure, nempe ex villa nostra.

S. Quando rus profecta est?

R. Superioribus diebus.

S. Quid agit ruri?

R. Curat nostra negotia rustica.

S. Quid potissimum?

R. Praeparat ea quae sunt opus ad proximam vindemiam.

S. Prudenter agit.

R. Unde istud probares?

S. Nam, ``omnibus in rebus adhibenda est praeparatio diligens.''\footnote{Cicero \emph{De Officiis} I.73: ``In omnibus autem negotiis priusquam adgrediare, adhibenda est praeparatio diligens.''}

R. Quis te istud docuit?

S. Quidam paedagogus dictavit e Cicerone.

R. Qua occasione?

S. Cum admoneret ut me diligenter pararem ad reddendam hebdomadem postero die.

R. Profecto recte monebat.

S. Sed ad propositum revertamur. Non habetis villicum, qui curet ruri negotia vestra?

R. Immo et villicum\footnote{S. villicam D. villicum} habemus, et famulos et ancillas.

S. Quid igitur opus est illic tuae matris opera?

R. Quia melius novit providere rebus omnibus, quam isti imperiti rusticolae.

S. Nihilne amplius?

R. Sine me finire propositum.

S. Putabam te absolvisse, perge.

R. Etiam (ut ex patre audivi) praecipua cura domini requiritur in re familiari administranda.

S. Ergo pater tuus nunc potius deberet ad villam esse.

R. Non potest.

S. Quid prohibet?

R. Quia totus est in arte sua occupatus.

S. Maiorem (ut opinor) ex ea re fructum percipit.

R. Quis dubitet?

S. Inde igitur fit ut relinquat uxori omnem curam rei domesticae.

R. Omnino sic est.

S. Sed mater quando est reversura?

R. Vix ante perfectam vindemiam.

S. Quid tu? Non ibis vindemiatum?

R. A matre (ut spero) brevi accesar. Sed quaeso te, quid cogitamus? Iam omnes in auditorium currunt.

S. Bene res habet, curramus et nos ne postremi simus.

\subsection{Colloquium 66}
\emph{Quidam occupatus in merenda, monetur instare praelectionis tempus. Sic pueros studiosos mutua decet admonitio.}

Riparius, Amedaeus

R. Audivistine horologium?

A. Dudum sonuit.

R. Dinumerasti horas?

A. Dinumeravi.

R. Quota est?

A. Fere sesquiprima.

R. Instat igitur praelectionis tempus, fac ut paratus sis.

A. Ubi merendam peredero, ecce me paratum.

R. Cur meridie non adfuisti nobiscum?

A. Prodieram cum bona venia praeceptoris.

R. Sed interim sum tibi impedimento.

A. Nihil impedis, ne bolum quidem perdidi interpellatione tua.

R. Bene habet. Perge, sed matura.

\subsection{Colloquium 67}
\emph{Ligarius Sarrasinum ad repetendam praelectionem invitat. Exemplum diligentiae in studio puerili.}

Ligarius, Sarrasinus

L. Fecistine officium tuum?

S. Qua in re?

L. In repetenda praelectione.

S. Nihil adhuc repetivi.

L. Quid causae fuit?

S. Exspectabam dum rediret compar meus.

L. Quo ille ivit?

S. Domum.

L. Quid eo?

S. Petitum merendam.

L. Quid si redibit serius?

S. Nescio, fieri potest.

L. Vis interea mecum repetere?

S. Equidem non recuso.

L. Secedamus igitur, ne quis molestus sit nobis.

S. Profecto bene mones, nemo studere potest in tanto ambulatorum strepitu et clamore.

L. Aspice illic locum remotissimum, ubi nulli sunt ambulantes.

S. Eamus illuc.

\subsection{Colloquium 68}
D., E.

D. Quo properas?

E. Eo cenatum. Quid tu?

D. Iam cenavi.

E. Quota hora?

D. Quinta, ut fere solemus.

E. Quid nunc ages?

D. Repetam aliquid eorum quae reddere habemus crastino die.

E. Ego didici ex paedagogo meo, Non esse tam cito a cibo studendum.

D. Istud ego quoque didici, sed volo nunc ediscere.

E. Quid ergo facies?

D. Ego, quasi animi causa, praelectionem particulatim aliquoties legam et perlegam.

E. Quid tum?

D. Ita paulatim fiet, ut sine cura, sine taedio, sine molestia, bonam partem praelectionis ediscam.

E. Ista non satis intelligo, et certe videris mihi supra aetatem sapere.

D. Non est res adeo difficilis, quin te docere possem, nisi ad cenam properares.

E. De cena in tempore admones: ego igitur eius causa me domum recipio. Vale.

D. Ducat te Deus, et reducat.

\phantomsection
\addcontentsline{toc}{subsection}{Admonitio}
\subsection*{Admonitio}
\emph{Illa septem quae proxime sequuntur colloquia usque ad finem libri huius, ex nostris \emph{Rudimentis}\footnote{M. Cordier, \emph{Rudimenta Grammaticae De Partium Orationis Declinatu}, Hennri Estienne, Geneva, 1566, pp. 36--39; Bibliothèque Publique et Universitaire Neuenburg (Neuchâtel), ZQ 630 C \url{doi.org/10.3931/e-rara-33161}} iampridem impressis, ad verbum transcripsimus: propterea quod non minus quam cetera pueris profutura videbantur. Sunt autem exordia facillima, repetendae praelectioni accomodata.}\footnote{Sic S., D. non habet.}

\phantomsection
\addcontentsline{toc}{subsection}{Praefaticuncula}
\subsection*{Praefatiuncula quae septem proximis colloquiis praefixa erat}
Ne pueri nihil agendo discant male agere, praesertim otioso sermone, pravisque aut ineptis colloquiis sese invicem corrumpentes, omnibus modis incitandi sunt ut in schola dum praeceptoris ingressum exspectant, assuescant interea bini ternive id quod ab eo praescriptum fuerit simul repetere. Plurimum proderit haec illis iucunda exercitatio, eosque interim ab otio, lascivia, multisque aliis rebus malis, quibus offenditur Deus, avertere poterit. 

Sed quia sine puerili colloquio eiusmodi repetitiones tractari non solent, pueri autem ipsi, nisi instituti fuerint, nihil aliud quam barbare loquuntur. Idcirco, ut Latine inter se loqui paulatim discant hac de re, hic aliquot breves colloquendi formulas proposuimus. Ceterum in praeceptoris diligentia situm erit ut haec ipsa colloquia discipulis aliquoties interpretetur, doceatque quomodo in his, et aliis eius generis (quae tradere ipse poterit) sese et domi et in schola debeant exercere, ad idque illos identidem cohortetur. Ita fiet, progressu temporis, ut promptiores semper et alacriores eos habiturus sit ad ea mature reddenda quae scripserit. Hinc etiam consequetur ut minore cum labore ac molestia docendi munus exsequatur.

\emph{Argumentum in sequens colloquium,\footnote{S. Argumentum in sequens colloquium \textsc{lviii}} et in septem reliqua\footnote{S. Corr. sex reliqua S. septem reliqua.} huius primi libri.}

\emph{Pueri declinationum exempla a magistro praescripta, se vicissim audientes, repetunut: idque, ne sint otiosi tantisper dum praeceptoris ingressum expectant}

\subsection{Colloquium 69}

A., B.

A. Visne repetamus una?

B. Quidnam?

A. Id quod nobis praescriptum est.

B. Equidem volo. Sed quo genere repetendi utemur?

A. Audiamus nos vicissim.

B. Sic praeceptor nos monet saepe.

A. Recte monet, sed male paremus.

B. Uter incipiet?

A. Ego, si ita tibi placet.

B. Maxime placet, incipe igitur.

A. Secundae declinationis nomina his exemplis Latine declinantur: \emph{magister}, \emph{puer}, \emph{dominus}, \emph{lanius}, \emph{Antonius}, \emph{regnum}.

B. Quae sunt hodie declinanda?

A. Hesternum quidem est \emph{lanius}, hodiernum vero \emph{Antonius}.

B. Cur nos id repetimus quotidie, quod pridie reddidimus?

A. Quia sic praeceptor iubet.

B. Id satis scio, sed cur iubet?

A. Ad confirmandam memoriam.

B. Age, declina \emph{lanius}.

A. Nominativus singularis, \emph{hic lanius}; genetivus, \emph{huius lanii}. \emph{et cetera ad finem usque.}

B. Verte Gallice.

A. \emph{Lanius}, \emph{lanii}, generis masculini: \fr{un bouchier}.

B. Declina \emph{Antonius}.

A. Nominativus singularis, \emph{hic Antonius}; genetivus singularis, \emph{huius Antonii}; dativus singularis, \emph{huic Antonio}. \emph{Et cetera ad finem usque.}

B. Verte Gallice.

A. \emph{Antonius}, \emph{Antonii}, generis masculini, est nomen viri; \fr{c'est le nom prope d'un homme}. Gallice: \fr{Antoine}.

B. Cur dicis nomen viri? Tu nondum es vir.

A. Fateor, sed sunt alii Antonii, qui viri sunt.

B. Utinam aliquando vir evadas!

A. Evadam, Deo iuvante.

B. Attende nunc, ut vicissim audias me.

A. Istic sum, dic audacter.

B. Secundae declinationis nomina.

A. Desine, praeceptor adest.

B. Audio tussientem, desinamus, ne putet nos garrire.

\subsection{Colloquium 70}
C., D.

C. Mox aderit praeceptor, repetamus.

D. Quid opus est mihi repetitione? Solus repetivi satis, omnia teneo memoria.

C. Quid tum? Quanto saepius repetes, tanto melius tenebis.

D. Bene mones, habeo tibi gratiam.

C. Incipe, tempus abit.

D. Quartae declinationis.

C. Erras, Daniel, incipiendum est ab exemplo hesterno.

D. Erravi, fateor.

C. Dic igitur nunc recte.

D. Nominativus singularis, \emph{hoc sedile}; genetivus singularis, \emph{huius sedilis}. \emph{Et cetera ad finem usque.}

Quartae declinationis nomina hoc exemplo declinantur: nominativus singularis, \emph{hic sensus}; genetivus singularis, \emph{huius sensus}; dativus singularis, \emph{hoc senso}.

C. Hactenus, audio praeceptorem.

\subsection{Colloquium 71}
E., F., G.

E. Quid agis, Francisce? Instat praeceptoris adventus.

F. Scilicet, instat. Nondum est semihora post secundam.

E. Tamen non debemus interim tempore sic abuti, repetamus.

F. Non stabit per me. Ego enim sum paratus.

E. Incipe igitur.

G. Exspectate parumper, quaeso, ego sum vestrae decuriae.

E. Matura.

F. Dicamus suum quisque casum ordine, ut interdum nos docet praeceptor.

E. Satis est verborum, attendite.

F. Quid aliud agimus?

E. Tertiae declinationis nomina his exemplis Latine declinantur: \emph{pater}, \emph{lumen}, \emph{rupes}, \emph{messis}, \emph{pars}, \emph{sedes}, \emph{vectigal}, \emph{laquear}.

F. Nominativus singularis: \emph{haec rupes}.
 
G. Genetivus: \emph{huius rupis}.\footnote{S. -- haec rupes -- rupis -- rupi D. -- haec rupes -- huius rupis -- huic rupi etc.}

E. Dativus: \emph{huic rupi}.

F. Accusativus: \emph{hanc rupem}.

G. Vocativus: \emph{o rupes}.

E. Ablativus: \emph{ab hac rupe}.

F. Nominativus pluralis: \emph{hae rupes}.

G. Genetivus: \emph{harum rupum}.

E. Errasti, Gabriel, corrige erratum.

G. Genetivus: \emph{harum rupium}.

E. Dativus: \emph{his rupibus}.

F. Accusativus: \emph{has rupes}.

G. Vocativus: \emph{o rupes}.

E. Ablativus: \emph{ab his rupibus}.

F. Verte Gallice.

G. Rupes, rupis, genus femininum: \fr{une roche}.

E. Pone in oratione.

F. Non est in libro nostro.

E. Sed praeceptor docuit.

F. Alta rupes: \fr{une 'aute roche}.

E. More patrio dicis, aspera fortiter: \fr{haute}.

F. \fr{Une haute roche}

E. Nominativus singularis: \emph{haec messis}.

G. Genetivus: \emph{huius messis}.

E. Dativus: \emph{huic messi}.

\emph{Et cetera usque ad finem huius nominis; deinde sic pergunt colloqui.}

E. Ambo errastis.

F. Erravi, fateor.

G. Ego quoque, sed uter erit victus?

E. Praeceptor iudicabit.

F. Aequum dicis.

E. Vultisne dicamus iterum, ad memoriam confirmandam?

F. Quidni?

G. Quid si praeceptor interveniat?

E. Quid tum? Laudabit nos ore pleno.

G. Sed mutandus est ordo.

E. Non est dubium. Incipe, Francisce.

F. Tertiae declinationis nomina, \emph{et cetera.}

\subsection{Colloquium 72}
H., I.

H. Visne repetere mecum?

I. Cur tam cito?

H. Ne observator nos deprehendat garrientes aut otiosos.

I. Age, repetamus. Sed uter incipiet?

H. Ego, quia victor sum.

I. Dic igitur.

H. \emph{Prudens} nomen adiectivum sic declinantur in genera: \emph{hic prudens}, generis masculini; \emph{haec prudens}, generis feminini; \emph{hoc prudens}, generis neutri. Idem nomen sic declinantur in casus: nominativus singularis, \emph{hic} et \emph{haec} et \emph{hoc prudens}.

I. Genetivus: \emph{prudentis}.

H. Dativus: \emph{prudenti}.

I. Accusativus: \emph{prudentem} et \emph{prudens}.

\emph{Et cetera ad finem usque.}

\subsection{Colloquium 73}
L., M.

L. Cur tu es otiosus?

M. Non sum omnino.

L. Quid agis igitur?

M. Cogito de lectione reddenda.

L. Ego quoque id ago, repetamus una.

M. Fiat, sed quam rationem tenebimus?

L. Age praeceptoris partes, ego discipuli.

M. Valde placet mihi conditio.

L. Sed ne sis mihi austerior.

M. Ne timeas, nosti me satis.

L. Novi.

M. Declina \emph{lego} in modo infinito.

L. Infiniti modi tempus praesens et praeteritum imperfectum, \emph{legere}; praeteritum persectum et plusquam perfectum, \emph{legisse}.

M. Perge.

L. Sine me paulisper respirare, quaeso, praeceptor.

M. Age, sino. Satisne respirasti?

L. Satis.

M. Perge nunc.

L. Futurum indefinitum, \emph{lecturum esse}; gerundia, \emph{legendi}. \emph{Et cetera ad finem usque.}

M. Gaudeo te recte fecisse officium.

L. Ego vero mihi gratulor, sed est Deo gratia.

M. Recte dicis, utinam ex animo.

L. Ex animo certe.

M. Bene habet, desinamus. Sentio praeceptoris adventum.

L. Eccum, adest. St!

\subsection{Colloquium 74}
N., O., P., Q., R.

N. Heus, pueri, nos hic sumus quinque: repetamus hodiernum verbum, ut solemus coram praeceptore.

O. Nemo, ut opinor, contradicet.

P. Quis contradiceret? Nostra omnium res agitur.

Q. Incipe igitur Nicolae, qui primus sedes.

N. Optativi et subiunctivi modi praesens tempus, singularis: \emph{audiam}.

O. \emph{Audias}.

P. \emph{Audiat}.

Q. \emph{Audiamus}.

R. \emph{Audiatis}.

N. \emph{Audiant}.

O. Praeteritum imperfectum, singularis: \emph{audirem}.

P. \emph{Audires}.

Q. \emph{Audiret}.

R. \emph{Audiremus}.

N. \emph{Audiretis}.

O. \emph{Audirent}.

P. Praeteritum perfectum: \emph{audiverim}.

Q. \emph{Audiveris}.

R. \emph{Audiverit}.

N. \emph{Audiverimus}. \emph{Et cetera usque ad finem verbi hoc pergunt ordine.}

\subsection{Colloquium 75}
S., T., V.

S. Hic dies nobis est feriatus, et iam satis lusimus.

T. Satis, opinor.

S. Vultis ergo ut animi gratia conferamus de studiis nostris?

T. Sane mihi gratum feceris.

V. Mihi vero gratissimum.

T. Sed quid tractabimus?

S. Tentemus declinare aliquod verbum Latine simul et Gallice.

V. Incipe igitur, qui nos provocasti.

S. Faciam, quando ita placet vobis.

T. Audiamus.

S. Indicativi modi praesens tempus, singularis: \emph{doceo}, \fr{j'enseigne}; \emph{doces}, \fr{tu enseignes}; \emph{docet}, \fr{il enseigne}; pluralis: \emph{docemus}, \fr{nous enseignons}; \emph{docetis}, \fr{vous enseignez}; \emph{docent}, \fr{ils enseignent}.

T. Praeteritum imperfectum, singularis: \emph{docebam}, \fr{j'enseignoye}; \emph{docebas}, \fr{tu enseignois}; \emph{docebat}, \fr{il enseignoit}; pluralis: \emph{docebamus}, \fr{nous enseignions}; \emph{docebatis}, \fr{vous enseigniez}; \emph{docebant}, \fr{ils enseignoyent}.

V. Praeteritum perfectum, singularis: \emph{docui}, \fr{j'enseignay}; \emph{docuisti}, \fr{tu enseignâs}; \emph{docuit}, \fr{il enseignâ}; pluralis: \emph{docuimus}, \fr{nous enseignásmes}; \emph{docuistis}, \fr{vous enseignástes}; \emph{docuerunt} vel \emph{docuere}, \fr{ils enseignérent}.

S. Aliter Gallice: \fr{j'ay enseigné}, \fr{tu as ensigné}, \fr{il a enseigné}, \fr{nous avons enseigné}, \fr{vous avez enseigné}, \fr{ils ont enseigné}.

T. Praeteritum plusquamperfectum: \emph{docueram}, \fr{j'avoye enseigné}; \emph{docueras}. \emph{Et cetera. Sic pergunt quatenus placet}

\section{Liber Secundus} % 72 colloquia
\subsection{Colloquium 1}
\emph{Interrogatiunculae de acceptis litteris.}

Cornelius, Martialis.

C. Quid legis?

M. Litteras.

C. A quo?

M. A patre. 

C. Quando accepisti?

M. Heri, vesperi.

C. Quis attulit?

M. Nescio.

C. Nescis? Quis tibi reddidit eas?

M. Ancilla quaedam a caupone. 

C. Unde sunt datae?

M. Lutetia, credo. 

C. Quo die?

M. Nondum licuit inspicere. 

C. Nempe ego te interpellavi. 

M. Parum refert, non adeo sum occupatus. 

C. Age, perlege tuam epistolam, ego interea studebo. 

M. Ego quoque mox idem faciam. 

\subsection{Colloquium 2}
\emph{Musicus Herardum percontatur de rebus ad illum pertinentibus.}

Musicus, Herardus

M. Quo in statu sunt res vestrae Lugdunenses?

H. Nescio, iampridem nihil audivimus. 

M. Nihilne scripsit frater tuus?

H. Post menses duos nihil litterarum misit, quod viderit pater. 

M. Fortasse aegrotat. 

H. Minime vero, nam tabellarii saepe nobis salutem nuntiant verbis eius. 

M. Libenter audio illum recte valere. Valde enim diligo, quia fuit mihi suavissimus condiscipulus. 

H. Ille (ut opinor) te vicissim diligit. 

M. Id vero mihi non est dubium. Sed nos hora vocat, eamus in auditorium. 

H. Maturemus. Iam recitatur catalogus.

\subsection{Colloquium 3} % In verse
\emph{Publica observatoris admonitio ad condiscipulos in aula communi scholae universae. Admonitionis summa, ne pueri tempore abutantur. Extemporalis responsio cuiusdam primae classis.}

Observator, Brisantellus

\settowidth{\versewidth}{Desinite, o pueri, garrire, absente magistro.}
\begin{verse}[\versewidth]
\flagverse{Ob.} Desinite, o pueri, garrire, absente magistro.\\
Verba quibus summus laeditur ille Pater.\\
De studiis potius tractate et rebus honestis:\\
Discite sermones aptaque verba loqui.\footnote{S. Discite colloquiis vera Latina loqui D. Discite sermones aptaque verba loqui.}\\
Discite et inter vos reddenda revolvere saepe.\\
Doctor enim pueris semper adesse nequit.\\
Discite sectari vestigia certa bonorum. \\
Otia vos fallant blanda, cavete, precor. \\
En, ego praemoneo, vos ne delectet abuti \\
Tempore, ne tergum verbera dura premant!\\ 
Ecce iterum vobis morum praedico magister: \\
Si quis erit caesus, ne mihi det vitio.\\!

\flagverse{B.} Desine plura loqui, nemo parere recusat. \\
Est monitor nobis optimus ille Pater, \\
Illiusque Patris natus, cui nomen Iesus, \\
Et qui nos renovans Spiritus intus alit.\\!

\flagverse{Ob.} Quem mihi sperassem tam respondere paratum?\\
Quis puer Angelicos mittit ab ore sonos?\\
O quam te memorem, nostrae doctissime classis! \\
Nam tibi divinum carmen ab ore fluit. \\
Non sum tam felix ut fundam ex tempore versus, \\
Sed modo quae dixi praemeditatus eram.\\!

\flagverse{B.} Si meditatus eras, qui nunc tam fundis aperte \\
Castalios latices? Qui furor iste novus?\\!

\flagverse{Ob.} Nam tua me tantis moverunt carmina flammis, \\
Ut mihi nunc videar posse movere feras.\\!

\flagverse{B.} Sed cur immeritum tantis me laudibus effers? \\
Est tribuenda uni gloria summa Deo. \\
Atque utinam eloquium nobis spatiumque daretur, \\
Et nostra in laudes solveret ora suas. \\
Sed quia tempus adest ut voce et mente precemur, \\
Idque iubet doctor, desino\footnote{S. desine S. Corr. et D. desino} plura loqui.\\!
\end{verse}

\subsection{Colloquium 4}
\emph{Observator de minutis rebus interrogans, tentat deprehendere puerum quondam in mendacio.}

Observator, Puer

O. Quid agis?

P. Scribo. 

O. Quid scribis? 

P. Sententias.  

O. Quas? 

P. Ex Novo Testamento.  

O. Bene facis; unde habuisti? 

P. Hypodidascalus dictavit nobis.  

O. Quando? 

P. Heri.  

O. Quota hora? 

P. Meridie.  

O. Ubi? 

P. In area.  

O. Qui aderant?

P. Omnes domestici, praeter primos et secundos.

O. Ubi erant illi?

P. In aula communi.

O. Quid agebant?

P. Disputabant.

O. Vale, et perge scribere.

\subsection{Colloquium 5}
\emph{Hic nihil invenit observator quod notet.}

Observator, Pueri studentes

P. Quid vos hic agitis, pueri? Mihi videmini garrire, et nugas agere. 

Quidam Puer. Longe falleris, nam repetimus una. 

O. Qua de re?

P. De verbis anomalis, id quod reddendum est hora tertia. 

O. Bene facitis. 

P. Vis audire nostrum colloquium?

O. Immo pergite, maius opus moveo. Volo tendere laqueos picis et graculis. 

P. In area multos ad solem invenies. 

O. Retibus est illic praeda parata meis.

\subsection{Colloquium 6}
\emph{Pueros otio deprehensos non deterret observator, sed eos humaniter admonet. Exemplum observatoribus imitandum in causis levioribus, praesertim cum pueri non sun contumaces.}

Observator, Pueri garrientes

O. Atat! Ecce nunc capti estis, non fatemini?

Quidam Puer. Certe fatemur ingenue, sed non dicebamus mala verba. Quaeso te, mi Nicolae, ne velis notare nos.
 
O. Quid garriebatis? Audivi nescio quid de ientaculo.

P. Illud est, loquebamur de ientaculo matutino, quia famulus non dederat nobis in tempore.
 
O. Puto id fuisse, nec certe est valde magnum malum, nisi quod sunt otiosa verba. 

P. Sed Latine loquebamur. 

O. Audivi, sed non erat fabulandi locus. Nam (ut scitis) hoc pusillum temporis a merenda debet vobis esse valde pretiosum, cum sit dicatum studio; scilicet ut se diligenter quisque praeparet ad reddenda magistris ea quae praescripserint. Nonne verum dico?

P. Certe verum dicis; debuissemus legere simul de Testamento quae mox oportebit reddere. Sed ignosce, precor, suavissime Nicolae. Posthac erimus prudentiores, et officium nostrum diligenter faciemus.
 
O. Si sic feceritis, praeceptor vos amabit tanquam minuta sua intestina. Nonne videtis quemadmodum diligit bonos pueros et studiosos? Nec amat solum, sed etiam laudat et praemiolis afficit. 

P. Ista scimus, et quotidie experimur.

O. Ergo mementote, et promissa facite.

P. Tacebis igitur hanc culpam?

O. Tacebo, sed ea lege ut caveatis recidere. 

P. Cavebimus, Christo favente. 

\subsection{Colloquium 7}
\emph{Nomenclator officium diligenter facit in requirendis absentibus.}

Nomenclator, Puer

N. Ubi est frater tuus?

P. Modo ivit domum.

N. Quid eo?

P. Petitum nobis obsonium. 

N. Quid nunc opus est obsonio?

P. In merendam.
 
N. Annon habetis in arca vestra?

P. Non. 

N. Quid ita non?

P. Quia mater non solet nobis dare obsonium nisi in praesens tempus.

N. Nempe quia novit vos esse gulosos. 

P. Quomodo gulosi sumus?

N. Quia fortasse uno convictu devoratis quod in tres datum fuerit.

P. Tace, ego dicam fratri te nos gulosos vocare.

N. Tace ego dicam praeceptori fratrem tuum nihil aliud quam discurrere. 

P. Atqui prodire non solet, nisi cum bona venia praeceptoris. 

N. Atqui praeceptorem fallit. 

P. Quomodo fallit eum?

N. Non enim mens est praeceptoris ut ter quotidie prodeat. 

P. Sine illum venire, videbis quid tibi respondeat. 

N. Immo videat quid praeceptori respondeat. 

\subsection{Colloquium 8}
\emph{Pecunia, cum adfertur, nuntium adfert, vulgi opinione, iucundisimum. Reprehensio libera inter familiars. Evangeium, id est, gratia Dei a Christo nuntiata. Cur evangelio non omnes credant.}

Pastor, Langius

P. Frater tuus venitne Lugduno?

L. Iam venit heri ante meridiem.

P. Nihilne litterarum tibi attulit?

L. Nihil. 

P. Quid igitur nuntiavit?

L. Prospera omnia. 

P. De patre quid narrat potissimum?

L. Ait illum, Dei beneficio, iam plane febri carere et paulatim convalescere. 

P. Gaudeo sane, Deumque precor ut pristinam valetudinem brevi recuperet. Sed cur ille (ut solet) nihil ad te scripsit?

L. Negat frater eum potuisse scribere. 

P. Quamobrem?

L. Quia nondum satis erat confirmatus. 

P. Nihil mirum cum tamdiu tam graviter aegrotaverit. Sed ille nihil ad te misit?

L. Immo pecuniam. 

P. Euge, nullus est iucundior nuntius. 

L. Ita aiunt. 

P. Tu vero sic respondes, quasi fabulam audias. 

L. Quin peius audio. 

P. Quidnam?

L. Merum mendacium. 

P. Egone mentitus sum?

L. Non dico te esse mentitum, sed falsum dixisti. 

P. Ego quid dicas non intelligo. 

L. Dabo operam ut intelligas. 

P. Obsecro te. 

L. Si nullus est iucundior nuntius quam de allata nobis pecunia, quid ergo est Evangeium Christi? Quis est iucundior nuntius quam gratia Dei, quam Christus attulit nobis per Evangeium?

P. Fateor nihil esse iucundius evangelio, iis duntaxat qui credunt ei, et ex animo amplectuntur. 

L. Equidem sic intelligo. 

P. At ego loquebar de humanis et terrenis rebus, tu vero statim ad caelum ascendisti. 

L. Ita solent boni contionatores. 

P. Non putabam te esse theologum maturum. 

L. Nihil dixi nisi quod est tritum et in medio positum. 

P. Utinam illud adeo vulgare et protritum foret ut omnes in Christum crederent. 

L. Nunquam credent omnes. 

P. Quid prohibet?

L. Quia multi sunt vocati, pauci vero electi, sicut Christus ipse testatur.\footnote{Secundum Mattheum 22:14: ``Multi enim sunt vocati, pauci vero electi.”}

P. Sed ne te diutius teneam, potesne facere ut fratrem tuum paucis conveniam?

L. Vix possum. 

P. Quid ita?

L. Nam habet a patre nostro mandata plurima, in quibus curandis totus est occupatus. 

P. Nonne cenabit domi apud vos?

L. Cenabit, opinor. 

P. Ibo igitur sub horam cenae. 

L. Veni, precor, et eadem opera nobiscum cenabis. 

P. Non recuso. 

L. Interim vale, sed fac memineris adesse tempori. 

P. Quota hora?

L. Ante sextam. 

P. Hora est mihi commodissima.

\subsection{Colloquium 9}
\emph{Quidam ex provectis dissimulanter recusat puero chartam in labellum compingere, sed oblata occasione, de minutis rebus cum illo fabulatur, ut eum sic in Latina lingua exerceat. Eadem opera, ut puer caritatis exemplum discat, compingit ei labellum chartaceum. Postremo, illum paucis et brevibus praeceptis instituit. Exemplum dissimulationis in bonum.}

Vingnolus, Angelinus

V. Rogo te, Angeline, compinge mihi hanc chartam. 

A. Quid me rogas? Non est ars mea. 

V. Et tamen saepe compingis aliis.\footnote{S. Atqui aliis compingere non recusas. D. Et tamen saepe compingis aliis.}

A. Quot habes schedas?

V. Octo; sed iam sunt complicatae, tantum restat insuere membranae. 

A. Quid dabis, si tibi compegero?

V. Nihil habeo quod possum dare. Nam pecunia nulla mihi est. 

A. Ergo quaere tibi alium opificem. Non enim gratis faciam. 

V. Mi Angeline, tu es tam bonus, denegabis mihi rem tantillam?

A. Scin' tu quid\footnote{S. quod S. Corr. et D. quid} habet proverbium?

V. Quod proverbium dicis?

A. ``Manus manum fricat.''\footnote{Erasmus \emph{Adagia} I.\textsc{i}.33}

V. Quid hoc sibi vult?

A. Da aliquid, siquid velis accipere. 

V. Si quid haberem,\footnote{S. habeberem S. Corr. et D. haberem} certe libenter darem. 

A. Cedo merendam tuam. 

V. Merendam! Ah me miserum, quid ederem? Mallem dare pileum, si auderem. 

A. Merenda tua parva res est. 

V. Sed vehementer esurio. 

A. Quid causae est?

V. Quia nihil prandi, nisi frustum panis et tres aut quattuor iuglandes. 

A. Eho, quid causae fuit?

V. Quia mater domi aberat. 

A. Quis ergo tibi dedit merendam?

V. Ipsa. 

A. Atqui dicebas eam abfuisse. 

V. Verum est. Aberat enim tempore prandii, nec rediit nisi paulo ante merendam. 

A. Nihilne edisti domi, antequam in ludum venires?

V. Nihil. 

A. Cur non?

V. Quia timebam non adesse in tempore. 

A. Tintinnabuli sonitus te satis admonere debet. 

V. Sed raro audimus a nostris aedibus. 

A. Quid ita?

V. Quia longe nimis ab hac schola distant. 

A. Suntne vera ista omnia quae mihi narras?

V. Vera profecto, Angeline. 

A. Age, da mihi tuam chartam, ego tibi compingam elegantem libellum. Tu interim ede merendam tuam. 

V. Ego petam a matre sextantem, quem dabo tibi. 

A. Cave petas, nihil volo. Quin potius darem tibi, si egeres.

V. Ago tibi gratias. 

A. Nonne putabas me serio petere abs te merendam?

V. Profecto sic putabam. 

A. Atqui dicebam ioco. 

V. Quamobrem?

A. Ut paulisper audirem te Latine fabulari, nam quod bene discas gaudeo. Quanti emisti chartam hanc?

V. Dedi pro codice assem cum semisse. 

A. Non emisti male, bona est, sed non recte complicasti. Habes membranam?

V. Ecce tibi. 

A. Bene res habet. Ego citius confecero quam tu merendam perederis. 

V. Habebo tibi maximam gratiam, mi Angeline. 

A. Honeste loqueris. Sed memento, fili, ut semper vivas in timore Domini, diligenter oboedias matri, sis frequens in schola, diligens in studio, ne verseris cum pravis ac dissolutis, denique quibus poteris benefacito, quomodo vides me fecisse tibi. Intellextin’?

V. Optime. 

A. Fac igitur ut saepe recorderis. 

V. Faciam, Deo volente. 

A. Ede nunc otiose. 

\subsection{Colloquium 10}
\emph{De carnium differentia, et beneficiorum divinorum agnitione. Mendacium vitandum est. Christus veritas. Christiani sunt Dei filii et Christi fratres. Caritas in pauperes.}

Davinus, Mayus

D. Quid tibi dedit mater in merendam?

M. Vide. 

D. Caro est, sed quaenam?

M. Bubula. 

D. Utrum recens an salsa?

M. Est bubula salita. 

D. Utrum pinguis an macra?

M. Eho inepte, non vides macram esse?

D. Annon malles esse vitulinam aut vervecinam?

M. Utraque bona est, sed in omni genere sapit mihi haedina, praesertim assa. 

D. Hem, delicatule! Iamne palatum tam doctum habes?

M. Dico ut sentio, nec enim est mentiendum. 

D. Absint a nobis mendacia; sumus enim filii Dei et Christi fratres, qui est ipsa veritas, ut ipse de se loquens testatur. 

M. Sed ad rem. Suilla quoque vescor libenter, modico sale aspersa, et bene cocta. 

D. O mirificam Dei gratiam, qui dat nobis tot obsoniorum genera et tam bona!

M. Quot putas esse hac in urbe pauperes qui solo pane hordeaceo victitant, neque tamen ad saturitatem?

D. Non dubito multos esse, tanta praesertim annonae caritate. 

M. Itaque nos in tanta bonarum rerum copia, quantas Deo gratias agere debemus? Quas laudes illi dicere?

D. Eius igitur beneficia magnifice ubique praedicemus, atque interim precemur, ut pauperum suorum misereatur inopiae. 

M. Utinam ipse corda nostra suo Spiritu ad eam rem penitus afficiat. 

D. Ita precor.

\subsection{Colloquium 11}
\emph{Oblata occasione quidam provectus sermonem ad parvulis captum, docendi gratia, accommodat. Exemplum est, quod nulla bene agenda ocasio, praesertim in docendo, sit praetermittenda.}

Arnaldus, Bessonus

A. Quid rides?

B. Nescio.

A. Nescis? Magnum signum stultitiae!

B. Me igitur stultum vocas?

A. Minime vero. Sed dico tibi argumentum esse stultitiae cum quis ridet et ridendi causam nescit. 

B. Quid est stultitia?

A. Si diligenter evolvas Catonem tuum, istud quod quaeris invenies. 

B. Nunc non habeo meum Catonem, et volo aliam rem agere. 

A. Quod habes negotium?

B. Habeo ediscere aliquid de \emph{Rudimentis}. 

A. Interim quaeris fabulari, ineptule?

B. Dic mihi (quaeso) de stultitia in Catone. 

A. ``Stultitiam simulare loco, prudentia summa est.''\footnote{Disticha Catonis II.18} Annon haec didicisti?

B. Immo, sed non recordabar. 

A. Cum domi eris, inspice librum tuum. 

B. O quantas gratias ago tibi! Ego proponam alicui istam quaestionem qui non poterit mihi respondere, et sic erit victus. 

A. Tace, puer, tace et stude, ne vapules. 

B. Non multum curo, ego fere teneo praelectionem. 

A. Nisi taceas, dicam observatori, qui te statim notabit. 

B. Mane, mane. Nihil dicam amplius. 

A. Sed memento id quod dixi tibi. 

B. Quid est?

A. Ne rideas unquam sine causa. 

B. Sed ridere non est malum. 

A. Non dico istud. 

B. Quid igitur?

A. Stultum est sine causa ridere. 

B. Nunc intelligo. 

A. Recordare saepe.

\subsection{Colloquium 12}
\emph{Nuntius laetus de evangelio.}

Cleophilus, Melchisedec

C. Ecquid habes novi?

M. Accepi litteras a patre, qui Lugduni habitat. 

C. Quando accepisti?

M. Heri sub noctem. 

C. Quis attulit?

M. Piscarius. 

C. Quidnam intellexisti ex istis litteris?

M. Omnia bene illic habere quod ad evangelium pertinet. 

C. Verane praedicas?

M. Exspecta, litteras ipsas ostendam tibi a prandio. 

C. Est profecto quod nostris fratribus gratulemur. 

M. Scilicet, quodque Deo nostro gratias maximas agamus. 

C. Id quidem praestare debemus omni tempore, sed nunc maxime, cum audimus ea quae ad gloriam eius praecipue pertinent. 

M. Utinam hoc semper habeamus in memoria. 

C. Communicabis igitur mecum tuas litteras?

M. Ut promisi. 

C. Ergo post prandium. 

M. Etiam dubitas?

C. Interea vale. 

M. Vale et salve, Cleophile.

\subsection{Colloquium 13}
\emph{Interrogatiunculae de profectione ad locum, et reditu ex loco.}

Israel, Matthaeus

I. Estne domi frater tuus?

M. Cur istud rogas?

I. Pater meus volebat eum convenire. 

M. Non est in hac urbe. 

I. Ubi igitur?

M. Peregre profectus est. 

I. Quando?

M. Nudiustertius. 

I. Quonam ivit?

M. Lutetiam. 

I. Qua iter facturus est?

M. Lugduno. 

I. Utrum pedes an eques ivit?

M. Ivit in equo. 

I. Quando est rediturus?

M. Nescio. 

I. Sed quem terminum constituit illi pater?

M. Iussit ut hic adesset ad vicesimum huius mensis diem. 

I. Ducat illum Deus, ac reducat. 

M. Ita precor. 

\subsection{Colloquium 14}
\emph{Quiddam profecturus ad vindemiam, promittit amico se curaturum ut rus accersatur. Exemplum puerilis amicitiae.}

Duaeus, Ballinus

D. Quando profecturus es domum? 

B. Cras, Deo iuvante.

D. Quis iussit? 

B. Pater.

D. Quando autem iussit?

B. Ad me scripsit superiore hebdomade. 

D. Quo die litteras accepisti?

B. Die Veneris. 

D. Quid continebant praeterea litterae?

B. Omnes recte valere, proximis diebus initium fore vindemiae. 

D. O fortunatum, qui vindemiatum properas!

B. Vis dicam patri meo ut te accersat?

D. Quam gratum mihi faceres! Sed vereor ut nolit. 

B. Immo gaudebit, cum propter nostram coniunctionem, tum vero quia et Latine colloquendo nos exercebimus et de studiis una interdum conferemus. 

D. Oh, gaudio totus exsilio. Amabo te id cura, mi animule. 

B. Senties. Interim Deum precemur ut dicta et consilia nostra vertat in gloriam sui nominis.

D. Bene mones, et certe ita expedit facere. 

\subsection{Colloquium 15}
\emph{Aurelius a Lamberto eunte domum petit sibi uvas adferri. Exemplum desideria puerilis.}

Aurelius, Lambertus

A. Siste parumper gradum, Lamberte, quo properas?

L. Recta domum. 

A. Quid eo?

L. Mater me vult paucis convenire. 

A. Nescis quamobrem?

L. Nescio, nisi forte ut vestimenta hiberna mihi facienda curet. 

A. Istud est verisimile. Iam enim instat hiems. 

L. Iam visa sunt gelicidia et glacies etiam alicubi. 

A. His diebus vidi in foro montanos quosdam qui dicebant magnam vim nivium decidisse superiore hebdomade, cum hic interea lentas pluvias tantum videremus. 

L. Ego quoque id ipsum audivi domi nostrae ex rusticis qui triticum nobis advexerant. Sed cogor abrumpere sermonem, ne mihi irascatur mater. 

A. Sed heus tu, mi Lamberte, adfer mihi domo aliquot uvas. Nam amplissima fuit vobis vindemia. 

L. Adferam (ut spero) utrique nostrum affatim, nisi si quid forte mater irata est mihi. 

A. Istud avertat Deus.

\subsection{Colloquium 16}
\emph{Pelignus et Bartholomaeus euntes ad tonsorem colloquuntur. Nemo gaudet se adire intempestive.}

Pelignus, Bartholomaeus

P. Quonam is tam celeriter?

B. Ad tonsorem. 

P. Ego quoque una tecum. 

B. Rogasti veniam?

P. Non rogavi, sed tantisper exspecta me dum eo rogatum. 

B. Festina igitur.

P. Mox rediero. --- Redii, eamus nunc iam. 

B. Quo vultu te praeceptor excepit?

P. Hilaro sane. 

B. Eodem me quoque acceperat. 

P. Non solet irasci nobis, nisi illum adeamus intempestive. 

B. Quotusquisque id non aegre fert?

P. Etiam nos, qui pueri sumus, saepius irascimur condiscipulis, cum studia nostra, quantulacumque sunt, interpellant. Sed iam desinamus; optime tonsorem ante officinam video. 

B. Euge, nulli sunt igitur exspectantes. Ita fiet ut minus diu moremur.

\subsection{Colloquium 17}
A., B.

A. Quo nunc solus abis?

B. Semper a versibus aliquid auspicaris.

A. Facile est carmen incipere, deducere non item. Sed dic, quo nunc is?

B. Recta in portum. 

A. Quid in portu habes negotii?

B. Viso ecquid alimenti mihi advectum sit. 

A. Vis me tibi comitem?

B. Immo etiam ducem, si tibi ita videtur. 

A. Nunquam dux esse didici, sed saepe egi comitem. 

B. Nihil ad rem pertinet pluribus verbis hic fabulari, eamus una. 

A. Eamus sane, confabulari licebit amplius ambulando in ripa lacus, si navis nondum appulsa sit. 

B. Quid si iam appulsa?

A. Tamen deambulare satis diu poterimus, dum exonerabitur. 

\subsection{Colloquium 18}
\emph{De quodam rus profecto cum patre.}

Nomenclator, Puer

N. Ubi est Petrus?

P. Ivit foras. 

N. Quo?

P. Abiit rus. 

N. Quicum?

P. Cum patre. 

N. Quis venerat accersitum?

P. Patris famulus. 

N. Quando est in urbem rediturus?

P. Hinc (ut dixit) ad octavum diem.

 
N. A quo petivit abeundi veniam?

P. Ab hypodidascalo. 

N. Cur non potius a ludimagistro?

P. Ad negotia prodierat. 

N. Sat habeo. 

\subsection{Colloquium 19}
Richardus, Niger

R. Quo ascendis?

N. In cubiculum nostrum.

R. Quid eo?

N. Petitum thecam scriptoriam. 

R. Adfer mihi cingulum eadem opera. 

N. Ubi est?

R. Super arcam meam. 

N. Adferam, sed tu me hic exspecta. 

R. Nusquam moveo.

\subsection{Colloquium 20}
Levinus, Gerardus

L. Heus, Gerarde. 

G. Quid vis?

L. Accerseris. 

G. Quis me vocat?

L. Frater tuus. 

G. Ubi est?

L. Prae foribus te exspectat. 

G. Certo scis esse fratrem meum?

L. Quidni sciam? Vidi illum, et sum alloquutus. 

G. Viso sane quid sit.

\subsection{Colloquium 21}
\emph{Sub praecepto de parentum oboedientia continetur omnes qui nobis praesunt. Contentiones vitandae.}

Puteanus, Vitactus

P. Nescis vetitum esse ne submisse loquamur inter nos?

V. Quidni scirem, cum praeceptor tam saepe nobis inculcet eius rei causas?

P. Cur igitur modo faciebas contra?

V. Quia Isaias ita me coeperat alloqui. 

P. Quid tum? Debuisti illum admonere, non imitari. 

V. Debui, sed tunc mihi non venit in mentem. 

P. Sed interim notandus es. 

V. Minime vero, nisi vis esse ipso praeceptore severior. 

P. Dic mihi causam. 

V. Quia praeceptor vetat quempiam notari qui sponte delictum agnoverit, modo ne tale sit factum quod verbo Dei interdictum sit. 

P. Nonne a Deo praeceptum est ut parentibus oboediamus?

V. Illud est quintum Decalogi praeceptum. 

P. Atqui (ut habemus in Catechismo) praeceptum illud patet latius. Nam parentum nomine praeceptores complectitur et magistratus et denique omnes quibus Deus ipse nos subiecit. 

V. Non equidem nego vera esse quae narras. Sed malo praeceptorem consulere quam tecum disputare, alioqui in maius malum me induceres, quod est contentionis vitium, multo magis a praeceptore vetitum. 

P. Aequum dicis. Memineris igitur praeceptorem admonere cum rationes a nobis exiget. 

V. Ne putes me obliturum, praesertim cum res mea agatur. 

\subsection{Colloquium 22}
\emph{Curiosus res novas audire semper captat.}

Rossetus, Ferrerius

R. Unde venis?

F. E foro. 

R. Quid illic audisti novi?

F. Nihil. 

R. Nihilne?

F. Nihil prorsus. 

R. Mirum est te nihil audisse de bello aut de ceteris rebus Gallicis. 

F. De iis quae nihil ad me pertinent, non soleo percontari. 

R. Esto. Sed tamen aliquid vel in transitu audiri solet. 

F. Ne mentiar, intellexi nonnihil in transitu. 

R. Narra, mihi quaeso. 

F. Nunc non est narrandi spatium. 

R. Cur non?

F. Est mihi alio properandum. 

R. Quonam?

F. Nihil ad te. 

R. Quando igitur revises nos, ut ex te istud audiamus?

F. A prandio, si licebit per otium. 

R. Amabo te, fac liceat. 

F. Dabo operam, sed me detines diutius. Vale. 

\subsection{Colloquium 23}
A., B.

A. Unde venis? 

B. A foro.  

A. Quis te illuc miserat? 

B. Mater. 

A. Quid egisti in foro? 

B. Emi pira.  

A. Nescis nobis vetitum esse emere fructus aliquos? 

B. Quis istud ignorat? Nam dictum est palam in aula.

A. Qui igitur ausus es emere pira? 

B. Mater dederat mihi sex tantem, ut mihi emerem in merendam. Quid mali feci, si parui matri? 

\subsection{Colloquium 24}
\emph{De reditu ex villa.}

Franciscus, Dionysius

F. Ubi fuisti his diebus?

D. Ruri. 

F. Quo in loco?

D. In villa nostra. 

F. Quid agebas illic?

D. Ministrabam patri. 

F. Quid vero ille?

D. Pastinabat vites nostras. 

F. Quando illinc rediisti?

D. Heri tantum. 

F. Quid pater?

D. Una mecum reversus est.

 
F. Bene factum, sed quo nunc is?

D. Recta domum. 

F. Sed quando repetes ludum litterarium?

D. Cras, iuvante Deo, aut summum perendie. 

F. Ergo interim vale. 

D. Et tu vale, mi Francisce.

\subsection{Colloquium 25}
\emph{Officium promittitur amici absentis nomine.}

 Fontesius, Curtetus

F. Quando rediturus est Blasius?

C. Non certo\footnote{S. certe S. Corr. et D. certo} scio, fortasse die crastino. Sed cur istud rogas?

F. Quia secum abstulit catalogum, et praeceptor irascetur si nemo sit qui recitet.

C. Relinque mihi istam curam, habeo catalogi exemplum. 

F. Recitabis igitur?

C. Recitabo. 

F. Bene facies, et noster Blasius referet gratiam siqua se offeret occasio. 

\subsection{Colloquium 26}
\emph{Garbinus et Furarius iocantes se exercent in usu et syntaxi huius nominis \emph{domus} et obiter alia simul tractant. Sine Dei voluntate nihil fit.}

Garbinus, Furarius

G. Quod est tibi domicilium?

F. Paterna domus. 

G. Unde nunc venis?

F. Domo. 

G. Ubi prandisti?

F. Domi. 

G. Ubi cenabis?

F. Domi vestrae, ut spero. 

G. Qui scis?

F. Pater ipse tuus hodie me invitavit. 

G. Ubi illum vidisti?

F. Domi Varronis. 

G. Quod illic erat tibi negotium?

F. Pater me miserat nuntiatum aliquid. 

G. Scire etiam velim ubi sis cubiturus. 

F. Domi fratris.\footnote{S. Domi fraternae S. Corr. et D. Domi fratris}

G. Quid habes negotii cum fratre tuo?

F. Dixit sorori nostrae se velle convenire me otiose. 

G. In qua domo habitat?

F. In quadam conductitia. 

G. Eho, nullamne habet propriam domum?

F. Habet quidem, sed eam locat quibusdam inquilinis. 

G. Locat igitur domum propriam, conducit alienam?

F. Scilicet, ut ex me audis. 

G. Quanti locat?

F. Octodecim aureis Italicis, quos nunc \emph{pistoletos} vocant. 

G. Cur illam non potius habitat?

F. Quia sita non est in loco satis commodo, sive (ut ita loquar) mercatorio. 

G. Sed illam alienam quanti conducit?

F. Longe pluris. 

G. Quanti igitur?

F. Quinque et viginti. 

G. Cara est habitatio. 

F. Carissima, sed quid agas? Loci commoditas id facit. 

G. Age (ut aliquando tandem finiamus) dic precor, scisne ubi cras futurus sis?

F. Domum revertar ut inde in scholam me conferam, si quidem permiserit Dominus. 

G. Cur addis, ``si Dominus permiserit''?

F. Quia nisi permissu Dei, ne domo quidem exire possumus. 

G. Istud audivi saepe ex praeceptore. 

F. Cur ergo rogabas?

G. Quia nunquam nimis dicitur quod bene dictum fuerit, praesertim ubi de rebus divinis agitur. 

F. Istud quoque ex praeceptore didicimus. 

G. Verum,\footnote{S. didicimus: sed utile S. Corr et D. didicimus. G. Verum sed utile} sed utile est talia saepe repetere ad memoriam exercendam. 

F. Vide quo nos sensim adduxerit tua prima interrogatio? 

G. Tantum volebam paucis verbis tecum iocari. 

F. Age, quoniam nunc satis animum exercuimus, non vis etiam corpus exercere valetudinis causa?

G. Quidni velim?

F. Eamus igitur lusum pila palmaria. Nam eo lusu scio te delectari. 

G. Delector sane, sed nunc pilam non habeo. 

F. Ecce tibi, sequere me. 

G. Ego te sequor libens, tu me bene ducito. 

\subsection{Colloquium 27}
\emph{Pecuniam creditam reposcit creditor a debitore. Deus omnia gubernat.}

Creditor, Debitor

C. Quoad patris reditum exspectas?

D. Hinc ad diem octavum. 

C. Qui scis diem?

D. Ipse pater ad me scripsit. 

C. Adventus eius (ut spero) te ditabit. 

D. Croeso ditior ero, si bene nummatus venerit.\footnote{Erasmus \emph{Adagia} I.\textsc{vi}.74}

C. Tunc mihi mutuum reddes?

D. Non est quod dubites quin, si tibi opus erit amplius, non modo reddam mutuum, sed etiam referam gratiam. 

C. Quomodo?

D. Pecuniam mutuam vicissim dabo. 

C. Nihil opus erit, spero. 

D. At nescis quid possit accidere. 

C. Tempus est brevissimum. 

D. Non eo dico quod tibi vellem ominari malum. 

C. Quicquid ominentur homines, Deus clavum tenet. 

D. Sed quid cessamus recipere nos in auditorium?

C. Opportune admones. 

\subsection{Colloquium 28}
\emph{Patris docti diligentia in filiis domi docendis. Beneficium Dei agnoscendum ad iis qui bonos parentes habent. Fratres ex fide in Christum.}

Picus, Marcuardus

P. Quando rediisti domo?

M. Tantum redeo. 

P. Ubi est frater?

M. Mansit domi. 

P. Cur mansit?

M. Ut pranderet cum patre. 

P. Tu vero, cur etiam non mansisti?

M. Iam pranderam cum patre. 

P. Quis vobis ministrabat?

M. Ancilla. 

P. Quid mater? Ubi erat?

M. Etiam domi, sed occupata. 

P. Qua in re?

M. In recipiendo tritico quod nobis advectum fuerat. 

P. Quando redibis domum?

M. Cum accersar a patre. 

P. Quo die istud erit?

M. Fortasse hinc ad quattuor dies. 

P. Cur vos tam saepe commeatis?

M. Sic volunt parentes. 

P. Quid agitis domi?

M. Quod iubemur a parentibus. 

P. Sed interim perit vobis studiorum tempus. 

M. Non omnino perit. 

P. Quid igitur?

M. Quoties pater non est necessario occupatus, omnibus horis exercet nos: mane, ante et post prandium, ante cenam, a cena satis diu, postremo etiam antequam cubitum eamus. 

P. Quibus rebus vos exercet?

M. Exigit a nobis ea potissimum quae tota hebdomade in schola didicimus: themata nostra inspicit ac de iis nos interrogat; saepe dat nobis aliquid modo Latine, modo Gallice describendum; interdum etiam nobis proponit brevem sententiam vernaculo sermone, quam Latine vertamus; interdum contra iubet aliquid Latinum Gallice reddere. Postremo, ante cibum et post, semper ex Bibliis Gallicis aliquid legimus, idque tota praesente familia.

P. Nihilne de Catechismo interrogat?

M. Id facit omni die Dominico, nisi forte domo absit. 

P. Mira narras, si modo vera. 

M. Immo sunt longe plura quam tibi narravi. Sum enim oblitus morum civilitatem, de qua etiam admonere nos solet in mensa.

P. Cur pater vester tantum sumit laborem in vobis docendis?

M. Ut sic intelligat num in schola operam ludamus et tempore abutamur. 

P. Mira hominis diligentia, atque adeo prudentia! O quam devincti estis Patri caelesti qui talem patrem in terra dedit vobis!

M. Faxit ille ut hoc et cetera eius beneficia nunquam obliviscamur. 

P. Bonum et pium est istud optatum; cura ut habeas non modo in ore, sed in animo etiam magis. 

M. Quod me tam fideliter mones, habeo tibi gratiam. 

P. Bene monendi officium debemus omnibus, sed maxime fratribus. 

M. Fratribusne igitur solis?

P. Eos potissimum fratres hic dico, qui ex fide in Christum nobis coniuncti sunt. 

M. Recte iudicas. Sed viso num frater domo tandem reverterit, nam ad cessandum promptus est plus satis. 

\subsection{Colloquium 29}
\emph{Res ad bonam scripturam necessariae. Probus puer bene merenda occasionem non modo expectat, sed etiam quaerit. Caritas Christiana. Spiritus autor eius. Exemplum diligenter notandum.}

Myconius, Petellus

M. Serione scribis, an tu ineptis?

P. Equidem scribo serio. Cur enim tempore abuterer? Tu vero cur istud rogas?

M. Quia vidi aliquando cum bene scriberes. 

P. Scribo interdum melius. 

M. Qui fit igitur ut nunc tam scribas male?

P. Desunt mihi bene scribendi adiumenta. 

M. Quae tandem?

P. Bona charta, bonum atramentum, bona penna. Nam haec mea charta (ut vides) misere perfluit, atramentum est aquosum et subalbidum, penna mollis et male parata. 

M. Cur ista omnia mature non providisti?

P. Pecunia mihi deerat, et nunc etiam deest. 

M. Incidisti in illud vulgare proverbium, ``cui deest pecunia, huic desunt omnia.''

P. Sic agitur mecum.

M. Sed quando te speras accepturum?

P. In mercatu proximo pater ad me missurus est, aut ipsemet venturus. 

M. Ego te interea iuvare volo. 

P. Si quidem id potes, magno beneficio me affeceris. 

M. Accipe hos sex asses mutuo ad chartam et cetera comparanda. 

P. Quam vere dictum est illud, ``amicus certus in re incerta cernitur''?\footnote{Ennius apud Ciceronem \emph{Laelius De Amicitia} 64: ``Quamquam Ennius recte, `amicus certus in re incerta cernitur,' tamen haec duo levitatis et infirmitatis plerosque convincunt, aut si in bonis rebus contemnunt aut in malis deserunt.''} Sed quid te impellit ut mihi ultro tam benigne facias?

M. Caritas illa Dei, quae (ut Paulus ait), ``effusa est in cordibus nostris.''\footnote{Ad Romanos 5:3: ``Spes autem non confundit, quia caritas Dei diffusa est in cordibus nostris per Spiritum Sanctum, qui datus est nobis.''}

P. Mira est vis divini Spiritus, qui eius auctor est caritatis. Sed mihi interim cogitandum quomodo tibi referam gratiam. 

M. Parva res est. Omitte istam cogitationem, ne\footnote{S. et D. ne te S. Corr. nec te} te impediat quo minus in utramvis aurem dormias. Tantum redde mutuum cum tibi commodum fuerit. 

P. Reddam (ut spero) propediem.

M. Eamus ad precationem, ne notemur. 

P. Adde unum, si placet. 

M. Quid est?

P. Ne hodie incenati mittamur cubitum. 

M. Ha, ha, he!

\subsection{Colloquium 30}
\emph{Docti adolescentis colloquium cum puero mediocriter provecto. Admonitio ad praestandum mutuas operas. Precatio. Quid pueris agendum cum mane surrexerint. Deus agnoscendus est adiutor. Spes certa in Deo nominem frustratur. Philosophi ethnici multa verbo divino consentanea dixerunt.}

Velusatus, Stephanus

 V. Quota hora surrexisti hodie?

S. Paulo ante quintam. 

V. Quis te expergefecit?

S. Nemo. 

V. An ceteri iam surrexerunt?

S. Nondum. 

V. Non ivisti illos excitatum?

S. Non ivi. 

V. Quamobrem?

S. Nescio, nisi quia non putabam illud ad me pertinere. 

V. An non te illi excitant interdum?

S. Immo saepissime. 

V. Debuisti igitur similiter facere. 

S. Debui, fateor. 

V. Memento igitur ut posthac facias. 

S. Meminero, Deo iuvante. 

V. Sed tu quid fecisti ex quo surrexisti e lecto?

S. Primum, flexis genibus, precatus sum Patrem caelestem in nomine Filii eius Domini nostri Iesu Christi. 

V. Factum bene. Quid postea?

S. Deinde ornavi me, et curavi corpus mediociter ut Christianum decet. Postremo, ad quotidiana studia me retuli. 

V. Si perges sic facere, ne dubites quin Deus tua iuvet studia. 

S. Adhuc me semper iuvit, quae est eius benignitas, nec me (ut spero) derelinquet. 

V. Recte loqueris, nec ille spem tuam frustrabit. 

S. Anno superiore didici in Catone, \begin{verse}``Spem retine: spes una hominem nec morte relinquit.''\footnote{Disticha Catonis II.25}\end{verse}

V. Quod retinueris, bene fecisti; est enim egregia sententia et homine Christiano digna. 

S. Atqui auctor eius libri non fuit Christianus. 

V. Non fuit, certa res est. 

S. Unde igitur sumpsit tot pulchras sententias?

V. Maxime ex philosophi ethnicis. Nam et ipsi divino Spiritu illuminati plurima dixerunt quae sunt verbo Dei consentanea. Quod tu quoque videre aliquando poteris, si litterarum studium prosequere. 

S. Ego prosequar (ut spero) dummodo ipse Deus dat patri meo vitam longiorem. 

V. Precare diligenter et ex animo ut illud contingat. 

S. Quotidie id precor saepe. 

V. Det tibi Dominus Deus in omni opere bono perseverantiam. 

S. Quod mihi optas, idem tibi precor; et gratias ago quod me tam fraterne monueris. 

\subsection{Colloquium 31}
\emph{Invitatio ad lusum. Eventus ludi. Quicquid nobis accidit, est boni consutendum. Ditare aliquem solius est Dei. Spe lucri non est ludendum. Ethnicorum doctrina de fortuna. Admonitio careat odio et invidia. Solus Deus est cordium scrutator. Exemplum admonitionis mutus.}

Dominicus, Barrasius

D. Ubi sunt iuglandes tuae?

B. De quibus loqueris iuglandibus?

D. Quas hodie ex praemio accepisti. 

B. Ubi sint rogas? Quasi vero tibi servare debuerim. 

D. Non sic intelligo, sed quaero quid feceris. 

B. Edi in merenda.

D. Edisti, miser? Cur potius non servabas ad ludendum?

B. Edere malui quam perdere. 

D. Non poteras perdere nisi duodecim. 

B. Fateor. 

D. Quod si sors tulisset, potuisses ducentas aut fortasse plures lucrifacere. 

B. Dubius est (ut vulgo dicitur) ludi eventus.

D. Quid tum? Ubique parati esse debemus in utranque partem et boni consulere quicquid nobis evenerit. 

B. Istud ego scio, sed non sum admodum ludendi peritus in eo genere. 

D. Abi, nunquam rem facies. 

B. Nemo rem facit nisi Deo volente, nec ego ditari ex ludo velim. 

D. Ergo (ut video) quaerendus est mihi collusor alius. 

B. Nihil sane impedio. Sed mane parumper. 

D. Quid vis?

B. Quid tu vocas fortem, de qua hic mihi mentionem fecisti?

D. Ipsam fortunam. 

B. Quid autem est fortuna?

D. Stultorum opinio. 

B. De fortuna quid opinantur stulti?

D. Nunc mihi non vacat de hoc tibi respondere, sed vide annotationem praeceptoris in Catonem. 

B. In quem locum?

D. In illum versiculum,\begin{verse} ``Indulget fortuna malis, ut laedere possit.''\footnote{Disticha Catonis II.23}\end{verse}

B. Ut video, non ignoras quid sit fortuna. 

D. Satis scio fortunam nihil esse. 

B. Cur ergo dixisti, ``Quod si sors tulisset''?

D. Excidit mihi sic loqui ethnicorum more. Nam eorum libri (ut saepe docet praeceptor) pleni sunt eiusmodi impia doctrina. 

B. Nihil mirum, nempe illi veram Dei cognitionem non habuerunt. 

D. Sed audi, mi Barrasi, si vis amplius disputare, quaere tibi alios disputatores. Nam mihi nunc serio ludendum est, volo tamen prius te vicissim admonere. 

B. O quam gratum mihi feceris!

D. Nonne tu dixisti haec verba, ``Dubius est ludi eventus''?

B. Fateor me dixisse, sed praemunivi. 

D. Quomodo istud intelligis?

B. Addidi enim haec tria verba, ``Ut vulgo dicitur.''

D. O astutam vulpeculam! Os occlusisti mihi. Sed haec inter nos sine odio aut malevolentia dicta sint. 

B. Novit Deus utriusque animum. 

D. Est enim solus cordium scrutator. Sed quid tu? Vis hic solus otio torpescere?

B. Cogito quo lusu me exerceam. 

D. Quasi vero sit diutius cogitandum. Age, sequere me, dabo tibi mutuo iuglandes. 

B. Amice nunc loqueris. Sed quando reddam?

D. Ad Kalendas Graecas, si non potes citius.\footnote{Erasmus \emph{Adagia} I.\textsc{v}.84 `Ad Graecas Calendas', Suetonius \emph{Augustus} 87.1}

B. O festivum\footnote{S. festum caput S. Corr et D. festivum caput} caput! Eamus.

\subsection{Colloquium 32}
H., I.

H. Iacobe. 

I. Quid vis?

H. Repetamus una praelectionem. 

I. Non est mihi otium. 

H. Quid habes negotii?

I. Volo scribere. 

H. Quid scribere paras?

I. Dictata praeceptoris. 

H. Cur heri non scripsisti?

I. Quo tempore scripsissem?

H. Cum luderetur. 

I. At ego nolebam amittere ludendi occasionem. 

H. Ah, piger! Nihil aliud quam lusiones meditaris? Nonne diebus Mercurii et Sabbati ternae sunt horae liberae?

I. Liberae quidem sunt, sed ad lusum destinatae, deputatae, attributae, assignatae. 

H. Immo partim lusibus aut alicui honestae remissioni, partim studio litterario sunt dicatae.
 
I. Fateor, duntaxat iis qui volunt immori studiis. 

H. Non poteras heri sesquihoram detrahere ludo tuo, aut saltem horam unicam?

I. Tu stude quantum voles, ego vero ludam quandiu licebit. 

H. Per me quidem licebit, sed interim parum promovebis in litteris. 

I. Nonne praeceptori satisfacio?

H. Utcunque. 

I. Tu igitur, studiose, vis esse praeceptore ipso severior? Omitte me tandem, age tuum negotium, ego meum. 

H. Age, age, ut libet. 

\subsection{Colloquium 33}
\emph{Hic probi adolescentis admonition contemnitur ab improbo. Scientia res inaestimabilis. Misericordia. Bona externa perdunt multos. Divitiae sunt Dei donum. Virtutis et scientiae possessio. Mens bona.}

Rufus, Castrensis

R. Quando repetes ludum litterarium?

C. Nescio. 

R. Cur de hac re patrem non admones?

C. Quid putas me curare?

R. Parum admodum, ut credo. 

C. Profecto verum dicis. 

R. Satis est signi te non amare litteras. 

C. Scio legere, scribere, Latine loqui, saltem mediocriter; quid opus est mihi tanta scientia? Ego plura scio quam tres sacerdotes Papistici!

R. O miserum adolescentem! Siccine rem contemnis inaestimabilem?

C. Quid tu sic exclamas? Unde tibi videor miser?

R. Amice, nulla tibi a me orta est iniuria. Quod enim tibi dixi non est convitium, ne tu in malam partem accipias. Sed misereor tui, quia id contemnis quod felicitatem parit. 

C. Lucrum, divitiae, et voluptas felicitatem pariunt. 

R. Immo ista multis fuerunt exitio, tametsi divitiae sunt donum Dei, nec nocent, nisi iis qui abutuntur. Verum enimvero nulla est homini pretiosior possessio quam virtus et rerum honestarum cognitio. 

C. Vis igitur contionari, ut video. 

R. Utinam divinas contiones audivisses diligenter!

C. Hem, obtundis me. Nunquid vis?

R. Ut bonam mentem det tibi Deus. 

C. Ea fortasse tibi magis est opus quam mihi. Vale.

\subsection{Colloquium 34}
A., B.

A. Miror ego quid tibi velis; tu fere semper es otiosus aut garris aut ineptis. 

B. Quid vis faciam?

A. Ut studeas diligenter. 

B. Cur me istud mones?

A. Pro meo in te amore, tuaque utilitate. 

B. Frustra mones. 

A. Quid ita?

B. Quia non est mihi animus in litteris. 

A. Quid ergo velles?

B. Discere aliquam artem aptam ingenio meo. 

A. Iamne cogitasti quaenam ars tibi placeat potissimum?

B. Iampridem. 

A. Cur ergo patrem non admones?

B. Nunquam ausus sum. 

A. Cur non?

B. Vereor ut mihi irascatur. 

A. Roga praeceptorem, ut illi dicat. 

B. Immo te oro, dic tu ipse praeceptori meis verbis, nam verecundia me impedit. Faciesne quod rogo?

A. Faciam certe, idque libentissime. Valde enim taedet me videre te adeo remissum. 

B. O quam mihi gratum feceris!

A. Sed praeceptor te appellabit. 

B. Quid tum? Oblata occasio me audacem reddet ad mentem meam libere aperiendam. 

A. Recte iudicas. 

B. Tu igitur fac promissi memineris, deinde renuntia mihi quid ille responderit.

A. Alioqui essem tibi nuntius inutilis. 

\subsection{Colloquium 35}
\emph{Euntes in scholam Latine colloquuntur. Simile gaudet simili. Improborum mores probis odiosi.}

Michael, Frisius

M. Salve, Frisi. 

F. Tu quoque salvus sis, Michael. Quota est hora?

M. Mox audies sonum semihorae post quintam. 

F. Bene habet, mature satis aderimus. 

M. Gaudeo me tibi occurrisse, ut euntes Latine tantisper colloquamur.

F. Ea sane est utilis et iucunda exercitatio. 

M. Quoties incido in aliquem ex istis dissolutis nebulonibus, mallem rhedarium offendisse. Non enim per eos mihi licet aliquid in via meditari, adeo mihi sunt eorum mores odiosi.

F. Nihil mirum. Nam fere sunt eiusmodi ut neque loqui velint quidpiam boni, neque audire sustineant. 

M. Quid cum illis agas, quibus nihil est curae nisi ut suas libidines expleant?

F. Nihil aliud crepant nisi suas cupedias et meras compotationes in secretis cauponulis. 

M. Etiam nos irrident plenis buccis, quod Latine per vicos loquamur. Illud vero est omnium pessimum, quod nusquam se patiuntur admoneri.

F. Quia scilicet (ut ait Propheta), ``Non est timor Dei ante oculos eorum.''\footnote{S. M. Etiam nos \ldots loquamur. F. Illud vero \ldots admoneri. M. Quia scilicet \ldots eorum. M. Si quid \ldots S. Corr et D. M. Etiam nos \ldots loquamur. Illud vero \ldots admoneri. F. Quia scilicet \ldots eorum. M. Si quid \ldots}\footnote{Psalmus 35:2}

M. Si quid occeperis amice commonere, statim audies, ``Tace, contionator, obtundis me.''

Quod si dixeris, ``Deferam te ad praeceptorem aut ad observatorem,''

 ``Oh, ego bene curo,'' inquiunt, ``tu non auderes. Nam si me accusares, non feres impune.'' 

F. Immo vero te continuo verberabunt, si locus erit semotus ab arbitris. 

M. Profecto cum quidam eorum me nuper offendisset in quodam recessu, impegit mihi in utranque malam duos ingentes colaphos et aufugit continuo. 

F. Quid tu, quaeso, interea?

M. Quid istud quaeris? Tam illud subitum fuit ut vix hominem aspicere potuerim. 

F. Sed qui tam cito et sensim ad scholam pervenimus?

M. Sic fere confabulantibus usu venire solet. 

F. Age, ingrediamur sine murmure et strepitu ne studentes offendamus.

\subsection{Colloquium 36}
Probus, Amiculus

P. Unde adfers istam tuniculam?

A. A domo. 

P. Quid vis facere?

A. Volo induere. 

P. Nunc non est mutandi tempus. 

A. Quando igitur?

P. Cras mane, cum surges e lecto. 

A. Bene mones, exspectabo.

\subsection{Colloquium 37}
\emph{Antonius et Daniel colloquuntur de matrimonio. Adolescens doctus et bene moratus. Puella bene instituta. Exemplum prosperi coniugii.}

Antonius, Daniel

A. Euge, audivi sororem tuam nupsisse. 

D. Verum audisti. 

A. Quis est maritus eius?

D. Quidam civis Lugdunensis, honestis parentibus progenitus. 

A. Estne dives?

D. Sic habetur, sed tamen pater meus haec longe pluris facit: primum, quod ille sit bene moratus adolescens; deinde, quod non solum doctissimus, sed etiam bonarum litterarum amantissimus; denique, quod verus Dei cultor et Christianae religionis summus observator. 

A. Mihi narras egregios adolescentis titulos. O felicem sororem tuam, quae Dei beneficio talem virum nacta est!

D. Felicem sane non abs re dixeris, siquidem bonum illud perpetuo sic agnoscat, ut semper meminerit ex Dei bonitate profectum esse, ob idque immortales agat eidem gratias. 

A. Credo id facturam. 

D. Ita spero quidem, sic enim a parentibus semper instituta est in doctrina Christiana. Sed me iam alio revocant domestica negotia. Ergo vale, mi Antoni. 

A. Tu quoque bene vale, suavissime Daniel. 

D. Nunquid vis?

A. Ut verbis meis dicas salutem plurimam tuis omnibus, praecipue patri matrique et ipsi novae nuptae, meque illi gratulari faustum illud coniugium. 

D. Ego vero id faciam, et quidem libentissime. 

\subsection{Colloquium 38}
\emph{Henericus conferre quaerit de contione sacra. Gerardus recusat atque offenditur. Exemplum in morum dissimilitudine.}

Henericus, Gerardus

H. Hodie te non vidi in contione; quid illud sibi vult?

G. Quid sibi velit nescio, ego tamen interfui. 

H. Narra mihi quae mandasti memoriae. 

G. Non est tuum a me rationem exigere. 

H. Ego quidem non exigo, verum id rogo ut memoriae causa conferamus una. 

G. Malim nunc solus recordari. Audies me (si voles) cum praeceptor ante prandium nos interrogabit. 

H. Quid mali esset si nunc inter nos ea de re conferemus?

G. Nihil mali esset, fateor. Sed nunc mihi non libet. 

H. Tua igitur te libido regit. 

G. Omitte me, cur molestus es?

H. Omitto sane. Sed audi unum verbum, non decet puerum esse tam morosum.

G. Nec puerum decet esse tam molestum. 

\subsection{Colloquium 39}
\emph{Admonitionem bonam puer bonus in bonam partem accipit et gratiam\footnote{S. gratias S. Corr. gratiam} habet monitori.}

Rublius, Lepusculus

R. Quid fecisti regula mea?

L. Reliqui in pergula superiore. 

R. Cur eam reliquisti?

L. Oblitus sum. 

R. Non recte factum, sed tu sic fere soles, si quid tibi fuerit commodatum. 

L. Piget me negligentiae meae. 

R. Non satis est dolere, nisi mores mutare velis. 

L. Deum precabor ut mihi mutare velit. 

R. Si sapis, alioqui nemo tibi posthac commodare volet. 

L. Habeo gratiam quod me tam amice monueris. 

R. I nunc petitum meam regulam; est enim ea mihi opus ad ducendas in charta lineas. 

L. Nunc eo. 

R. Refer ad me in cubiculum. 

L. Mox habebis. 

\subsection{Colloquium 40}
\emph{Admonitus monitorem, vicissim monet. Suspectus caveas, ne sis miser omnibus horis. Nam timidis et suspectis aptissima mors est. Insuspiciosos.\footnote{Sic}}

Emericus, Baldus

E. Cur solus rides?

B. Quid tua?

E. Quia fortasse rides me. 

B. Unde tibi orta est ista suspicio?

E. Quia malus es. 

B. Omnes quidem mali sumus, at ego te peior non sum. Nemo igitur ridet nisi aliquem irrideat?

E. Non sic intelligo, sed qui solus ridet (ut saepe audivi) aut stultus est aut aliquid mali cogitat. 

B. Ista sententia cuius sit nescio, sed cuiuscunque sit, non est perpetuo vera. Tamen admonitionem tuam in bonam partem accipio; teque moneo vicissim ut caveas suspiciosus esse, ``Nam timidis et suspectis aptissima mors est,'' ut est in morali nostro carmine.\footnote{Disticha Catonis IV.43}

E. Memini. Boni tamen consulo admonitionem tuam.

\subsection{Colloquium 41}
\emph{Nathanael Mercerium monet ne sit curiosus alienarum rerum percontator. Horatius, Percontatorem fugitio, nam garulus idem est. Nec reticent patulae commissa fidiliter aures. Exemplum in percontatores.}

Nathanael, Mercerius

N. Unde venis?

M. Domo. 

N. Quid agitur domi vestrae?

M. Nihil tua refert. 

N. Fateor, sed familiares sic rogare fere solemus, perinde quasi rogemus, ``Ut valetis? Quomodo se vestrae res habent?''

M. De re aliena nimium percontari non decet. 

N. Taceo. Sed videris mihi pro aetate nimis sapere. 

M. Nihil meum dico, id audivi saepe. 

N. Ego quoque non semel audivi. 

M. Cur ergo non uteris?

N. Quia non semper in mentem venit. 

M. Immo quia tu es percontator, quae res non sine causa datur vitio. 

N. Habeo gratiam quod me adeo amice moneas. Posthac, adiuvante Deo, cavebo ineptus esse. 

M. Ita paulatim sapies.

\subsection{Colloquium 42}
\emph{Admonitio de civilitate morum.}

Humbertus, Plantinus

H. Heus tu, praeceptor adest. 

P. Quid tum?

H. Respice ad illum. 

P. Quamobrem?

H. Ut ei caput aperias et venientem salutes. 

P. Ita decet facere, sed aliud cogitabam. 

H. Tace. 

\subsection{Colloquium 43}
\emph{Latinus sermo in colloquiis exercetur. Colloquiorum materia. Otiosus sermo qui fit. Docilitas.}

Pontanus, Marcus

P. Unde redis?

M. Foris. 

P. Cur exieras?

M. Redditum urinam. 

P. Qualis est caeli facies?

M. Nebulosa. 

P. An regelat?

M. Sic resolvitur gelu ut nives omnino liquescant. 

P. Etiamne pluit?

M. Sensi aliquid superne distillare. 

P. Fortasse in transitu e stillicidio tecti. 

M. Immo e nubibus scio; quod si non credis, vide tu ipse. 

P. Quasi ego tibi non credam in re tantilla. 

M. Cur igitur dubitare videbaris?

P. Ut pluribus verbis tecum fabularer. 

M. Quorsum id pertinet?

P. Ad Latinum sermonem exercendum. 

M. Sed interim saepe otiosa verba dicimus, a quibus omnino abstinendum Christus praecepit. 

P. Tota erras via in praecepti intellectu. 

M. Cur istud dicis?

P. Quia non est otiosus sermo qui ad aliquam institutionem refertur, praesertim ubi agitur de bonis et honestis, qualia sunt Dei opera in rebus naturalibus. 

M. Videris mihi recte sentire, proinde facile tibi assentior. 

P. Sed haec hactenus, instat nobis aliud negotium. 

M. Age, desinamus.

\subsection{Colloquium 44}
\emph{Exemplum oblivionis.}

Trollietus, Bolanus

T. Scin' tu quota sit hora?

B. Non certum scio, sed video instare cenae tempus. 

T. Me miserum! Oblitus sum adire matrem, quae iusserat. 

B. Curre, curre! Opportune venies, ut cenes domi. 

T. Recte mones, eo rogatum veniam. 

B. Eccum hypodidascalum!

T. Optime adest. 

\subsection{Colloquium 45}
\emph{Rolandus et Languinus de scalpello recens empto colloquuntur. Res inventa restituenda est domino, si cognosci potest. Inventori dandum aliquid praemium. Deus precandus et laudandus.}

Rolandus, Languinus

R. Quid ais de scalpello quod emeram tibi nudiustertius? Estne satis bonum?

L. Immo vero est optimum, sed (me miserum!) parum abfuit quin perdideram. 

R. Eho, quid ais? Quomodo id accidit?

L. Cum redirem foris, exciderat mihi in vico. 

R. Unde exciderat?

L. E theca mea, quam imprudenter apertam reliqueram. 

R. Quomodo recuperasti?

L. Affixeram statim chartulam valvis ianuae.\footnote{S. valvis templi D. valvis ianuae}Post prandium quidam puer sextae classis mihi retulit. 

R. Utinam omnes tam fideles essent, qui res amissas reperiunt!

L. Profecto pauci sunt qui restituant, si modo sit res alicuius pretii. 

R. Et tamen id verbo Dei nominatim praecipitur. 

L. Quidni? Est enim furti species si quis rem alienam inventam retineat, modo sciat cui reddenda sit. 

R. At plerique putant se iure possidere quicquid amissum invenerint. 

L. Errant illi quidem gravissime. 

R. Verum (ut redeamus ad inceptum sermonem) quid dedisti puero qui scalpellum tuum invenerat?

L. Dedi sextantem et nuces aliquot iuglandes. Laudavi praeterea et paucis admonui idem semper esse faciendum. 

R. Recte fecisti; sic enim libentius reddet alias si quid repererit. Sed quid si perdidisses?

L. Aequo animo tulissem, et mihi emissem aliud. 

R. Itane aequo tulisses animo?

L. Certe non sine aliqua molestia. 

R. Non igitur aequo animo, sed nolo te arctius urgere. 

L. Non sumus theologi. 

R. Quid ergo?

L. Grammaticuli. 

R. Et quidem imperiti. 

L. Tanto diligentius Deum precari debemus, ut per evangelium suum nos liberet ab ignorantiae tenebris, in quibus versati sumus et adhuc versamur. 

R. Id vero faciemus, si sanctis admonitionibus pareamus quas audimus quotidie a praeceptore et saepe a contionatoribus, divini verbi administris. 

L. Vide quantum profuerit nobis scalpelli mei amissio. 

R. Ob eam rem tibi dupliciter gratulor: primum, quod tibi recte emerim; deinde, quod amissum recuperaveris. 

L. Habeo tibi gratiam, mi Rolande. 

R. Quin Patri nostro caelesti sit laus, et gratiarum actio. 

L. Amen. 

\subsection{Colloquium 46}
\emph{De pennis ad scribendum acuendis. Pennarum bonarum notae.}

Macarius, Calvinus

 M. Mihi non videris nimis occupatus. 

C. Mediocriter. 

M. Quid si mihi exacuas duas aut tres pennas?

C. Satis sit tibi si unam acuero in praesentia. Suntne novae?

M. Novae quidem, sed paratae usque dum acuantur. Iam enim laevigavi, caudam rescidi, detraxi plumulas. 

C. Ostende. Profecto sunt optimae et ad scribendum aptissimae. 

M. Unde id nosti?

C. Quia sunt caule amplo, firmo et nitido. Nam molles, et quae caulem breviorem habent, parum sunt ad scribendi usum habiles. 

M. Gaudeo me utiliter emisse. 

C. Non abs re, sed quanti?

M. Pro his tribus dedi quadrantes duos. 

C. Singulas igitur binis emisti denariolis. 

M. Res apparet. 

C. Est vile pretium pro rei bonitate, de quo emisti?

M. De quodam circumforaneo. 

C. Apud huius oppidi mercatores singulae, et quidem minus bonae, venduntur sextantibus. 

M. Et tamen audent interdum dicere pluris sibi constare Lugduni. 

C. Ea fere est mercatorum consuetudo: “Nihil enim proficiunt, nisi admodum mentiantur,” ut ait Cicero.\footnote{\emph{De Officiis} I.150: ``Sordidi etiam putandi, qui mercantur a mercatoribus, quod statim vendant; nihil enim proficiant, nisi admodum mentiantur; nec vero est quicquam turpius vanitate.''}
 
M. Sed age, ne te diutius remorer, agamus quod instat. 

C. Cito expediero, aspice me diligenter ut discas aliquando. 

M. Aspicio intentis oculis, sed mihi opus esset spatio paulo longiore. 

C. Istud ergo fiet in cubiculo, si quando me velis invisere. 

M. Quo tempore?

C. Post scholae missionem, hoc est hora nona matutina, vel quarta pomeridiana. Nunc habes pennas duas recte (ni fallor) in usum tuum accommodatas. Hanc tertiam in aliud tempus tibi integram servabis. 

M. Accipe tibi, si placet. 

C. Quin tibi serva? Domo adferuntur mihi satis multae. 

M. Ago tibi quantas possum gratias. Vale. 

C. Incolumem te conservet Deus. Sed heus, ne parcas unquam labori meo!

M. Tu quoque et me et rebus meis vicissim utere, si quid opus fuerit. Iterum vale.

\subsection{Colloquium 47}
\emph{Brevis confabulatio, sed doctrinae plena. Vir bonus et prudens occasionem docendi capit e re minima.}

Puteanus, Burela

P. Unde veniebas modo?

B. E culina. 

P. Quid illuc iveras?

B. Ut me calefacerem. 

P. Tu, credo, libentius es in culina quam in schola, nonne?

B. Nihil mirum; in schola non est ignis, sicut in culina. 

P. Abi, sapis. 

B. Utinam tam saperem in divinis rebus quam in cura corporis!

P. Fac sapias. 

B. Quomodo?

P. Studio, cura, labore et diligentia. 

B. Non parco labori. 

P. Recte facis. Sed est tempus exspectandum, cuius progressu fiunt omnia. Interea precandus est Deus assidue. 

B. Bene mones, utinam ille studia nostra promoveat in gloriam sui nominis. 

P. Id faciet, si pergamus eum colere diligenter. 

\subsection{Colloquium 48}
Puteanus, Capusius

P. Quid tecum cogitas, Capusi?

C. Libenter irem domum. 

P. Quid eo?

C. Ut me his diebus parumper recrearem cum matre. 

P. Quid obstat quo minus eas?

C. Praeceptor non vult permittere. 

P. Melius tibi consulit quam ipse putas. 

C. Quomodo?

P. Nam interea perderes multum temporis, et cum rediisses tibi doleret. Nonne verum dico?

C. Profecto sic est. 

P. Mane igitur, si sapis. 

C. Parebo tuo consilio, quia mihi rectum videtur. 

P. Nollem sciens malum tibi consilium dare. Atque utinam quod suadeo succedat tibi prospere. 

C. Spero ita fore, Deo volente.

\subsection{Colloquium 49}
\emph{Ab amico quidam mutuo assem impetrat. Obiter mentio fit de providentia Dei.}

Martialis, Blancus

M. Quantum habes pecuniae?

B. Assem cum semisse, tu vero?

M. Non tantum. 

B. Quantum igitur?

M. Unicum assem. 

B. Vis mihi dare mutuo?

M. Est mihi opus. 

B. In quem usum?

M. Ad emendam chartam. 

B. Hodie reddam tibi. 

M. Addendum fuit, ``Deo iuvante.''
 
B. Sic docet praeceptor ex verbo Dei, sed non possum assuescere. 

M. Fac assuescas. 

B. Quomodo id fiet?

M. Si saepe cogites nos a Deo sic pendere ut nihil possimus sine eius auxilio. 

B. Bonum mihi das consilium. 

M. Quale mihi dari velim. 

B. Sed (ut ad propositum redeamus) dabis mutuo istum assem?

M. Miror te mutuo petere, qui plus habeas quam ego. 

B. Est quidem scholasticus hac transiens qui librum venalem ostentat. 

M. Quid tum?

B. Cupiebam emere, quia vilius indicat quam noster bibliopola. 

M. Accipe. Sed, quaeso, unde tam cito reddes?

B. A cena ibo domum ut a matre petam. 

M. Quid si dare nollet?

B. Nihil cunctabitur, cum librum ostendero.

\subsection{Colloquium 50}
\emph{Montanus et Eusebius conferunt annos suos. Laudatur docti paedagogi diligentia. Item boni patris prudentia in liberorum instituendorum cura. Agnoscitur in ea re Dei beneficium. Exemplum exercitationis in lingua Latina a teneris annis.}

Montanus, Eusebius

M. Quot annos habes?

E. Tredecim, ut a matre accepi. Tu vero?

M. Equidem non tot habeo. 

E. Quot igitur?

M. Deest unus. 

E. Sunt ergo duodecim. 

M. In promptu est ratio. 

E. Sed frater tuus quotum agit annum?

M. Quintum. 

E. Quid ais? Iam Latine loquitur. 

M. Quid miraris? Semper habemus domi paedagogum et doctum et diligentem. Is semper nos Latine loqui docet; nihil Gallicum effert, nisi aliquid declarandi causa. Quinetiam patrem non audemus nisi Latine alloqui.

E. Nunquam igitur Gallice loquimini?

M. Solum cum matre, idque certa quadam hora, cum illa nos ad se vocari iubet. 

E. Quid agitis cum familia?

M. Cum familia rarus est nobis sermo et quidem tantum in transitu, et tamen famuli ipsi nos Latine alloquuntur. 

E. Quid ancillae?

M. Si quando usus postulat ut eas alloquamur, utimur sermone vernaculo ut solemus cum ipsa matre. 

E. O vos felices, qui tam diligenter docemini!

M. Est Deo gratia, cuius dono patrem habemus qui curet nos tam accurate erudiendos. 

E. Certe eius rei laus et honor unico caelesti Patri debetur. 

M. Sed quid agimus? Iam audio recitari catalogos. 

E. Ergo festinemus. 

\subsection{Colloquium 51}
\emph{Sylvius Ludovico aegrotanti suadet ut quiescat. Paret Ludovicus.}

Sylvius, Ludovicus

S. Quid tristis es, Ludovice?

L. Aegroto. 

S. Quid morbi est?

L. Nescio. 

S. Sed tamen, estne gravis morbus?

L. Non admodum, gratia Deo. 

S. Quidnam tibi dolet?

L. Caput. 

S. Quid? Totumne caput?

L. Non certe. 

S. Quid ergo? Utrum sinciput an occiput?

L. Haec pars anterior. 

S. Est ergo sinciput. 

L. Quid igitur faciam?

S. Quiesce bene, mox sanus eris. Sic enim a matre accepi, nullum esse praesentius remedium capitis doloribus quam quietem. 

L. Atqui sunt varii morbi capitis. 

S. Et varia fortasse remedia. Sed quid est facilius quam id tentare quod dixi tibi?

L. Experiri quidem nihil (ut spero) nocebit. Sed ubi quiescam?

S. Domi vestrae in lecto. 

L. Mater non sinet. 

S. Immo si dixeris te aegrotare. 

L. Atqui me putabit simulare. 

S. Fieri potest, sed quid dubitas periculum facere?

L. Bonum consilium. 

S. Utere, si vis. 

L. Faciam profecto. 

S. Enimvero, si sapis. 

L. Sed unum restat. 

S. Quid est?

L. Impetranda est a praeceptore venia. 

S. Adi, et pete. 

L. Quid si nolit dare?

S. Immo facillime. 

L. Qui scis istud?

S. Quia satis est credulus nobis, nisi qui aliquoties illum fefellerunt. 

L. Nunquam sciens illum fefelli. 

S. Ito igitur fidenter. 

L. Nunc eo. 

S. Sed heus, prius meditare quid sis dicturus, ne forte loquendo haereas. 

L. Bene mones, non accedam imparatus. 

\subsection{Colloquium 52}
\emph{Paulus et Timotheus certant uter melius pronuntiet aliquid eorum quae in schola edidicerint. Solomon iudex. Exemplum honesti studiosorum certaminis.}

Paulus, Timotheus, Solomon iudex

P. Optatus mihi ades, Timothee; quaerebam qui mecum certare vellet, sed omnes ad lusus certamen currunt. Tu vero quid ais?

T. Quid ego malim quam tecum de nostris studiis pacifice contendere?

P. Sed quod petis certandi argumentum? An de repetendis Ciceronis \emph{Epistolis}?

T. Malo de Catone. 

P. Quamobrem?

T. Quia restant mihi ediscendae aliquot praelectiones de Cicerone. Scis enim me aegrotasse fere duas hebdomadas. 

P. Memini. Vis igitur dicamus secundum librum \emph{Moralium Distichorum?}

T. Est longus nimis in hanc horam. 

P. Quid ita?

T. Quia nobis aliquandiu ludendum est ut corpus exerceamus ad valetudinem conservandam. 

P. Dicamus ergo librum tertium, qui est brevissimus. 

T. Sed iudicem volo. 

P. Praesto est Solomon, qui me ob eam rem sequitur.

T. Vis igitur, Solomon, audire nos?

S. Quid dicturi estis?

P. Tertium librum \emph{Distichorum Moralium}. 

S. Nonne alternis dicetis?

T. Scilicet suum uterque distichum. 

S. At pueri (ne erretis) nolo vos audire tanquam iudex. 

P. Cur non?

S. Ne forte mea sententia alteruter amicorum offendatur. 

T. In quo igitur nobis eris adiutor?

S. Notabo in chartula diligenter utriusque lapsus, deinde referetis ad praeceptorem. 

T. Quid tum fiet?

S. Ille utri videbitur et victoriam et praemium adiudicabit.\footnote{S. adiucabit S. Corr. et D. adiudicabit}

P. Eris igitur nobis tantum testis?

S. Sic intelligo. 

T. Optima sane videtur mihi ratio. 

P. Mihi quoque valde probatur. 

S. Sed unum restat. 

T. Quid est?

S. Vultisne, praeter lapsus manifestos, haesitationes etiam notari?

T. Sic volunt praeceptoris leges super hac re. 

S. Date mihi librum in manum, ut certius observare possim. 

P. Tene meum. 

T. Incipiamne?

P. Aequum est, quia tu a me provocatus es. 

T. Audi (quaeso) Solomon, sed diligenter. 

S. Tu vero cave dicas negligenter. 

\settowidth{\versewidth}{Hoc quicunque cupis carmen cognoscere, lector,}
\begin{verse}[\versewidth]
\flagverse{T.} ``Hoc quicunque cupis carmen cognoscere, lector,\\
Haec praecepta feres, quae sunt gratissima vitae.''\footnote{Disticha Catonis III praefatio 1--2:\\ ``Hoc quicumque voles carmen cognoscere, lector,\\ Cum praecepta ferat, quae sunt gratissima vitae.''}\\!

\flagverse{P.} ``Instrue praeceptis animum, nec discere cesses; \\
Nam sine doctrina vita est quasi mortis imago.''\footnote{III.1: ``Instrue praeceptis animum, nec discere cessa; \\ Nam sine doctrina vita est quasi mortis imago.''}\\!

\flagverse{T.} ``Commoda multa feres, sin autem spreveris illud, \\
Non me scriptorem, sed te neglexeris ipse.''\footnote{III praefatio 3--4:\\  ``Commoda multa feres, sin autem spreveris illud,\\ Non me scriptorem, sed te neglexeris ipse.''}\\!

\flagverse{P.} ``Cum recte vives, ne cures verba malorum, \\
Arbitrii nostri \ldots''\footnote{III.2: ``Cum recte vivas, ne cures verba malorum, \\Arbitri non est nostri, quid quisque loquatur.''}\\!
\end{verse}

\emph{Sic pergunt ad finem usque libri tertii.}

\subsection{Colloquium 53}
\emph{Porralis et Macardus de reditu a vindemia deque uvis allatis colloquntur.}

Porralis, Macardus

P. Gratulor tibi reditum, Macarde. Quando redisti rure?

M. Heri post meridiem. 

P. Quid mater?

M. Quemadmodum illa me secum duxerat, ita reduxit. 

P. Nonne venit in equo?

M. Et quidem tolutario. 

P. Tu vero?

M. Quid rogas? Eram illi a pedibus. 

P. Non tibi fuit molestus labor itineris?

M. Nulla mihi fuit via difficilis, adeo erat iucunda in urbem reditio. Quid quaeris? Noluissem eques venire. 

P. Quantum distat hinc villa vestra?

M. quattuor milliaribus, iisque non admodum longis. 

P. Sed iam satis de reditu. Nunc aliud agamus, fuistine memor promissi tui? Num redisti vacuus?

M. Attuli uvarum quantum potui. 

P. Quantum igitur?

M. Quasillum. 

P. Hui, quasillum! Tibi igitur uni. 

M. Immo nobis duobus.\footnote{S. ambobus S. Corr. et D. duobus}

P. Quid duobus tantillum?

M. Non poteram ferre amplius pro viribus corpusculi mei. Quod si robustus essem, asini onus asportassem; mater enim facile permittebat. 

P. Quam vellem adfuisse!

M. Ego et mater te plurimum desideravimus. Sed esto animo bono, ea reliquit famulum ruri, qui amplissima corbe onustus veniet. Tum illa tibi dabit affatim. 

P. Aha, nunc optata loqueris, mi Macarde. 

M. Eamus domum ad nos, videbis quasillum nostrum adhuc (ut spero) integrum. 

P. O lepidum caput! Nam et cupiebam ire salutatum matrem tuam mihi carissimam. 

M. Profecto illi gratissimum feceris. 

P. Eamus igitur. 

\subsection{Colloquium 54}
\emph{Dives ex occasione pauperi opem pollicetur. Deus hominum opera nos nec opinantes iuvat. Deus afficit animos ad bene agendum. Deus hominum affectus gubernat. Exemplum insigne caritatis erga pauperes.}

Antonius, Bernardus

A. Quid hic solus cogitas?

B. Meam deploro miseriam. 

A. Quaenam te afficit miseria?

B. Heu (me miserum!) ecce, mutavimus classem, nec est mihi pecunia unde libros emam. 

A. An non tibi dat pater?

B. Dat quidem interdum, sed parce nimis. 

A. Est igitur avarus. 

B. Non sequitur. 

A. Quid igitur impedit quominus tibi pecuniam suppeditet?

B. Paupertas. Praeterea, cum peto, miratur tot nobis opus esse libris. 

A. Nihil mirum, praesertim cum sit pauper. Sed interim esto animo bono, nec te afflictes, quaeso. Dabo operam ut te iuvet pater meus. Libenter enim largitur pauperibus, praesertim iis quos novit bonarum litterarum studiosos esse. 

B. O me felicem, si tua opera me Deus adiuverit!

A. Iuvabit, spero. Sed tu interim precare illum diligenter ut mei patris animum erga te affectum reddat. 

B. Bene mones. Nam (ut saepe audivi e sacris contionibus) solus est Deus qui hominum corda gubernat ac dirigit.

A. Ita res habet. 

B. Vale, mi Antoni, qui mihi animum reddidisti. 

A. Tu quoque, Bernarde, vale. Sed dic mihi, quaeso, quantum nummorum tibi opus est?

B. Si duos haberem decusses, abunde mihi esset in praesentia. 

A. Tace. Cras (ut spero) divinum auxilium senties. 

\subsection{Colloquium 55}
\emph{De frigoris acerbitate et hypocausti incommodo.}

Philippinus, Vuherius

P. Quo nunc is?

V. In hypocaustum. 

P. Quid eo?

V. Hoccine rogandum est? Non frigus sentis?

P. Quotusquisque nunc non sentiat, cum sit adeo acerbum? Sed ego malim me in culina calefacere. 

V. Atqui praeceptor vetuit. 

P. Non ignoro, sed rogabo veniam. 

V. Cur non vis in hypocausto calefieri?

P. Vapores clibani tentant mihi caput, quod alioqui infirmum satis habeo. Unde fit, ut facile ex capite laborem. 

V. Ego quoque sic aliquando fui, sed paulatim assuefeci me ad ferenda hypocausti incommoda. 

P. Et ego (ut spero) me assuefaciam. Verum praestat id fieri horis pomeridianis, ubi tantus aestus deferbuerit. 

V. Sed nunc tempus non est hic philosophandi diutius, iam mihi dentes frigore crepitant.

\subsection{Colloquium 56}
\emph{Attingitur differentia inter hortum et hortos. Rerum nomina tenere non satis est, nisi res ipsas cognoscamus.}

Stratanus, Theobaldus

S. Quae sunt arbores in horto vestro?

T. Hortum habemus suburbanum\footnote{S. surburbanum S. Corr. et D. suburbanum} in quo sunt olera, quibus vescimur quotidie, praeterea sunt in fundo nostro bini horti variis arboribus consiti. 

S. In horto quae sunt olera?

T. De hoc mater melius respondere posset. Nam illic saepe versatur aut serendi causa aut sarriendi aut aliquid colligendi. 

S. Sed tamen dic mihi aliquot olerum nomina. 

T. Parum prodesset nomina tibi recensere, nisi res ipsas videres. Quin eamus in rem praesentem. 

S. Potes ire quando libet?

T. Possum quidem, matre permittente. 

S. Fac, amabo, permittat. Sed ea lege ut me tibi comitem assumas. 

T. Id facillime fiet. Tantum hic me exspectes, mox rediero. 

S. Quid si ea domi non est?

T. Tamen hoc tibi renuntiabo. 

S. Bene vertat Deus. 

\subsection{Colloquium 57}
\emph{Amicus amico pecuniam ultro offert.\footnote{S. affert S. Corr. offert} Beneficium voluntarium. Humana beneficia diligentius quam divina expendimus. Exemplum liberalitatis rarissimum.}

Praepositus, Caulonius

P. Hodie pecuniam a patre accepi, si tibi forte est opus. 

C. Nihil nunc opus est mihi, sed tamen gratiam habeo maximam quod pro tua liberalitate ultro mihi offers beneficium. Quotus enim quisque id faciat?

P. Credo esse paucissimos, tu me tamen non semel beneficiis provocasti. 

C. Adeo parva illa fuerunt ut non sint commemoratione digna. 

P. Non est parvum beneficium quod ab optima voluntate profectum sit. 

C. Utinam Dei erga nos beneficia tam expenderemus quam solemus hominum!

P. Faxit ille ut in ea cogitatione nos exerceamus et saepius et diligentius. 

C. Illud profecto necesse est, si volumus eius benignitatem saepius experiri.

\subsection{Colloquium 58}
\emph{Fatonus Barbarino vult dictata describere. Barbarinus permittere non audit, ne a praeceptore arguatur. Nec fallendum nec mentiendum est. Praestat puerum timidiorem esse quam audaciorem.}

Fatonus, Barbarinus

F. Quid agis?

B. Scribo. 

F. Quid scribis?

B. Describo dictata praeceptoris. 

F. Quaenam?

B. Hesterna. 

F. Quid? Non aderas?

B. Immo aderam, sed non poteram magistrum dictantem assequi. 

F. Quae res te impediebat?

B. Quod satis commode non sederem. 

F. Veneras ergo serius?

B. Istud est. 

F. Cedo commentarium tuum, egomet tibi scribam. 

B. Quid faciam lucri?

F. Ego citius quam tu descripsero, post ludemus una, ut concessit praeceptor. Da (inquam) libellum tuum. 

B. Libenter id quidem facerem, sed non audeo. 

F. Quid times?

B. Edictum praeceptoris. 

F. Quod mihi edictum narras?

B. Nescis eum vetuisse ne quis sine permissu eius alteri scribat?

F. Id ego probe memini, sed unde hoc sciret?

B. Rogas? Cum emendandi causa scripturae rationem exiget, tum captus ero; novit enim manum meam. Praeterea, neque fallendum est neque mentiendum. 

F. Verbo Dei utrunque vetamur. 

B. Quid ergo responderem praeceptori, cum ille negaret me ista scripsisse?

F. Non eo res evadet, spero. 

B. Nolo tua spe tantum subire periculum. 

F. Vah! Nimium timidus es, nunquam rem facies. 

B. At tu forsitan audacior. 

F. Tu igitur scribe quantum voles, ego ad ludendum me confero. 

B. Abi, quaeso, iam unam paginam descripsissem nisi me interpellasses. 

F. At interim aliquid proficimus, dum Latine fabulamur. 

\subsection{Colloquium 59}
\emph{De accepta pecunia et mutuo facile data. Amicus verus. Exemplum animi ad bene merendum promptissimi.}

Berthinus, Probus

Q. Venitne pater ad mercatum hodiernum?

R. Hodie mane convenit me, cum adhuc e lecto surgerem. 

Q. Nihil ab eo petisti?

R. Immo pecuniam. 

Q. Et numeravit?

R. In praesentia. 

Q. Quantum, obsecro?

R. Viginti asses. 

Q. Papae! Asses viginti? Qui fit ut audeat tibi tantum pecuniae committere?

R. Quia novit me dispensatorem frugi. Siquidem semper illi reddo rationem usque ad teruncium. 

Q. Sed aegre fortasse impetrasti. 

R. Immo facillime, atque adeo cum gratia. 

Q. O mitem parentem!

R. Certe mitissimum. 

Q. Sed (ut ad rem) quid facies ista pecunia?

R. Emam libros et alia mihi necessaria. 

Q. Potesne mihi aliquid mutuo dare?

R. Possum, si modo eges. 

Q. Nisi egerem, non peterem. 

R. Quantum vis a me accipere?

Q. Quinque asses. 

R. Accipe. 

Q. O vere amicum animum!

R. Amicus verus non est nisi qui amicum iuvat in tempore, si tamen habet unde iuvet. 

Q. ``Amicus certus,'' (ut est in proverbio), ``in re incerta cernitur.''\footnote{Ennius apud Ciceronem \emph{Laelius De Amicitia} 64}

R. Quando reddes mutuum?

Q. Ubi primum pater in hanc urbem venerit. 

R. Quando venturum speras?

Q. In mercatu proximo, nempe ad octavum diem Octobris.

\subsection{Colloquium 60}
\emph{Paedagogus, ad exercedndam suorum discipulorum memoriam, a quodam eorum exercitationis matutinae rationem exigit. Narrat puer ordine et distincte singula, idque memoriter, et audientibus condiscipulis. Exemplum ad pueros in simplici narratione exercendos, ut puro sermoni paulatim assuescant.}

Paedagogus, Puer

Pae. Hodie mane quota hora expergefactus es?

Pu. Ante lucem, quota hora nescio. 

Pae. Quis te expergefecit?

Pu. Venit excitator hebdomadarius cum laterna sua, pulsavit acriter ostium cubiculi, quidam aperuit, excitator accendit nostram lucernam, elata voce inclamavit; experrecti sunt omnes. 

Pae. Narra mihi ordine quid egeris ex illo tempore usque ad finitum ientaculum. Vos, pueri, auribus atque animis diligenter attendite, ut discatis hunc vestrum condiscipulum imitari. 

Pu. Experrectus surrexi e lecto, indui tunicam cum thorace, sedi in scabello, accepi femoralia et tibialia, utraque indui, calceos calceavi, femoralia ligulis astrinxi thoraci, tibialia periscelide ligavi super crura, cingulo me praecinxi, caput diligenter pexui, aptavi capiti pileolum, togam indui. Deinde egressus cubiculo, descendi infra, urinam in area reddidi ad parietem, accepi aquam frigidam e situla, manus et faciem lavi, os et dentes collui, detersi mantili manus et faciem. 

Interea signum ad precationem datur minore tintinnabulo, in aulam privatam convenitur, precamur una, accipimus ordine ientaculum a famulo culinario, ientamus in triclinio sedentes quieti, sine murmure et strepitu. Quos audivi inepte garrientes aut verba loquentes otiosa aut etiam lascivientes vidi, amice admonui, qui non paruerunt admonitioni, detuli ad observatorem ut eos notaret. 

Pae. Nemone vobis praeerat dum ientaretis?

Pu. Immo hypodidascalus. 

Pae. Quid agebat interea?

Pu. Ille per mediam aulam ambulabat tenens librum in manibus et identidem monens observatorem ut notaret inepte garrientes. 

Pae. Nullumne igitur verbum tunc licet emittere?

Pu. Immo licet, verum ii demum notari solent qui diu et multis verbis inepte et sine ullo fructu confabulantur. Ceterum licet omnibus iucundos inter se tractare sermones de bonis et honestis rebus, dum tamen id modeste fiat citra clamorem et contentionem. 

Pae. Hactenus satisfecisti mihi, cetera narrabis a prandio nisi aliquod negotium intervenerit. Eamus nunc in aulam ad prandium, ne magistro in mora simus. 

Pu. Audivi modo signum dari. 

Pae. Opportune datum.

\subsection{Colloquium 61}
\emph{Prosequitur puer narrationem ante prandium inceptam.}

Paedagogus, Puer

Pae. Ubi finivisti narrationem ante prandium?

Pu. Cum vellem finem imponere de ientaculo, tu me interpellasti, praeceptor. 

Pae. Perge igitur narrare ordine reliqua. 

Pu. Dum ientandi finem facimus, datur publicum signum posterius. Sumit quisque libros, imus in aulam communem, recitantur de more catalogi singularum classium. Qui adsunt, ad nomen respondent. Ego quoque respondeo, absentes notantur in catalogis ab ipsis nomenclatoribus. Finita catalogorum recitatione, ludimagister pulpitum ascendit ut precetur, iubet nos attentos esse, tumque publice precatur. 

Ubi precatus est, ``Recipite,'' inquit, ``vos in suum quisque auditorium.'' Conveniunt omnes, ego item venio cum meis condiscipulis. Sedeo in loco meo. Praeceptor ingreditur, inquirit de absentibus, deinde sedet in cathedra sua et iubet pronuntiari auctoris scriptum. Pronuntiavimus terni clara voce, ut solemus quotidie. Tum iubet ut reddamus interpretationem: aliquot ex rudioribus legunt singuli, nos alii reddimus terni, idque memoriter, praeter eum qui verba ipsa auctoris praeit nobis ordine. Tandem praeceptor exigit Gallicam verborum significationem. Doctiores, quibus nominatim id praecepit, respondent, ego quoque iussus ab eo respondeo, laudat illos qui bene responderint, de quorum numero ego (quod sine iactantia dictum sit) unus eram. Postea iubet singulas orationis partes ordine tractari ad rationem grammaticam. Postremo palam praescribit quid sit a prandio reddendum. 

Audita hora octava, precationem imperat. Qua finita monet ut officium sedulo faciamus; tandem nos missos facit. Eo spectante, eximus ordine et sine strepitu, laetique discedimus. Satisne tibi feci, praeceptor?

Pae. Immo cumulatissime. 

Pu. Placetne tibi ut sub cenae tempus idem faciam de reliquis huius diei actionibus?

Pae. Nihil opus erit. Nam de iis quae horis pomeridianis aguntur, alias te audivi satis. 

Pu. Nunquid vis praeterea?

Pae. Estne tempus eundi in aulam communem ad Psalmorum cantionem?

Pu. Tempus est. 

Pae. Ito igitur.

\subsection{Colloquium 62}
\emph{Paedagogus puerum in divinis rebus per otium instituit. Divinorum benficiorum commemoratio. Christi mors salus est omnium credentium. Laudatur pueri studium ac diligentia Locus ex Ioannis Evangelio.}

Paedagogus, Puer

Pae. Ades, Carole?

Pu. Adsum, praeceptor. 

Pae. Quid agunt duo condiscipuli tui?

Pu. Adhuc docentur a subdoctore.

Pae. Tu vero iamne pronuntiasti contextum praelectionis in crastinum mane?

Pu. Pronuntiavi. 

Pae. Satisne recte?

Pu. Satis, gratia Deo. 

Pae. Quis te audivit?

Pu. Ludimagister. 

Pae. Bene habet, sed est quod monere te velim. 

Pu. Ego istud audire percupio. 

Pae. Saepenumero cogitandum tibi est quantum debeas bonorum omnium largitori Deo, qui et ingenium et memoriam tam felicem tibi dederit. 

Pu. Quid illi non debeam qui mihi dedit omnia?

Pae. Dic aliquot eius beneficia praecipua, quemadmodum docui te aliquando. 

Pu. Dedit mihi caelestis ille Pater corpus, animam, vitam, mentem bonam, parentes bonos, locupletes, nobiles, bene erga me affectos et qui non modo suppeditant mihi copiose omnia ad hanc vitam necessaria sed etiam (quod est longe maximum) me bonis litteris bonisque moribus tam diligenter instituendum curant, ut nihil sit praeterea requirendum. 

Pae. Vere omnia ista dixisti, sed unum praetermisisti quod est singulare Dei beneficium. Scin' tu quid sit?

Pu. Sine me paulisper cogitare. 

Pae. Otiose cogita. 

Pu. Nunc ego reminiscor. Sed pro magnitudine rei, nescio quibus verbis id possum exprimere. 

Pae. Dic tamen quo poteris modo. 

Pu. Cogito etiam atque etiam. 

Pae. Dic tandem. 

Pu. Innumerabilia sunt Dei optimi maximi erga me beneficia: in corpore, in animo, in externis rebus, sed nullum maius nec dici nec cogitari potest, quam quod Filium suum unicum gratis mihi dederit, qui me miserrimum peccatorem, et sub Satanae tyrannide captivum ac morti aeternae destinatum, redemit; idque morte sua, omnium crudelissima et maxime ignominiosa. 

Pae. Satis apte dixisti, et totidem fere verbis quot alias te docueram. Sed nunquid Deus tibi uni hoc tantum beneficium praestitit?

Pu. Minime vero. 

Pae. Quibus praeterea?

Pu. Omnibus quotquot evangelio fideliter ac vere crediderint. 

Pae. Age, profer locum ex Ioannis Evangelio in eam sententiam. 

Pu. ``Sic Deus dilexit mundum ut Filium suum unigenitum daret, ut omnis qui credit in eum non pereat sed habeat vitam aeternam. Non enim misit Deus Filium suum in mundum ut condemnet mundum sed ut servetur mundus per eum. Qui credit in eum, non condemnatur; qui vero non credit, iam condemnatus est, quia non credidit in nomen unigeniti Filii Dei. Haec est autem---''\footnote{Secundum Ioannem 3:16-19.}

Pae. Hactenus satis. Sed cuius sunt verba ista?

Pu. Ipsius Christi de se ipso loquentis. 

Pae. Quem alloquitur?

Pu. Nicodemum, qui ad eum noctu\footnote{S. nocte S. Corr. et D. noctu} venerat. 

Pae. Faxit ipse Christus, unicus servator noster ut magis ac magis in ipsius cognitione proficias. 

Pu. Faciet, spero. 

Pae. Perge igitur, ut coepisti, alacriter, quod bene vertat Deus in gloriam sui nominis. 

Pu. Ita precor.

Pae. Eamus cenatum.

\subsection{Colloquium 63}
\emph{Faceta observatoris confabulatio cum puero vere candido et simplicis ingenii.}

Observator, Puer

O. Tu nunquam studes, quando eris doctus?

P. Id fiet progressu temporis, Deo iuvante. 

O. Recte dicis, sed interim laborandum tibi est. 

P. Atqui ego non sum arator. 

O. Etiam rides? Quasi laborare idem sit quod arare. 

P. Scio non idem esse. 

O. Cur ergo sic respondisti? Nonne istud ridere est?

P. Et ridere non est malum, cum sit naturale omnibus hominibus. 

O. Pergin’ tu nugas dicere?

P. Quod dixi verum est, et verum dicere non est nugari. Cur me immerito reprehendis?

O. Iure te arguo. 

P. Quo iure?

O. Quia non ignoras ridere pro irridere usitatum esse, et tamen sic accepisti quasi de risu sim locutus. 

P. Si defendo causam meam, quid mali facio?

O. Pergis igitur esse pertinax? Profecto serio notaberis. 

P. Ne, quaeso, mihi irascaris, mi Martine. 

O. Non irascor, sed officium meum facio. 

P. Sed audi, quaeso. 

O. Quid audiam? Tuas nugas?

P. Audi, inquam, nihil mentiar. 

O. Dic breviter, est mihi alibi negotium. 

P. In primis, cum tu me monuisti, non eram otiosus. 

O. Quid ergo? Si nihil faciebas, nonne otiosus eras?

P. Non eram, pace tua dixerim. 

O. Qui potest illud fieri?

P. Dicam tibi, etsi tute melius hoc intelligis quam ego. Nihil faciebam, ut apparebat. Sed tamen cogitabam aliquid boni. 

O. Declara istud mihi. 

P. Cum tu facis versus, saepe meditaris diu, quasi sis otiosus, quanvis nunquam sis minus otiosus. 

O. Pro ista aetate nimis acutus es! Etiamsi tibi, ut ais, otiosus non eras, tamen qui te viderent possent aliter iudicare. 

P. At solus eram. 

O. Verum, sed poterant aliqui intervenire. Denique, non fateris culpam? 

P. Siqua fuit culpa, in eo fuit quod primo aspectu videbar tibi esse in otio, cum revera non essem. 

O. In eo nihil requiro. Sed de irrisione quid respondes?

P. Certe nihil dixi irridendi animo. 

O. Quo igitur?

P. Iocabar, crede mihi. 

O. Quorsum?

P. Ut paucis verbis fabulando, aliquid ex te addiscerem. 

O. Non is sum a quo multa doceri queas. 

P. Immo tecum multum boni saepe didici. 

O. Quid tandem vis concludere?

P. Ut mihi ignoscas. Quando (ut vides) malo animo nihil peccavi, quod equidem sciam. 

O. Age, ignosco, quia videris mihi candidus et apertus, neque adhuc vidi te mendacem esse. 

P. Ago tibi gratias, Martine suavissime.

\subsection{Colloquium 64}
N., O.

N. Ergone abis in patriam?

O. Cogor abire, nempe accersitus a patre. 

N. Nunquamne es reversurus?

O. Non, spero. 

N. Quando profecturus es?

O. Crastino die, ut opinor. 

N. Siccine igitur me relinquis?

O. Ita necesse est. 

N. O me miserum! Ubi et quando amicum talem reperiam, talem studiorum meorum socium?

O. Ne doleas, esto animo bono, meliorem tibi dabit Deus!

N. Ille quidem potest, scio; at ego vix sperare possum. 

O. Noli, obsecro, te affligere tantopere, nec enim hac separatione corporum interitura est amicitia nostra quin potius accrescet magis, et absentes corpore, praesentes animis erimus. Quid epistolae, quas ultro citroque dabimus? Quantam vim speras habituras esse? Quid quod mutuo illo desiderio amor ipse noster fiet iucundior?

N. Verisimilia sunt quae dicis omnia, sed interim non lenitur dolor meus. 

O. Ah, reprime lacrimas!

N. Non queo prae dolore. 

O. Siccine agis? An putas me minore dolore tangi? Sed quid agas? Divinae voluntati parendum est. Nunc ipse te collige, obsecro, ac potius ad hilariter cenandum te para. Pluribus a cena colloquemur. 

N. O quam triste divortium!

\subsection{Colloquium 65}
\emph{Doctrina pueris maxime necessaria, de fugienda videlicet pravorum consuetudine et maxime impostorum. Exemplum amici fidi in dando consilio.}

Messor, Vallensis

M. Non meministi praeceptorem tam saepe monere nos de fugiendis pravis sodalibus?

V. Ego vero probe memini. 

M. Tamen alicubi satis negligenter uteris eius monitis. 

V. In quo videor tibi ea negligere?

M. Dicam tibi, modo attente audias. 

V. Dic, obsecro, audiam attentissime.

M. Nunquam vis cavere ab illo impostore?

V. Cur caveam?

M. Ne illius contagione depraveris. Nosti enim esse pessimum. 

V. Atqui non sponte sequor, ad me accurrit undique. 

M. Nimirum quia novit te habere quod des, et dare libenter ac saepe. 

V. Quid igitur mihi faciendum suades?

M. Dic semel et serio, et quasi animo irato, ``Quid vis, amice? Cur me ubique sequeris? Omnes clamitant te esse pessimum, adeo sodales tui esse nolunt. Proinde omitte me posthac, quaeso, ne tua causa virgis palam cedar.''

V. Quid si velit aliquid contra respondere?

M. Abrumpe illi sermonem, teque recipe celeriter.

V. Ago tibi gratias, quod me tam fideliter monueris. 

\subsection{Colloquium 66}
G., H.

H. Visne permanere in ista ignorantia?

H. Avertat Deus!

G. Quid igitur facies?

H. Da mihi super hac re consilium, quaeso. 

G. In primis Deum saepissime et ex animo precare, deinde semper attentus esto: hoc est, diligenter audito quicquid docetur, sive praeceptor loquatur sive aliquid reddant condiscipuli tui; postremo caritatem diligenter cole. 

H. Quibus modis?

G. Neminem neque laedito neque offendito, nemini invideto, neminem odio habeto. Sed contra, omnes dilige tanquam fratres, ac bene omnibus, quoad poteris, facito. 

H. Quid illa mihi conferent ad studiorum profectum?

G. Plurimum. 

H. Quomodo?

G. Sic enim Deus tibi illuminabit ingenium, memoriam ac ceteras animi dotes augebit. Denique studia tua ita promovebit ut maiores in ea re progressus in dies facias. 

H. Consilium mihi sane das optimum. Utinam in Dei ipsius gloriam uti perpetuo valeam, tibique aliquando referre gratiam!

G. Non opto ut mihi aliud gratiae eo more referas, nisi ut Deum saepenumero laudes, studiaque honesta semper prosequaris, atque ita ad divinarum litterarum cognitionem tandem pervenias.

\subsection{Colloquium 67}
\emph{Interrogatiunculae de morbo. Deus est medicorum princeps.}

Castellanus, Mossardus

C. Quid egisti per hos quindecim dies?

M. Ministravi matri, quae graviter aegrotabat. 

C. Ain' tu?

M. Sic est profecto. 

C. Quo laborabat morbo?

M. Febre tertiana. 

C. An convaluit?

M. Paulatim convalescit, gratia Deo.
 
C. Quis eam sanavit?

M. Medicorum summus. 

C. Quis ille?

M. Ipse Deus. 

C. De hoc nihil dubito, sed cuius opera?

M. Domini Sarrasini. 

C. Is habetur maximi nominis in medicinae professione. 

M. Id quotidie probant egregiae curationes eius. 

C. Quibus remediis utebatur in curanda matre tua?

M. Medicamentis. 

C. Satis istud intelligo, etiam te tacente. Sed dic plane, quae fuerunt ista medicamenta?

M. Sine me aliquantisper recordari. 

C. Sino, dic tandem quae reminisceris. 

M. Duo tantum nomina mihi occurrunt, clysteres et potiones. 

C. Quid ista conferunt?

M. Eho inepte, ita rogas, quasi ego medicinae operam dederim! Itaque si cupis amplius scire, quaere tuteipse ab iis potius qui ita profitentur: hoc est, a medicis et pharmacopolis. 

C. Ne mihi succenseas, oro. 

M. Cur tu es adeo curiosus?

C. Ut ediscam semper aliquid. 

M. At vide interim ne voceris percuntator. 

C. Audi tamen item pauca. 

M. Loquere. 

C. Quandiu aegrotavit mater?

M. Fere duas hebdomadas. 

C. Intera ubi erat pater?

M. Profectus erat Lugdunum ad mercatum. 

C. Sed tu, qua hora rediisti in gymnasium?

M. Hodie mane. 

C. Dedistine excusationem praeceptori?

M. Dedi. 

C. Quid tibi respondit?

M. ``Factum bene,'' inquit. Tu vero ubi eras?

C. Hesterno die rus iveram cum patruo. 

M. Age, videamus quid simus reddituri hora secunda. Nam ego quodammodo nunc novus sum discipulus.

\subsection{Colloquium 68}
\emph{Meio, meis, meiere, minxi, mictum.}

Grangerius, Toquetus

G. Visne ire mictum?

T. Satis otiose minxi. 

G. Eamus una, quaeso, ut parum fabulemur. 

T. Tace, inepte, nisi vis accusari. Non est fabulandi tempus. Nonne debuisti meiere cum ientaretur?

G. Debui, sed oblitus sum. 

T. Ito igitur solus, cum bona venia praeceptoris, ne sis posthac tam obliviosus. 

G. Parebo tibi, et meminero.

\subsection{Colloquium 69}
\emph{Non solum nobis nati sumus: sed alius alium sollicite curare debet in tempore. Exemplum tempestivae admonitionis.}

Malagnodus, Gassinus

M. Quid cogitas, Gassine? Cave tibi obsecro. 

G. Quid mihi cavebo?

M. Ne in morbum incidas. 

G. Qua ex causa?

M. Ex nimia lusus intemperantia. 

G. Unde apparet periculum?

M. Quia totus aestuas, totus sudore mades. 

G. Recte et in tempore admones. Profecto non sentiebam. 

M. Desiste, si me audis. 

G. Audio vero libenter, ac tibi morem gero. Quis enim respuat tam fidele consilium?

M. Deterge faciem sudariolo et indue te celeriter, ne subitum frigus contrahas. 

G. Habeo tibi gratiam, nam fere morbis sum obnoxius. 

M. Quid est causae?

G. Infirmitas meae valetudinis. Vides enim quam imbecillo sim corpore. 

M. Tanto magis debes tibi cavere. 

G. Istud probe novi, et parens uterque me monet saepissime. Sed quid agis? Natura proni sumus in nostram perniciem. 

M. O mi Gassine, non est voluptati serviendum, sed temperantia valetudini consulendum.

G. Est in promptu carmen Catonis in eam sententiam. 

M. Teneo, sed de his alias. Iam satis indutus es, non est quod hic morere diutius. 

G. Vale, Malagnode, monitor amicissime. 

M. Vin’ tu, ut domum te deducam?

G. Nihil opus est deductione. Ego belle me habeo, Dei beneficio. 

M. Mi Gassine, cura ut valeas.

\subsection{Colloquium 70}
\emph{Omnis victoria, nisi ad Dei gloriam referatur, mera est vanitas. Victoria modesti adolescentis recte utuntur. Stulti contra faciunt. Curiositas. A pueris tria potissimum curanda. Sine Deo nihil fit boni. Spiritus divinus in nobis operatur. Omne bonum a Deo esse confitendum est. Deus auctor fidei, idem et perseverantiae. Spes bona. Adolescentum ingenuorum consensio in bonum.}

Robineriatus, Bobussardus

R. Valde miror cur hodie mane non adfueris. 

B. Quid miraris tantopere? Nihil hic est novi; multi absunt quotidie, immo fere horis singulis. 

R. Atqui victoria tibi erat in manibus. 

B. Quid ego curo? Eiusmodi victoria (ut bene dicebat quidam) nihil aliud est quam brevis gloria. 

R. Sed interim modesti adolescentes hinc ad studia magis incenduntur, nec tamen inani gloria tumescunt sed ad honorem Dei referunt quicquid inde laudis accesserit. 

B. Istud certe raro contingit. Plures enim sunt qui victoriis abutantur ad privatam gloriam quam qui divini honoris rationem habeant. 

R. Verisimile dicis. 

B. Immo verissimum. 

R. Sed velim mihi dicas cur abfueris. 

B. Scripsi ad patrem litteras. 

R. Cuius nomine?

B. Matris. 

R. Dictavitne tibi ipsa?

B. Quid scripsissem, nisi dictasset?

R. Quid continebant litterae?

B. Longum esset tibi narrare. 

R. Saltem dic earum argumentum. 

B. Varium erat et multiplex; et quid tua, quaeso, scire refert?

R. Nihil. 

B. Cur ergo tam avide quaeris?

R. Animi causa, ut fere curiosi sumus novi aliquid audiendi. 

B. Nihil aliud quam garris. Omitte me. 

R. Ausculta paucis. 

B. Age, ausculto, loquere quid velis.

R. Scire cupio ubi sit pater tuus.

B. Quasi vero nescias. 

R. Unde scirem?

B. Cum tibi sit notissimus et cum simus vicini, non putassem te ignorare. 

R. Dic tandem quaeso. 

B. Est Lugduni. 

R. Quando est profectus?

B. Abhinc dies quattuor. 

R. Quid illic agit?

B. Negotiatur. 

R. Quando rediturus est?

B. Finito mercatu. 

R. Ad quod tempus finietur?

B. Roga mercatores, non est meum curare talia. 

R. Quid igitur curas?

B. Ut Deum timeam, parentibus oboediam, bonas artes cum pietate discam. 

R. Nae tu magnifice loqueris. Sed dic mihi serio, potesne res tantas efficere?

B. Egone istud mihi assumo? Quin potius fateor, ne incipere quidem penes me esse. 

R. Quid ergo de te fiet?

B. Deus ipse Spiritu suo in me operabitur. 

R. Optime sentis. Nihil ex te praeterea requirebam. 

B. Est Deo gratia, cui acceptum refero quicquid inest in me boni. 

R. Istud recte, et laudo equidem, ne tibi videar nihil aliud quam garrire. 

B. Cum illud dicerem, iocabar sane. 

R. Ego sic accepi. Sed tu (ut coepisti) perge discere et sapere. 

B. Qui mihi dedit fidem, idem (ut spero) perseverantiam dabit. 

R. Bene speras, et ego idem spero tecum. Itaque pergamus vivere inter nos coniunctissime, ut adhuc fecimus. 

B. Per me quidem non stabit, nisi ope divina prorsus ero destitutus. 

R. Avertat ipse Deus! Sed audin' tu horologium?

B. Ut in ipso tempore sermonem finivimus!

\subsection*{Duo quae sequebantur colloquia de consilio auctoris sunt translata ad finem huius secundi libri. (S. I.20, I.21 = D. II.71, II.72.)}

\subsection{Colloquium 71}
\emph{Duo pueri de usu chartarum pueriliter disputant.}

Rossetus, Monarchus

R. Unde venis?

M. Foris. 

R. Quid prodieras?

M. Ut emerem chartam. 

R. Emistine?

M. Emi. 

R. Quantum emisti?

M. Scapum. 

R. Quanti?

M. Quinque quadrantibus. 

R. Cuius formae?

M. Minoris. 

R. Ostende. 

M. Vide num bona sit. 

R. Bona est profecto. In quem usum emisti?

M. Inepte quaeris. Quis est chartarum usus nisi ad scribendum?

R. Immo alius. 

M. Quis, cedo?

R. Ad merces involvendas. 

M. Intelligebam de charta scholastica, non de emporetica; non enim sum mercator. 

R. Utimur etiam charta ad siccandam recentem scripturam. 

M. Satis scio, sed charta illa est bibula. 

R. Et tamen charta est. 

M. Esto. 

R. Est ergo multiplex chartae usus etiam in schola. 

M. Cogor fateri. 

R. Etiam dicam tibi alium usum, et quidem in schola frequentissimum. 

M. Quem?

R. Non ausim dicere sine praefatione honoris. 

M. Quid opus est inter nos honorem praefari? Non enim verba fetent. 

R. Dicam igitur, quando ita vis. 

M. Dic libere. 

R. Ad tergendas nates in latrina. 

M. Illuc non referuntur chartae purae, sed iam scriptae, eaeque inutiles. 

R. Quid tum? Chartae sunt tamen. 

M. At ego de charta pura et nova loquebar. 

R. Sed interim victus es. 

M. Sit ita sane, non me paenitet disputatiunculae huius nostrae. Sed iam a lusu disceditur. 

R. Et nos ergo loco cedamus. 

\subsection{Colloquium 72}
\emph{Hugo et Blasius experiuntur atramenti sui mixturam. Mediocritas. Experientia. Ex duobus corruptis aliquid boni componitur.}

Hugo, Blasius

H. Habesne bonum atramentum?

B. Cur istud rogas?

H. Ut mihi des aliquantulum. 

B. Eho, non habes?

H. Immo, sed eo non possum scribere. 

B. Quid obstat?

H. Quia nimis spissum est. 

B. Nescis diluere?

H. Non est mihi aqua. 

B. Dilue vino. 

H. Multo minus. 

B. Quid si aceto dilueres?

H. Inde charta perflueret. 

B. Qui scis?

H. Audivi e quodam magistro, qui me docebat scribere. 

B. Ego vero aliud audivi magis mirum. 

H. Narra mihi, sodes. 

B. Quid mihi dabis?

H. Bonam aciculam. 

B. Audi igitur quod ego didici ex quodam paedagogo meo. Atramentum, quod aceto liquefactum est, aegre eluitur.

H. Fieri potest. Sed interim da mihi parum in usum praesentem. 

B. Tene atramentarium tuum bene apertum, ego infundam tibi. 

H. Ecce, infunde. Vah! Quam liquidum est. 

B. Fortasse quia non est gummi satis. 

H. Sed quam decolor!

B. Utere, si vis, quale quale est. Non enim habeo melius. 

H. Quid igitur faciam?

B. Hem, inepte, non potes penna tua bene miscere?

H. Miscui satis. Quid possem praeterea?

B. Infunde rursus in cornu meum. 

H. Admove propius, estne satis?

B. Comprime penna linteolum. 

H. Ita compressi ut fere sit aridum. Quid erit tandem?

B. Atramentum bonum, aut certe mediocre. 

H. Bona est mediocritatis regula, ut ex praeceptore didicimus. Sed nunquid ex duabus malis rebus confici potest aliquid boni?

B. Ubi miscuero, et tibi rursus infudero, videbis experimentum. 

H. Ardeo istud videndi desiderio. 

B. Porrige nunc atramentarium tuum. 

H. Ecce, infunde. Ohe, iam satis est! Quae isthaec est profusio? Plus mihi dedisti quam tibi retinueris. 

B. Commisce iterum, etiam atque etiam. 

H. Nunquam posset coquus sua iura et condimenta melius confundere. 

B. Iam tandem facito periculum. 

H. Dicta mihi aliquam sententiam, ut interim discam aliquid. 

B. ``Experientia,'' ut vulgo dicitur, ``est rerum magistra.'' Habes?

H. Dicto citius. 

B. Videlicet iampridem tenebras. 

H. Quis illud ignoraret, quod est adeo vulgare?

B. Nunc videamus. 

H. Res apparebit melius ubi scriptura bene desiccata fuerit. 

B. Quid vis expectare? Iam siccata est plus satis. 

H. Oh, vide quam nigra sit. 

B. Dixine vere?

H. Aliquando periculum feceras scilicet. 

B. Constabit igitur experientiam esse rerum magistram. 

H. Quinetiam hinc experimur, ex rerum commixtione bonum fieri temperamentum. 

B. Iam incipis altius philosophari, itaque discedo. 

H. O longum sermonem de nihilo!

B. Nihil me paenitet, alioqui inerti otio torpebamus.

\section{Liber Tertius, cui insunt magistri colloquia cum discipulis} % 42 colloquies
\subsection*{Admonitio}
\emph{Haec a pueris ita legenda erunt, ut ex duobus legentibus unus discipulum, alter praeceptorem agat.}

\subsection{Colloquium 1}
\emph{Duo aut tres domesticis discipulis praeceptorem mane salutatum veniunt, ut sciat an omnes surrexerint. Admonitio praeceptoris. Exemplum diligentiae observatu dignum.}

Unus ex discipulis, Praeceptor

D. Salve, Praeceptor.

P. Salvus per Iesum Christum. An surrexerunt omnes?

D. Omnes, praeter parvulos.

P. Nunquis aegrotat?

D. Nemo, gratia Deo.

P. Quid agitur?\footnote{S. igitur D. agitur}

D. Alii se induunt, alii iam student naviter.

P. Adestne vobis hypodidascalus?

D. Iamdudum.

P. Ite igitur precatum, vosque diligenter commendate Domino Deo per Iesum Christum deprecatorem nostrum. Deinde pergite in studiis vestris, usque ad horam ientaculi.

D. Ita solemus, Praeceptor.

P. Credo equidem. Sed, quia fere somniculosi estis ac negligentes, idcirco ego vos admoneo saepius.

D. Gratiam habemus, praeceptor humanissime. Nunquid vis praeterea?

P. Dic famulo, ut mihi togam adferat.

\subsection{Colloquium 2}
\emph{Interrogatus quidam de contione sacra, nihil scit respondere. Obiurgatur a praeceptore. Parum abest quin vapulet, sed precibus exorat veniam.}

Praeceptor, Discipulus

P. Adfuistine hodie contioni sacrae?

D. Adfui.

P. Qui sunt testes?

D. Multi ex condiscipulis qui me viderunt testari possunt.

P. Sed producendi erunt aliquot.

D. Producam cum iubebis.

P. Quis habuit contionem?

D. Dominus N.\footnote{S. et D. Dom. N.}

P. Quota hora incepit?

D. Septima.

P. Unde sumpsit thema?

D. Ex Epistola Pauli Ad Romanos.

P. Quoto capite?

D. Octavo.

P. Adhuc bene respondisti, nunc videamus quid sequatur. Ecquid memoriae mandasti?

D. Nihil quod referre possim.\footnote{S. possem S. Corr. et D. possim}

P. Nihilne? Cogita paulisper, et vide ne turberis; quin esto animo bono.

D. Certe praeceptor, nihil possum reminisci.

P. Ne verbum quidem?

D. Nihil prorsus.

P. Hem, verbero! Quid igitur profecisti?

D. Nescio, nisi quod fortasse interim a malis abstinui.

P. Istud quidem est aliquid, si modo fieri potuit ut malo omnino abstinueris.

D. Abstinui quoad potui.

P. Fac ita esse; non tamen satisfecisti Deo, cum scriptum sit, ``Declina a malo, et fac bonum.''\footnote{Psalmus 36:27} Sed dic mihi (quaeso) qua gratia illuc iveras potissimum?

D. Ut aliquid addiscerem.

P. Cur id non fecisti?

D. Non potui.

P. Non potuisti nebulo? Immo noluisti, aut certe non curasti.

D. Cogor fateri.

P. Quae res te cogit?

D. Conscientia mea, quae me accusat apud Deum.

P. Recte dicis, utinam ex animo!

D. Equidem ex animo dico.

P. Ita fieri potest. Sed age, quid fuit causae quamobrem nihil memoriae mandaveris?

D. Negligentia mea, non enim diligenter audiebam.

P. Quid igitur faciebas?

D. Identidem dormiebam.

P. Ita soles. Sed quid agebas reliquo tempore?

D. Cogitabam mille ineptias, ut solent pueri.

P. An tu adeo puer es ut non debeas attentus esse ad verbum Dei audiendum?

D. Si attentus essem, possem aliquid proficere.

P. Quid igitur meruisti?

D. Verbera.

P. Meruisti profecto, idque largissime.

D. Ingenue confiteor.

P. Verbo tenus, opinor.

D. Immo certe ex animo.

P. Fortasse, sed interim para te ad plagas accipiendas.

D. Ah praeceptor, ignosce obsecro. Peccavi, fateor, sed nulla ex malitia.

P. Atqui tam supina ista negligentia proxime ad malitiam accedit.

D. Non equidem inficior, sed tuam imploro clementiam per Iesum Christum.

P. Quid igitur facies, si tibi ignovero?

D. Faciam posthac officium meum, ut spero.

P. Addendum erat, ``Deo iuvante," sed id parum curas.

D. Immo praeceptor, adiuvante Deo, praestabo posthac officium.

P. Age, condono culpam tuis lacrimis, tibique ea lege ignosco ut promissi memineris.

D. Gratias ago, praeceptor humanissime.

P. Eris apud me in maxima gratia si promissa servaberis.

D. Faxit Deus optimus maximus ut possim.

P. Faxit precor.

\subsection{Colloquium 3}
\emph{Admonitio ludimagister de classe doctore vacua, statim providet.}

Martinus famulus, Praeceptor

M. Praeceptor, nemo est qui doceat in sexta classe.

P. Quid hoc rei est? Ubi est magister Philippus?

M. Morbo detinetur in lecto.

P. Qui scis?

M. Nuntiavit quidam ex discipulis eius domesticus.

P. Dic hypodidascalo meo.

M. Non est in musaeolo suo.

P. Qui scis?

M. Nam ego ter aut quater pulsavi ostiolum.

P. Dic primae classis doctori ut mittat e suis aliquem.

M. Quid si nolit mittere?

P. Abi inepte, an putas eum esse tam imprudentem ut recuset? Abi, propera.

\subsection{Colloquium 4}
\emph{Praeceptor docens evocatur auditorio.}

Bordonus, Praeceptor, Discipuli

B. Praeceptor!

P. Hem, quid est?

B. Sunt quidam qui te conventum volunt.

P. Ubi sunt?

B. Te expectant in vico.

P. Nunc adibo.

B. Atqui urgent.

P. Praecurre tu atque eos intromitte in aream, ego te sequar. Vos interim expectate cum silentio. Mox ego adero, ut vos ad cenam dimittam.

D. O quam bonum verbum!

\subsection{Colloquium 5}
\emph{Stulte interrogatur magister a quodam discipulo de eo quod palam dictum est. Exemplum est puerilis imprudentiae ut discant pueri ad verba doctoris attentiores esse.}

Canellus, Praeceptor

C. Praeceptor, quid reddemus cras mane?

P. Hodie mane palam dixi ante scholae missionem.

C. At ego non aderam, praeceptor.

P. Roga igitur condiscipulos. Nam si vellent singuli me interrogare de rebus a me palam dictis, quaeso, quando finis esset? Itaque fac sis posthac prudentior.

C. Curabo pro viribus.

P. Sed tu ubi eras?

C. Prodieram.

P. Quid prodieras?

C. Ut curarem negotium aliquod de quo pater ad me scripserat.

P. A quo petivisti veniam?

C. Ab hypodidascalo.

P. Cur non a me potius?

C. Quia eras occupatus.

P. Quid agebam?

C. Alloquebaris in area quosdam viros honoratos, qui te conventum venerant.

P. Abi, nunc recordor.

\subsection{Colloquium 6}
\emph{Magister observatores publicos admonet de ipsorum officio.}

Praeceptor, Famulus, Observatores

P. Heus, Martine!

F. Hem, praesto sum, here.

P. Accerse mihi huc quinque publicos observatores quos hesterno die in hunc mensem elegi. Nostin'?

F. Optime, nam egomet aderam.

P. Sunt (opinor) in suo quisque auditorio. Festina.

F. Quamprimum rediero.

O. Adsumus omnes, praeceptor. Quid tibi placet imperare?

P. Satis erat iubere, nec enim sum imperator nec magistratus. Ego vos huc accersendos iussi, ut vos officii vestri commonefacerem. Vos igitur attentis auribus, atque animis audite: non ignoratis quanto cum timore Domini hesterno die palam in aula nostra communi vos elegerim. Auspicati sumus a sacris precibus, secuta est admonitio nostra atque exhortatio ad omnem coetum scholasticum de timore Domini deque moribus, qui deceant studiosos in schola quotidie versantes; deinde, non sine optimorum adolescentium testimonio vos elegi quinque quos ad hoc munus idoneos existimavi.

Postremo ventum est ad secundam cum gratiarum actione precationem: ne igitur putetis ludum fuisse aut iocum actionem illam, in qua nomen Domini tam studiose fuerit invocatum. Ac licet apud imperitos aut arrogantes hoc munus et vile et abiectum videatur, vos tamen credite, cum honorificum, tum sanctum esse vestrum illud ministerium. Quod si aliter existimabitis, fieri non potest ut munere vestro recte fungamini. Itaque ego vos hortor, quantum possum, et per Iesum Christum obtestor, ut cum Dei timore atque reverentia diligentiam praestetis in iis omnibus quae intelligeris ad officium vestrum pertinere. 

A vobis igitur absit omnis favor, odium, gratia, studium vindicandi et similia, quae transversos agunt homines, et sincerum corrumpunt iudicium. Ne timeatis improborum minas, quibus illi animos adolescentium ab officio solent absterrere. Quam enim habent in vos potestatem? Potius eum timete, qui vester est Dominus, qui vitae ac necis potestatem habet. Illius (inquam) tanti Principis timor vobis ob oculos semper observetur. Incurretis (scio) in aliquot improborum ac dissolutorum odium, sed pluris sit vobis unius Patris vestri caelestis amor et caritas, quam omnes omnium hominum inimicitiae. Estote semper memores verbi illius, quo Servator noster et summus praeceptor suos discipulos ad constantiam hortabatur: ``Si vos,'' inquit, ``odit mundus, scitote quod me prius odio habuerit.''\footnote{Secundum Ioannem 15:18} Vos igitur propter ipsum Christum, omnes flocci facite nebulonum minas, offensiones, inimicitias, dummodo gloriae Dei possitis inservire fideliter. Haec sunt de quibus nunc pro temporis brevitate vos admonendos esse existimavi, praeter illa quae vos in aula hesterno die audivistis.

Primus Ob. Maximas tibi gratias agimus, praeceptor humanissime, et Christum precamur, ut sua dona tibi semper adaugeat. A te vero vehementer petimus ut (si tibi molestum non est) praescriptam des nobis hortationem tuam, quo illam inter nos quandoque relegentes memoriae tenacius infigamus.

P. Id ego faciam primo quoque tempore, quandoquidem rem sane honestissimam postularis.

Primus Ob. Optamus etiam a te (si placet) commentariolum scriptum habere de praecipuis officii nostri capitibus ut simus certiores quid potissimum sit nobis hac in re observandum.

P. In ipso tempore de hoc admones et sic ego iampridem in animo habebam, sed me quotidie aliud ex alio impedivit. Dabo igitur eiusmodi commentariolum quod videlicet contineat quicquid ad observatorum publicorum officia pertinebit. Id autem describetis ex ipso archetypo meo, quod ideo servare volo ut ceteris quoque tradere possim futuris observatoribus. Nunc redite in suum quisque auditorium.

O. Recta imus, praeceptor.

\subsection{Colloquium 7}
\emph{Ituri ad nuptias, pridie veniam petunt ut domi vestimenta mutent. Admonentur paucis a praeceptore.}

Clericus, Praeceptor

C. Licetne, praeceptor, ut ego et patruelis eamus domum?

P. Quid eo?

C. Ad nuptias consobrinae.

P. Quando est nuptura?

C. Crastino die.

P. Cur tam cito vultis ire?

C. Ut mutemus vestimenta.

P. Per me licet eatis, hac tamen lege ut cras huc redeatis cubitum.

C. Quid si volet patruus ut expectemus repotia?

P. Non detinebit vos, satis scio, dummodo dicatis ei qua lege dimiserim.

C. Verum fatebimur.

P. Abite, et ab omni cavete intemperantia. Faciteque ut luceat lux vestra coram hominibus ut glorificetur vester ille caelestis Pater.

C. Ita quidem speramus fore, ipso nos in omnibus adiuvante.

\subsection{Colloquium 8}
\emph{Iturus ad forum. Proverbium de pane.}

Portanus, Praeceptor

Po. Licetne exire, praeceptor?

P. Quae tibi est exeundi causa?

Po. Ut quaeram in foro aliquem ex nostratibus.

P. Quid istud opus est?

Po. Mandare illi volo ut meos admoneat de pane mihi aut adserendo aut mittendo.

P. Ubi panis deficit, omnia sunt illic venalia.

Po. Istud vulgatum est apud nos proverbium.

P. Immo ubique pervulgatum, adeo panis mortalium vitae est necessarius. Sed ad rem: tu nunc prodire vis?

Po. Si tibi placet, praeceptor, ne mei negotii occasionem amittam.

P. Abi, et festina ante prandium redire.

Po. Dabo equidem operam.

\subsection{Colloquium 9}
\emph{Duo fratres ituri domum, monentur a praeceptore de testimonio adferendo aut teste adducendo.}

Buetus, Praeceptor

B. Licetne mihi exire una cum fratre?

P. Quid causae est?

B. Ut mater emat nobis calceos, deinde ut tonsorem adeamus.

P. Quid eo?

B. Resectum capillos.

P. Quid nunc opus est?

B. Ut cras (si Dominus permiserit) invisamus patruum.

P. Ite, et mature redite ad studium. Sed, heus pueri, adferte mihi a matre testimonium in crastinum diem aut testem adducite.

B. Deo iuvante, id curabo diligenter. Nunquid vis, praeceptor?

P. Ut meis verbis matrem officiose salutetis.

\subsection{Colloquium 10}
\emph{Quidam prodire volentes obiurgantur ac reiiciuntur in postremum diem.}

Albertus, Praeceptor

A. Praeceptor, licetne nobis ire ad tonsorem?

P. Quid eo?

A. Ut capillum tondeamus.

P. Libenter quotidie exiretis sexies, quin expectate in crastinum diem ut eatis una cum ceteris?

A. Atqui propter forum turba erit in tonstrina.

P. Quid tum? Satis habebitis otii ad expectandum. Recipite vos ad studium.

A. Ut libet, praeceptor.

\subsection{Colloquium 11}
\emph{Iturus ad patrem in cauponam.}

Bargius, Praeceptor

B. Praeceptor, accersor a patre.

P. Ubi is est?

B. In diversorio.

P. Quando venit?

B. Advenit modo.

P. Quis tibi tam cito nuntiavit?

B. Misit ad me famulum.

P. Ubi est?

B. Prae foribus me expectat.

P. Cur illum non intromisisti?

B. Noluit intrare.

P. Quid ita?

B. Quia (ut ait) festinatione urgetur.

P. Voca illum, ut paucis conveniam, deinde abi. Sed cura ut quam primum huc adsis.

B. Eo vocatum.

\subsection{Colloquium 12}
\emph{Quaerit praeceptor de famuli absentia.}

Praeceptor, Ruscinaeus

P. Ubi est Martinus?

R. Ivit ad forum.

P. Quid eo?

R. Emtum (ut dixit) cingulum.

P. Iniussu meo exire non debuit, sed hoc nihil ad te. Quis dabit vobis merendam?

R. Dixit se hora secunda reversurum ut det nobis.

P. Quid si fallat?

R. Id non est moris eius.

P. Nisi ad horam adfuerit, admone uxorem de vestra merenda. Habet enim clavem alteram cellae penuariae.

\subsection{Colloquium 13}
\emph{Arguitur qui exierat inussu praeceptoris sed veniam exorat.}

Praeceptor, Scarronus

P. Demiror unde nunc venias.

S. Domo redeo, praeceptor.

P. Cur iveras domum?

S. Petitum merendam.

P. Quamobrem non attuleras?

S. Mater erat occupata.

P. Quid tum? Debuisti exire iniussu meo?

S. Non debui, fateor.

P. Quid igitur meruisti?

S. Plagas accipere. Sed ignosce mihi (quaeso) praeceptor.

P. Cur non petivisti eundi potestatem?

S. Quia non audebam te interpellare.

P. Quid agebam?

S. Tenebas libellum quendam et legebas aliquid.

P. Fieri potest, sed vos tamen saepe me interpellatis ob rem leviorem. Nunc igitur para te ad vapulandum.

S. Parce mihi, obsecro, praeceptor.

P. Sine ut prius cogitem aliquantisper. Age, parco: tum quia ingenue confiteris, tum quod satis studiosus mihi videris.

S. Gratias ago maximas, praeceptor humanissime.

\subsection{Colloquium 14}
\emph{A praeceptore chartam petens, ad eius famulum mittitur.}

Gulielmus, Praeceptor

G. Praeceptor, non restat mihi charta ad scribendum. Visne dare codicem?

P. Quem in usum?

G. Partim ad colloquia, partim ad exemplaria.

P. Retulisti in codicem tuum?

G. Retuli.

P. Ostende.

G. Ecce tibi, praeceptor.

P. Quid istud? Retulisti octodecem, vis ergo de maiore?

G. Si tibi placet.

P. Pete a famulo. Ac, ne dubitet, ostende illi tuum codicem ut idem in suum referat.

G. Audio.

P. Audi item: cave abutaris charta ne tibi pater graviter succenseat.

G. Faxit Deus ut bene utar.

\subsection{Colloquium 15}
\emph{Empturus cultellos admonetur ut sibi imponi caveat. Deus cultores suos iuvat.}

Grivetus, Praeceptor

G. Praeceptor, licetne prodire?

P. Quamobrem?

G. Ut emam cultellos mensarios.

P. Ubi sunt quos habebas?

G. Reliqui domi.

P. Quid ita?

G. Quia iam obtusi erant et inutiles.

P. Habesne pecuniam ad emendos alios?

G. Mater dedit mihi.

P. Quis erit adiutor ad emendum?

G. Gerardus.

P. Ite sane, et cavete ne vobis imponatur.

G. Cavebimus, Deo iuvante.

P. Omnes quidem iuvat sed eos potissimum qui ad eius honorem omnia referunt.

\subsection{Colloquium 16}
\emph{De suburbana demabulatione. Vernetus et Spatula veniam impetrant eundi ambulatum ut ceteris ludentibus, liberius et iucundius colloquantur. Dei laudandi argumentum cultoribus eius nunquam deest.}

Vernetus, Praeceptor, Spatula

V. Praeceptor, licetne pauca?

P. Loquere.

V. Nos duo proponebamus, si tibi ita videretur, ire, dum ceteri ludunt, foras ambulatum.

P. Quo vultis exire?

V. In proxima suburbana.

P. Quid autem agetis ambulantes?

S. Tractabimus colloquium aliquod.

P. Sed de bonis et honestis rebus?

S. Haec temporis serenitas et tam pulchra terrae facies praebebunt nobis honestum aliquod argumentum.

P. Nunquam deest Dei laudanda materia, duntaxat veris eius cultoribus.

V. Nunquam profecto. Sed ut ad propositum revertamur: permittis nobis, praeceptor, extra urbem prodire?

P. Nisi mihi perspecta esset vestra perpetua fidelitas et verus amor litterarum, nunquam permitterem, praesertim cum pravi adolescentes me saepe in hoc genere fefellerint. Vos igitur prodite, et mature ad cenam revertimini.

\subsection{Colloquium 17}
\emph{Iturus ad sartorem.}

Isaias, Praeceptor

I. Praeceptor, licetne prodire?

P. Quo prodire cupis?

I. Ad sartorem.

P. Quid eo?

I. Petitum femoralia.

P. Iamne facta sunt?

I. Sunt, opinor.

P. Recte (opinor) dicis, quia res incerta est.

I. Atqui promiserat mihi in hunc diem.

P. Quid si fallat?

I. Nihil mirum fuerit.

P. Nunc quoque vere locutus es, nam raro ad promissum tempus fidem praestant artifices.

I. Viso tamen, praeceptor, si mihi permittis.

P. Nihil impedio.

I. Nunquid vis, praeceptor?

P. Immo, ut properes, ne desis praelectioni.

I. Bene mones, abeo.

\subsection{Colloquium 18}
\emph{Iturus domum ob levem causam, non impetrat, adeoque obiurgatur. Exemplum observatu dignum ne pueri cursitare temere permittantur.}

Caius, Praeceptor

C. Licetne prodire?

P. Quo?

C. Domum.

P. Hem, tam saepe ire domum?

C. Mater iusserat ut ego et frater se adiremus hodie.

P. Cuius rei gratia?

C. Ut ancilla vestimenta nobis excuteret.

P. Quid istud? Suntne vobis pediculi?

C. Et multi quidem.

P. Cur uxorem meam non admonuistis?

C. Non ausi sumus.

P. Quasi vero illa sit usque adeo difficilis. Ancillam habet ea potissimum gratia, ut vestram omnium curet munditiem. Nec vos ignoratis illud, sed gaudetis matris invisendae occasionem vobis dari. Vos igitur manete, cras ego curabo ut vobis excutiantur vestes.

C. Sed mater nos obiurgabit.

P. Egomet eam placabo, quiescite.

\subsection{Colloquium 19}
\emph{Tornatur et frater petitum panem ituri, diligenter a praeceptore, ne fallant eum, interrogantur. Exemplum notandum ad idem quod superius.}

Tornator, Praeceptor, Pueri

T. Praeceptor, licetne cras ire domum?

Prae. Quid eo

T. Petitum panem.

Prae. Non tibi restat?

T. Restat quidem, sed parum admodum.

Prae. Quid frater? Estne tecum iturus?

T. Iussit pater.

Prae. Quando convenisti illum?

T. Die Iovis, cum venisset in hanc urbem.

Prae. Ubi illum vidisti?

T. Apud forum.

Prae. Non mentiris?

T. Non mentior.

Prae. Unde probabis?

T. Sunt ex condiscipulis, qui aderant.

Prae. Qui tandem?

T. Adsunt Blasius et Audax.

Prae. Estne verum, pueri?

Pu. Omnino verum.

Prae. Qui scitis?

Pu. Vidimus eius patrem, audivimus ipsa verba.

Prae. Si ita est, permitto ut eas domum cum fratre.

Pu. Vale praeceptor.

Prae. Vos servet Dominus Deus.

T. Idem tibi precamur ex animo.

Prae. Sed heus! Quando huc aderitis?

T. Crastino die vesperi, Deo iuvante.

Prae. Cura, ut promissi memineris.

T. Curabo.

Prae. Scilicet, ut soles.

T. Immo melius, spero. Nunquid vis?

Prae. Ut verbis meis salutem dicas parentibus.

T. Faciam libenter; iterum vale, praeceptor.

Prae. Vos quoque valete, et lento gradu ambulate, propter aestum solis.

T. Ita facere solemus.

\subsection{Colloquium 20}
\emph{Puer interrogatus de absentia, causam adfert honestam, eamque scriptam profert.}

Praeceptor, Villarianus.

P. Quid sibi vult, quod abfueris hac tota hebdomade?

V. Oportuit me manere domi.

P. Quamobrem?

V. Ut matri adessem, quae aegrotabat.

P. Quod illi officium praestabas?

V. Saepius ei legebam.

P. Quid legebas?

V. Aliquid ex Sacris Litteris.

P. Sanctum istud et laudabile ministerium. Utinam sic omnes studerent verbo Dei! Sed quid? Nihil agebas praeterea?

V. Quoties opus erat, illi ministrabam cum ancilla.

P. Haeccine vera sunt omnia?

V. Habeo testimonium.

P. Profer illud.

V. Ecce!

P. Quis scripsit?

V. Famulus noster, matris nomine.

P. Agnosco eius manum, quia saepe ab illo mihi attulisti.

V. Licetne igitur redire in sedem meam?

P. Quidni liceat, cum mihi satisfeceris?

V. Gratias ago, praeceptor.

\subsection{Colloquium 21}
\emph{Lucetus ad forum iturus paucis interrogator ab hypodidascalo.}

Lucetus, Hypodidascalus.

L. Praeceptor, licetne mihi prodire?

H. Quae tibi est prodeundi causa?

L. Est mihi eundum ad forum.

H. Quid eo?

L. Ut emam corium.

H. In quem usum?

L. Ad calceorum soleas.

H. Quis te adiuvabit in emptione?

L. Quidam oppidanus, cui hoc mandavit pater meus.

H. Debueras adire me cum ceteris, qui ad forum prodierunt.

L. Occupatus eram.

H. Qua in re?

L. In scribendis ad patrem litteris.

H. Quando eas dabis?

L. Hodie, si quem in foro offendero.

H. Abi, et memineris ad horam sextam adesse.

L. Meminero.

\subsection{Colloquium 22}
\emph{Arator a negotio reversus praeceptori, ut moris erat, se sistit.}

Arator, Praeceptor.

A. Praeceptor, tuo permissu hora prima prodieram, nunc redeo.

P. Curasti negotium tuum?

A. Curavi, gratia Deo.

P. Factum bene. Quota est hora?

A. Instat secunda.

P. Voca mihi famulum, deinde ito ad merendam cum ceteris.

\subsection{Colloquium 23}
\emph{Hugo a praeceptore petens mutuo pecuniam, mittitur ad hypodidascalum.}

Hugo, Praeceptor.

H. Praeceptor, visne mihi mutuo dare aliquantulum pecuniae?

P. Quid opus est tibi pecunia.

H. Ut Sylvio satisfaciam.

P. Quantum debes illi?

H. Assem cum semisse.

P. Quo nomine?

H. Quia scripsit mihi aliquot colloquia.

P. Ostende.

H. Vide, si placet.

P. Adi hypodidascalum, dic ut det quantum petis.

H. Gratias ago, praeceptor.

P. Non est quod agas, sed refer in codicem tuum.

H. Quin iam retuli.

P. Factum bene, ostende ipsi hypodidascalo.

\subsection{Colloquium 24}
\emph{Blasius, scribendi causa iturus ad tutorem, veniam impetrat, sed prius diligenter interrogatus.}
Blasius, Praeceptor.

B. Licetne mihi, praeceptor, adire tutorem?

P. Quae te causa movet?

B. Iusserat ille ut se hodie convenirem, si liceret per otium.

P. Quando iusserat?

B. Nudiustertius.

P. Ubi illum vidisti?

B. In area quae est e regione templi.

P. At vide ne mentiaris.

B. A me absit mendacium. Si vis dabo testes ex condiscipulis qui mecum aderant.

P. Qui sunt illi?

B. Daniel et Corberius. Visne ut eos accersam?

P. Mane, ego illos conveniam. Sed dic, quid eget tutor opera tua?

B. Ad aliquid describendum.

P. Qua igitur hora vis illum adire?

B. Nunc, si tibi placet.

P. Quando huc redibis?

B. Cum primum me dimiserit.

P. Nunc abi, atque illi ex me dic salutem plurimam.

B. Faciam libenter.

\subsection{Colloquium 25}
\emph{Magister ad prandium invitatus excusationem mittit.}

Scriba, Magister.

S. Praeceptor, pater te invitat prandium, si tibi placet.

M. Estne solus?

S. Solus (opinor) praeter domesticos.

M. Excusa me illi, iam enim aliunde invitatus eram. Age tamen illi meis verbis gratias.

S. Nunquid vis aliud?

M. Nihil nisi ut mature ad scholam redeas.

S. Mature, iuvante Deo.

\subsection{Colloquium 26}
\emph{Gaspar ad sartorem iturus, et eadem opera ad tonsorem.}

Gaspar, Praeceptor.

G. Licetne prodire, praeceptor?

P. Quo?

G. Primum ad sartorem, deinde ad tonsorem.

P. Cur ad sartorem?

G. Ut curem tibialia reficienda.

P. Suntne lacerata?

G. Adeo lacerata ut vix induere possim.

P. Cur ad tonsorem?

G. Ut illi ostendam ulcus quod mihi his diebus subortum est in femore.

P. Detege, ut videam.

G. Vide, quando ita placet.

P. Est furunculus.

G. Ita coniciebam.

P. Cum aperueris tonsori, roga illum ut emplastrum ulceri aptum adhibeat.

G. Faciam quod suades.

P. Sed nunquis est qui tecum adire velit?

G. Immo Ioannes Fluvianus.

P. Quod habet negotium?

G. Tonsorem quoque vult adire.

P. Ite igitur una et redite similiter.

G. Nunquid vis praeterea?

P. Ut maturetis reditum, ne merenda vestra mulctemini.

\subsection{Colloquium 27}
\emph{Latomus multorum nomine veniam petit proeudendi ad mercatum.}

Latomus, Praeceptor.

L. Praeceptor, licetne nobis prodire?

P. Estisne multi qui prodire vultis?

L. Ferme omnes.

P. Quid hoc sibi vult?

L. Est hodie mercatus, inde fit ut fere sibi quisque velit aliquid emere.

P. Nunc ego sum occupatior quam ut singulorum prodeundi causam possim cognoscere; adite igitur subdoctorem, qui cognoscat, et si vacat vos deducat ipse.

L. Gratias agimus, praeceptor humanissime.

\subsection{Colloquium 28}
\emph{Theophilus iubetur scribere ad patrem cuiusdam pueri de remittendo ad scholam filio.}

Praeceptor, Theophilus.

P. Hodie igitur Petrum convenisti?

T. Hodie.

P. Ubi?

T. In templo.

P. Quota hora?

T. Octava matutina.

P. Nunquid rogasti quando sit rediturus scholam?

T. Rogavi.

P. Quid ille?

T. ``Nescio,'' inquit.

P. Debuisti illum ad reditum exhortari.

T. Id ego feci, et multis quidem verbis.

P. Bene fecisti, sed quid ille respondit?

T. Se adhuc a patre detineri ad fructus colligendos.

P. Quid si ad ipsum patrem scribas de statu nostro scholastico? Fortasse enim movebitur ut filium citius remittat.

T. Si tibi ita videtur, faciam, idque diligenter.

P. Fac igitur primo quoque tempore. Sed audi, scribe planissime, deinde litteras tuas mihi ostende priusquam des perferendas.

T. Sedulo faciam, praeceptor.

\subsection{Colloquium 29}
\emph{Petanellus de variis negotiis obeundis veniam petit.}

Petanellus, Praeceptor.

Pet. Praeceptor, licetne mihi exire?

Prae. Quo tibi eundum est?

Pet. Ad tonsorem.

Prae. Non est tibi aliud negotium?

Pet. A tonsore, volo ire emptum ligulas, illinc me ad sutorem conferre.

Prae. Cur ad sutorem?

Pet. Ut uni ex calceis meis annectat corrigam.

Prae. Ista omnia quando confeceris?

Pet. Intra horae spatium, ut spero.

Prae. Erunt multi fortasse in tonstrina expectantes.

Pet. Fieri potest, sed si videro diutius tibi morandum illic esse, expectabo in diem Sabbati.

Prae. Estne alius qui prodire velit?

Pet. Pontanus ait se velle chartam emere.

Prae. Scisne illi opus esse?

Pet. Scio.

Prae. Ite igitur una. Curate diligenter suum uterque negotium, ne sitis cessatores.

Pet. Deo iuvante, cavebimus.

\subsection{Colloquium 30}
\emph{Carbonarius sartorem aditurus.}

Carbonarius, Praeceptor.

C. Licetne exire?

P. Quo?

C. Ad sartorem.

P. Quid eo?

C. Ut mihi tunicam faciendam metiatur.

P. Quae tibi est materia?

C. Niger pannus.

P. Ubi est?

C. In arca mea.

P. Sartor autem quis tibi est?

C. Petrus Sylvius.

P. Estne peritus artifex?

C. Sic audivi, et est notus patri meo, qui iussit ut illum adirem.

P. Ubi habitat?

C. In vico xenodochii.

P. Non nimis longe est, cave discurras.

C. Cavebo.

P. Facile a me veniam impetrant qui nunquam fallunt.

C. Avertat Deus ut unquam fallam.

\subsection{Colloquium 31}
\emph{Magister a discipulo interrogatus, respondet humaniter. Bonus praeceptor bene optat discipulis. Solus Deus dat bene agenda voluntatem. Omnis boni gratia Deo debetur. Regula citra exceptionem vix ulla est. In dubiis grammaticis consulendae sunt regulae. Studiosus doctos observare ad imitandum debet. Praeclara opera longiori tempore indigent. Experientia. Patrum indoctorum stultitia. Tempus expectandum. Scientia magna laboribus comparator. Exemplum boni praeceptoris et discipuli studiosi.}

Luternius,\footnote{S. Laternius S. Corr. et D. Luternius} Praeceptor.

L. Praeceptor, licetne pauca?

P. Loquere quid velis.

L. Cum interdum dicis alicui nostrum, ``Ubi est follis?'' vel, ``Cedo follem,'' non apparet utrum \emph{follis} sit masculini an feminini generis.

P. Non apparet, fateor. Quid tum?

L. Unde igitur scire possumus?

P. Cur me de hoc nunquam rogasti?

L. Tam multa tam saepe interrogamus ut vereamur ne tibi molesti simus.

P. Quasi vero istud unquam prae me feram. Contra, eo magis amo vos, quo me rogatis saepius. Quid enim magis cupio quam ut aliquando vos videam et optimos et doctissimos?

L. Habemus gratiam maximam, praeceptor humanissime.

P. Eam gratiam ego et vos Deo nostro debemus, qui solus sua bonitate utrisque bonam dedit voluntatem.

L. Faxit ille ut hoc beneficio recte semper utamur in ipsius gloriam. Sed dic, quaeso, cuius generis est \emph{follis}?

P. Masculini.

L. At ego potius feminini dixissem.

P. Quamobrem?

L. Quia tale est \emph{pellis}, quod est in \emph{Rudimentis} pro exemplo positum.

P. Non abs re id coniciebas. Nam \emph{-is} finita, quale est \emph{pellis}, magna ex parte feminina sunt.

L. Non igitur omnia?

P. Vix ulla est tam generalis regula quae exceptione careat. \emph{Follis} igitur sub exceptione cadit, quia masculinum est. Sic aliquot alia, ut \emph{ignis}, \emph{piscis}, \emph{axis}, et cetera.

L. Sed unde illa dignoscam?

P. Facile cognosces cum perveneris ad grammaticae regulas. Sed interim Latine loquentes attente observa, teque ad eorum imitationem diligenter accommoda.

L. At istud longum est, praeceptor.

P. Non fiunt nisi longo tempore praeclara aedificia.

L. Experientia nos istud docet. At pater meus vellet me annuo spatio doctum videre.

P. Ego vero istud unius diei spatio videre vellem. Sed quid agas? Omnibus in rebus expectandum tempus est. Pater tuus, quia non didicit litteras, nescit quid doctrina valeat, neque quantis laboribus illa comparetur.

L. Verum dicis. Sed quid illi respondere possum, cum apud me conqueritur de temporis longi spatio in discendis litteris?

P. Docebo te inter cenandum. Nunc ito lusum cum ceteris ut me in museum meum recipiam.

L. Ignosce mihi (quaeso) praeceptor, quod te interpellaverim.

P. Nihil me interpellasti, non enim occupatus eram. Praeterea, si te audire mihi molestum fuisset, nonne poteram te in aliud tempus reicere?

L. Tuo iure id poteras.

P. Abi igitur.

\subsection{Colloquium 32}
\emph{Quidam puer, patris nomine, praeceptorem in hortos invitat. Praeceptor aliquid obiter docet ad mores pertinens.}

Castrinovanus, Praeceptor.

C. Salvus sis, praeceptor.

P. Auspicato advenis, quid nuntias?

C. Orat te pater meus ut animi causa eamus una in hortos suos suburbanos.

P. Ad eam rem nos invitat serenitas et nunc sumus feriati. Sed quid illic aspectu iucundum videbimus?

C. Varias et pulchras arbores cum suis fructibus item herbarum et florum miram varietatem.

P. Nihil est illis rebus hoc tempore iucundius.

C. Ea est Dei erga nos beneficentia.

P. Quam quidem assiduis laudibus prosequi debemus.

C. Sed vereor ne patri in mora simus.

P. Tantisper expecta dum togam muto ut sim ad ambulandum expeditior. --- Iam paratus sum, nunc eamus. Sed estne domi pater?

C. Prae foribus nos expectat.

P. Bene res habet. Vide ut eum decenter salutes.

C. De hoc, te docente, saepe admoniti fuimus.

\subsection{Colloquium 33}
David, Praeceptor

D. Pater meus tibi salutem plurimam dicit.

P. Ain' tu? Quando rure rediit?

D. Heri tantum.

P. Ut valet?

D. Optime.

P. Mater vero ubi est?

D. Adhuc est in Gallia.

P. Ubi in Gallia?

D. Aureliae.

P. De illa quid auditis?

D. Esse bona valetudine praeditam, Dei beneficio.

P. Dominus Deus conservet eam.

D. Ita precor.

P. Dic vicissim patri salutem plurimam verbis meis.

D. Faciam sedulo.

\subsection{Colloquium 34}
\emph{Discipulus magistrum de dubio quodam consulit. Magister benigne respondet. Analogiae non ubique fidendum. Latinitas quibus constet maxime. Aetatis progressus quid requirat. Omnis profectus noster Deo tribuendus. Citra favorem divinum frustra laboramus. Exemplum anomaliae in genere et accentu sub eadem terminatione.}

Bachodus, Praeceptor.

B. Praeceptor, licetne pauca?

P. Dic libere.

B. Cur non dicimus \emph{hic arbor}, sicut \emph{hic labor}? Item cur genitivum \emph{arboris} non proferimus penultima longa, ut fere in ceteris nominibus terminationis eiusdem?

P. Quia loquendi usus aliter probavit. Nec enim ubique locum habet analogia, sed, ubi ea deficit, sequendus est eorum usus qui recte et pure locuti sunt. Nam ipsa Latinitas usu et auctoritate magis quam ratione constat.

B. Da igitur auctoritatem de nomine \emph{arbor}.

P. ``\'Arboris ex\'esae truncus,'' apud Vergilium.\footnote{Not in Vergil, but Seneca \emph{Epistulae} 90.7 ``exesae arboris trunco.''} Nonne hic manifeste vides et genus et accentum?

B. Video praeceptor. Sed suntne alia eodem accentu?

P. In primis Graeca omnia, ut \emph{Castor}, \emph{Castoris}; sic \emph{Hector}, \emph{Nestor}, et similia. Item haec duo neutrius generis: \emph{aequor}, \emph{aequoris}; \emph{marmor}, \emph{marmoris}. Sunt et adiectiva quaedam, ut \emph{memor}, \emph{memoris}, et ex eo compositum \emph{immemor}. Talia quoque sunt ex \emph{decus} et \emph{corpus} composita, ut \emph{indecor}, \emph{indecoris}; \emph{tricorpus}, \emph{tricorporis}. Sed haec apud grammaticos annotata facile per te invenies. Nam ista aetas tua maiorem in dies requirit diligentiam. Huc accedit, quod haec ipsa, quae tuo labore et diligentia inveneris, firmiore tenebis memoria.

B. Ago tibi gratias, humanissime praeceptor, quod me tanta humanitate non solum doces, sed etiam admones.

P. Bene facis. Sed interim volo memineris, soli Deo acceptum referre quicquid boni ex labore meo in te proficiscitur. Frustra enim docendo laboramus, nisi laboribus nostris divinus favor accesserit. Nosti illud Apostoli: ``Neque qui plantat, est aliquid, neque qui rigat, sed qui dat incrementum, Deus.''\footnote{Ad Corinthos I 3:7}

B. Essemus profecto plumbo stupidiores, si ista ignoraremus quae nobis tam saepe inculcas, tamque diligenter.

P. Tanto diligentius vos oportet tum meminisse, tum recordari. Sed iam tempus est ut ad quotidianum pensum te referas. Ego vero interea me abdo in musaeolum.

\subsection{Colloquium 35}
\emph{Pueri, qui in quotidiano sermone ad Latine loquendum diligenter exercentur, multa discunt sine regulis grammaticis. Bene precandum omnibus, maxime nostris.}

Praeceptor, Olivarius

P. Dic Latine, \emph{un livre}.

O. Liber.

P. Liber cuius generis?

O. Masculini.

P. Qui scis?

O. Ex bene loquendi usu et consuetudine.

P. Ostende usum.

O. Nam quotidie loquentes, sic dicimus, ``Hic liber cuius est?'' Dicimus item saepenumero ``Liber meus,'' ``liber tuus,'' ``bonus liber,'' et similia.

P. Bene respondisti. Sed quis loquendi usum te docuit?

O. Tu ipse, praeceptor.

P. Ergone tenes omnem usum linguae Latinae?

O. Si tenerem, iam non essem discipulus.

P. Quid igitur?

O. Magister, fortasse.

P. Abi, responso tuo contentus sum.

O. Gaudeo sane.

P. Age gratias Deo, qui tibi dedit ingenium et mentem bonam.

O. Utinam semper agnoscam eius in me beneficia!

P. Utinam ille favore suo tua studia prosequatur. Quid hoc sibi vult, Olivari?

O. Quod mihi bene precaris.

P. Ergo tu quoque memento bene ex animo precari omnibus, praecipue vero condiscipulis tuis.

O. Meminero, praeceptor.

P. Addendum fuerat, ``Deo iuvante.''

O. Oblitus sum, fateor.

\subsection{Colloquium 36}
\emph{Exemplum ad pueros in Latinis Gallice vertendis paulatim.}

Praeceptor, Daniel

P. Attende, Daniel, ut discas Latina bene Gallice vertere.

D. Attendo praeceptor.

P. At diligenter.

D. Immo diligentissime, et ex animo.

P. Bene facis.

D. Propone igitur mihi Latina, ut nobis interdum soles.

P. Quid opus est?

D. \emph{Que faut-il?}

P. Gallinae.

D. \emph{A une poule.}

P. Ut.

D. \emph{Ce que.}

P. Illa.

D. \emph{Elle.}

P. Sit.

D. \emph{Soit.}

P. Bona.

D. \emph{Bonne.}

P. Recte vertisti. Nunc ad singulas partes huius orationis responde nominatim.

D.  Respondebo quoad potero, dummodo mihi praeieris.

P. \emph{Quid.}

D. Est nomen.

P. \emph{Opus.}

D. Nomen.

P. \emph{Est.}

D. Verbum.

P. \emph{Gallina.}

D. Nomen.

P. \emph{Ut.}

D. Coniunctio, in hoc loco.

P. \emph{Illa.}

D. Pronomen.

P. \emph{Sit.}

D. Verbum.

P. \emph{Bona.}

D. Nomen.

P. Age, dicamus iterum ut singula paulo plenius intelligas.

D. Quid nunc respondebo?

P. Indica breviter singularium partium declinatum, ut vos docere soleo.

D. Praeito igitur ut coepisti.

P. \emph{Quid.}

D. \emph{Quid}, \emph{cuius}: nomen substantivum anomalum.

P. \emph{Opus.}

D. \emph{Hoc opus}, \emph{operis,} ut \emph{hoc onus}, \emph{oneris.}

P. Falleris, Daniel.

D. Quid ita?

P. Quia \emph{opus} hic est adiectivum.

D. Eho, adiectivum! Quomodo declinatur?

P. Est indeclinabile.

D. Me miserum! Nunquam istud audieram.

P. Addendum fuit, ``quod sciam,'' vel, ``quod meminerim.''

D. Quamobrem?

P. Quia fortasse audieras, sed memineras male.

D. Fieri potest, sed perge (quaeso) me docere. Quid Gallice significat istud nomen?

P. Non solet Gallice verti, nisi iunctum cum verbo \emph{sum}, \emph{es.}

D. Da exemplum.

P. Quotidie in ore habes exempla.

D. Nunc mihi non occurrunt.

P. Nonne soles dicere et audire ex condiscipulis, ``Opus est mihi charta, atramento, pecunia,'' et similia?

D. Saepe dico, fateor, et saepe audio, sed parum adverto.

P. Nunc igitur adverte, et manda memoriae, ``Opus est mihi pecunia ad libros emendos:'' \emph{``I’ay besoin d’argent pour achepter des livres,''} vel sic, \emph{``Il me faut de argent,''} vel, \emph{``I’ay a faire.''}

D. Da item aliud exemplum, quaeso.

P. ``Opus est tibi virgis, ut tua expellatur pigritia:'' \emph{``Tu as besoin de verges, afin que ta paresse soit chassee.''}

D. Fateor equidem, praeceptor, sed Deus (ut spero) mei miserebitur.

P. Omnium miseretur Deus qui pie illum invocant. Sed de nomine \emph{opus} iam satis multa quod ad vos attinet. Ad cetera redeo: \emph{est}.

D. \emph{Sum}, \emph{es}, \emph{esse,} verbum anomalum.

P. {Gallina.}

D. \emph{Gallina}, \emph{gallinae}, ut \emph{mensa}, \emph{mensae.}

P. \emph{Ut.}

D. Non declinatur, quia est coniunctio. Gallice: \emph{que, afin que, a ce que}.

P. \emph{Illa.}

D. \emph{Ille} generis masculini, \emph{illa} feminini, \emph{illud} neutrius.

P. Declina in feminino.

D. \emph{Illa}, \emph{illius}, \emph{illi}, et cetera.

P. \emph{Sit.}

D. Iam dictum est.

P. \emph{Bona.}

D. \emph{Bonus}, generis masculini, \emph{bona} feminini, \emph{bonum} neutrius; nomen adiectivum.

P. Confer ad exemplum.

D. \emph{Iustus}, \emph{iusta}, \emph{iustum}; \emph{bonus}, \emph{bona}, \emph{bonum}.

P. Nunc mutuo vos interrogare, ut plenius omnia tractetis.

\subsection{Colloquium 37}
Banderius, Praeceptor

P. Quid vis?

B. Licetne mihi ire domum?

P. Cut ante horam?

B. Pater iussit ut nunc adirem.

P. Quid eget opera tua?

B. Vult me in villam mittere.

P. Quid eo?

B. Petitum uvas et eadem opera nuntiatum aliquid villico nostro.

P. Quid si me fallis?

B. Adferam testimonium, ut soleo.

P. Quando redibis?

B. Hora prima, ut spero.

P. Qui tam cito?

B. Villa nostra non longe hinc est.

P. Ito sane.

\subsection{Colloquium 38}
\emph{Excusat se quidam quod inconsulto praeceptore prodierit, cuius rei testes producit.}

Discipulus, Praeceptor

D. Praeceptor, placetne audire excusationem meam?

P. Quando abfuisti?

D. Hesterno die.

P. Quota hora?

D. Prima.

P. Quae fuit causa?

D. Accersitus fui.

P. A quo?

D. A patre.

P. Quis tibi nuntiavit?

D. Famulus noster.

P. Cur me non adisti?

D. Quia dicebat ille se urgeri festinatione.

P. Suntne tibi testes?

D. Adsunt, praeceptor.

P. Abi, sede in loco tuo. Ego interim eos interrogabo.

\subsection{Colloquium 39}
Praeceptor, Michael.

P. Cur non venisti citius?

M. Expectabam fratrem.

P. Ubi est?

M. Restitit in foro.

P. Cur eum non adduxisti?

M. Volebat emere atramentum.

P. Immo pira, mala aut aliquid e ceteris fructibus.

M. Nescio, tamen illud dicebat.

P. Cum sitis fratres, cur non habetis domi commune atramentum in ampulla?

M. Frater nihil vult habere commune mecum.

P. Vult igitur omina sibi propria?

M. Illud est.

P. Admone me cum venerit, ego illum docebo quid sit fraternitas.

M. Faciam, praeceptor.

P. Abi in locum tuum.

\subsection{Colloquium 40}
D., P.

D. Licetne abesse hora secunda?

P. Quid habes negotii?

D. Pater eget opera mea.

P. Qua in re?

D. Ut sibi aliquid scribam.

P. Sed interim non reddes quae praescripta sunt vobis.

D. Iam edidici.

P. Factum bene.

D. Placetne tibi audire me?

P. Cras audiam, cum licebit per otium.

D. Permittisne igitur ut absim, praeceptor?

P. Age, permitto. Sed ita ut crastino die scriptum adferas testimonium.

D. Ego semper tibi adfero aut a patre scriptum aut a nostro famulo, patris nomine.

P. Recte facit pater. Sunt enim multi qui me pascunt mendaciis. Nunc abi, et patri dic salutem verbis meis.

D. Faciam, praeceptor.

\subsection{Colloquium 41}
Praeceptor, Caperonus

P. Heus, Caperone!

C. Hem, praeceptor!

P. Quid flet frater tuus?

C. Aegrotat.

P. Qui scis?

C. Satis apparet.

P. Quo signo?

C. Quia vomuit.

P. Quid illi dolet?

C. Caput et stomachus, ut dicit.

P. Cur non recepit se domum?

C. Non ausus est.

P. Tu vero non audebas me admonere? Age, duc illum tu ipse domum usque, et matri narra diligenter ut ille se habeat. Propera, quid cessas? Duc eum lento gradu.

C. Ducam, praeceptor.

\subsection{Colloquium 42}
\emph{Nimio somno deditus puer reprehenditur de veniendi in gymnasium tarditate. Mendacio et causa frivola purgare se conatur. Illi, ut verum fateatur, promittitur impunitas. Poenarum metu, post aliquot tergiversationes, rem omnem confitetur. Diligentiam promittit in posterum. Docetur qua illam ratione praestare queat. Multis verbis ad mutandos mores humanissime commonetur. Ex paenitentiae signo quod prae se fert, bonam spem promittit.}

Praeceptor, Tiliacus.

P. Heus, Tiliace! Sequere me in cubiculum, est quod ego te seorsum monere velim.

T. Adsum, praeceptor.

P. Nunquamne in scholam mature venies?

T. Non possum venire citius.

P. Semper istud dicis, quid impedit?

T. Nemo est domi nostrae qui me expergefaciat.

P. Nemo?

T. Prorsus nemo.

P. Non habetis ancillam?

T. Habemus quidem, sed non curat me excitare.

P. Immo, tu (ut opinor) non curas surgere. Nonne verum dico? Quid taces? Responde nunc tandem aliquid.

T. Me miserum! Quid agam?

P. Nihil est quod terrearis, fatere verum.

T. Quid si confessus ero?

P. Ego tibi agnoscam, crede mihi.

T. Ah pudet.

P. Ne pudeat verum fateri, quaeso, alioqui vapulabis. Pergin' tacere? Heus observator, vise ad matrem eius et roga.

T. Ne mittas, oro, praeceptor. Dicam tibi rem omnem, nihil reticebo.

P. Age, esto animo bono.

T. Sic est profecto, ut dixisti.

P. Non satis istud est, volo audire sigillatim omnia. Narra mihi plane quemadmodum sese res habeat.

T. Cum venit ancilla me excitatum, primum nihil respondeo, quasi serio dormiam; deinde, si magis urgeat, attollo aegre caput. Sedeo in lecto, thoracem iniicio humeris quasi statim surrecturus.

P. Quam pulchre narras! Ita me Deus amet, nunc te magis amo quam unquam feci. Perge!

T. Cum primum egressa est ancilla cubiculo, tum ego reclino caput in pulvinum ac demitto pedes.

P. Etiamne redormis?

T. Ego vero redormio bene placideque.

P. Quandiu?

T. Donec ancilla secundo veniat.

P. Cum redit, quid tibi dicit?

T. Exclamat, vociferatur, insanit.

P. Quibus utitur verbis?

T. ``Hem nebulo!'' inquit, ``quando eris in schola? Ego dicam praeceptori tuo ut te bene verberet! Tu nunquam vis surgere, nisi bis aut ter excitatus fueris.''

P. Bona fide promittis facturum te posthac officium?

T. Si unquam recidero, causam non dico quin palam cedar virgis, idque acerbissime.

P. Belle quidem promittis. Sed quomodo praestabis promissa?

T. Adiuvante Domino Deo.

P. Qua ratione flectes illum?

T. Fide et assiduis precibus.

P. Alioqui nihil possis obtinere.

T. Credo equidem.

P. Non satis est credere, nisi cures efficere diligenter.

T. Curabo pro viribus, ac dies noctesque id unum meditabor.

P. Optime loqueris, dum tamen memor esse pergas.

T. Quomodo possum oblivisci? Nunquam desinunt istud monere contionatores; tu vero fere quotidie ad id nos hortaris, et bene facis, praeceptor, quia omnes sumus valde negligentes, sed ego primus omnium.

P. Da igitur operam ut tu omnium primus mores istos mutes, ac memento praecipue semper verax esse.

T. Faxit Deus ut nunquam mentiar.

P. O quam felix esses!

T. Satis in praesentia felix ero, si tantum me absolveris.

P. Faciam quod tibi sum pollicitus. Sed ea lege, ut promissi tui memineris et re ipsa praestes, quemadmodum nunc mihi recitasti.

T. Quid igitur restat quo minus abeam liber?

P. Immo aliquid restat. Mane, et audi etiam nunc.

T. Quandiu voles, praeceptor.

P. Inter cetera, excutias oportet istam pigritiam, quae te in lecto detinere solet. Non enim decet studiosum adolescentem somniculosum et inertem esse, sed alacrem et experrectum, cuiusimodi vides aliquos ex condiscipulis. Non tenes memoria divinum Petri Apostoli praeceptum?

T. Quid illud est?

P. ``Sobrii,'' inquit, ``estote, et vigilate.''\footnote{Epistula Petri I 5:8 ``Sobrii estote, vigilate.''}

T. O quoties audieram! Sed (proh dolor!) nunquam usurpavi.

P. Fac ut studiose usurpes in posterum, neque illud solum sed etiam cetera bene vivendi praecepta, quae toties audivisti. Quod quidem si diligenter feceris, tibi in primis bene consules: iucundus eris parentibus et mihi et condiscipulis tuis, denique (id quod est praecipuum) sic carus eris Deo, qui studia in gloriam sui nominis magis in dies promovebit.

T. O quantum fructum sentio ex ista admonitione tua!

P. Vehementer sane gaudeo et tua et condiscipulorum causa.

T. Quid si narras illis meam paenitentiam?

P. Ego vero narrabo primo quoque tempore ut exemplo tuo discant nihil esse acceptius Deo quam culpam agnoscere et ad frugem bonam redire. Vale, fili, et adesto hora tertia in auditorio.

T. Ago tibi gratias ingentes, amantissime praeceptor.

\section{Liber Quartus, paulo graviora continens, praesertim in moribus et Christiana doctrina} % 39 colloquies
\subsection{Colloquium 1}
\emph{De iis quae in oculos incidunt. Amicus amico nulla spe mercedis adductus opitulatur. Oculus membrum inter exteriora nobis carissimum. Zachariae locus de pupilla oculi. Experientia. Iocus oblique notatus. Iocandum interdum, modo cum fructu. Studia nostra ad gloriam Dei referenda.}

Perialdus, Samuel

P. Obsecro te, Samuel, da mihi operam paulisper.

S. Quid istud est?

P. Nescio quid incidit mihi in oculum, quod me habet valde male.

S. In utrum oculum incidit?

P. In dextrum.

S. Vis inspiciam?

P. Inspice, amabo te.

S. Aperi quantum potes, ac tene immobilem.

P. Non queo a nictu continere.

S. Mane, egomet tenebo sinistra manu.

P. Ecquid vides?

S. Video aliquid minutum.

P. Exime (quaeso) si potes.

S. Quin, iam exemi.

P. O factum bene! Quid est?

S. Cerne tu ipse.

P. Est mica pulveris.

S. Et quidem usque adeo pusilla, ut vix cerni possit.

P. Vide quantum doloris adferat oculis res tam exigua.

S. Haud mirum quidem. Nullum enim e membris exterioribus oculo tenerius esse dicitur. Inde etiam fit, ut experiamur, nihil esse nobis carius.

P. Hoc Deus approbat, cum de sua in nobis caritate loquens apud Zachariam secundo capitulo\footnote{S. et D. Zachariam 2. Cap.} sic ait, “Qui vos tangit, tangit pupillam oculi mei.''\footnote{Zecharia 2:8 ``qui enim tetigerit vos, tangit pupillam oculi mei.''}

S. O immensam Dei bonitatem, qui nos tantopere caros habet!

P. Non mihi rubet oculus?

S. Aliquantulum, nempe quia fricuisti.

P. Credin' tu mihi adhuc dolere?

S. Quidni credam, qui toties talem molestiam sum expertus?

P. Experientia est rerum magistra.

S. Ita vulgo dicitur.

P. Quid pretii dabo isti medico pro labore?

S. Quantum pacti sumus.

P. Brevis est conclusio, ergo nihil. Sed tamen habeo tibi gratiam, atque utinam detur referendi locus!

S. Quin potius avertat Deus.

P. Bene correxisti, dixeram imprudenter at sine dolo.

S. Sic accepi. Sed interim iocari licet, praesertim ut in Latina lingua nos exerceamus.

P. Faxit Dominus Deus ut omnia studia nostra ad gloriam ipsius referantur.

S. Faxit precor.

\subsection{Colloquium 2}
\emph{Alexander Carolo commodatum reddit cum gratiarum actione. Beneficia ultro citroque oblata. Exemplum aulicae civilitati propemodum simile.}

Alexander, Carolus

A. Ecce reddo tibi commodatum, et gratias ago maximas.

C. Non est quod agas. Sed tu satisne usus eras?

A. Satis diu usum concessisti, quae tua est humanitas.

C. Quoties opus erit, quaeso ne parcas rebus meis.

A. Non parcam, quando ita iubes.

C. Pergratum mihi feceris.

A. Gratiam habeo maximam. Tu vero nostris utere, si quis usus fuerit.

C. Non est quod moneas, satis mea sponte sum impudens.

A. Immo verecundus nimis.

C. Esto, aliquando tamen senties.

A. Ita velim. Bene vale.

C. Te servet Dominus Deus.

\subsection{Colloquium 3}
Paulus, Quintinus.

P. Cur diligenter audire debemus evangelium?

Q. Ut discamus Deum colere ex volunate eius.

P. Nihil aliud respondes?

Q. Quid responderem? Nihil enim scio praeterea.

P. Nonne etiam ut sobrie et iuste vivamus?

Q. Declara mihi illa duo adverbia, quaeso.

P. Sobrie, id est continents; iuste, id est in iustitia, nempe ut suum cuique tribuamus. Ita fit, ut sobrie ad vitam cuiusque privatam pertineat; iuste autem ad caritatem quam proximo debemus.

Q. Sed audi, mi Paule, nonne cultus Dei omnia illa complectitur?

P. Probe sentis, Quintine. Sed volui experiri an responsionem tuam recte intelligeres.

Q. Bene fecisti. Nam de sincero verbi divini intellectu nihil nimis dici potest. Et de hac re quidem tecum pluribus verbis libenter agerem. Sed nos hora ipsa admonet, ut discedamus.

\subsection{Colloquium 4}
\emph{Ab observatore monetur quidam adolescens ut in admonendo fratre natu minore sit diligentior. Pergendum constanter in admonendis omnibus, praecipue qui nobis coniuncti sunt. Bonae indolis signum quoddam in puero. Exemplum fraternae admonitionis.}

Observator, Baptista

O. Frater tuus semper in contione aut garrit aut ineptit aut aliquem incitat. Ex quo fit ut saepe notandus sit, deinde vapulet.

B. Quid vis faciam?

O. Cur non saepe mones?

B. Nunquam desisto monere.

O. Perge, precor.

B. Nihil est quod me preceris. Nunquam cessabo, donec (volente Deo) aliqua ex parte se correxerit.

O. Sic usurpabis Catonis praeceptum: ``Quando mones aliquem\ldots,''\footnote{Disticha Catonis I.9:\\ ``Cumque moneas aliquem nec se velit ille moneri,\\si tibi sit carus, noli desistere coeptis.''} nosti cetera.

B. Sed oro te, mi Nicolae, ut quoties eum notaveris, id mihi renunties.

 
O. Nunquam finis esset, adeo frequens est nomen eius in meis commentariolis.

B. Saltem fac me semel certiorem cum primum commiserit quo sit accusandus, tum ego dicam patri cuius verba magis timet quam verbera.

O. Istud non est parvum bonae indolis argumentum.

B. Ita spero quidem. Facies igitur quod rogo?

O. Ego vero, atque libens.

\subsection{Colloquium 5}
\emph{Patritius Meloco arcanum commissum extorquere frustra conatur. Amicus, alter idem. Exemplum taciturnitatis celandis amicorum consiliis. Item in curiosos arcanorum scrutatores.}

Patritius, Melocus

P. Quid consilii tractabas modo cum praeceptore?

M. Si scire cupis, illum percontare.

P. Cur me celas?

M. Ut ne palam facias.

P. Non ideo (crede mihi) te rogo ut id proferam. Quid enim proficerem?

M. Quamobrem igitur tam cupide rogas?

P. Nimirum ut mecum tacitus gaudeam, si quid boni audieris.

M. Itane paratus advenis ut a me extorqueas quod mihi uni, idque a praeceptore, creditum est?

P. Quod mihi dixeris, surdo et muto dictum puta.

M. Egone tergum meum in fidem tuam committam?

P. Id profecto potes, et quidem sine periculo.

M. Nunquam dices tam commode ut istud mihi persuadeas.

P. Dabo fidem me taciturum.

M. Etiamsi ter quaterve sanctissime iuraveris, non prodam. Proinde, tu desine percontari.

P. Hem! Ubi est illa nostra amicitia?

M. Nescis illud dictum sapientis, ``Quod tacitum esse vis, nemini dixeris''?\footnote{Proverbia Senecae/De Institutione Morum/De Moribus 16}

P. Audivi aliquoties. Sed quod amico dictum sit, nemini dictum videtur. ``Est enim amicus quasi alter idem.''\footnote{Cicero \emph{Laelius De Amicitia} 80: ``Quod nisi idem in amicitiam transferetur, verus amicus nunquam reperietur; est enim is, qui est tanquam alter idem.''}

M. Eadem tibi dicet qui scire ex te volet, et item alius qui ex illo; atque ita ad aures omnium perveniet. Itaque si me tibi posthac vis amicum esse, me missum facito.

 
P. Non sum imperator ut te missum faciam.

M. Pergin' molestus esse?

P. Abire malim quam tibi molestiam exhibere.

\subsection{Colloquium 6}
M., N.

M. Quando vis abire domum?

N. Nescio. Ubi Deo visum fuerit, id enim pendet ex voluntate eius, non mea.

M. Quid si te accersat pater?

N. Tunc intelligam Deum sic velle, ideoque parebo.

M. Quid si alia fuerit Dei, alia patris tui voluntas?

N. De hoc non est meum disputare. Sed, ut confido, pater non temere me accerset.

M. Ego quoque non aliter sentio, sed volui tantisper tecum fabulari.

N. Gaudeo hunc nostrum sermonem non fuisse fabulosum.

M. Utinam in scholis frequentiores essent sermones eiusmodi!

N. Tum erunt, cum Deus ipse puerorum animos timore sui affecerit.

M. Ergo precemur eum ut id brevi contingat.

\subsection{Colloquium 7}
\emph{Hieremias narrat Marthoraeo de inopinato patris adventu, ob falsum de morbo rumorem. Citro irascens facile placatur. Ira puerorum brevis. Exemplum paternae diligentiae in liberorum cura.}

Marthoraeus, Hieremias

M. Qua pecunia emisti librum istum?

H. Qua censes, nisi mea?

M. Miror unde habueris.

H. Quid miraris? An tibi debeo reddere rationem?

M. Egone exigo?

H. Videris exigere.

M. Non exigo, inquam. Sed sic solemus inter nos familiariter et libere fabulari ut Latine semper condiscamus aliquid.

H. Ea res, fateor, plurimum confert nobis ad Latine loquendi facultatem. Sed nemo est tam lenis quin interdum subirascatur.

M. Est ut dicis. Sed est brevis puerorum ira.

H. Quod autem de pecunia rogabas, eam a patre acceperam.

M. Quando venerat?

H. Abhinc octo dies.

M. Miror quod eum non viderim.

H. Non est quod mireris.

M. Quid ita?

H. Quia vix sesquihoram hic moratus est. Cum enim de equo descendisset, meque paucis esset allocutus. ``Ascendamus,'' inquit, ``in tuum cubiculum ut tecum liberius colloquar.''

M. Sed antequam narres cetera, velim scire quid sibi vellet adventus eius tam inopinatus.

H. Falso quodam rumore (ut fit) audierat me aegrotum esse.

M. Quid ille, cum praeter spem te valentem invenit?

H. Mirifice affectus est gaudio.

M. Quis dubitet?

H. Praeterea, Deo optimo maximo maximas egit gratias.

M. Libenter haec audio. Perge, quaeso.

H. Tunc me de valetudine percontatur, una precamur, non sine gratiarum actione. Tandem quaerit ecquid mihi opus sit.

``Opus,'' inquam, ``pater?''\footnote{D. pater? S. pater.}

``Qua,'' inquit, ``re eges?''

``Libro,'' inquam, ``decem assium.''

Tum ille promit ex marsupio decussem,\footnote{S. decussim S. Corr. et D. decussem} tum mihi in manum dat. Et vale dicto, statim conscendit equum, atque abit.

M. Cur tibi plus dedit quam petiveras?

H. Istud inepte quaeris. Scilicet, ita laetus erat quia me, praeter spem, bene sanum offenderat. Quod si vel coronatum petivissem, tam facile dedisset mihi.

M. O quantum debes illi summo Patri, qui tibi adeo bonum patrem dederit!

H. Ne cogitari quidem potest quantum debeam. Nam etiamsi malum dedisset, deberem tamen non parum. Sed quid cessamus auditum ire praelectionem?

M. Iam instat hora tertia.

H. Parata sunt mihi omnia.

M. Mihi quoque.

H. Eamus ergo in auditorium.

\subsection{Colloquium 8}
\emph{Reprehenditur profectio in militiam. Militum vita corruptissima. Bonus paterfamilias semper aliquid per occasionem docet. Timor Dei. Custodit parvulos Dominus ac dirigit. Patres et praeceptores diligendi et qua re. Exemplum adolescentis stulti et parentum contemptoris.}

Sonerius, Villaticus

S. Ubi nunc est frater tuus natu maximus?

V. Ivit in militiam!

S. Quid ais? In militiam!

V. Sic res est.

S. Sic ergo valedixit litteris?

V. Iampridem litterarum satietas eum ceperat.

S. Quid ita?

V. Nescio, nisi quia volebat liberius vivere.

S. Quomodo permisit pater?

V. Quid? Putas permisisse? Patre absente, matre invita profectus est.

S. O miserum adolescentem!

V. Immo vero miserrimum.

S. Quid faciet?

V. Id quod ceteri qui sequuntur illud vitae genus: nempe spoliabit, rapiet, ludet alea, potabit, scortabitur.

S. Estne istaec militum vita?

V. Omnino.

S. Unde scis istud?

V. Audivi nuper ex patre cum cenaremus.

S. Quorsum narrabat talia?

V. Docebat nos nihil esse certius quam Deum timere, qui custodit parvulos eosque in viam rectam paulatim inducit.

S. Et praeceptor ipse noster de iis rebus saepe nos admonet.

V. Tanto magis debemus esse soliciti ut parentes caros habeamus, et praeceptores quorum opera Deus ad nostram institutionem utitur.

S. Utinam utrisque praestemus quod ipse nobis in sua lege praecepit!

V. Ita faxit ille.

S. Faxit, precor.

\subsection{Colloquium 9}
\emph{Quidam reversus ex Germania praeter opinionem et voluntatem patris, domo expellitur. Post aliquod tempus, amicorum opera redit in gratiam. Exemplum eius quod per iram peccaturm. Item stultae matrum indulgentiae in liberos.}

Lucas, Orosius

L. Audio fratrem tuum iam revenisse ex Germania.

O. Sic est.

L. Solusne rediit?

O. Non omnino.

L. Quis igitur cum illo?

O. Quidam civis huius oppidi, qui fere biennium illic habitaverat.

L. Cur iverat frater?

O. Missus illuc fuerat a patre ut Germanice disceret loqui.

L. Quam igitur ob rem non fuit illic diutius?

O. Iam non poterat ferre matris desiderium.

L. O tenellum adolescentem! Quotum agit annum?

O. Septimum decimum, si recte mater meminit, ex qua id audivi saepe.

L. Age, quo vultu a patre acceptus est eius adventus.

O. Rogas? Pater non sustinebat illum aspicere. Quinetiam nec salutatione dignatus nec alloquio, iussit eum abire e conspectu.

L. Quid praeterea?

O. Nisi mater cum lacrimis intercessisset, iubebat apparitorem accersi qui miserum in carcerem coniceret.

L. Atqui non poterat iniussu magistratus.

O. Nescio, tamen conabatur.

L. Quid postea factum est? Cubuitne domi vestrae?

O. Minime vero.

L. Ubi igitur?

O. Sororis meae virum nostin'?

L. Tanquam digitos.

O. Eo missus est a matre, dum patris ira defervesceret.

L. Quid tandem accidit?

O. Egit mater cum propinquis et amicis nostris ut iratum patrem mitigarent.

L. Sic igitur frater tuus cum patre in gratiam rediit?

O. Id non fuit magni negotii. Iam enim patrem coeperat paenitere quod sic excanduisset quodque tam graviter accepisset filium.

L. Nempe dies eius iram lenierat.

O. Ea tamen lege recepit illum ut promitteret se in Germaniam rediturum statim a vindemia.

L. Vide quam ineptus sit iste in matres nostras affectus.

O. Atque ipsae matres sunt in causa. Cur enim adeo tenere nos adamant?

L. Naturam cogere difficile est.

O. In hanc sententiam tenesne versum ex Horatio?

L. ``Naturam expellas furca licet,\footnote{S. tamen S. Corr. et D. licet} usque recurret.''\footnote{\emph{Epistulae} I.10.24-25: ``Naturam expelles furca, tamen usque recurret\\ et mala perrumpet furtim fastidia victrix.''}

O. Sed quid hoc? Dum fabulamur, a lusu cessatum est.

 
L. Nihil inde nobis accidit mali. Ad disputationes conveniamus iam.

 
\subsection{Colloquium 10}
\emph{Olerum quorundam nomina enumerantur. \emph{Moretum} Vergilianum. Laetitia ob rem culinariam. A puero fructus et olera carnibus praeferuntur. Studiosis satis est modicum alimentum. Parentes aegre ferunt nisi filios in schola bene curatos videant. Matrum indulgentia. Parentum amore in liberos non facile immutatur.}

Conradus, Linus

C. Ubi fuisti hodie post prandium?

L. In horto praeceptoris.

C. Quid illuc iveras?

L. Ille me miserat petitum olera.

C. Quae tandem olera collegisti?

L. Vix enumerare possum omnia.

C. Saltem quae occurrunt memoriae.

L. Sed cur illud quaeris?

C. Ut interim recordemur aliqua rerum nomina, quae parvuli didicimus.

L. Pulchra est exercitatio, praesertim cum aliquid nobis suppetat otii. Audi igitur, collegi: allia, serpyllum, porros, caepas, nasturtium, cuminum, faeniculum, thymum, amaracum, hyssopum, apium, salviam, satureiam.

C. Herbae sunt olentes, quas adhuc numerasti.

L. Sic institueram memoriae gratia.

C. Perge porro.

L. Paucae quidam restant ut beta, cichorium, lactuca, oxalis, eruca, brassica, portulaca. Plures non occurrunt.

C. Qui potuisti tot meminisse?

L. Praeceptor mihi dederat scriptum catalogum.

C. Et noveras omnes?

L. Noram, alioquin eum interrogassem.

C. At ego non novi omnes, quamvis nomina didicerim.

L. Ego tibi plures etiam demonstrabo, cum licebit nobis in hortum ire.

C. Multumne attulisti?

L. Plenum calathum.

C. Sed de singulis quantum?

L. Nimis es curiosus, quid vis? De singulis attuli quantum opus fuit.

C. Tametsi curiosus tibi videor, tamen hoc mihi velim respondeas: scin' tu in quem usum praeceptor tot olerum genera curaret apparanda?

L. Partim ut ius bene condiretur, partim ut minutal ex oleribus fieret.

C. Quam bene consulebat nobis!

L. Optime, sed tamen ea non erat causa praecipua.

C. Quaenam igitur?

L. Legistine unquam \emph{Moretum} Vergilii?

C. Carmen quidem legi, sed moretum nunquam edi nec vidi, quod sciam.

L. At videbis, spero. Nam praeceptor uxorem docuit conficere et illa confecit ex eius praescripto.

C. An apponetur nobis aliquid in cenam boni?

L. Saltem ius carnium pingue, bene conditum, carnes optimae et minutal ex oleribus.

C. Unde scis ista?

L. Omnia vidi in culina cum iussus essem adiuvare, praecipue in meis oleribus repurgandis.

C. Quid moretum? Non saltem gustabimus?

L. Immo dabitur nobis. Nam confectum est quod satis sit omnibus.

C. Ista mihi sapiunt magis, praesertim aestate, quam carnes ipsae aut pisciculi.

L. Optarem profecto servari carnes in hiemem, ut tota aestate olera et fructus esitaremus.

C. Atqui (ut accepi) ista minus alunt.

L. Id ego audivi aliquoties. Sed quid opus est tanto studiosis alimento?

C. Non tanto fateor. Si tamen parentes nos viderent pallidos et macilentos, statim eius rei culpam assignarent praeceptori. Nonne sic est?

L. Non est dubium. Sed quid agas? Fere parentes (praecipue matres) nobis indulgent nimium.

C. Vera quidem dicis. Sed tu interim matris indulgentia libenter frueris.

L. Quasi vero tu minus.

C. Ne mentiar, quod tibi ascribo, in me quoque frequenter experior.

L. Non possumus parentum erga nos affectum, nisi nostris vitiis, immutare. Tantum caveamus eorum abuti benevolentia. Sed in primis laudemus illum Patrem nostrum benignissimum, qui nobis tales progenitores dedit.

C. Ista libens audio. Sed nos hora vocat.

L. Age, finem imponamus.

\subsection{Colloquium 11}
\emph{De carnis emptione. Divinatio ex coniecturis. Contra vulgi opinionem de fortuna. Mali mores tenacius retinentur. Solus Deus adiutor. Linguae lapsus ex imprudentia. Memoria exercenda. Dociles proficient admonitu.}

Molerius, Dothaeus

M. Unde redis?

D. A foro.\footnote{S. Foro S. Corr. A foro D. E foro}

M. Quid emisti?

D. Carnem.

M. Qualem?

D. Vitulinam.
 
M. Ostende, quaeso, fere nova res est hoc tempore.

D. Vide.

M. Bona videtur mihi.

D. Non falleris, opinor.

M. Quot sunt librae?

D. Nolunt lanii appendere vitulinam.

M. Cur non?

D. Propter novitatem.

M. Vide astutiam. Scilicet quisque vendit quam potest carissime.

D. Rem acu tetigisti.

M. Quantum putas pendere?

D. Duas libras et paulo amplius.

M. Quanti emisti?

D. Age, divina.

M. Non sum divinus.

D. Atqui multi divinant qui tamen divini non sunt.

M. Fieri potest, sed ex quibusdam coniecturis. Alioqui divinatio vetita est in Divinis Litteris.

D. Divina igitur ex coniectura.

M. Emisti totum duobus assibus?

D. Paulo minoris.

M. Quanti ergo?

D. Tenta iterum.

M. Viginti denariolis.

D. Nolo te diutius torqueri de nihilo.

M. Dic igitur, sodes.

D. Hoc totum constitit mihi sex quadrantibus.

M. Profecto fortuna tibi pulchre favit.

D. Quam mihi fortunam narras?

M. Hic mos est loquendi.

D. Mos, ut dicitur, tyrannus est pessimus. Atque utinam bonos mores tam studiose coleremus quam obstinate retinemus malos!

M. Tunc melius se haberent omnia.

D. Nos igitur fortunam istam ethnicis et impiis relinquamus. Fortuna nihil est, solus est Deus, qui favet nobis, solus est adiutor et protector noster.

M. Istud quidem certo scio, fideliter credo et vere confiteor. Sed quid agas? Saepe labitur lingua, nihil mali cogitante animo.

D. Opportebat te illud meminisse ne lingua praecurrat mentem.

M. Istud quidem didicimus e Septem Sapientum dictis. Sed non semper occurrunt eiusmodi pulchre dicta, licet ea mandaverimus memoriae.

D. Tanto igitur magis illa est exercenda ut nobis, cum opus est, suppetat.

M. Isto tuo admonitu discam esse alias prudentior.

D. Sed audio signum dari, desinamus.

\subsection{Colloquium 12}
Herus, Famulus

H. Fuistine hodie in foro?

F. Fui.

H. Quando?

F. Post contionem factam.

H. Quid emisti nobis?

F. Fere nihil.

H. Quid autem?

F. Butyrum.

H. Quanti?

F. Quadrante.
 
H. Tantillum?

F. Non ausus sum amplius emere.

H. Quid timebas?

F. Ne bonum non esset.

H. Satis prudenter factum.

F. Cur istum dicis, here?

H. Quia malim te esse hac in re timidiorem quam audaciorem. Sed nunquid emisti praeterea?

F. Nihil.

H. Eho, nihilne?

F. Nihil prorsus.

H. Vah, quam parce nobis obsonatus es!

F. Quid aliud emere potuissem?

H. Quasi nescias quibus cibis oblectari soleam.

F. Scio te amare caseum molliusculum et pira et alios fructus recentes.

H. Recte dicis. Cur ergo non emebas?

F. Caseus ipse carior erat pro nostra pecuniola.

H. Quid fructus?

F. Alii erant non satis maturi, de aliis ego dubitabam essentne boni.

H. Miser, non poteras gustare?

F. Atqui istae mulieres nihil gustare permittunt, nisi te empturum affirmes.

H. Nihil mirum, multi enim gustarent animi tantum gratia. Tu igitur esto alias prudentior.

F. Quomodo?

H. Si videris pulchrum aliquem fructum, eme aliquantulum denariolo ut facias periculum.

F. Quid tum praeterea?

H. Si tibi sapuerit, tum emito amplius; sin minus, relinquito et alio te conferre.

F. Bona est ista cautio.

H. Memineris igitur ut ipse postea utaris.

F. Ego, ut spero, meminero diligenter. Nunquid vis praeterea?

H. Ut cures quae tui sunt officii, deinde litteris incumbas.

\subsection{Colloquium 13}
\emph{Praeceptoris accusatio in Gallice loquentes. Eiusdem ad Latine loquendum exhortatio. Interpretatio edicti de vetito Gallico sermone. Male acturi quaerunt latebras. Iniquorum perveristas. Nitimur in vetitum. Doctrina Christi et timor Dei avocant a malo. Exhoratio ad recte vivendi studium.}

Caroletus, Quintinus

C. Adfuisti matutinae precationi?

Q. Adfui. Tu vero ubi eras?

C. Iveram ad patrem in cauponam.

Q. Quid eo?

C. Heri vesperi iusserat ut se convenirem bene mane.

Q. Qui ausus es tam mane prodire, idque inconsulto praeceptore?

C. Iam heri veniam impetraram antequam iretur cubitum. Sed dic mihi, quid ille a precatione palam admonuit?

Q. ``Audivi,'' inquit, ``esse inter vos, qui saepenumero Gallice fabulentur, et nemo interea vestrum mihi quicquam indicat. Quod est argumentum consensionis omnium in eodem peccato.'' Haec fuit accusationis summa. Deinde in eam sententiam multa dixit, quae meminisse non potui.

C. Sed quae tandem fuit conclusio?

Q. ``Quamobrem,'' inquit, ``admoneo vos ut alius alium ad Latine loquendum cohortemini diligenter, et eorum nomina, qui parere noluerunt, ad me quamprimum deferatis ut huic malo remedium adhibeam.''

C. Nullum igitur verbum licebit efferre Gallicum?

Q. Quantum ex verbis eius colligere potui, non ita rem intelligit. Non enim (ut scis) usque adeo est severus exactor ut statim puniat si cui verbum aliquod inter colloquendum exciderit.

C. Aliquoties (ut memini) palam dixit edictum suum ad eos demum pertinere qui cum Latine sciant, tamen semper latibula quaerunt ut Gallice fabulentur, idque de rebus ineptissimis.

Q. Ea est quorundam pertinacia, ut malint saepissime vapulare repugnando praeceptis honestissimis quam laudari atque etiam diligi obsequendo.

C. Meministin’ audire ex ipso praeceptore, ``Nitimur in vetitum''?

Q. Memini, atque adeo est verissimum. Tamen qui doctrinam Christi libenter amplectuntur, non studio peccant neque malitia

C. Istud praestat verus ille timor Domini.

Q. Tales igitur, quoad licet per naturae infirmitatem, sedulo cavere nituntur ne quid scientes faciant dicant aut cogitent quo Deus vel minimum offendatur.

C. Ergo studeamus et nos idipsum cavere, studeamus recte vivere et Dei nostri parere voluntati. Non modo ne vapulemus sed magis ut illi optimo Patri nostro placeamus.

Q. Ita fiet ut vere simus non tenebrarum sed lucis filii.

C. Sed de his alias pluribus. Ad disputationes nos recipiamus.

Q. Ecce, vocat signum.

\subsection{Colloquium 14}
\emph{Aliquot nomina carnis generum.\footnote{S. de carnis generibus S. Corr. nomina carnis generum} Reprehenditur inperite locutus. Vivarium ferarum. Hortatio ad agnoscenda beneficia Dei. Eleemosyna secreta. Locus ex Evangelio.}

Albertus, Tirotus

A. Nunc demum redis a foro?

T. Quid demum? Tanta est ad lanienam turba ut vix accedere potuerim.

A. Quas attulisti nobis carnes in diem crastinum?

T. Bubulam et vervecinam.

A. Estne in foro carnium magna copia?

T. Tanta profecto ut mirer adeo caras esse.

A. Nihil mirum. Ut multae sunt carnes, ita multi qui carnes edunt quotidie. Sed quae potissimum vidisti carnis genera?

T. Vidi bubulam, vitulinam, ovillam, vervecinam, suillam, haedinam, agninam.

A. Nihilne amplius?

T. Quid velles praeterea?

A. Nihil igitur erat ferinae?

T. Non queo referre simul omnia. Immo, etiam ferinam vidi.

A. Qualem?

T. Cervinam et aprugnam. O quam pinguis est aprugna!

A. O quam ineptus es!

T. Quid ita?

A. Quia falleris in rerum nominibus. Nam quod in sue domestico dicitur \emph{arvina} id in sue fero (id est apro) \emph{callum} vocatur, et est in eo genere durissimum.

T. Istud quidem audire non memini.

A. Nunc audisti, manda (si vis) memoriae.

T. Tu vero doctor, unde illud didicisti?

A. Domi nostrae ferina caro res est frequentissima.

T. Unde vobis tanta est copia?

A. Pater habet ruri ferarum multarum vivarium, ex quo interdum solidi apri in urbem adferuntur.

T. Quale est vivarium istud?

A. Locus est fere quadrangulari forma, amplissimus, muris altissimis septus undique, consitus multis et proceris arboribus, inter quas sunt dumeta maxime densa.

T. Quales sunt illic arbores? Utrum urbanae an silvestres?

A. Silvestres fere omnes. Sed in his potissimum quercus et fagi, quarum glande pascuntur cervi, apri, damae.

T. O quantas merito gratias Deo debetis, qui vobis largitus est tantam rerum omnium abundantiam!

A. Non sumus immemores beneficiorum eius. Pater enim piurima bona in pauperes erogat, quod tamen tibi uni dictum esse velim.

T. Cur ita?

A. Quia talia non vult praedicari.

T. Tanto magis laudandus quod Christi praeceptum vere sequitur, cuius verbae apud Mattheum scripta sunt, ``Cum facis,'' inquit, ``eleemosynam,'' et quae sequuntur.\footnote{6:2: ``Cum ergo facies eleemosynam, noli tuba canere ante te, sicut hypocritae faciunt in synagogis et in vicis, ut honificentur ab hominibus. Amen dico vobis: receperunt mercedem suam.''}

A. Quoto capite?

T. Sexto, nisi me fallit memoria.

A. Sed haec hactenus. Satis enim sumus collocuti, et iam ad disputandum convenitur.

T. Eamus igitur.

A. Sequere me, aut (si mavis) praecede.

T. Ego neutrum faciam, sed una ibimus.
 
\subsection{Colloquium 15}
\emph{De studiorum tempore amisso. Eius recuperandi spes et quibus modis id fieri possit. Viva magistri vox. Hortatio ad animum bonum.}

Guinandus, Monerotus

G. Tantum igitur hodie e villa revertisti?

M. Hodie tantum, idque paulo ante prandium.

G. Atqui dixeras te futurum illic modo biduum.

M. Ita sperabam fore et sic pater promittebat.

G. Quid igitur obstitit quominus redieris citius?

M. Mater me detinuit, tametsi etiam cum lachrimis eam obsecrarem ut me missum faceret.

G. Sed cur te tamdiu remorata est?

M. Ut se comitarer in reditu.

G. Quid vero agebas interea?

M. Colligebam fructus cum rusticis nostris.

G. Quos fructus?

M. Quasi non sint tibi noti fructus autumnales et serotini: pyra, mala, iuglandes, castaneae.

G. O iucunda exercitatio!

M. Non est iucunda solum, sed etiam frugifera.

G. Sed hoc malum, quod interim quinque aut sex praelectionum fructus tibi periit.

M. Non omnino periit, spero. Curabo pro viribus ut aliqua ex parte recuperem.

G. Quid facies?

M. Describam quam potero diligentissime.

G. Quid tum postea?

M. Ediscam ipsam auctoris orationem.

G. Sed sententiam non satis intelliges.

M. Ipsa me iuvabit praeceptoris interpretatio ut sensum magna ex parte assequar.

G. Nec tamen id satis erit.

M. Tu (si placet) aderis mihi per otium ut conferamus una.

G. Libenter equidem faciam. Sed ne istud quidem sufficiet.

M. Non habeo quod possim amplius.

G. Quanto praestitisset vivam audire magistri vocem!

M. Multo sane praestiterat. Sed quando id mihi non contigit, nec mea culpa factum est, nihil habeo quod me accusem in hac parte.

G. Recte dicis. Fac igitur habeas animum bonum. Nam quod ego tecum pluribus verbis de hac re disputavi, non ideo feci ut vellem te adducere in desperationem. Sed totum illud profectum est ex meo in te amore singulari.

M. Haud mihi dubium istud est, quo fit ut maiorem tibi habeam gratiam.

G. Sed ecce, vocat vos ad cenam tintinnabulum!

M. Nuntius opportunus.

\subsection{Colloquium 16}
Ioannes, Petrus

I. Salve, Petre!

P. O Ioannes, auspicato advenis. Valesne bene?

I. Optime, gratia Deo. Tu vero, ut vales?

P. Recte sane, Dei beneficio. Sed quando redisti domo?

I. Nudiustertius.

P. Bene habet. Opportune venisti.

I. Nempe sciebam instare vacationis terminum.

P. Placetne ut otiose aliquandiu confabulemur?

I. Maxime, dummodo semoti simus ab hac turba clamosa ludentium.

R. Bene mones, secedamus in auditorium illud quod est apertum.

I. Quam apte hic sedamus! Age, loquamur libere.

R. Suntne peractae vestrae vindemiae?

I. Omnino.

R. Quantum temporis posuistis in toto opere?

I. Dies circiter quindecim.

R. Tu igitur semper interfuisti?

I. Nullum intermisi diem.

P. Quid agebas?

I. Uvas saepius colligebam.

R. Cum verbo \emph{colligebam} debuisti aliquid addere.

I. Quidnam, quaeso?

R. ``Et edebam.''

I. Quid opus fuit? De hoc nemo dubitare potest. Quis enim fructus bonos et maturos legit quin edat etiam ex optimis?

R. Profecto recte loqueris. Euge! Responsum laudo.

I. Iamne putabas os occlusisse mihi?

P. Istud non cogitavi quidem.

I. Quid igitur?

P. Non expectabam tam promptum, tamque prudens responsum.

I. Nihil est quod mireris. Nam, ut est in proverbio, ``Saepe etiam est olitor verba opportuna locutus.''\footnote{Polydore Vergil \emph{Adagiorum Libellus} A249 (1498/1521): ``Saepe etiam est olitor valde opportuna locutus,'' ``Even a kitchen-gardener has often spoken to good point.''}

P. Cui debes hoc proverbium?

I. Magistro Iuliano. Is enim dictat nobis interdum proverbia eiusmodi et pulchras sententias ex bonis auctoribus.

P. Optime vobis consulit. Sed quibus horis solet id facere?

I. Nonnunquam a cena, saepius autem cum in auditorio nihil habemus reddere.

P. Utinam sic omnes facerent dummodo non essent impedimento quotidianis scholae exercitationibus!

I. Bene subiunxisti istam exceptionem per adverbium dummodo. Sunt enim quidam paedagogi qui suis dictatis et privatis lectionibus sic onerant pueros suos ut non possunt in schola satisfacere.

P. Inde fit ut ipsi praeceptores interdum conquerantur de talibus paedagogis. Sed quid agimus? Redeamus ad sermonem institutum.

I. Placet.

P. Fuistine semper occupatus in uvis colligendis?

I. Opus illud in paucis diebus absolvitur, propterea quod ita magnus operariorum numerus ad id locari solet.

P. Quid deinde fit?

I. Calcantur uvae, vinum hauritur e cupis maioribus eadem opera effunditur in dolia, deinde uvae ipsae nondum expressae subiiciuntur prelo in torculari, postremo exportantur et abiiciuntur vinacea.

P. Atqui non curabas ista.

I. Immo curabam, aliqua ex parte. Nam agendis omnibus inteream, maxime ut solicitarem operarios.

P. Eras ergo illic tanquam praefectus et quasi magister operum.

I. Immo etiam revera magister operum et praefectus; pater enim me praefecerat.

P. Quam gaudebas isto magisterio! Quam pulchrum erat videre te cum tua gravitate aliis imperantem, alios adhortantem, alios denique arguentem!

I. Profecto si vidisses me, dixisses alium esse quam in schola discipulum.

P. Ut video, non eras otiosus.

I. Immo, ut bonus paterfamilias, adhibebam saepenumero manus operi ut ipsos operarios meo exemplo instigarem.

P. Non abs re (ut apparet) pater te huic muneri praefecerat.

I. Nempe aliis in rebus expertus erat meam diligentiam.

P. Absit tamen verbo iactantia.

I. Ego sic intelligo. Sed libere sic loquor quia cum familiari meo.

P. Sed pergamus. Quanta est vobis vini copia?

I. Mediocris, qualis fere hoc anno ubique esse dicitur, tametsi vinum rubellum habemus affatim, album non item. Verum quicquid est, contenti sumus et de manu Domini cum gratiarum actione recipimus.

P. Quot implevistis dolia?

I. Plus minus quadraginta, sed sunt alia aliis maiora.

P. Papae! Non tibi videtur magnus proventus?

I. Satis quidem, sed non pro ratione anni superioris.

 
P. Quid refert? Quanto minor est quantitas, tanto etiam pluris vendetur.

I. Sic fere solet evenire. Sed nonne tibi videor satis narrare de vendemiis? Quid praeterea desideras?

P. Quoniam sic abundamus otio, volo etiam aliquid ex te audire de fructibus arborum. Est enim quasi altera vindemia. Nonne habetis multos?

I. Plenis tabulatis, quae est Dei benignitas.

P. Quando collecti sunt?

I. Quo tempore vinum faciebamus, familia colligebat.

P. Quae sunt vobis eiusmodi fructuum genera?

I. Mala, pira, castaneae, iuglandes, sed malorum et pirorum multa est varietas.

P. Quid cotonea? Non etiam habetis?

I. Immo habemus, sed ea sub malorum genera continentur, unde et alio nomine appellantur mala cydonia.

P. Quid autem attulisti huc rediens?

I. Nihil nisi quasillum uvarum selectarum. Sed singulis posthac hebdomadibus mihi afferentur plenis saccis mala, pira, castaneae.

P. Interea da mihi, quaeso, aliquot ex uvis tuis.

I. Eamus in cubiculum meum, illic dabo tibi.

P. Equidem paratus sum, eamus.

I. Illic etiam agemus de repetendo ultimo in diem Lunae colloquio. Nam, ut opinor, id praeceptor in primis exiget.

\subsection{Colloquium 17}
\emph{Puer a patre commendatus praeceptori. Verus amor patris in filium. Correctio necessaria puero. Poena etiam iniusta ferenda est, et qua re. Christus pro nobis passus immerito. Optatum pium et sanctum. Hortatio ad diligentiam. Exemplum pueri studiosi et oboedientis.}

Eustathius, Bosconellus

E. Audivi patrem tuum venisse hodie in gymnasium.
 
B. Verum audivisti.

E. Qua venerat gratia?

B. Ut pro meis alimentis praeceptori numeraret pecuniam,\footnote{D. pecuniam S. pecunia} simul ut me illi commendaret.

E. Nunquamne te commendarat?

B. Immo saepissime.

E. Quid sibi vult ista commendatione tam frequenti?

B. Amore vero me prosequitur.

E. Quid tum?

B. Ideo cupit me diligenter erudiri.

E. Quid si commendet ut saepius vapules?

B. Ea est fortasse causa. Sed quid inde? Non propterea me diligit minus.

E. Unde istud colligis?

B. Quia puero tam necessaria est correctio quam alimentum.

E. Verum quidem dicis, sed pauci ita iudicant. Nemo enim est quin panem quam virgam malit.

B. Illud est naturale omnibus, quis negat? Sed tamen patienter ferenda est poena, praesertim iusta.

E. Haec habetur in libello morali sententia: ``Quod merito patiris patienter ferre memento.''\footnote{Disticha Catonis III.17: \\``Quod merito pateris, patienter perfer id ipsum,
\\cumque reus tibi sis, ipsum te iudice damna.''} Sed, quid si poena sit iniusta?

B. Ea quoque patienda est nihilominus.

E. Cuius causa?

B. Propter Iesum Christum, qui mortem iniustissimam eamque acerbissimam tulit pro peccatis nostris.

E. Utinam id nobis in mentem veniret quoties aliquid patimur!

B. Praeceptor id nos saepe monet, quoties occurrit occasio. Sed, ``surdis narratur fabula,'' ut est in proverbio.\footnote{Terentius \emph{Heauton Timorumenos} 222: ``ne ille haud scit quam mihi nunc surdo narret fabulam.''}

E. Ergo demus operam ut simus posthac diligentiores.

B. Ita faxit Deus.

\subsection{Colloquium 18}
\emph{Divinis rebus omnia sunt postponenda, sed fere contra fit. Homo semper homo est nisi divino spiritu renovatus. Exemplum hominis qui eadem facit quae in aliis reprehendit. Turpe est docotri, etc.}

Grandinus, Thomas

G. Cur non interfuisti hodiernae contioni?

T. Occupatus eram in scribendis litteris.

G. Non poteras differre negotium?

T. Urgebat tabellarii festinatio.

G. Atqui praeceptor docet nos omnia postponenda esse Dei negotiis.

T. Docet quidem, neque id mihi dubium est. Sed nunquam sumus adeo perfecti quin saepe Deum terrenis istis postponamus.

G. Istud malum.

T. Pessimum vero. Semper homines sumus, nisi Deus nos Spiritu suo immutaverit. Sed dic (quaeso) fuitne frequens auditorium?

G. Non admodum, pro more solito.

T. Unde fit istud?

G. Ignoras populum nunc esse occupatum in vendemiis?

T. Non ignoro, sed non possunt homines divinis rebus unicam horam \footnote{S. horulam S. Corr., D. horam} impendere?

G. De hoc non est meum tibi reddere rationem, hoc tantum dico, ``Turpe est doctori, cum culpa redarguit ipsum.''\footnote{Disticha Catonis I.30:\\ ``Quae culpare soles, ea tu ne feceris ipse:
\\turpe est doctori, cum culpa redarguat ipsum.''}

T. Papae! Quantum colaphum impegisti mihi! Vale, verbum non amplius addam.

G. Esto igitur alias prudentior.

\subsection{Colloquium 19}
\emph{Discessurus quidam ante vacationem scholasticam, reprehenditur a condiscipulo. Consilia humana a Deo reguntur. Sua cuique tractanda scientia. Modeste sapiendum.}

Molinaeus, Carrarius

M. Tu igitur cras (ut audio) discessurus es.

C. Cras, si Dominus permiserit.

M. Eho, cur tam cito?

C. Urget me pater.

M. Immo tu urges patrem.

C. Itane tibi videtur? Quomodo patrem urgere possum?

M. Assidua missione litterarum.

C. Tantum semel scripsi instare vacationem scholasticam.

M. Quando misisti litteras?

C. Hebdomade superiori.

M. Quo die?

C. Veneris.

M. Quid facies domi?

C. Instat vindemia, interim colligendi sunt fructus arborum.

M. Poteras expectare dimissionis diem.

C. Nescio quando sit futurus.

M. Spero fore ad finem proximae hebdomadis.

C. Sed istud non est in nostro situm arbitrio.

M. Ne in praeceptoris quidem.

C. Cuius igitur?

M. Solius Dei, qui hominum consilia suo nutu gubernat.

C. Atqui Satanas videtur interdum gubernare.

M. Quantum Deus ipsi permittit. Sed ista sapientioribus relinquamus.

C. Tutius est. Monet enim proverbium, ``Ne sutor ultra crepidam.''\footnote{Erasmus \emph{Adagia} I.vi.16 ``Ne sutor ultra crepidam.''}

M. Saepe istud ex praeceptore audivimus.

C. Idem quoque non semel docuit nos illam Pauli sententiam, ``Noli altum sapere, sed time.''\footnote{Ad Romanos 11:20}

M. Illud etiam frequenter habet in ore: ``Altiora te ne quaesieris.''

O. Sed audin' tu ad cenam signum dari?

M. Adhuc pulsat aures meas tintinnabulum.

C. Eamus in aulam, ne desimus precationi.

M. Cras ante discessum te salutabo.

\subsection{Colloquium 20}
\emph{Patientia in parvis exercitata, paulatim crescit. Volente Deo nobis accidunt mala. Providentia Dei. Memoria fluxa eam non exercentibus. Memoriae execendae ratio. Negligentia. Spiritus Dei nos excitat. Brevis exhortatio ad excutiendam negligentiam, pigritiam, tarditatem.}

Petrinus, Croseranus

P. Quo ludi genere hodie te exercuisti?

C. Iuglandium.

P. Ecquid lucri fecisti?

C. Immo, perdidi.

P. Fortuna igitur tibi adversa fuit.

C. Nescio qua fortuna. Tantum scio mea culpa accidisse, sed ita volente Deo.

P. Cur Deus id voluit?

C. Ut hinc discam ferre graviora cum acciderint.

P. Quasi vero Deus lusiones puerorum curet.

C. Curat profecto. Quinetiam nihil fit in rerum natura sine divina providentia.

P. Siccine philosopharis? Quisnam te ista docuit?

C. Non tute audivisti toties ex contionatore nostro?

P. Fieri potest ut audierim. Sed quid agas? Fluxa est mihi memoria.

C. Nimirum, quia illam non exerces.

P. Quomodo exercenda est?

C. Primum diligenti attentione, hoc est diligenter advertendo ea quae audivimus aut legimus, deinde eadem saepe repetendo, denique docendis aliis ea quae didicimus.

P. Ista nobis saepius inculcantur a praeceptore. Sed (me miserum!) quam supina est haec mea negligentia!

C. Sic sumus omnes, nisi Spiritus ille Dei nos excitet.

P. Quid igitur faciam?

C. Expergiscere, mi Petrine. Toto animo totisque viribus ad Deum adspira, illum assidue et pio affectu precare, vigilans esto, pravos fugito, versare cum bonis, tum moribus facillimis effice ut eos tibi familiares reddas.
 
P. Quid tandem consequar?

C. Rogas? Si te istis moribus assuefeceris, Dominus Deus sua clementia tui miserebitur, brevique animum tuum immutatum senties.

P. O quam opportunus hic mihi congressus fuit! Obsecro te, mi Croserane, ut saepius colloquamur.

C. Per me non stabit, quoties utrique licebit per otium.

P. Gratias ago maximas.

C. Non est quod agas. Recipiamus nos in auditorium.

\subsection{Colloquium 21}
\emph{Quidam reprehenditur quod animi causa nouerat. Peccatum per ignorantiam. Deus ad usum nostrum creavit omnia. Beneficiis Dei non abutendum. Ignoscendum puero sponte delictum confitenti. Ad precandum admonitio. Singulorum secretae preces. Perseverantia in verbo Dei. Exemplum rarisimum pueri qui admonitus culpam libenter confitetur.}

Aegidius, Massuerus

Ae. Cur hic dispergebas pisa?

M. Quando?

Ae. Post prandium.

M. Id faciebam animi causa.

Ae. Sed pisa illa unde habueras?

M. Acceperam e conchula, ubi reposita erant ut crastino die coquerentur.

Ae. Debuistine animi causa malum facere?

M. Non putabam id esse malum.

Ae. An non est malum conculcare panem pedibus?

M. Istud ego nollem facere.

Ae. Cur nolles?

M. Quia panis est nobis maxime necessarius.

Ae. Et pisa ipsa et cetera quae eduntur, Deus in usum nostrum creavit.

M. Non ignoro illud, quinetiam piscis libenter vescor, si bene cocta et condita sunt.

Ae. Praeterea, vellesne abuti rebus tuis?

M. Minime.

Ae. Tanto minus alienis debes.

M. Istud satis intelligo.

Ae. Ergo non recte fecisti.

M. Non recte, fateor. Non tamen animo malo.

Ae. Cur igitur fecisti?

M. Mea ineptia me ad illud incitavit.

Ae. Quid inde meruisti?

M. Plagas.

Ae. Recte dicis, sed (opinor) non ex animo.

M. Immo certe, ne me accuses oro.

Ae. Quando quidem sponte fateris, non accusabo. Sic enim velle se dixit praeceptor saepissime.

M. Quid ille dixit?

Ae. Ut de rebus eiusmodi levioribus neminem ad ipsum deferamus, qui modo culpam libens agnoverit.

M. Istud ergo beneficium tibi debeo, mi Aegidi.

Ae. Nihil velim mihi debeas hoc nomine, sed mecum praecare Deum ut a malo nos liberet.

M. Quotidie in schola quater aut quinquies palam precamur.

Ae. Quid tum?

M. Praeterea privatim, quoties cibus sumitur, quoties cubitum itur, quoties cubitu surgitur. Nonne satis haec sunt?

Ae. Praeter illa, saepe monet praeceptor ut interdum pro se quisque precandi causa secedat aliquo in secretum locum. Memenistin'?

M. Memini probe. Sed (ut scis) difficile videtur esse ut pueri secretis precibus assuescant.

Ae. Et tamen paulatim assuescere optimum fuerit.

M. Progressu temporis Deus ipse noster ad eam rem nos incitabit atque assuefaciet.

Ae. Ita fore sperandum est, si tamen in verbi eius tum lectione, tum auditione sedulo prosequamur.

\subsection{Colloquium 22}
\emph{De quodam convivio. Qui convivae, quae ferculorum varietas et copia, quis ordo fuerit, etc. Notatur obiter conviviorum luxus. Gratiarum actio. Obiurgatur ipse convivator de luxu.}

Varro, Castrinovanus

V. Quid est quod hodie tam cito a patruo redieris, praesertim cum fuerit convivium?

C. Quid illic fecissem diutius?

V. Cenam expectasses ut ederes de prandii reliquiis.

C. Satis ederam in prandio. Praeterea, iussit patruus ut domum reducerem praeceptorem, quem ego ad convivium deduxeram.

V. Quid frater tuus patruelis? Cur in ludum vobiscum non rediit?

C. A matre retentus est in unum aut alterum diem.

V. Quamobrem?

C. Ut illi resarciantur vestimenta.

V. Illa est mulierum cura. Sed age, quoniam nunc otiosi sumus, narra mihi (quaeso) aliquid de convivio.

C. Quid de illo scire cupis?

V. Primum, qui convivae fuerint; deinde, quam lautum et opiparum convivium.

C. Convivae fuerunt hi praecipui: quattuor syndici, suppraefectus urbis, et alii duo primae notae e senatorum numero.

V. Nostine?

C. De facie quidem. Sed eorum non teneo nomina.

V. Nulline praeterea?

C. Duo ex patrui mei familiaribus.

V. Quotus accumbebat praeceptor.
 
C. Quotus esset non observavi. Sed erat in media fere mensa, e contraria parti mei patrui.

V. Tu vero?

C. Hui inepte, qui istud roges? Egone homunculus cum tantis viris epularer? Satis hoc mihi honorificum fuit quod ministrarem.

V. Nullaene erant mulieres?

C. Nullae, praeter uxorem patrui quae quidem sedebat in mensa extrema.

V. Quid illa tam remota?

C. Sic voluit ipsa, ut identidem commodius surgeret propter ordinem ministerii.

V. Quid filius?

C. Iuxta matrem assidebat.

V. Habeo de convivis, nunc expecto de convivio.

C. Onus mihi valde magnum imponis ac difficile, maxime propter memoriam. Sed quando (ut dixisti) plusculum otii nacti sumus hoc pomeridiano tempore, dabo equidem operam ut aliqua ex parte expleam desiderium tuum.

V. Pergratum mihi feceris.

C. Ea tamen lege ut par pari referas, si quando dabitur occasio.

V. De hoc nihil est quod dubites. Incipe.

C. At ego interea sedere volo, quia longa est narratio.

V. Eamus sub pergulam ut in umbra commodius fabulemur.

C. Audi nunc iam.

V. Quia (ut ais) longa est narratio, dic mihi primum quota hora accubitum est?

C. Fere decima.

V. Quota resurrectum?

C. Paulo ante meridiem.

V. Sedebantne omnes commode?

C. Commodissime.

V. Nunc ad rem aggredere.

C. Accipe igitur mensae praeludia.

V. Appone, cum voles.

C. In primis apposita sunt tenella crustula, mellita opera pistoris cum aromate.

V. Optimum sane exordium et ad conciliandos animos aptissimum.

C. Omitte quaeso istas interpellationes, ne mihi perturbetur memoria.

V. Posthac non interpellabo, nisi si quid opus erit requirere.

C. Secutae sunt pernae salitae, hilla infumata, lucanicae, linguae bubulae sale quoque et sumo induratae. Atque haec ad excitandam appetentiam et sitam acuendam.

V. Quasi vero non satis acueretur solis aestu et fervore.

C. Sic docti solent facere convivatores.

V. Istaec audio libenter, praesertim cum exprimas omnia propriis et significantibus verbis. Perge porro.

C. Eodem ordine interposita sunt acertaria e lactucis capitis, avium intestina frixa, minutalia ex vitulina, cum ovorum vitellis integris. Et haec hactenus de praeludiis, qui missus primus fuit.

V. Nihil interim bibitum est?

C. Indigna homine quaestio. Quis enim vino hic parceret? Vix teres, et quidem strenuissimi, sundendis potibus sufficiebamus. Sed de potibus agam posterius, sine me cibos expedire.

V. Age, sino.

C. In secundo missu haec fere fuerunt: artocreae, pulli gallinacei elixi cum lactucis, bubula, vevercina, vitulina, suilla recens salsa, ius carnium ovorum vitellis croco et omphacio suavissime conditum, aliquot item iuscula ex oleribus.

V. Hic plus opinor fabulatum quam esum fuisse, quia scilicet assa expectabantur.

C. Vix mensam attigerant cum illa tollere iussi sumus. Venio igitur ad tertium missum in quo assa haec fuerunt exposita: pulli gallinacei, pulli columbini, anserculi fratriles et suculi, item cuniculi, anni vervecini; postremo, ferina duorum generum opere pistorio incrustata.

V. Quid praeterea?\footnote{Sic D., S. C. Vix mensam ... incrustata. Quid? Quid praeterea? Hem (paene praetermisi) ... et pisa cocta cum siliquis. V. Nihilne piscium}

C. Quid? Hem, (paene praetermisi) duae perdices interiectae cum lepusculo, fabae virides frixae et pisa cocta cum siliquis.

V. Nihilne piscium?

C. In tempore admones. Erat trocta ingens quae divisa fuerat in partes quattuor praeter caudam, praegrandis item lucitis ad eundem modum quadripartitus. Taceo minutos pisces et mediocres, partim elixos, partim assos aut frixos, item cancros fluviatiles, magno omnia numero. Sed haec magis ad ostentationem quam ad necessitatem. De iis enim fere nihil gustatum est.

V. Unum mihi videris praetermisisse.

C. Quid illud est?

V. Nulline erant intinctus?

C. Immo singulis propemodum ferculis addita erant scitissimi saporis embammata, quae coquus ipse miro artificio concinnaverat. Nec vero defuerunt capparides ex oleo et aceto, mala citrea, olivae conditivae cum sua muria, acetum rosaceum, succus oxalidis.

V. O quot et quanta gulae irritamenta!

C. Adde etiam corporis et animi impedimenta.

V. Sed quis (quaeso) fuit ultimus actus fabulae?

C. Tandem, cum iam nec carnes nec pisces ullus ederet, iubet patruus apponi bellaria in quibus haec fuerunt praecipua: caseus recens pinguissimus, itemque vetus multiplex, scriblitae, placentulae, oryza in lacte cocta et bene saccarata, persica praecocia, ficus, cerasa, uvae passae, caryotae, tragemata serotina, salgama multorum generum, et alia quae nunc mihi non occurrunt. Tanta denique fuit omnium esculentorum affluentia ut mensa ipsa vix sustineret. Quid quaeris? Quater aut quinquies mutati sunt orbes et quadrulae. Cibos crassos et duriores magna ex parte integros in culinam referebamus, adeo pauci attingebant propter delicatiorum\footnote{S. delicatiorem S. Corr. D. delicatiorum} copiam.

V. Quid confert tanta cibariorum abundantia et diversitas?

C. Ad gravandum stomachum et morbos complures generandos. Sed quid agas? Sic fere hoc tempore vivitur.

V. Qui talibus viris convivia faciunt, certare videntur de copia, de apparatu, de splendore et lautitia.

C. Et tamen extant inter huius civitatis leges quaedam (ut audivi) sumptuariae.

V. Silent leges inter convivia, ut obiter e Cicerone detorqueamus aliquid.\footnote{Cicero \emph{Pro Milone} 11: ``Silent enim leges inter arma.''}

C. An putas omnes convivas illa sumptuum immanitate delectari?

V. Non puto, nisi forte sint lurcones aut asoti aut Apiciani (ut ita dicam) ventricolae. Ceterum quid est causa?

C. Rogas? Convivatorum non modo stultitia, sed etiam insania.

V. “Maxima pars hominum morbo iactatur eodem,” ut ait Horatius.\footnote{Horace \emph{Sermones} II.3.121: ``maxima pars hominum morbo iactatur eodem."}

C. Sed nos hanc censuram omittamus iuxta proverbium, ``Ne sutor,'' et cetera.\footnote{Erasmus \emph{Adagia} I.vi.16 ``Ne sutor ultra crepidam.''}

V. Redeamus igitur ad propositum: quale vinum appositum est?

C. Si de colore quaeris: album, nigrum, fulvum, sanguineum, deque singulis coloribus vina multiplicia. Si de bonitate, omnia fere generosissima. Sed in primis commendabant\footnote{S. commendabat S. Corr et D. commendabant} illud vini genus ex Burgundia, quod vulgo vocatur Arbosium.

V. Unde hoc petebatur?

C. E patrui cella vinaria.

V. Habetne multum eiusmodi?

C. Duos cados vini Helvetii,\footnote{S. et D. heluelli} duos item albi limpidissimi.

V. Quis fuit prandii exitus?

C. Ubi videt patruus convivas omnes paene defessos edendo, bibendo, colloquendo tunc vinum fundi iubet singulis, omnes ad extremum potum invitat. Hinc ordine tolluntur omnia, insternuntur mensis latiora mantilia ex lino tenuissimo, datur aqua odorifera manibus leviter abluendis. Ego et patruelis de more Deo gratias agimus, ipse vero patruus clara voce agit universo coetui. Tandem primus syndicus, convivarum nomine, satis accurato sermone, publicas agit gratias convivatori. Simul obiurgat eum quod tam munifico et sumptuoso apparatu convivatus fuerit. ``Immo,'' inquit patruus, ``mihi quaeso ignoscite, quod vos pro dignitate non satis ampliter tractaverim.'' His dictis, surgunt e mensa universi, magna pars valedicto statim discedit, ceteri manent stantes et colloquentes in aula.

V. Quid vos interea, qui ministraveratis ad mensam?

C. In culinam ad prandium nos recepimus, corpus illic otiose et ex animi sententia curavimus.

V. Ubi erat interim praeceptor?

C. Patruus illum seorsum vocaverat ad colloquium.

V. Credo ut te et filium suum de meliore nota commendaret.

C. Istud est verisimile.

V. Nescis quae fuerit causa tanti convivii?

C. Quid mea scire refert?

V. Tanto minus igitur mea.

C. Recte colligis, et sic expectabam.

V. Atqui non sum dialecticus.

C. Tibi est naturalis dialectica.

V. Eandem habent et rustici.

C. Sed age, dic mihi serio, non ego te tractavi magnifice?

V. Tali convivio nunquam interfueram.

C. Gaudeo stomacho tuo satis esse factum.

V. Est Deo gratia, qui dedit nobis otium tam iucundum.

C. Surgamus tandem. Nam audio pueros a lusu discedere.

\subsection{Colloquium 23}
\emph{De scholastica societate in conferendis merendae cibis. Auspicantur merendam a carmine in Dei laudem. Cura de pauperibus. Matrum in filios amor ineptus. Divinare ex coniecturis non est malum. Exhortatio ad divinis laudes et gratiarum actionem. Notatur voracitas in edendo. Reprehenditur abusus in verborum notione. Latine loquendi servanda est properitas. Ex prava institutione vitia contrahantur. Deteriora pertinacious haerent.\footnote{S. haerut S. Corr. D. haerent} Exemplum permissa hilaritatis in Christianorum conviviis.}

Pastorculus, Poesatus

\begin{verse}[\versewidth]\settowidth{\versewidth}{Tityte, qui patulae resides sub tegmine mori,}
\flagverse{Pa.} Tityte, qui patulae resides sub tegmine mori,\\
Tune hic solus eris, tam laetus tamque supinus?\\!

\flagverse{Poe.} O Meliboee, Deus nobis haec otia fecit.\\
Ille Deus magnus, qui nostrum fecit in usum\\ 
Omnia, dans propriam cunctis animalibus escam.\\
Qui mare, qui terras, et quod tegit omnia caelum\\
Condidit, ille opifex rerum, qui summus habetur.\\!

\flagverse{Pa.} Carmina mitte loqui, nunc me liquere camoenae.\\
Est mihi mens alibi, cupio certare merenda.\\!

\flagverse{Poe.} Sed tamen hac mecum poteris residere sub umbra,\\
Namque hic (ut cernis) locus est satis amplus utrique.\end{verse}
 
Pa. Mittamus ergo versiculos, et merendas nostras conferamus.

Poe. Per me non stabit. Scrutemur peras. Age, explica tuam.

Pa. Expecta parumper. Dic mihi prius quid habes in merendam.

Poe. Panem.

Pa. Quasi vero sine pane merenda esse soleat.

Poe. Ne panem quidem pauperes semper habent.

Pa. In tempore admones, ponendae erunt reliquiae nostrae in eorum corbulam.

Poe. Quid si reliquiarum nihil fuerit?

Pa. Saltem restabit panis, et hoc satis erit. Sed dic tandem, nunquid habes obsonii?

Poe. Etiam dubitas? Mater mea nunquam committeret ut me in scholam mitteret sine aliquo obsonio.

Pa. Dic ergo, quid est?

Poe. Divina.

Pa. Non sum vates, nec velim hanc merendae horam in nugis terere.

Poe. Saltem periculum facies quam valeas coniecturis, qua de re iam aliquid in rhetoricis audivimus.

Pa. Aut caseus aut caro est residua ex prandio.

Poe. Neutrum.

Pa. Dic sodes, ut accingamus nos operi.

Poe. Ne te diutius torqueam, sunt pira praecocia.

Pa. Ain' tu? Res nova. Nondum hoc anno videram.

Poe. Vide nunc.

Pa. Quam matura sunt!

Poe. Cur non addis etiam ``quam bona''?

Pa. Sed nondum gustavi.

Poe. Satis acute me reprehendis. Accipe et gusta.

Pa. Papae, quam mitia! Quam boni succi!

Poe. Nonne merito maximas gratias agere debemus Deo nostro, tam benigno Patri, qui nobis indignis tot bona tamque varia largitur?

Pa. Qui non facit, is profecto est ingratissimus.

Poe. Agedum, vescamur bonis eius cum gaudio et gratiarum actione.

Pa. Iamdudum esurio.

Poe. Sed tu nullum habes obsonium?

Pa. Vide frustum crassum vetusti casei.

Poe. Edamus primum pira, caseo claudemus stomachum.

Pa. Sed maturemus, ne forte hora nos opprimat.

Poe. Neutrum cessare video. Certe quod ad me pertinet, non queo comesse citius.

Pa. Ne tamen ita devores porcorum more. Ecquid pudet?

Poe. Quia dicebas esse festinandum.

Pa. ``Maturemus,'' dixeram, non autem, ``festinemus.''

Poe. Ego non adeo scrupulose inter haec verba discrimen facio.

Pa. Vult tamen praeceptor ut proprie loquamur, quantum per ingenii captum licebit. Nam bene loquendo, bene etiam scribere condiscimus.

Poe. Contra diligenter scribendo consuescimus etiam recte loqui.

Pa. Haec duo inter se coniuncta sunt. Sed heus, otiose, (inquam) edamus, satis habemus temporis.

Poe. Nonne tota haec hora ad merendam libera est?

Pa. Hodie quidem libera. Sed tamen desinamus ne panis deficiat nos et nihil reliqui fiat pauperibus.

Poe. Eamus ergo ad puteum ut aliquantum potemus.

Pa. Hem, verborum proprietate semper abuteris! Istud \emph{bibere} est, non \emph{potare}.

Poe. Quod mihi non parcas, habeo sane gratiam. Ex prava institutione prima haec vitia contraxi.

Pa. Verissime igitur Quintilianus dixit, ``Haec ipsa magis pertinaciter haerent quae deteriora sunt.''\footnote{\emph{Institutio Oratoria} I.1.5} Meministin'?

Poe. Memini. Sed interim iuva me, ut hauriamus aquam e puteo.

Pa. Enitere validius, nimium me laborare sinis.

Poe. Tanto bibes iucundius.

Pa. Bibitum satis est. Recipiamus nos in aulam, ne precationi desimus et actioni gratiarum.

Poe. Tu praecede, dum urinam illuc eo redditum.

\subsection{Colloquium 24}
\emph{Reprehenditur negligentia in cultu corporis. Excusatur a penuria rei numariae. Verecundia in petendo. Mediocritas et modus. Verecundia ubi sit utendum. Hominum societas. Abusus in rebus. Hortatio ad petendi fiduciam. Exemplum admonitionis liberae inter familiares.}

Leonardus, Pellio

L. Demiror tuam negligentiam.

P. Qua in re tandem?

L. Quod te non curas diligentius.

P. Ego vero me curo fortasse nimis. Satis edo, bibo, dormio, quae est Dei erga me benignitas. Praeterea pecto capillum, lavo manus, faciem, dentes, oculos, et haec mane praecipue. Quinetiam, cum tempus postulat, corpus exerceo, relaxo animum et ludo cum ceteris. Quid vis amplius?

L. Mittamus ista. Non ea sunt, quae in te reprehendo.

P. Quid igitur?

L. Circumspice vestimenta tua, a calce ad verticem nihil integrum invenies. Omnia sunt lacerata et obsoleta. Ista profecto nequaquam vestrum genus decent. Si saltem curares vestitum tuum sarciendum aut quoquo modo instaurandum.

P. Loqueris tu quidem quae libet, quod si parentes haberes tam procul remotos fortasse non esses elegantior. Si mihi pecunia suppeteret, non paterer me usque adeo pannosum esse.

L. Nec ideo tamen cares negligentia. Cur enim non petis alicunde mutuo?

P. Unde peterem?

L. Si non alicunde, certe a praeceptore posses.

P. Quid si dare nollet?

L. Nemini denegat ex discipulis domesticis, siquidem videt opus esse.

P. Id ego non ignoro, sed sum verecundior quam ut audeam ex eo petere.

L. Ah, rusticus est iste pudor!

P. Malo tamen verecundus esse quam impudens.

L. Verecundia (ut dixit quidam) est bonum in adolescente signum. Sed ubique adhibenda est mediocritas.\footnote{Seneca \emph{Epistulae Morales} XI.1: ``Ubi se colligebat, verecundiam, bonum in adulescente signum, vix potuit excutere.''}

P. Ego eo sum ingenio ut semper vereor offendere quempiam.

L. Ingenium laudo, sed est modus in rebus. Nam ille offendendi metus habere locum debet in rebus turpibus aut certe indecoris, hic vero nihil tale video. Est usitatum in hominum societate ut alii aliorum opera indigeant. Quis igitur mihi vitio dabit, si quid ab amicis aut commodato petam aut mutuo?

P. Nemo reprehendet, nisi forte rebus eiusmodi abuti velis.

L. Tu vero (quantum ego te novi) abuti nolles.

P. Apage istum abusum!

L. Quid ergo iam obstat, quominus petas, praesertim ab homine facillimo tuique (ut apparet) amantissimo?

P. Age, petam. Sed per epistolium quod ut reddas, tibi dabo.

L. Reddam profecto libentissime, teque illi commendabo diligenter.

P. Equidem non parvam tibi gratiam habeo, quod me tanti feceris ut ad hanc fiduciam hortarere.

L. Nunc restat ut scribas quod dicis epistolium, reliqua mihi committas.

P. Bene vertat Deus quod coepimus.

L. Ne dubites, res prospere succedet.

\subsection{Colloquium 25}
\emph{Quidam ultro vult venire habitatum in gymansio. Gaudes amicus eius, eumque ad maturandum hortatur. Patris lenitas in filium. Consilium ex Deo profectum. Plerique pueri habitationem in schola tanquam carcerem refugiunt. Improbi adolescents a studio probos absterrent. Aliquot genera excertationis scholasticae, praecipue in Latine loquendo. Hominum maledictis augetur opus Dei, tantum abest ut retardetur. Praeceptor in schola severior quam inter domesticos et qua re. Magistri gravitate schola in officio continetur. Disciplinae scholasticae bona. Quibus rebus inducantur adolsecentes ad amorem litterarum. Praeceptor bonus a studiosis diligatur. Disciplina severior quibus adhibenda. Mens bona et sincerum iudicium a Deo. Exemplum rarissimum adolescentis, qui ultro quaerit in scholam habitare.}

Clavellus, Quercetanus

C. Nescis quid mihi his diebus in mentem venerit.

Q. Quid istud, quaeso?

C. Cogito in gymnasium me recipere.

Q. Quid! In gymnasium? Habitandi causa?

C. Non ut inquilinus habitem, sed ut vobiscum vivam in mensa praeceptoris.

Q. O utinam istud ex animo ac vere diceres!

C. Ex nostra mutua consuetudine atque amicitia deberes satis intelligere, me nihil apud te neque simulare, neque dissimulare solitum.

Q. Plane istud iampridem intelligo. Sed, audito ex te isto verbo, me rapuit affectus in eam exclamationem, ita prorsus oblitus eram mei.

C. Non aliter dictum tuum accipio. Sed ad rem, pater quidem non cogit me, sed ex vultu et verbis eius satis video id illi placere maxime.

Q. Nempe hoc illud est, novit pater tuus, vir prudentissimus, liberalia ingenia cogi nolle, duci facile. Sed tamen non dubito id fieri divini instinctu numinis. Tu vero quid iudicas?

C. In eadem sum sententia, praesertim cum ego quoque in eam partem sponte propendeam.

Q. Magnum argumentum, istud esse ex Deo consilium.

C. Id ego crediderim, nam qui ad eam rem coguntur a parentibus, fere quaerunt subterfugia quibus detrectent imperium.

Q. Ego (ut fatear ingenue) aliquando id in me sum expertus, ante scilicet quam ingressus essem. Quid enim censes? Audiebam ex istis Satanae mancipiis tot maledicta, tum de praeceptore, tum de disciplina, ut mihi viderer carcerem aut pistrinum ingredi verius quam gymnasium. Quod si tu quoque in aliquam eiusmodi pestem incidisses, non dubium est quin ille manibus pedibusque obnixe conatus esset absterrere te ab isto tam sancto proposito.

C. Nemini adhuc palam feceram.

Q. Quod mihi dicis, non est palam facere.

C. Satis scio, sed mihi istud excidit ex Terentio. Nunc pergamus ad reliqua.

Q. Quid restat, nisi ut Deum preceris et pergas graviter in sententia? 

C. Audies. De\footnote{S. Audies: de\ldots~ S. Corr. et  D. Auides. De\ldots} quotidiano victu in vestra mensa, de praeceptoris et hypodidascali familiaritate in audienda praelectionum repetitione, de facilitate praeceptoris in ipso convictu, de illo a cena placido studiorum certamine, de libertate colloquendi per otium de honestis, de libera inter nos reprehensione. De his (inquam) omnibus satis multa narrasti mihi alias, quae quidem valde placent omnia. Mihi tamen nihil videtur utilius quam illa perpetua Latine loquendi exercitatio. Quis enim maior est studiorum fructus? Quid honestius? Quid iucundum magis, praesertim ubi alius alium sine odio aut invidia libere reprehendit.

Q. Quid, quod victi in earum rerum certamine, pudore tantum puniuntur, victores etiam praemio donantur?

C. Omnes denique vestrae excogitationes, vel auditu ipso, me delectant mirifice.

Q. Quanto magis illud diceres, si videres ista, si, dum geruntur, interesses?

C. Ego (Deo volente) interero.

Q. Utinam propediem!

C. Ita spero quidem. Unus tamen restat mihi scrupulus, quo me facile (si voles) liberabis.

Q. Faciam bona fide, si potero. Sed interim vide ne, ``nodum in scirpo quaeras,'' ut ante vidimus in Terentio.\footnote{\emph{Andria} 940--1: ``dignus es
\\cum tua religione, odium: nodum in scirpo quaeri'.''} Age, qui tandem scrupulus iste est?

C. Suspecta est mihi vestra disciplina scholastica, non quod velim ob id incepto desistere sed ut alacrius veniam et iucundius, ubi ea de re te audiero.

Q. Nisi tu is esses, qui (ut ego novi) me nec vanum nec mendacem existimas, equidem mallem de hoc omnino tacere quam ea tibi referre quae sentio.

C. Quid ita tandem?

Q. Quia difficile est ea verisimilia facere, quanquam alioqui verissima, cum de his vulgus hominum pessima quaeque dicat et iudicet.

C. Quorsum istud, quaeso?

Q. Nam improborum mendaciis adeo male audit nostra disciplina inter istos idiotas, ut mirum sit unum aliquem habitare in schola nostra velle. Tametsi (quod est opus Dei) quo peius audimus, eo plures ad nos conveniunt.

C. Nihil opus est tam longa insinuatione. Dic mihi plane omnia, ne timeas, non me absterrebis. ``Omnia praecepi atque animo mecum ante peregi.''\footnote{Vergilius \emph{Aeneis} VI.105}

Q. Adverte igitur animum, dicam brevissime quicquid alicuius esse momenti videbitur. In primis, illud velim tibi persuadeas praeceptorem esse nobis multo humaniorem quam in schola palam apparet. Tam enim familiariter versatur inter nos quam prudens pater solet inter liberos. Cur igitur (inquies) est palam tam severus? Respondeo: quia sine tali severitate (ut ex ipso audivi, cum aliquando familiari cuidam narraret) tanta turba scholastica, tamque variis moribus praedita, nec coerceri, nec in officio contineri posset. Suo enim quisque more, suo quisque arbitratu vellet vivere.

C. Quinetiam miror ego mecum saepissime, tantam esse in tota schola reverentiam, tantum silentium, tantam denique modestiam.

Q. Multo magis mirarere, si unquam vidisses scholas paganicas.

C. Vidi aliquando atque consideravi diligenter. Plus est silentii in gymnasio nostro sexcentario quam quadraginta puerorum, immo triginta in istis scholis trivialibus. Sed perge, quaeso, vereor ut sis orator Asianus. Iam enim incipit oratio tua quasi longius aberrare.

Q. Atqui etiam interpellas ipse.

C. Nihil extra propositum dixi. Sed nunc prosequere.

Q. Vis in summa dicam tibi? Disciplina haec domestica, licet paucis improbis otiosa videatur, bonis tamen et studiosis propter utilltatem valde placet. Nam, si domi res esset dissoluta, quid nobis tutum foret contra lasciviorum et petulantium insultus? Quae nobis quies? Quae studiorum tranquillitas? Itaque disciplina ipsa veris studiorum amatoribus est perfugium et quasi propugnaculum, non secus ac nautis portus in tempestate. Denique quisquis est inter nos pacificus et ad recte semper agendum paratus, is est a disciplina tutissimus. Nec vero id agit praeceptor ut nos plagis et verberibus trahat ad litterarum studia, sed nos potius inducere nititur his maxime rebus: honesta et liberali tractatione, benevolentia, humanitate, facilitate morum, beneficiis, denique virtutis et studiorum amore. Ex quo fit ut maior pars nostrum sic affecta sit ut illi ex animo parere studeat, cum caveat offendere, cum tanquam parentem diligat, observet, revereatur.

C. Alioqui discipulorum officio non fungeremini.

Q. At sunt quidam verberones, qui nec Deum timent, nec parentes, nec verbera, qui et litterarum studia oderunt ``cane peius et angue,'' ut verbis utar Horatii.\footnote{Horatius \emph{Epistulae} I.17.30: ``alter Mileti textam cane peius et angui.''} Talibus (inquam) severa interdum adhibitur disciplina, quia videlicet necessitas cogit.

C. Sat habeo. Nunc enim video quorsum spectat vestrae disciplinae severitas.

Q. Nimirum ut mores bonos tueatur, malos autem ut corrigat aut expellat.

C. Ego disciplinam istam sane deosculor, tantum abest ut reformidem. Te autem, mi Quercetane, amo equidem de ista loquendi libertate qua mihi stimulos acriores addidisti.

Q. Ego vero immortales gratias ago Patri caelesti, qui tibi mentem bonam dederit et sincerum iudicium.

C. Vale igitur, et me (si Dominus permiserit) expecta in proximam hebdomadem. Iterum vale, et inceptum nostrum tuis precibus Christo commenda etiam atque etiam.

Q. Precor tibi noctem quietam et somnium placidum.

\subsection{Colloquium 26}
\emph{Basilius a praeceptore deprehensus in delicto, amicum Florentium consulit quo remedio poenas evadat. Deo volente adversa nobis accidunt. Falso nobis adscribimus quae soli Deo tribuenda sunt. Cupediae tentant pueros. Magistri in pueris deprehensis prudentia. Excusationes frivolae. Poenarum metus ad remedium quaerendum impellit. In adversis ad Deum in primis confugiendum est. Consilium sine Deo nihil prodest. Salus omnis a Deo speranda. Praeceptoris prudentia in tractanda ingeniorum varietate. Consilium utile ad poenas evitandas. Occasio captanda in adeundis hominibus. Exemplum fidelis amicorum consilii in adversis rebus.}

Florentius, Basilius

F. Quid est quod ita te tristem video?

B. Me miserum! Vix sum compos animi, ita sum timore perculsus.

F. Quid (inquam) istud est?

B. Praeceptor nos deprehendit.

F. Qua in re? Furtone?

B. Ah, minime gentium!

F. In quo igitur?

B. In secreta compotatione.

F. Indignum facinus!

B. Heu me miserum! Me miserum! Quid agam?

F. Ah! Ne te afflictes tantopere. Qui aderant tecum?

B. Flavianus et Forensis. O perditos, qui me huc impulerunt!

F. Ubi id factum fuit?

B. In cubiculo Flaviani, quod utinam hodie e lecto non surrexisset!

F. Quomodo tandem deprehensi estis? Non erat observatum cubiculum?

B. Immo erat, sed (ut scis) praeceptor claves habet omnium conclavium. Praeterea, putabamus illum in conventum hodiernum fratrum accessisse, alioqui pessulum obdidissemus ostio.

F. Utcunque res habet, Deo volente accidit.

B. Sic omnino sum persuasus. Sed fere ita solemus nostra omnia sive bona sive mala, vel prudentiae nostrae attribuere vel imprudentiae.

F. Qua occasione compotationem istam occeperatis?

B. Illi duo (quod satis nosti) non sunt convictores.

F. Quid tum?

B. Domo allatae fuerunt eis ad victum, nescio quae cupediae, quae utinam in via periissent.

F. Sed noluit Deus. Perge.

B. Et quia illos interdum per otium doceo, heri post cenam ad istas epulas me invitarunt in hodiernum ientaculum.

F. Tibi infaustum ientaculum. Sed quid? Eratne caninum?

B. Immo plus satis erat vini. Habent enim vini doliola in cella praeceptoris.

F. Quid praeceptor, ubi vos sic epulantes vidit? Annon excanduit gravissime?

B. Nihil prorsus commotus est, sed subridens, ``Volo,'' inquit, ``esse de isto symposio, si placet.''

F. O molestum convivam! Quid vos?

B. Attoniti omnes obmutescimus. Nam, eo dicto, statim se proripuit.

F. Omnia signa video hic esse pessima.

B. Quis ita non iudicet?

F. Oportet praeceptorem aliunde rem olfecisse.

B. Suspectus est mihi quidam, qui nos indicaverat.

F. Quem suspicaris?

B. Dicam tibi post rei exitum.

F. Quod invenietis huic malo remedium?

B. Nescio, valde stupemus omnes, tametsi duo illi convivatores mei nihil videntur esse soliciti. Habent enim in promptu causam.

F. Quid tandem causari queunt?

B. ``Iveramus ientatum,'' inquient, ``quia mane, cum ceteri ientarent, eramus cum parentibus qui heri nobis attulerant alimenta.''

F. Istud quidem est aliquid. Sed non tamen sic poenas evadent.

B. Cur non?

F. Quia id fieri debuit in triclinio palam, non autem clanculum in cubiculo. Illud etiam causam gravabit quod te, cum sis convictor, quasi ad crapulandum adduxerint. Quod quidem ipsi praeceptori est maxime odiosum. Tua tamen causa est longe pessima.

B. Cedo igitur. Quid faciam, mi Florenti?

F. Age, conferamus consilia quibus (si fieri potest) absolvaris.

B. Nihil est quod a me expectes in eo genere. Neque enim consilii locum habeo, neque ullam remedii copiam. Sed tu (obsecro) explicaveris ingenii tui ad me cito iuvandum. Animus pendet mihi, instat poena, satis vides. Ea (nisi quid impediat) post cenam exigetur.

F. Id nequaquam fiet. Cum enim crimen publicum non sit, non erit poena publica.

B. Sive publice sive privatim id futurum est, non differetur.

F. Recte iudicas.

B. Id causae est quamobrem nunc ad opem tuam confugerim, quod si me deseris, actum est. Dabo poenas gravissimas.

F. Ad Deum potius confugiendum fuit, sine quo nullum valet consilium, a quo item salus omnis et petenda et speranda est.

B. Res est manifesta satis, et ego priusquam te adirem, non praetermisi precandi officium. Sed tamen vult ille Pater clementissimus ut iis utamur auxiliis, quae nobis offert ipse et quasi in manum dat. Quare te oro, per nostram arctissimam necessitudinem ut consilio tuo nunc mihi succurras. Hic non est cessandi aut tardandi locus. Huc igitur vires tuas intende, hanc causam suscipe, obsecro.

F. Quando quidem sic instas ut vim adferre quodammodo videaris, dicam ex animo et vere quod sentio. Nostin' praeceptoris ingenium?

B. Novi (opinor) ex parte aliqua.

F. Ergo memoria tenes, nihil esse illo clementius iis quidem quos videt submissos esse, culpam ingenue consiteri. Superbis autem et contumacibus vix aliquid potest ignoscere.

B. Egomet ista non semel observavi.

F. Scin' igitur quid facias?

B. Dic mihi, quaeso.

F. Dum tua res adhuc integra est, tibi suadeo, ut hominem solus adeas in museum et orationem habeas in hanc sententiam: ``En ego, praceptor humanissime, graviter peccavi, fateor. Sed (ut scis) prima est noxa quam admiserim. Nunquam enim antea quicquam admiseram quod flagris dignum videretur. Quamobrem clementissime praeceptor, hanc primam culpam (si tibi placet) pristinae integritati meae condonabis. Quod si unquam posthac recidero, causam non dico quin plectar severissime.'' Hac aut simili oratione illum (ut spero) exorabis.

B. O consilium prudens et opportunum!

P. Utere, si tibi videtur. Tale certe est, quale mihi dari vellem in eiusmodi malo.

B. Sed unum restat, de quo sum solicitus.

F. Eho, quid restat?

B. Non potero tam breviter et commode dicere in conspectu praeceptoris quam tu nunc dixisti.

F. Non debes adeo de verbis meis esse solicitus, modo teneas sententiam.

B. Propemodum teneo.

F. Bene habet, scribe nunc crasso modo ut poteris, deinde conferemus una. Postea edisces ad verbum diligenter.

B. Hoc nihil est neque tutius neque certius. Sed quo tempore censes illum adiri posse opportune?

F. Cum primum videbis eum a prandio se recipere in bibliothecam, aut si forte (ut solet) ambulatum in horto solus iverit, tunc illum statim sequere.

B. Quonam utar exordio?

F. Non opus est alio quam quo apud illum uti solemus omnes.

B. Quid\footnote{S. Quid D. Quod} est illud?

F. ``Praeceptor, licetne pauca?''

B. Bene vertat Deus omne consilium tuum. Nunc eo scriptum quod dixisti, deinde te revisam.

F. Matura, nam instat hora prandii.

\subsection{Colloquium 27}
\emph{Quidam consolatur amici desiderium ac dolorem de patris absentia. Exemplum caritatis in afflictis consolandis.}

Ambrosius, Gratianus

A. Subtristis mihi videris, Gratiane. Quid rei est?

G. Dies noctesque de patre absente cogito et ob eam rem interim maerore conficior.

A. Quam pridem abest?

G. Quattuor abhinc menses.

A. Nihil ad vos interea scripsit?

G. Ex quod discessit ne verbum quidem audivimus.

A. Fieri potest ut scripserit, sed aut litterae interciderunt aut fuerunt interceptae.

G. Quod dicis verisimile est. Nam antea, ubicunque esset, solebat nobis scribere.

A. Nonne hac aestate iverat Lutetiam?

G. Eo certe consilio tunc sese in viam dederat.

A. Confido salvum esse.

G. Ego quoque non diffido. Sed non ideo meus allevatur dolor.

A. Quid ait mater?

G. Fere semper lamentatur. Inde fit, ut mihi duplicetur aegritudo.

A. Sed tamen non usque adeo te macerare debes. Potius enim de illo bene sperare oportet quam te sic macerare. Quid enim sunt quattuor menses? Quotula est pars eorum qui tot annos domo absunt, interim iactati, ``Per varios casus, per tot discrimina rerum?''\footnote{Vergilius \emph{Aeneis} I.214}

G. Sed quid me consolantur aliorum casus in pericula?

A. Vide tamen ne sis nimium delicatus, qui patris desiderium tantillo tempore ferre non queas.

G. Ferrem equidem aequo animo, modo illi bene esse saltem intelligerem.

A. Cui potest esse male, qui in Deo spem omnem collocavit? ``Sive enim vivimus, sive morimur, Dei semper sumus,'' ut divus Paulus ait.\footnote{Ad Romanos 14:8 ``Sive enim vivimus, Domino vivimus: sive morimur, Domino morimur. Sive ergo vivimus, sive morimur, Domini sumus.''}

G. Nihil dubito istud esse verissimum. Sed ea est carnis huius infirmitas.

A. Quid faceret igitur, qui in Christo spem nullam habet?

G. Nescio.

A. Sed cave tamen ne ista tua impatientia Deus ipse offendatur.

G. Delictum meum subinde agnosco, et supplex imploro veniam.

 
A. Facis ut debes. Sed audi, quaeso, quid mihi nunc in mentem venerit.

G. Quid istud est?

A. Quid si pater tuus navigaverit in Britanniam negotiandi causa? Illic enim nunc est libertas maxima.

G. Quam libertatem dicis?

A. Evangelium, quod illic auditur liberrime.

G. Ain' tu evangelium nunc esse in Britannia?

A. Certa res est.

G. Atque idololatriam profligatam?

A. Omnino.

G. O auditu iucunda evangelia!

A. Immo iucundissima.

G. Sed unde scis ista?

A. Unde sciam, rogas? Miror ego te adhuc ignorasse quod in ore est omni populo.

G. Mirari desines si scias ubi nos habitemus.

A. Ubi quaeso?

G. In angulo totius urbis remotissimo.

A. Atqui putabam vobis esse domicilium in vico ad Molardum.

G. Iam ante migraveramus quam peregre pater esset profectus.

A. Quod igitur ignorabas, nunc habeto certissimum. Utque magis credas, hoc audi praeterea: maior pars Britannorum, qui ob evangelium in hanc urbem tanquam ad asylum sese receperant, in patriam remigrabant iam ante dies quindecim.

G. O mi Ambrosi, quantum me isto nuntio de Britannicis rebus hodie recreasti! Qualem adhibuisti meo dolori medicinam?

A. Sic solet Deus noster suis adesse in extremis angustiis.

G. Mirum, ni pater est in Britannia. Iampridem enim saepius querebatur quod non esset tutus illuc accessus ad mercaturas obeundas.

A. Quinetiam Britannus quidam homo, neque levis neque nugator, narrabat his diebus patri meo se litteras certas illinc accepisse in quibus hoc erat inter cetera, omnes undique ob Christi nomen profugos in ipsa Britannia excipi humanissime tractarique benignissime.

G. Quid est igitur, quod amplius dubitemus?

A. Nulla (ut audis) restat ea de re dubitatio.

G. Tantum superest, ut in primis Dei optimi maximi bonitatem extollamus quanta possimus laude et gratiarum actione, deinde sedulo atque assidue precemur ut sua beneficia non modo confirmet, sed etiam in dies magis ac magis augeat.

A. Igitur mi Gratiane, memineris patris salutem ipsi Deo per Christum saepissime commendare idque votis et precibus ardentissimis.

G. Utinam ille suo Spiritu sic afficiat animum meum ut ex immo pectore eiusmodi preces effundere valeam, quas ipse dignetur exaudire!

A. Votum sanctissimum, modo (quod quidem credo) ex animo profectum.

G. O mirabilem consolatorem Deum nostrum! O quantum valet in adversis rebus veri amici consilium et consolatio!

A. Sed quo nunc is?

G. Domum recta propero ut haec matri quamprimum nuntiem atque eius animum omni expleam gaudio.

A. Faxit Deus ut illa serio exhilaretur.

G. Ita precor.

\subsection{Colloquium 28}
\emph{De exercitationibus post vacationem scholasticam revocatis. De classium probatione ante promotiones. De spe praemii. De eadem in adversis rebus. De fructibus autumnalibus. De maximo vini proventu. De abusu vini in summa copia. De admonitionum contemptu. De iis qui se dociles praebent ad verbum Dei audiendum.}

Dessynangaeus, Ionas

D. Salve, Iona\footnote{S. Ionas S. Corr. et D. Iona} optatissime.

I. Salve et tu, vindemiator iucundissime.

D. Quomodo se habet schola?

I. Optime.

D. Iamne redierunt multi?

I. Vix quarta pars nunc abest.

D. Quid igitur\footnote{S. agitur S. Corr. et  D. igitur}?

I. Docetur, legitur, scribitur, repetitur.

D. Ista sunt generalia et quotidiana. Sed quid fit in nostra classe?

I. Idem quod in fit ceteris, et quod fiebat ante vacationem.

D. Ergone iam serio habentur praelectiones?

I. Iam ab hinc octo dies. Nam quid aliud fieret?

D. Solebamus per aliquot dies exerceri in iis repetendis quae antea didiceramus.

I. Tribus diebus totis nihil aliud actum est.

D. Quid probatio, iamne incepta est?

I. Hui, incepta! Propemodum finita est. Cras sexta classis interrogabitur.

E. Me miserum! Exclusus sum a praemio.

I. Etiamne praemium sperabas?

D. Quidni sperarem? Unicuique sperare licet, praesertim studioso.

I. Malim ego nihil sperare.

D. Quid ita?

I. Nam\footnote{S. Iam S. Corr. et  D. Nam} si nihil contigerit, nulla erit mihi frustratio neque molestia, sin consequar aliquid id ego in lucro deputabo.

D. Nunquamne legisti, ``Vivere spe vidi, qui moriturus erat''?

I. Immo vero legi, et memoria teneo. Sed hic nihil ad propositum.

D. Cur non?

I. Quia illic loquitur Ovidius de spe illa quae in rebus adversis retinenda est, quem sensum expressit Cato noster hoc disticho: \begin{verse} ``Rebus in adversis animum submittere noli. \\ Spem retine. Spes una hominem nec morte relinquit.''\footnote{Disticha Catonis II.25}\end{verse}

D. Tu igitur nihil speras?

I. Spero vitam aeternam.

D. Dicebas modo sperandum esse in adversis rebus. Quae tibi sunt adversa?

I. Ea quae me oppugnant quotidie.

D. Quae tandem sunt illa?

I. Propria caro, Satan et improbi homines, qui me iniuriis afficiunt.

D. Ex quo tempore factus es theologaster?

I. Nec sum theologus, nec theologaster. Sed quod dico, id ego didici e sacris contionibus.

D. Laudo equidem. Sed age, dic mihi serio, estne acta classis nostrae probatio?

I. Iam tibi serio dictum puta.

D. Dolet mihi non interfuisse.

I. Citius venisses. Quid agebas?

D. Mater me coegit vindemiae exitum expectare.

I. Credo. Sed tu tua voluntate coactus es.

D. Ut fatear ingenue, libenter expectavi. Sed quid agas? Homines sumus, ut aiunt.\footnote{Petronius \emph{Satyrica} 75.1 ``Post hoc fulmen Habinnas rogare coepit ut iam desineret irasci et, `nemo,' inquit, `nostrum non peccat. 	Homines sumus, non dei.' ''}

I. Immo pueri.

D. Sed vix credas quam sit iucundum rusticari, presertim ubi tanta est omnium fructuum affluentia.

I. Estne vobis magnus vini proventus?

D. Tantus profecto ut maiorem vidisse non meminerim.

I. Quid aiunt rustici in hac tanta ubertate?

D. Nihil aliud quam potationes et crapulas crepant. Quinetiam vino iam perinde abutuntur, quasi aqua fontana sit.

I. Ea est stultae plebis dementia ut Dei beneficiis nunquam recte uti sciat.

D. Scilicet istud est quod dicitur, ``Nunquam sapiunt stulti, nisi in angustiis.''

I. Ergo plectuntur merito.

D. Quod admonitiones etiam irrident.

I. Sunt qui nullam omnino ferre possunt, quin bene et comiter etiam monentibus irascuntur.

D. Audi aliquid gravius: fuerunt qui mihi verbera minarentur cum eos humanissime admonerem.

I. Istud mihi novum non est. Nam et quidem mihi aliquando pugnum intentavit quod nisi veritus esset mei patris auctoritatem profecto vapulassem accerrime.

D. Sed omittamus eos qui caeci sunt, nec tamen se tales esse intelligunt.\footnote{S. nec tamen intelligunt S. Corr. et D. nec tamen se tales esse intelligunt} Iactant se probe callere quid sit evangelium, cum revera evangelio adversentur.\footnote{cum revera \ldots~ adversentur. S. omisit.  Habent S. Corr. et  D.}

I. Si verbi divini ministros et pastores suos non audiunt, qui putas fieri posse ut nos audiant?

D. Sequamur ergo quod praecepit Christus, ``Nolite sanctum dare canibus.''\footnote{Secundum Mattheum 7:6: ``Nolite dare sanctum canibus, neque mittatis margaritas vestras ante porcos, ne forte conculcent eas pedibus suis et conversi dirumpant vos.''}

I. At sunt quidam simplices et benevoli qui auditione verbi divini mirum in modum delectantur. Hi sunt, quos docere est iucundissimum.

D. Quam gaudeo quoties in tales incido ut ego illos complector libenter!

I. Nec immerito. Nam ubi eos docueris, tum demum frueris laboris tui fructu non sine voluptate maxima.

D. Sed nos quid cogitamus? Non vides, ut nox fere nos oppresserit?

I. Discedamus igitur et cras (si Dominus permiserit) pluribus verbis agemus una de studiis alacriter repetendis.

\subsection{Colloquium 29}
\emph{De moribus adolescentis ingenui tam absente patre quam praesente. De quinto Decalogi praecepto. Locus Pauli de honore liberorum erga parentes. Exemplum simulatae repugnantiae ad eliciendum colloquium.}

Rufinus, Sylvester

R. Pater tuus (ut accepi) rediit e Gallia.

S. Rediit sane.

R. Quando?

S. Die Lunae vesperi.

R. Adventus eius non tibi molestus fuit?

S. Quid molestus? Immo vero iucundissimus. Sed cur istud rogas?

R. Quia forsitan illo absente liberius vivendi est tibi potestas.

S. Nescio quam mihi libertatem narres.

R. Potandi, ludendi, cursitandi.

S. An igitur putas me nihil aliud agere, dum pater abest?

R. Sic\footnote{S. Corr. et  D. Omisit S.} fere solent omnes.

S. Dissoluti quidem. Nam quod ad me attinet, absente patre sic vivo ut eo praesente. Non poto, sed bibo quantum satis est. Ludo, cum tempus postulat. Non discurro, sed cum bona matris venia in publicum prodeo cum aliquid habeo negotii.

R. Etiamne matri tantopere subditus es?

S. Aeque ac patri. Quid enim putas? Nonne de utroque equale est praeceptum Domini?

R. ``Honora,'' inquit, ``patrem tuum, et matrem tuam.''\footnote{Exodus 18:20}

S. Quid ethnicus ille auctor distichorum moralium?\begin{verse} ``Dilige non aegra caros pietate parentes.\\Nec matrem offendas, dum vis bonus esse parenti.''\footnote{Disticha Catonis III.24:\\``Aequa diligito caros pietate parentes\\nec matrem offendas, dum vis bonus esse parenti.''}\end{verse} Quid Paulus noster? ``Filii,'' inquit, ``oboedite parentibus in Domino.''\footnote{Ad Ephesios 6:1: ``Filii, oboedite parentibus vestris in Domino; hoc enim iustum est.''} Nonne parentis nomine pater et mater continentur?

R. Istud a Latinis observatur auctoribus.

S. Quinetiam, si quod esset reverentiae discrimen, videretur optimo iure plus deberi matribus ut quae dolores tantos ac labores propter nos pertulerunt.

R. Novi ego ista, et quae dixisti placent mihi omnia.

S. Cur ergo mihi repugnabas?

R. Ut ea quasi repugnantia nobis accerserem sermonis materiam. Nam (ut tute nosti) praeceptor saepe nos exhortatur ut nostrum otium eiusmodi sermonibus impendamus.

S. Bonum sane otium quod honesto in negotio consumitur.

R. Huc pertinet illud Africani apophthegma qui dicebat, ``Se nunquam minus otiosum esse, quam cum otiosus esset,'' ut ex Cicerone didicimus.\footnote{Cicero \emph{De Officiis} III.1: ``P. Scipionem, Marce fili, eum, qui primus Africanus appellatus est, dicere solitum scripsit Cato, qui fuit eius fere aequalis, nunquam se minus otiosum esse, quam cum otiosus, nec minus solum, quam cum solus esset.''}

S. Sed iam tempus monet ut huic sermoni finem imponamus.

R. Recte mones. Fortasse enim tua causa cena tardatur domi.

S. Plura (si Dominus permiserit) in proximo congressu.

R. Precor tibi noctem prosperam.

S. Et ego tibi placidam per membra quietem.

\subsection{Colloquium 30}
\emph{De negligentia quidam admonetur. Admonitio ad somniculosos. Praeceptorum moralium usus. Quotidie proficiendum. Vitia naturalia. Armis spirtualibus pugnandum cum vitiis.}

Nomenclator, Desiderius

N. Non satis mirari possum te non esse diligentiorem.

D. Qua in re videor tibi negligens?

N. Quod mane fere nunquam ades in tempore, atque inde sit ut paene quotidie noteris in catalogo. Cur tu es adeo somniculosus?

D. Mea sic fert natura.

N. Corrige naturam istam, hoc est naturae vitium. Quid tibi profuit Catonis nostri dictum?

D. Quodnam, quaeso?

N. ``Plus vigila semper, nec somno deditus esto.''\footnote{Disticha Catonis I.2:\\``Plus vigila semper nec somno deditus esto;\\nam diuturna quies vitiis alimenta ministrat.''}

D. Ne plura, ego probe memini.

N. Nihil prodest meminisse nisi ad usum tuum accommodes.\footnote{S. Corr. et  D. Omisit S.}

D. Utinam et hoc et alia praecepta salubria tam facile ad bene vivendi usum transferemus quam ea facile ediscimus!

N. Ut verum fatear praecipere quam praestare multe est facilius. Sed tamen eniti debemus ut vel monitis vel precibus proficiamus aliquid et in dies evadamus meliores.

D. Qui id non facit, eius salus desperanda est. Sed nihil difficilius emendatur quam naturale vitium.

N. Omnia fere vitia nobis sunt naturalia et nisi Dei bonitas servaret nos, essemus omnes sceleratissimi.

D. Quid igitur faciendum?

N. Fortiter cum vitiis nostris pugnandum est.

D. Quo duce?

N. Ipso Deo.

D. Quibus armis?

N. Divinis et spiritualibus.

D. Ubi tandem inveniuntur?

N. In epistola Sancti Pauli Ad Ephesios.

D. Quoto capite?

N. Sexto.

D. Quid si locum per me non intellexero?

N. Non omnino intelliges, satis scio. Sed praeceptor erit consulendus.

D. Quid si mecum adfueris?

N. Adesse volo, certum est. Verum captanda erit opportunitas.

D. De hac igitur alias consilium capiemus.

N. Quando istud erit?

D. Proximo die Mercurii, si tibi ita videtur.

N. Quota hora?

D. Post meridiem prima.

N. Placet sententia.

D. Nunc igitur discedamus.

\subsection{Colloquium 31}
\emph{Non est malum malo, sed bono malum compensandum.}

Trapezita, Raemundus

T. Licetne malum malo rependere?

R. Cur istud quaeris?

T. Ut iis respondere possum, qui in hac re mihi contradicunt.

R. Breviter interrogasti, respondebo brevissime: non licet.

T. Cur non?

R. Quia vetuit Christus et post eum Apostoli.

T. Quid igitur faciendum est?

R. Malum bono compensandum.

T. An non satis est bonum bono rependere?

C. Non satis Christiano quidem.

T. Quamobrem?

R. Nam Christianum oportet Christum praeceptorem imitari.

T. Quid fecit Christus in eo genere.

R. Sanavit eum, qui ipsi colaphum impegerat. Peccatus est pro iis, qui ipsum in crucem sustulerant. Alia item multo fecit eiusmodi.

T. Nihilne igitur facit qui gratiam pro gratia rependit?

R. Tantum facit, quantum faciunt ethnici.

T. Quid illi faciunt?

R. Amicos diligunt et referunt gratiam illi, a quibus acceperunt.

T. Nihilne amplius?

R. Nihil. Nam quid amplius expectes ab iis qui verum Deum non noverunt?

T. Quid faciunt inimicis?

R. Quibuscunque modis possunt eos persequuntur.

T. Estne Christiano peccatum?

R. Si non licet (ut iam tibi probavi) conclude id peccatum esse.

T. Atqui (ut vulgo dicitur) vim vi repellere licet.\footnote{Iustinianus \emph{Digesta} XLIII.16.1.27 ``Vim vi repellere licere Cassius scribit idque ius natura comparatur: `apparet autem,' inquit, `ex eo arma armis repellere licere.' ''} Quam sententiam novus quidam poëta etiam latius interpretatus est his verbis: ``Pellere vim vi iura sinunt, et vulnere vulnus.''\footnote{Canon Law}

Quid ad hoc respondes?

R. Istud quidem permittunt ethnicorum leges. Sed lex divina longe aliter loquitur.

T. Quid est divina lex?

R. Idem quod Dei verbum.

T. Quicquid ergo fit contra Dei verbum, estne peccatum?

R. Procul dubio peccatum est.

T. Potesne ista probare ex divinis litteris?

R. Quidni possim? Nihil facilius.

T. Adfer mihi, quaeso, sententias.

R. Non credis id quod est omnibus in confesso?

T. Immo vero indubitanter credo.

R. Quid igitur opus est testimoniis?

T. Ut contradicentibus certo respondere possim.

R. Recte intelligis. Sed quia non possis ex tempore memoriae mandare, expecta dum tibi describam in chartula, in qua etiam sit locorum annotatio ut possis quibus voles etiam digito demonstrare.

T. Optimum ex tempore consilium. Quanto istud mihi commodius fuerit? Sic enim per otium ediscam. Ut ad manum habeam, quoties erit opus. Sed quando mihi dabis?

R. Cras (si libet) huc ad me revertere.

T. Quota hora?

R. A meredie prima.

T. Sat habeo, interea vale.

R. Te servet Dominus Deus.

\subsection{Colloquium 32}
\emph{Simulatione quadam hic quaesitum est colloquendi argumentum. Lex scholastica, ne pueri res suas alienare audeant. Christi exemplo adversa ferenda propter ipsum. Nisi Divini Spritius auxilio nihil fieri potest boni. Precibus autem illud est impetrandum.}

Grimomdus, Blevetus

G. Vis emere hoc cingulum?

B. Cur emerem? Unum mihi satis est. Tu vero cur vis vendere?

G. Quia sunt mihi duo.

B. Nec tamen licet ut vendas, nisi vis in poenam incurrere.

G. Quid vetat me vendere res meas? Nihil adhuc tuum habes.

G. Eho, nihil? Unde probas istud?

B. Quia nondum es tui iuris, sed sub patria potestate. Denique vis audire breviter tibi istud non licere?

G. Maxime velim.

B. De hoc lex est scholastica, cuius haec est sententia: ``Pueri iniussu parentum, nec vendant aliquid, nec emant, nec permutent, nec alienent quovis alio modo. Qui contra fecerit, verberibus plectetur.''

G. Istud ego non ignorabam. Sed volebam periculum facere an constans esses in observandis legibus.

B. Tu igitur es simulator.

G. In hac simulatione nihil video esse mali, num tu interpretaris male.

B. Minime vero, nihil enim nocuisti mihi.

G. Quid si nocuissem?

B. Tulissem aequo animo ut Christianum decet.

G. Utinam adversa omnia sic feramus propter Christum, qui nihil non tulit salutis nostrae causa!

B. Feremus certe, si exemplum eius semper nobis proponamus ob oculos.

G. Difficile id quidem est.

B. Immo impossibile nisi illius Spiritu semper adiuvemur, quod quidem assiduis precibus est impetrandum.

G. O quam suavi sermone tantillum otii consumpsimus!

\subsection{Colloquium 33}
\emph{Non est mentiendum, ne in rebus quidem levioribus, apud Deum. Nullum est leve peccatum. Peccata nostra Deus Christo condonat. Christus deprecator noster et advocatus. Nemo non peccat saepissime, etiam in bonis operibus. Quid agendum sit ut peccata remittantur. Paulus veniam consecutus et qua re. Divini verbi studium inter exercitationes scholasticas annumerandum. Quorsum in litteris humanis instituendi sint pueri.}

Sartor, Odetus

S. Ecquid hodie precatus es domi?

O. Cur quaeris istud?

S. Quia non interfuisti matutinae precationi.

O. Quid scis?

S. Observavi.

O. Atqui tu non es observator.

S. Non sum.

O. Quamobrem igitur observabas?

S. Quia es mihi carissimus.

O. Quid tum?

S. Dolebit mihi si vapulabis.

O. Quid? An ego notatus fui?

S. Etiam dubitas?

O. Cum recitaretur catalogus, nemo me excusavit?

S. Nemo, quod sciam.

O. Si me tantopore amas (ut dicis) cur tute non excusasti me?

S. Quid causae dixissem?

O. Commentus esses aliquid.

S. Ergo mentitus essem?

O. Quid inde?

S. Sed ex verbo Dei mentiri vetitum est.

O. Fateor, sed erat leve mendacium.

S. Nihil leve iudicandum est, quo Deus offenditur.

O. Negare id non possum. Sed levia ista facile remittet nobis propter Iesum Christum, qui est deprecator noster et advocatus. Nam quotusquisque non peccat quotidie saepissime?

S. Profecto nullus. Quinetiam vix precamur aut aliquid boni operamur in quo non insit aliqua peccati species.

O. Quid ergo futurum esset nobis, nisi Deus facile ignosceret? Unde et quotidie precantes dicimus, ``Remitte nobis debita nostra.''\footnote{Secundum Mattheum 6:12}

S. Nihil dubitare debemus quin remittat, si errata nostra serio et vere agnoscamus, si ex animo petamus veniam, si fidem habeamus nobis ignosci.

O. Quid igitur restat?

S. Illud restat ut neque delectemur peccatis, neque in his persistamus, neque malitiose, scienter, et de industria peccatum committamus. Nimis enim multa sunt quae per carnis infirmitatem peccamus et etiam per ignorantiam.

O. Nihil dubito de peccatis illis quae ex carnis imbecillitate perpetramus, qualis fuit Petri abnegatio. Sed qui fit ut peccemus\footnote{D. peccemus S. peccamus} per ignorantiam?

S. De hoc Pauli exemplum habes, qui scribens ad Timotheum, in priore epistola palam profitetur se, quamvis ecclesiam Dei persecutus fuerit, tamen veniam esse consecutum quia ignorans id fecerit.

O. Isto exemplo mihi abundo satisfecisti (nam et ego id legisse memini). Sed scin’ tu quoto epistolae capite id scriptum sit?

S. Equidem non soleo vexare memoriam in retinendis capitum numeris. Mihi hoc tempore satis esse videtur aliquot sententias quasi raptim annotare quas memoriae mandare liceat, si quid interdum otii suffurari possim ex quotidiana studiorum praesentium occupatione.

O. Utinam ego quoque id facere possem!

S. Quid impedit?

O. Vix ego queo satis praeceptori facere in exercitationibus scholasticis, tantum abest ut aliis studiis aliquid temporis impertiri concedatur.

S. Satis profecto quotidie sumus occupati. Sed haec tamen ferenda sunt quandiu nobis opus esse, ipsi Deo et parentibus nostris visum fuerit.

O. Ego propterea libenter fero ac tolero omnes eiusmodi labores.

S. Qua spe toleras?

O. Quia futurum spero ut primi sint gradus quibus ad maiora aliquando perveniam.

S. Sed de his alias pluribus. Nunc agamus quod instat.

O. Quid istud\footnote{D. istud  S. istuc} est?

S. Audin' tu ad cenam signum dari?

O. Bonum signum nuntias, iam sentiebam esuriem.

S. Nimirum quia merendam praetermisisti.

O. Utinam ea tam facile semper abstinere possem quam hodie carui libenter!

S. Ego vero non libenter careo nisi occupatus necessario negotio.

\subsection{Colloquium 34}
\emph{Confabulatio de profectionis in Italiam causa. Locorum fama et rerum novarum cupiditas nos inducit ad peregrinandum. Laus Italiae. Vini Italici praestantia. Dei laudandi materiae ex eius beneficiis. Amavit nos Deus usque ad delicias. Humanae cogitationes nos a divinis rebus avocant. Humana inconstantia. Roma, fons abominationum. Bestia magna, Papa, ut spectetur, per urbem portatur. O monstrum horribile! O idolum execrandum!}

Phrygio, Stephanus

P. Salve, ambulator optatissime.

S. Et tu salvus sis, cessator occupatissime.

P. Satisne recte vales, mi Stephane?

S. Immo rectissime, quae est Dei optimi maximi benignitas.

P. Equidem ex animo gaudeo, tibique vehementer gratulor reditum istum incolumem. Ubi fuisti hoc toto anno?

S. In Italia.

P. Quam ob causam animum induxeras illuc proficisci?

S. Ob famam regionis, de qua tam multa ubique praedicantur. Nec ignoras quam simus rerum novarum cupidi.

P. Sic est natura comparatum. Sed quid illic invenisti?

S. Certe multo plura quam ex fama audiveram.

P. Sed multa (credo) vidisti quae minime velles.

 
S. Nempe scelera. Sed, quod ad regionem attinet, terra est longe fertilissima, omni optimorum fructuum genere valde abundans, praecipue vino praestantissimo.

P. Scilicet ea potissimum res tibi arridebat.

S. Ut verum fatear, mire afficiebat palatum; nam quale dicas hoc nostrum vinum esse? Villum vere dicas, si cum illo compares.

P. Inde ergo tibi se afferabat pulchrum Dei laudandi argumentum.

S. Pulcherrimum. Sic enim saepe cogitabam quam bonus es, Domine Deus, qui nos amavisti usque ad delicias! Non enim solum ea creasti nobis ad victum quae terra sponte sua producit, sed etiam tot genera rerum delicatissimarum, quae (si moderate sumamus et cum gratiarum actione) et corpus suavissime nutriunt, et ipsum animum mirifice exhilarant. O quibus verbis, quibus operibus, satis digne glorificemus nomen tuum Domine? Denique sic afficiebar animo ut nihil magis cuperem quam divinas laudes semper in ore habere. Sed (proh dolor) aliis atque aliis subinde conceptis cogitationibus ignis ille paulatim extinguebatur.

P. Istud apud me non est novum, nam saepe tale quid mihi solet accidere.

S. Ea est naturae nostrae inconstantia.

P. Horis fere omnibus istud experimur. Sed quid tandem egisti in tua Italia?

S. Invisi animi grata, aliquot urbes celebriores. Alicubi etiam studui aliquandiu.

P. Quas urbes invisisti potissimum?

S. Multas quidem vidi in transitu. Sed paucas contemplatus sum otiosus, nimirum: Genuam, Florentiam, Venetias, denique Romam illam, quae olim mundi caput dicebatur. Nunc autem est omnium abominationum fons et origo.

P. Vidistine magnam illam bestiam?

S. Vidi obiter, cum per vicos (opinor) spectaculi gratia portaretur.

P. Sed (ut ad rem) in quibus tandem oppidis commoratus es studiorum gratia?

S. Roma rediens transivi Bononia, Patavio, Mediolano. In eorum oppidorum singulis menses circiter tres versatus sum in vario litterarum genere. Volui enim e singulis paucula quasi degustare.

P. Quid autem vidisti novi in tot celeberrimis oppidis?

S. Rogas? Fere mihi omnia nova videbantur. Sed longum foret tibi omnia narrare, praesertim nunc, cum mihi est aliquo properandum.

P. Quo tandem?

S. Ad patruum, qui me ad cenam invitavit.

P. Nolo igitur te remorari diutius. Sed quando licebit nobis magis otiose confabulari?

S. Cras a prandio, si volueris.

P. Ego vero id percupio.

S. Ad horam igitur primam expecta me in cubiculo.

P. Fiet, hora est ad merendam opportuna.

\subsection{Colloquium 35}
\emph{Officia domestici apud ludimagistrum hypodidascali.}

Magister, Hypodidascalus

M. Quid quod hisce diebus inter nos egeramus? Satisne cogitasti?

H. Etiam atque etiam cogitavi.

M. Ecquid placet tibi conditio quam obtuli?

H. Maxime.

M. Quid mensa, seu convictus?

H. Nihil in ea re desidero.

M. Quid restat igitur?

H. Ut (si tibi non molestum est) praescribas mihi quas operas a me tibi praestari velis.

M. Id vero est aequissimum. Accipe igitur praecipua officii tui capita, quorum hoc primum est: 
quotidie mane diligenter curare ut omnes domestici mei discipuli mature cubitu surgant pro ratione temporis, tum hiberni, tum aestivi. Ubi surrexerint, ea curent quae ad cultum et munditiem corporis pertinent. Postremo, ut adsint privatae nostrae precationi.

Secundum, ter quotidie in aulam deducere; mane scilicet, et ante horam undecimam ac tertiam pomeridianam. Illic (nisi egomet adero) expectare donec aliquis ex doctoribus adfuerit. Interea curare recitandos catalogos et precationem dicendam. Item observare sedulo nunquis ex ipsis doctoribus absit ab auditorio suo; si quis aberit, mihi statim renuntiare aut eius partes agere.

Tertium, manere cum pueris domesticis quoties non docentur in scholis suis. Interea minores ad lectionem et scripturam instruere, ceterorumque repetitiones audire, quantum tempus et opportunitas patietur. Omnes denique in officio retinere, admonere, arguere, obiurgare, virgis etiam, ubi opus fuerit, castigare.

Quartum, feriatis diebus eos ad sacras contiones ordine deducere ac domum similiter reducere.

Quintum, quoties ludere permissum erit, subinde observare ne quid praeter officium et bonos mores vel factis vel dictis admittant.

Sextum, suppeditare illis ex pecunia quam tibi in manus dabo chartam, pennas, atramentum, et alia quaedam duntaxat parvi pretii necessaria. Eaque omnia in expensorum codicem referre. Id autem Mercurii et Sabbati potissimum diebus fieri solet.

Septimum, quae ad eorum libros, vestimenta, et curam corporis pertinebunt non negligere. Hoc est, interdum ab illis librorum et vestimentorum rationem exigere, valetudinis et cultus coporis rationem habere, et alia eiusmodi (in pueris praesertim minoribus) curanda et observanda.

Octavum, docere pueros tum in classe mea, tum in ceteris, praeter tres superiores, si quando necessitas postulabit.

Nonum, interdum (si opus fuerit) me et domi et foris in privatis negotiis adiuvare.

Hactenus audisti quae mihi abs te praestari velim officia, quaeque etiam ab aliis domesticis hypodidascalis exigere soleam. Eorum tamen omnium adeo severus exactor fuero quin ipse, quoties per otium licebit, aliqua tibi remittam in quibus ego quasi vicarii partes agam. Intellextin' haec omnia?

H. Ego vero diligenter omnia. Sed unum te oro ut ad memoriam renovandam, des mihi eorum commentariolum et simul cogitandi ac deliberandi spatium.

M. Quantum temporis postulas?

H. Diem unum naturalem.

M. Ego vero duos integros dabo. Interea (ut coepisti) nullo tuo sumptu nobis perges convivere et commorari tam libere, quam si esses domi tuae.

H. Istud non sine humanitate facis, quo fit ut maiori beneficio me devincias.

M. Habebis a prandio quod requiris commentarium, cum primum mea tibi manu conscripsero.

H. Quid si mihi dictares?

M. Malim egomet scribere, ne quid forte inter dictandum excidat.

H. Ut libet.

\subsection{Colloquium 36}
\emph{Studiosus quidam adolescens apud familiarem conqueritur de habitationis suae in commoditate ad litterarum studia. Cupit in aedibus scholasticis habitare. Narrat cur non liceat. Accenditur a consdiscipulo. Consultant una de ineunda ratione ad rem perficidendam. Consilium proponit amicus, docetque exequendi modum. Hortatur ad Deum invocandum. Non prodest consilium nisi iuvante Deo. Deus qui dat consilium bonum, dat etiam effectum. Exemplum quale est supra in colloquio huius libri primo\footnote{S. tertio S. Corr. primo} et vicesimo. = D. 4.25}

Quaestor, Benignus

Q. Quam doleo me non interfuisse mane repetitioni vestrae!

B. Cur non venisti in ludum citius, ut fere soles?

Q. Me miserum! Non surrexi in tempore.

B. Quamobrem?

Q. Quia nemo me expergefecit.

B. Quis te solet excitare?

Q. Hospes noster aut eius ancilla. Sed absente illo ancilla saepe obliviscitur, aut certe negligit.

B. Ubi erat hospes?

Q. Sub auroram prodierat ad sua negotia, ut postea rescivi.

B. Quid hospita, nihilne curat?

Q. Quid putas eam curare? Quotidie ex quo surrexit, semper intenta est partim curandis filiolis, partim ceteris domesticis rebus.

B. Nullosne habes contubernales scholasticos?

Q. Prorsus nullos.

B. Ah puer infelix, qui neminem habes quocum de studiis conferas!

Q. Ob eam rem mea est miserrima conditio, quantum ego iudico. Non enim possum arbitrio meo studere propter tantam mercatorum turbam, qui domum illum frequentant et mihi toto die obstrepunt.

B. Non habes tibi cubiculum?

Q. Quid mihi prodest habere? Est enim ita coniunctum gradibus et cochleis ut ne felis quidem aut ascendat aut descendat quin feriat aures meas aliquis strepitus.

B. Magna profecto molestia.

Q. Illa vero multo maior, quod supra meum cubiculum est amplissimum conclave, ubi merces asservantur. Unde fit ut horis omnibus aliquae graves sarcinae vel importentur vel exportentur.

B. O Deum immortalem! Quomodo illic potes vivere?

Q. Quid ais, ``vivere''? Equidem non vivo, sed langueo potius. Neque unquam mihi videor esse liber, nisi cum sum in schola tecum una et cum ceteris nostris condiscipulis.

B. Quam doleo vicem tuam!

Q. Utinam liceret mihi tecum habitare in his aedibus scholasticis!

B. Nihil esset mihi iucundius. Sed quid impedit?

Q. Patris vetus amicitia cum illo hospite meo.

B. Deberes patrem admonere studiorum tuorum incommodis.

Q. Saepe quidem monui et coram et per litteras.

B. Quid ille respondit?

Q. Frustra monetur, quasi surdo narratur fabula.\footnote{Terentius \emph{Heauton Timorumenos} 222: ``ne ille haud scit quam mihi nunc surdo narret fabulam.''}

B. Quid ita?

Q. Quia nunquam in discendi ludo versatus est. Ideoque in studiorum ratione nihil intelligit.

B. Ego tamen, si mea res ageretur, omnem moverem lapidem ut voti compos efficeret.

Q. Quid si praeceptor ipse ad patrem meum scriberet?

B. Nunquam istud illi persuaderes.

Q. Cur non?

B. Quia non vult ambire quempiam ut discipulorum turbam sibi comparet. Abhorret enim ab omni tum ambitione, tum avaritia.

Q. Quid igitur mihi faciendum suades?

B. Unicum habeo consilium.

Q. Ne mihi, obsecro, reticeas.

B. Ea res per amicos tentanda est.

Q. Idem mihi quoque aliquando in mentem venerat sed nunquam ausus sum experiri.

B. Quid dubitas?

Q. Vereor ut hoc parum succedat.

B. Rei exitus in manu Domini. Sed quid tentare nocebit?

Q. Tentemus sane. Nihil enim mali (ut confido) inde potest accidere. At ego nescio qua ratione hic utendum sit.

B. Dic mihi, non expectas ut brevi pater in hanc urbem veniat?

Q. Spero venturum propediem.

B. Quando igitur?

Q. Ad Kalendas Quintiles.

B. Optime est. Scin' igitur quid sit opus facto?

Q. Doce, quaeso.

B. Fac singulatim convenias duos aut tres ex paternis amicis praecipuis, qui sunt viri graves et honorati. Nempe, ut plus valeat eorum auctoritas apud patrem tuum.

Q. Bene mones, quid illis dicam?

B. Narrabis diligenter omnes incommoditates studiorum tuorum.

Q. Nihilne amplius?

B. Docebis insuper quonam modo tibi provideri possit ut tempus redimis, quod apud istum hospitem tam misere hactenus perdidisti, et nisi eo remedio tibi mature consulatur, actum esse de studiis tuis et eorum progressu. Denique ne ante destiteris monere, orare, obsecrare, donec persuaseris ut tibi promittant se aucturos esse serio cum patre tuo negotium.

Q. Quid si recusabunt?

B. Vix fieri potest ut recusent omnes.

Q. Non est veresimile, praesertim cum sint mei amantissimi et mihi patris nomine gratificentur adeo libenter.

B. Ad haec res ipsa urgebit eos nempe tanta studiorum tuorum iactura.

Q. Pluribus verbis opus non est. Auxilio Dei fretus aggrediar primo quoque tempore.

B. Sed interim memor esto, ut in divinas preces dies noctesque incumbas.

Q. Ipso volente Deo, id curabo pro viribus. Satis enim scio nullum consilium mihi esse profuturum, nisi quoad ille iuverit.

B. Sed iam tempus est ut domum te recipias ne forte hospes offendatur. Quid cessas?

Q. Cogito ne quid praetermiserim de quo esses admonendus.

B. Si quid alterutri nostrum praeterea occurrerit, cras otiose tractabimus.

Q. Vale igitur, mi Benigne, et perge, quaeso, me tuis precibus adiuvare quemadmodum iuvisti optimo consilio.

B. Domino Deo profecta sunt omnia, qui ut consilium dedit, sic dabit effectum.

Q. Ita fore confido. Iterum vale.

B. Vale, Quaestor suavissime.

\subsection{Colloquium 37}
\emph{Simulata oratione quidam condiscipulum familiariter obiurgat quod pecuniam a patre non petierit. Condiscipuli super ea re prudens responsum. Oblata occasio non est negligenda. Locus ex \emph{Distichis Moralibus}. Natura tenacissimi sumus eorum quae rudibus annis percepimus. Beneficiorum divinorum agnitio. Iocari et irridere. Patrum prudentia in suppeditando filiis quae opus sunt. Pueri tractandae pecuniae minime sunt idonei. Improbi pueri neque obiurgationibus neque flagris moventur. Omnis boni laus soli debetur Deo. Fallendi consilium nemo dare debet. Simulatio vultu tegitur. Honestae confabulationes in otio utiles sunt. Latina lingua quibus adiumentis potissimum comparetur.}

Athanasius, Beniaminus

A. Siccine, me insciente, abiit pater tuus ut mihi non licuerit eum convenire?

B. Cur a prandio non venisti in diversorium eius?

A. Quia putabam tantum cras illum esse discessurum.

B. Ego quoque idem arbitrabar; sed noluit occasionem praetermittere quae se ex tempore obtulerat.

A. Adhuc ille meminit Catonis distichon illud: \begin{verse} ``Quamprimum captanda tibi est occasio prima,\\ Ne rursus quaeras, quae iam neglexeris ante.''\footnote{Disticha Catonis IV.45:\\``Quam primum rapienda tibi est occasio prima,\\ne rursus quaeras, quae iam neglexeris ante.''}\end{verse}

B. Illud opusculum sic memoria tenet, ut in eo videatur aetatem contrivisse.

A. Vide quanta vis sit memoriae in iis quae rudibus annis didicimus.

B. Ea est Quintiliani super hac re sententia, cuius verba (ut opinor) meministi.

A. Memini. Sed (ut ad rem) quae fuit patri occasio, ut ante discerit quam instituerat?

B. Quidam Lugdunenses quibuscum ad mercatum huc venerat.

A. Aderasne cum profectus est?

B. Praestolabar illum in diversorio.

A. Unde scieras mutasse consilium de profectione?

B. Eram in prandio cum inter ipsos convenerat ut expeditis quibusdam reliquis in urbe negotiis equos sub horam secundam conscenderent.

A. Quod superest, satisne ex animi sententia rem suam fecit?

B. Ita feliciter, ut me ob eam rem ad divinas laudes vehementer hortatus fuerit.

A. Tu nunc igitur (opinor) bene nummatus redis.

B. Mene rides?

A. Cur ego id facerem?

B. Pro tua libidine.

A. Quasi vero animi gratia soleam irridere ceteros.

B. Atqui ita putabam.

A. Longe aberras. Nam aliud est iocari, aliud irridere. Alterum caret vitio, estque inter amicos satis frequens; alterum est vitiosum atque odio dignum, utpote quod ex contemptu fere proficiscitur.

B. Ignosce igitur mihi.

A. Non gravis\footnote{S. levis S. Corr. et  D. gravis} est culpa. Sed dic, rogo, nihil tibi pecuniae dedit pater?

B. Ne petivi quidem.

A. Tamen sponte dedit.

B. Aliquantulum.

A. Quantum igitur?

B. Perpusillum.

A. Dic, sodes.

B. Cur tam avide inquiris?

A. Ut amicorum more tibi gratuler.

B. Nihil est gratulatione dignum.

A. Fatere tandem quid sit.

B. Soli asses quinque.

A. Hui tantillum! O stulte, qui non petieris duos aut tres decusses argenteos.

B. Non ausus sum.

A. Quid verebare?

B. Ut plane denegaret atque aegre ferret quod peterem.

A. Nunquam id fecisset, modo petendi causam addidisses.

B. Credo equidem. Sed quid causae attulissem?

A. Rogas? Nonne sunt res sexcentae quibus indiget usus scholasticus!

B. Multae sunt, fateor.

A. Tu vero adeone abundas rebus omnibus ut tibi desit nihil?

B. Immo desunt plurima, sed quibus facile caream. Praeterea satis novit pater quae mihi opus sunt\footnote{D. sunt S. sint} tum studiorum causa, tum ad victum cultumque corporis.

A. Novit quidem sed alia multa sunt illi et curanda et cogitanda.

B. Credo esse illi praecipuam liberorum curam.

A. Sed nimis ab eo remotus es.

B. Sine me pervenire quo volo.

A. Age, sino.

B. Novit etiam pater me nondum esse idoneum ad recte tractandam pecuniam.

A. Cur non? An tu ad eam rem non satis aetatis habes atque prudentiae?

B. Istinc absum longissime. Itaque pater dedit praeceptori in mandatis ut omnia mihi suppeditet ad usus vitae et studiorum necessaria, ad quam rem praebet illi quantum satis est pecuniae.

A. Esto.

B. Ergo si quid a patre peterem, me statim ad praeceptorem remitteret, fortassis etiam irasceretur et me graviter obiurgaret.

A. Facile est obiurgationem pati, modo ne sequantur verbera.

B. Facile est, credo. Sed iis duntaxat quos neque pudor movet neque ulla parentum reverentia. Ego autem ipsa verbera ferre malim quam patris irati obiurgationem. Ex quo fit ut sedulo caveam ne quid illi praebeam causae ad irascendum, id enim sub quinto praecepto Divinae Legis continetur.

A. Facis ut pium decet adolescentem.

B. Eius rei laus non mihi sed soli Deo tribuenda est.

A. Nempe a quo proficiscitur quicquid nobis inest boni.

B. Faxit ille ut quae bona inspirat nobis, ea sequamur animo promptissimo. Sed, ut ad te redeam, serione reprehendebas quod nullam pecuniam a patre rogassem?

A. An ego te vellem ad fallendum patrem inducere?

B. Mihi quidem non fit verisimile, me tamen ipsum fefellisti.

A. Quomodo?

B. Quia serio loqui videbaris adeo apte vultum verbis ipsis accommodabas.

A. Sed quid censes de hac nostra confabulatione?

B. Argumentum satis apertum dedisti nobis in hoc otio nostro vespertino.

A. Ecquid habuit sermo noster quod reprehendisset observator, si forte (ut solet) nos observasset ex insidiis?

B. Nihil, ut opinor.

A. Profecto verum est quod saepe nobis praeceptor inculcat.

B. Quid illud est?

A. Latinae linguae copiam et facultatem comparari his potissimum rebus: saepe scribendo, confabulando, legendis auctoribus, Gallica Latine aut Latina Gallice convertendo.

B. Ergo his rebus diligenter nos exerceamus, adiutore Domino Deo, in cuius manu sita sunt studia nostra omnia.

A. Idem faxit ut eius erga nos beneficia vero cultu dignisque laudibus perpetuo celebremus.

B. ``Hoc opus hoc studium parvi properemus et ampli,''\footnote{Horatius \emph{Epistulae} I.3.28} haec sit votorum, summa suprema, precor.

A. Sed audi horologium.

B. Nos opportune admonet, itaque desistamus.

A. Alioqui solis occasus nos hic opprimet.

\subsection{Colloquium 38}
\emph{Tres pueri paedagogum suum quasi deimproviso adeunt. Suam quisque pronuntiat sententiam ut ipsos ducat ambulatum. Ille concedit libenter. Sic studiosos pueros tractare decet ut sint ad suum munus obeundum alacriores.}

Honoratus, Vivianus, Pratensis, Paedagogus

\begin{verse}[\versewidth] \flagverse{H.} ``Quod caret alterna requie, durabile non est,\\ Haec reparat vires, fessaque membra novat.'' --- Ovidius.\footnote{\emph{Heroides} IV.89}\end{verse}

V. ``Nec me offenderit lusus in pueris, est et hoc alacritatis signum.'' --- Quintilianus.\footnote{\emph{Institutio Oratoria} I.3.10}

 
Pr. ``Nulla res est, quae perferre possit continuum laborem.'' --- Quintilianus.\footnote{I.3.8}

Pae. Video quorsum spectant ista, nimirum ut vos ambulatum ducam. Sed eandem cantilenam semper fere recantatis ut solent nostrae aviculae.

H. Quid ergo vis dicamus, praeceptor?

Pae. Dicite posthac suam quisque sententiam ex Novo Testamento.

V. Euge, nihil erit nobis facilius. Habemus enim in promptu multam earum copiam. Vis ergo, praeceptor, ut iam nunc incipiamus?

Pae. Sane velim, quando (ut ais) tanta est vobis copia.

V. Quis incipiet?

P. Tune, Honorate, vis honoris tui causa huius rei speciem edere?

H. Id ego libenter faciam, sed Dei honoris causa.

Pae. Laudo istud verbum. Divinus enim honor et gloria omnibus in rebus est praeferenda. Eia, incipe si quid habes.

H. ``Nisi abundaverit iustitia vestra plus quam scribarum et Pharisaeorum, non potestis ingredi in Regnum Caelorum.'' Mattheum quinto capite.\footnote{5:20}

V. ``Deposito mendacio, loquimini veritatem quisque proximo suo.'' Ad Ephesios capite quarto.\footnote{4:25}

Pr. ``Filii, oboedite parentibus in omnibus, hoc enim placet Domino.'' Ad Colossenses tertio.\footnote{3:20}

Pae. Euge, bonam speciem! Videte ut progressus respondeat. Hoc est, ut pergatis in posterum diligenter.

H. Qui nobis principium dedit, dabit idem successus prosperos.

Pae. Ita sperandum est. Parate vos ut prodire maturemus.

H. Mox aderimus paratissimi.

Pae. Sumite suum quisque pallium ut prodeatis honestius. Sed heus pueri!

Pr. Quid vis, praeceptor?

Pae. Videte ut Psalmos etiam adferatis, alicubi in umbra cantabimus.

Pr. Ita fiet ambulatio nostra iucundior.

\subsection{Colloquium 39}
\emph{In vacatione scholastica studiosi adolescentes libenter quidem relaxant animos, non tamen omnio cessant, nec otio illo per ignaviam abutuntur. Fides etiam inimico servanda. Agendi affectus addit celeritatem. Solus Deus suorum ductor. Exemplum studiosis observatu dignissimum.}

Unchetus, Marrellus

V. Fuistine hodie in gymnasio?

M. Ubi ergo fuissem? Tu vero quid agebas?

V. Eram domi occupatus.

M. Id evenit praeter morem tuum. Soles enim abesse rarius.

V. Quam possum rarissime. Quid autem actum est?

M. Nihil prorsus.

V. Ergone remissionem habemus?

M. Certo.

V. Quamobrem?

M. Propter mercatum hodiernum.

V. Quis dedit?

M. Ludimagister, permissu tamen rectoris.

V. Quid concessit?

M. Vacationem ab omni munere scholastico.

V. An in totum diem?

M. A mane ad occasum usque solis.\footnote{S. A mane usque ad occasum usque solis S. Corr. \ldots occasum solis D. A mane ad occasum usque solis} Tametsi diligenter et multis quidem verbis admonuit ut in otio de negotio cogitaremus ne cras in ludum veniremus imparati.

V. Quid igitur nos? Hoccine abutemur otio?

M. Id vero aetatem nostram decet minime.

V. Tu ergo quid putas facere?

M. Me recipere in musaeolum, nisi forte tibi magis placet in sesquihoram aliquo prodeamus ambulatum.

V. Egone recusarem? Immo nihil est quod nunc magis velim. Nam et nos interea tractabimus aliquem sermonem litterarum et simul corpus exercebimus.

M. Eamus igitur extra muros.

V. Quonam?

M. Usque ad ripam lacus.

V. Valde istud mihi arridet. Sed tu (si placet) me expectabis.

M. Quandiu?

V. Tantisper dum crepidas eo mutatum calceis.

M. Ubi vis expectari?

V. Ad portam Franciscanam.

M. At vide ne me fallas.

V. An ego amicum fallerem cum sciam etiam inimico servandam esse fidem?

M. Abi, festina! Ego dum te opperior aliquid interim legam.

---

V. Salve, Marcelle!

M. Quis iste salutator?

V. Ecce, redii.

M. Eho, tam cito! Mihi videris volasse.

V. Nimirum affectus ipse pedibus alas addidit.

M. Eamus nunc, ducente Deo.

V. Solus Deus est, qui suos ducit ac reducit.

M. Maturemus, satis longe hinc lacus\footnote{S. locus S. Corr. et D. lacus} abest.

V. Tanto melius prandebimus, perge.

\begin{center}\textsc{Finis}\end{center}
\pagebreak
\phantomsection
\addcontentsline{toc}{section}{Maturini Corderii admonitiones de suis colloquiis, versibus scriptae}
\section*{Maturini Corderii admonitiones de suis colloquiis, versibus scriptae}

\poemtitle*{Quomodo legenda sint a pueris haec colloquia\footnote{Haec poëtmata desunt in D.}}

\settowidth{\versewidth}{Nam citra studium leviter quaecunque leguntur,}
\begin{verse}[\versewidth]
Ad verbum non sunt haec ediscenda, sed horum\\
Perfaciles\footnote{S. Fiet ut S. Corr. Perfaciles} ediscas plurima, saepe legens.\\
Quae post succurrant nullo quaesita labore,\\
Si tamen attente singula verba notes.\\
Nam citra studium leviter quaecunque leguntur,\\
Rarius in pueri pectore fixa manent.\\
Quod si comparibus iunctis haec ipsa revolves\footnote{S. Quod si cum caro malis recitare sodali S. Corr. Quod si comparibus \ldots}\\
Maior erit frucuts, gratia maior erit.\\
Per te ipsum poteris magnam cognoscere partem,\\
Praeceptor faciet cetera plana tibi.\\
\end{verse}

\poemtitle*{Excusandae esse in colloquiis scholasticis eadem saepe repetita}

\settowidth{\versewidth}{Nil mirare eadem pueros iterare frequenter,}
\begin{verse}[\versewidth]
Nil mirare eadem pueros iterare frequenter,\\
Usu etiam doctis ista venire solent.\\
Complures hic sunt iisdem doctoribus usi,\\
Inde fit ut tales saepe loquantur idem.\\
\end{verse}

\poemtitle*{Ficta non esse nomina puerorum qui in praecedentibus colloquiis loquentes inducuntur}
\settowidth{\versewidth}{Horum quisquis adhuc vitali vescitur aura,}
\begin{verse}[\versewidth]
Nomina ne credas puerorum hic facta loquentum,\\
Discipuli ex magna parte fuere mei.\\
Horum quisquis adhuc vitali vescitur aura,\\
Si legat haec, nomen commeminisse potest.\\
\end{verse}
\end{document}